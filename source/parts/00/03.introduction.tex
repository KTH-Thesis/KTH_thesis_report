Research into decentralized control of interacting systems has been well
underway for the past three decades by academic, industrial or commercial,
civilian or military parties. Decentralized Control schemes and strategies are
now used in such areas as those concerned with controlling the trajectory and
formation of mobile agents, the area of jurisdiction of an air traffic
controller\cite{4459797}, or in the field of micro-robotics
\cite{iswarm}\cite{micron}. Focus has been given on research regarding unmanned
autonomous vehicles $-$ as in unmanned ground,
underground and air vehicles, better known as UGV's, UUV's and
UAV's \cite{1429425}\cite{Dunbar2006549}\cite{4389042}\cite{1470732}.
In principle, all above areas are concerned with controlling a network of
mobile agents under certain conditions, such as avoiding collisions with their
environment or with one-another, and with the actuation forces applied to each
agent having to satisfy certain constraints.

The nature of such problems is suitable for approach by the strategy of
Model Predictive / Receding Horizon - Control since it can directly incorporate
the consideration of such limitations, while being able to handle linear and
non-linear, continuous-time or discrete-time underlying systems $-$ see
\cite{FINDEISEN2003190}\cite{262032}\cite{grune2016nonlinear}, and the thorough
elementary review \cite{Mayne2000789}. In principle, under this method, each
agent solves a finite horizon optimal control problem (FHOCP) on-line given
information on its configuration and that of its neighbours and, in general,
information pertaining to its environment. The optimization problem is posed in
a way such that all relevant and necessary constraints are explicitly
included. In decentralized approaches, the dynamics of each system may not be
directly influencing the dynamics of other neighbouring systems (as a direct
source of uncertainty for instance), but may only do so indirectly, as
information on their state is incorporated in the cost function used, or in the
constraints considered (such as in the case of maintaining a set distance
between agents), or in both.

The literature on approaching the problem of decentralized stabilization of a
compound system of neighbouring agents is large and evolving by branching out.
In \cite{1470732} (\cite{00207170600867123}), the authors consider agents modeled
as discs whose sensing range is not unlimited, and whose dynamics are in
continuous time and are described by the single (double) integrator. In this case,
all agents are assigned a desired configuration that need not be known to
any other agent than the one assigned to. The problem is approached by
designing and utilizing decentralized navigation functions \cite{Lav06}.
However, this method is problematic in that, in practice, it requires
preposterously large actuation forces (up to $10^5$), it may easily give rise to
numerical instability, and it requires harsh assumptions on the initial
conditions of the problem (for instance, when an agent needs to bypass an
obstacle and the problem is symmetric, the method will direct it onto the
obstacle instead of making it overtake it). In \cite{Gustavi2010133}, the
assignation regime is altered to a leader-follower configuration, leading to the
added benefit of limiting the overall required sensor load, as resources are freed
by having only leading/anchor agents being able to discern global positions or
relative positions to landmarks.  In this case the dynamics of both leaders and
followers are based on the Laplacian consensus equation, and they directly
include the state of neighbouring agents.

In \cite{Dunbar2006549}, what is demonstrated in detail is a decentralized
multi-agent MPC stabilization approach that couples agents chosen as neighbours
during the solution of their optimization problems in the running cost but not
through their constraints. An interesting compatibility constraint is introduced
in order to keep a degree of consistency between an agent's states predicted by
its neighbours and their real values. Plant-model mismatch or uncertainties are
omitted.  By contrast, in \cite{1383977} and \cite{1429425}, robust centralized
and decentralized control laws are designed for UAV's. In order to attenuate
disturbances, they use a tightening of constraints regime that, in general
contrast to this work, propagates among agents, instead of through each of them.
This work extends that of \cite{1185106} in the multi-agent dimension,
and it translates the constraints' tightening regime in the face of disturbances
in the continuous-time case. \\[0.5ex]

This work considers the stabilization of a network of multiple non-point agents
whose dynamics are in continuous time and non-linear, whose sensing
capabilities are not unlimited, in the absence and in the presence of additive
bounded disturbances. In both cases, it designs control regimes which under
sufficient conditions stabilize the compound multi-agent system.
The problem is approached through the MPC framework in a
decentralized manner in that each agent solves its own FHOCP, having
availability of information on the current and planned actions of all agents
within its sensing range. Agents needed to maintain connectivity with one-another
are termed neighbours, and their determined interactivity ranges are explicitly
defined through the constraints of the optimization problems that each of them
solve.

This document is structured as follows:
Chapter \ref{chapter:notation_reliminaries} introduces the notation used
henceforth, and auxiliary prerequisites of the rest of the work.
Chapter \ref{chapter:prob_formulation} poses its motivating problem.
Part \ref{part:advocated_solutions} is composed of the detailed feasibility
and stability guarantee proofs for the cases of absent and present
disturbances.
Part \ref{part:simulations} verifies the reality of the designed control regimes
through simulations on four discrete scenarios, in ascending order of
difficulty.
Part \ref{chapter:conclusions_future_work} concludes this work and poses
relevant future directions of it.
Appendix \ref{chapter:proofs_of_lemmas} is composed of proofs of various lemmas
that this work introduces.
Appendices \ref{chapter:simulation_figures_without_disturbances} and
\ref{chapter:simulation_figures_with_disturbances} include the complete
collection of results derived from the all determined simulations.
