%-------------------------------------------------------------------------------
\section{The error model}

A feasible desired configuration
$\vect{z}_{i,des} \in \mathbb{R}^9 \times \mathbb{T}^3$
is assigned to each agent $i \in \mathcal{V}$, with the aim of agent $i$
achieving it in steady-state:
$\lim\limits_{t \to \infty} \|\vect{z}_i(t) - \vect{z}_{i,des}\| = 0$. The
interior of the norm of this expression denotes the state error of agent $i$:
\begin{align}
  \vect{e}_i(t) : \mathbb{R}_{\geq 0} \to \mathbb{R}^9 \times \mathbb{T}^3,\
  \vect{e}_i(t) = \vect{z}_i(t) - \vect{z}_{i,des}
\end{align}
The error dynamics are denoted by $g_i(\vect{e}_i, \vect{u}_i)$:
\begin{align}
  \dot{\vect{e}}_i(t) = \dot{\vect{z}}_i(t) - \dot{\vect{z}}_{i,des} =
  \dot{\vect{z}}_i(t) = f_i\big(\vect{z}_i(t), \vect{u}_i(t)\big) = g_i(\vect{e}_i(t), \vect{u}_i(t)\big)
  \label{eq:position_based_error_model}
\end{align}
with $\vect{e}_i(0) = \vect{z}_i(0) - \vect{z}_{i,des}$.

In order to translate
the constraints that are dictated for the state $\vect{z}_i(t)$ into constraints
regarding the error state $\vect{e}_i(t)$, we define the set
$\mathcal{E}_{i,t} \subset \mathbb{R}^9 \times \mathbb{T}^3$ as:
$$\mathcal{E}_{i,t} \triangleq \big\{\vect{e}_i(t) \in \mathbb{R}^9 \times \mathbb{T}^3 :
\vect{e}_i(t) \in \mathcal{Z}_{i,t} \ominus \vect{z}_{i,des}\big\}$$
as the set that captures all constraints on the error state with dynamics
\eqref{eq:position_based_error_model} dictated by problem \eqref{problem}.

On functions $g_i$ we make the following assumption:\\[1ex]
\begin{bw_box}
  \begin{assumption} (\textit{$g_i$ is Lipschitz continuous in $\mathcal{E}_{i,t} \times \mathcal{U}_i$})
  \label{ass:g_i_Lipschitz}

  Suppose that $\vect{e}_1, \vect{e}_2 \in \mathcal{E}_{i,t}$ and
  $\vect{u} \in \mathcal{U}_i$. Functions $g_i$ are Lipschitz continuous in
  $\mathcal{E}_{i,t} \times \mathcal{U}_i$ with Lipschitz constants $L_{g_i}$:
  \begin{align}
    \big\|g_i\big(\vect{e}_1, \vect{u} \big) - g_i\big(\vect{e}_2, \vect{u} \big)\big\|
      \leq L_{g_i} \big\|\vect{e}_1 - \vect{e}_2\big\|
  \end{align}

\end{assumption}
\end{bw_box}

If we design control laws $\vect{u}_i \in \mathcal{U}_i$,
$\forall i \in \mathcal{V}$ such that the error signal $\vect{e}_i(t)$ with
dynamics given in \eqref{eq:position_based_error_model}, constrained under
$\vect{e}_i(t) \in \mathcal{E}_{i,t}$, satisfies
$\lim\limits_{t \to \infty} \|\vect{e}_i(t)\| = 0$, while all system related
signals remain bounded in their respective regions,$-$ if all of the above are
achieved, then problem \eqref{problem} has been solved.

In order to achieve this task, we employ a Nonlinear Receding Horizon scheme.
