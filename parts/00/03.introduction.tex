Research into decentralized control of interacting systems has been well
underway for the past three decades by academic, industrial or commercial
parties. Decentralized Control schemes and strategies are now used in such areas
as those concerned with controlling the trajectory and formation of mobile
agents, the area of jurisdiction of an air traffic controller\cite{4459797}, or
in the field of micro-robotics \cite{iswarm}\cite{micron}. Focus has been given
on research regarding unmanned autonomous vehicles $-$ as in unmanned ground,
underground and air vehicles, better known as UGV's, UUV's and
UAV's \cite{1429425}\cite{Dunbar2006549}\cite{4389042}\cite{1470732}.
In principle, all above areas are concerned with controlling a network of
mobile agents under certain conditions, such as avoiding collisions with their
environment or with one-another, and with the actuation forces applied to each
agent having to satisfy certain constraints.

The nature of such problems is suitable for approach by the strategy of
Model Predictive / Receding Horizon - Control since it can directly incorporate
the consideration of such limitations, while being able to handle linear and
non-linear, continuous-time or discrete-time underlying systems
\cite{FINDEISEN2003190}\cite{262032}\cite{grune2016nonlinear}.
In principle, under this method each agent solves a finite horizon optimal
control problem (FHOCP) on-line given information on its configuration and that
of its neighbours and, in general, information pertaining to its environment.
The optimization problem is posed in such a way such that all relevant and
necessary constraints are explicitly included. In decentralized approaches,
the dynamics of each system may not be directly influencing the dynamics of
other neighbouring systems (as a direct source of uncertainty for instance), but
only indirectly, as information on their state is incorporated in the cost
function used, or in the constraints considered (such as in the case of
maintaining a set distance between agents), or in both.

The literature on approaching the problem of decentralized stabilization of a
compound system of neighbouring agents is large and continuing to soar in volume.
In \cite{1470732} (\cite{00207170600867123}), the authors consider agents modeled
as discs whose sensing range is not unlimited, and whose dynamics are in
continuous time and are described by the single (double) integrator. In this case
all agents are assigned a desired configuration that need not be known to
any other agent than the one assigned to. The problem is approached by
designing and utilizing decentralized navigation functions \cite{Lav06}. In
\cite{Gustavi2010133}, the assignation regime is altered to a leader-follower
configuration, leading to the added benefit of limiting the overall required
sensor load, as resources are freed by having only leading/anchor agents
being able to discern global positions or relative positions to landmarks.
In this case the dynamics of both leaders and followers are based on the
Laplacian consensus equation, and they directly include the state of
neighbouring agents.
