\section{Simulation results}

For compatibility with real situations, we assume that in the case
of disturbances, the signals affecting the agents are of the same nature
(consider for instance the case of UAV's affected by wind); the disturbance
signal considered was $\delta(t) = 0.1 * \sin 2t$. Therefore,
$\overline{\delta} = 0.1$





In this case the initial configurations of the three agents are
$\vect{z}_1$ $=$ $[-6, 3.5, 0]^{\top}$,
$\vect{z}_2$ $=$ $[-6, 2.3, 0]^{\top}$ and
$\vect{z}_3$ $=$ $[-6, 4.7, 0]^{\top}$.
Their desired configurations in steady-state are
$\vect{z}_{1,des}$ $=$ $[6, 3.5, 0]^{\top}$,
$\vect{z}_{2,des}$ $=$ $[6, 2.3, 0]^{\top}$ and
$\vect{z}_{3,des}$ $=$ $[6, 4.7, 0]^{\top}$.
Obstacles $o_1$ and $o_2$ are placed between the two at $[0, 2.0]^{\top}$
and $[0, 5.5]^{\top}$ respectively. The penalty
matrices $\mat{Q}$, $\mat{R}$, $\mat{P}$ were set to
$\mat{Q} = 0.7 (I_3 + 0.5\dagger_3)$, $\mat{R} = 0.005 I_2$ and
$\mat{P} = 0.5 (I_3 + 0.5\dagger_3)$, where $\dagger_N$ is a $N \times N$
matrix whose elements are chosen at random between the values $0.0$ and $1.0$.
The sampling time is $h = 0.1$ sec, the time-horizon is $T_p = 0.5$ sec, and
the total execution time given was $10$ sec.

Figure \eqref{fig:d_ON_res_3_2_errors_agent_1} depicts the evolution of the
error states of agent 1 through time.
Figures \eqref{fig:d_ON_res_3_2_distance_agents_13} and
\eqref{fig:d_ON_res_3_2_distance_obstacle_1_agents} show the evolution of the
distance between agents 1 and 3 through time, and the evolution of the
distance between all agents and obstacle $o_1$ respectively.
Figure \eqref{fig:d_ON_res_3_2_inputs_agent_2} shows the input signals
directing agent 2 through time. Last but not at all least, figures
\eqref{fig:d_ON_res_3_2_V} and \eqref{fig:d_ON_res_3_2_V_zoom} depict
the evolution of the quadratic Lyapunov function
$\vect{e}^{\top} \mat{P} \vect{e}$ through time for all three agents.
Boundary values for the inputs and distances are portrayed in the colour
\textcolor{cyan}{cyan}. The evolution of the trajectories of the agents in the
$x-y$ plane are omitted; they are (with minor variations) equivalent to those
in the case where disturbances are absent.


\noindent\makebox[\linewidth][c]{%
\begin{minipage}{\linewidth}
  \begin{minipage}{0.45\linewidth}
    \begin{figure}[H]
      \scalebox{0.7}{% This file was created by matlab2tikz.
%
%The latest updates can be retrieved from
%  http://www.mathworks.com/matlabcentral/fileexchange/22022-matlab2tikz-matlab2tikz
%where you can also make suggestions and rate matlab2tikz.
%
\definecolor{mycolor1}{rgb}{0.00000,0.44700,0.74100}%
\definecolor{mycolor2}{rgb}{0.85000,0.32500,0.09800}%
\definecolor{mycolor3}{rgb}{0.92900,0.69400,0.12500}%
%
\begin{tikzpicture}

\begin{axis}[%
width=4.133in,
height=3.26in,
at={(0.693in,0.44in)},
scale only axis,
xmin=1,
xmax=100,
xmajorgrids,
ymin=-2,
ymax=2.1,
ymajorgrids,
xlabel={time [iterations]},
ylabel={component magnitude},
axis background/.style={fill=white},
legend style={legend cell align=left,align=left,draw=white!15!black}
]
\addplot [color=mycolor1,solid]
  table[row sep=crcr]{%
1	-12\\
2	-11.0357002192746\\
3	-10.1245109521708\\
4	-9.18738732523736\\
5	-8.23154924519137\\
6	-7.25600583467408\\
7	-6.2631659074083\\
8	-5.25780136980059\\
9	-4.27333244553113\\
10	-3.27763742421729\\
11	-2.29853251285211\\
12	-1.37198786491555\\
13	-0.513946566055827\\
14	0.0866290836164753\\
15	0.33138041035715\\
16	0.278211220698197\\
17	0.118688830590016\\
18	0.00647574150103004\\
19	-0.0187887386409353\\
20	-0.0230860384846833\\
21	-0.0229816026277388\\
22	-0.0220125737276301\\
23	-0.0207292913011952\\
24	-0.0190744192791464\\
25	-0.0169487860226661\\
26	-0.0143111755836636\\
27	-0.0111535158476398\\
28	-0.00754790205743027\\
29	-0.00357592463467942\\
30	0.000622416356660126\\
31	0.00488990857758534\\
32	0.0090349251142785\\
33	0.0128596961588329\\
34	0.0161650305197812\\
35	0.01889382322331\\
36	0.0208567536679179\\
37	0.0219375796994614\\
38	0.0220554135827261\\
39	0.0211936365662433\\
40	0.0193926703104417\\
41	0.0167494567010094\\
42	0.0134104722062194\\
43	0.00949248645968904\\
44	0.00525068288979966\\
45	0.000859582999084783\\
46	-0.00346448609936722\\
47	-0.00753751154861172\\
48	-0.0111798934646757\\
49	-0.0142066685205217\\
50	-0.0165286485024373\\
51	-0.0183531372023017\\
52	-0.0194912850394756\\
53	-0.0199116128350599\\
54	-0.0196282442416332\\
55	-0.0186589490012905\\
56	-0.0170059877359394\\
57	-0.0147147348561698\\
58	-0.011827999830291\\
59	-0.00842174825563343\\
60	-0.00458837382858168\\
61	-0.00045853321938582\\
62	0.00381447036461368\\
63	0.00805035275773843\\
64	0.00260940547172175\\
65	-0.0010425697935171\\
66	0.00331403052141352\\
67	0.0143429171556353\\
68	0.0193439893720467\\
69	0.021457602898198\\
70	0.0217498498904845\\
71	0.0206919815817076\\
72	0.0185609308292246\\
73	0.0155790514150914\\
74	0.0119231608827535\\
75	0.00780552599280427\\
76	0.00345547618620625\\
77	-0.000921581656099039\\
78	-0.00513126751647384\\
79	-0.0089942051275916\\
80	-0.012325501528429\\
81	-0.0149730509867233\\
82	-0.0170513630039659\\
83	-0.018474108853382\\
84	-0.0191999246442107\\
85	-0.0192301116595906\\
86	-0.0185521324775858\\
87	-0.0171808123047706\\
88	-0.0151548791196427\\
89	-0.0125009973310031\\
90	-0.00928881202597453\\
91	-0.00560969043210593\\
92	-0.00156989624882284\\
93	0.00267437991835866\\
94	0.00696751609690567\\
95	-0.00218156594682402\\
96	-0.013323868471841\\
97	-0.0165252589239539\\
98	-0.00325341223677723\\
99	0.0115841437095531\\
100	0.0182144353958477\\
};
\addlegendentry{$\text{e}_{\text{1,1}}$};

\addplot [color=mycolor2,solid]
  table[row sep=crcr]{%
1	0\\
2	0.232537001594962\\
3	0.653275471281589\\
4	1.01945667498219\\
5	1.33945117250341\\
6	1.59848664757482\\
7	1.78426245146688\\
8	1.88166309297038\\
9	1.82649458201547\\
10	1.66845754155366\\
11	1.43764527864374\\
12	1.05249457773444\\
13	0.535389952281192\\
14	0.145657847269637\\
15	0.0348306291766969\\
16	0.0494237401720461\\
17	0.0557030359381731\\
18	0.0503819613603888\\
19	0.0463416831641861\\
20	0.041051999233332\\
21	0.034116320280306\\
22	0.0258175194564708\\
23	0.0165053459903434\\
24	0.00656213106080076\\
25	-0.00360611652788407\\
26	-0.0135849311528754\\
27	-0.0229695438984098\\
28	-0.0313818350926987\\
29	-0.0384873269599104\\
30	-0.0440081632653786\\
31	-0.0477339843523771\\
32	-0.0495291069379937\\
33	-0.0493364544489917\\
34	-0.0471781352666223\\
35	-0.0431512013286606\\
36	-0.0374256230917382\\
37	-0.0302352064829072\\
38	-0.0218685722633768\\
39	-0.0126574065995775\\
40	-0.00296391991424396\\
41	0.0068326142744975\\
42	0.0163494972472264\\
43	0.0252136920915775\\
44	0.0330771153003483\\
45	0.0396266275425059\\
46	0.0445970965967659\\
47	0.0477812248114942\\
48	0.0490386592950794\\
49	0.0483027635395975\\
50	0.0455860464606418\\
51	0.0409855863223959\\
52	0.034673461153531\\
53	0.0268965546175507\\
54	0.0179659830833399\\
55	0.00824344825448849\\
56	-0.00187515498637592\\
57	-0.0119767370166533\\
58	-0.0216502700895608\\
59	-0.0305045360937578\\
60	-0.0381853523506449\\
61	-0.0443899972248424\\
62	-0.0488791705755401\\
63	-0.0514856031421122\\
64	-0.0521709285600031\\
65	-0.0508759175599756\\
66	-0.0475916442887905\\
67	-0.0424043165131101\\
68	-0.0357495462336667\\
69	-0.0278248279246323\\
70	-0.0189230058576981\\
71	-0.00938817138520225\\
72	0.000407961091129701\\
73	0.010083121499107\\
74	0.0192587297701045\\
75	0.0275747567241138\\
76	0.0347022468263333\\
77	0.0403548907754757\\
78	0.0443002709415408\\
79	0.0463694310620696\\
80	0.0464646188549605\\
81	0.044564934256154\\
82	0.0407323440076896\\
83	0.0351078070268537\\
84	0.0279086513983767\\
85	0.01942072446536\\
86	0.00998590185421279\\
87	-1.22126683045102e-05\\
88	-0.0101654783856231\\
89	-0.0200602255875285\\
90	-0.0292949373538215\\
91	-0.03749827907979\\
92	-0.0443448081768035\\
93	-0.0495677211982028\\
94	-0.0529691426309335\\
95	-0.0544777977620498\\
96	-0.0540221801501761\\
97	-0.0516231872946929\\
98	-0.0470406302528498\\
99	-0.0407400444133052\\
100	-0.0332567957636802\\
};
\addlegendentry{$\text{e}_{\text{1,2}}$};

\addplot [color=mycolor3,solid]
  table[row sep=crcr]{%
1	0\\
2	0.472104827277756\\
3	0.390487647064949\\
4	0.349386174381163\\
5	0.288855376623065\\
6	0.219327328026152\\
7	0.136554574904723\\
8	0.0393923258160251\\
9	-0.172830913729049\\
10	-0.16468642388795\\
11	-0.32119317260279\\
12	-0.489441693798394\\
13	-0.615973879791215\\
14	-0.558681773693759\\
15	-0.334368656413803\\
16	-0.0984343812848431\\
17	0.0248035037497457\\
18	0.0425624138646781\\
19	0.00658511771093399\\
20	-0.00716728316980002\\
21	-0.0130072265804752\\
22	-0.0155601477230558\\
23	-0.0163599005346946\\
24	-0.0159835690026965\\
25	-0.0146869451941437\\
26	-0.0126729252601799\\
27	-0.0100847351796093\\
28	-0.00710092457684795\\
29	-0.00389097135358437\\
30	-0.000607118063854719\\
31	0.00259531131753384\\
32	0.00560245868498839\\
33	0.00829012885279613\\
34	0.0105240978562074\\
35	0.0123560775721962\\
36	0.0137198665072179\\
37	0.0146276821487434\\
38	0.0150627345378021\\
39	0.0150145292331992\\
40	0.0144758130191995\\
41	0.0134459732243992\\
42	0.0119456510571471\\
43	0.00995984821292621\\
44	0.00754873279013412\\
45	0.00476087187901179\\
46	0.00168339546464416\\
47	-0.00157473392900512\\
48	-0.00488212792155524\\
49	-0.00806865813958548\\
50	-0.0108547089882912\\
51	-0.0132973680171312\\
52	-0.0151179966331691\\
53	-0.0162471657322007\\
54	-0.0166180817822246\\
55	-0.0161994517321243\\
56	-0.0150310809658162\\
57	-0.0131844989548241\\
58	-0.0107629555282994\\
59	-0.00790084802253239\\
60	-0.00476277654601265\\
61	-0.00149347891174318\\
62	0.00174775672987662\\
63	0.00483831637474835\\
64	0.00520806849598277\\
65	0.00920822264775293\\
66	0.0186663489854961\\
67	0.0183130113106983\\
68	0.0173021002758017\\
69	0.0166396944975856\\
70	0.0160921557345657\\
71	0.0154000425882646\\
72	0.0143957354404437\\
73	0.0129963943598882\\
74	0.0111763519656662\\
75	0.00892271191165527\\
76	0.00628438612477311\\
77	0.00332553054683562\\
78	0.000142506772081861\\
79	-0.00314370990953125\\
80	-0.00638859839292801\\
81	-0.00935772136388043\\
82	-0.0119634774389988\\
83	-0.0140126452097575\\
84	-0.0154219142417219\\
85	-0.0160894984962923\\
86	-0.0159907841058687\\
87	-0.0151241509084665\\
88	-0.0135460241635221\\
89	-0.0113498859956052\\
90	-0.00865306085212153\\
91	-0.00561677919821012\\
92	-0.00239301195103154\\
93	0.000873783112918691\\
94	0.00402388699429119\\
95	0.00182564427600397\\
96	0.00671066141054492\\
97	0.0201170822979769\\
98	0.0330878078344038\\
99	0.0252289816508756\\
100	0.0207268413160135\\
};
\addlegendentry{$\text{e}_{\text{1,3}}$};

\end{axis}
\end{tikzpicture}%
}
      \caption{The evolution of the error states of agent 1 over time.}
      \label{fig:d_ON_res_3_2_errors_agent_1}
    \end{figure}
  \end{minipage}
  \hfill
  \begin{minipage}{0.45\linewidth}
    \begin{figure}[H]
      \scalebox{0.7}{% This file was created by matlab2tikz.
%
%The latest updates can be retrieved from
%  http://www.mathworks.com/matlabcentral/fileexchange/22022-matlab2tikz-matlab2tikz
%where you can also make suggestions and rate matlab2tikz.
%
\definecolor{mycolor1}{rgb}{0.00000,1.00000,1.00000}%
\definecolor{mycolor2}{rgb}{0.00000,0.44700,0.74100}%

%
\begin{tikzpicture}

\begin{axis}[%
width=4.133in,
height=3.26in,
at={(0.693in,0.44in)},
scale only axis,
xmin=0,
xmax=100,
xmajorgrids,
ymin=0.8,
ymax=2.2,
ymajorgrids,
xlabel={time [iterations]},
axis background/.style={fill=white},
axis x line*=bottom,
axis y line*=left,
legend style={at={(0.699,0.582)},anchor=south west,legend cell align=left,align=left,draw=white!15!black}
]
\addplot [color=mycolor1,solid]
  table[row sep=crcr]{%
0	1.01\\
100	1.01\\
};
\addlegendentry{$\text{d}_{\text{max}}$};

\addplot [color=mycolor1,solid]
  table[row sep=crcr]{%
0	2.01\\
100	2.01\\
};
\addlegendentry{$\text{d}_{\text{min}}$};

\addplot [color=mycolor2,solid]
  table[row sep=crcr]{%
1	1.2\\
2	1.13028748544624\\
3	1.20843567591419\\
4	1.09973251897809\\
5	1.40168431045553\\
6	1.68380298292713\\
7	1.66084436659695\\
8	1.88360008816731\\
9	1.98829687135464\\
10	1.98999051923358\\
11	1.9937575649837\\
12	1.99767488367426\\
13	2.00314221004763\\
14	2.0088428549773\\
15	1.68157856599251\\
16	1.33841237603773\\
17	1.22725239728932\\
18	1.20035658505603\\
19	1.19875891080645\\
20	1.19870178112698\\
21	1.19864519288565\\
22	1.1986146256676\\
23	1.19860457277368\\
24	1.19860255567031\\
25	1.19860340923153\\
26	1.19860513598442\\
27	1.1986070880808\\
28	1.1986090613695\\
29	1.19861091871173\\
30	1.19861256499516\\
31	1.19861390515348\\
32	1.19861491810557\\
33	1.19860317370551\\
34	1.19865097355751\\
35	1.19867294467597\\
36	1.19867842171633\\
37	1.19867772557985\\
38	1.19867610143894\\
39	1.19867446205803\\
40	1.1986729007508\\
41	1.19867137677415\\
42	1.19866987128414\\
43	1.19866834186544\\
44	1.19866718226884\\
45	1.1986662971627\\
46	1.19866505699853\\
47	1.1986641399038\\
48	1.19866350809043\\
49	1.19866318250216\\
50	1.19866320613601\\
51	1.19866361179797\\
52	1.19866438752901\\
53	1.19866697029939\\
54	1.19866764423678\\
55	1.19866899996185\\
56	1.19867094217547\\
57	1.1986727382721\\
58	1.19867473832284\\
59	1.19867675109324\\
60	1.19867863493054\\
61	1.19868032091887\\
62	1.19868172997167\\
63	1.19868283745241\\
64	1.19870246965061\\
65	1.19874360193547\\
66	1.19876449341507\\
67	1.19871554247192\\
68	1.1987007854248\\
69	1.19869480384684\\
70	1.19869189478297\\
71	1.19869002999072\\
72	1.19868846858383\\
73	1.19868697581124\\
74	1.19868551005639\\
75	1.19868402231687\\
76	1.19868240698117\\
77	1.19868150274237\\
78	1.19868041868908\\
79	1.19867965330894\\
80	1.19867920066431\\
81	1.19867906962599\\
82	1.19867931143717\\
83	1.19867992329953\\
84	1.19868089297335\\
85	1.19868213669356\\
86	1.19868379829671\\
87	1.19868526792606\\
88	1.19868718652181\\
89	1.19868948809198\\
90	1.19869152288882\\
91	1.19869344698404\\
92	1.19869520778262\\
93	1.19869673976627\\
94	1.19872529146706\\
95	1.19879776261745\\
96	1.19907764632389\\
97	1.19947516133022\\
98	1.19916886360199\\
99	1.19892234186361\\
100	1.19895242470188\\
};
\addlegendentry{$\text{d}_{\text{13,a}}$};

\end{axis}
\end{tikzpicture}%
}
      \caption{The distance between agents 1 and 3 over time. The maximum allowed
        distance has a value of $2.01$ and the minimum allowed distance a value
        of $1.01$.}
      \label{fig:d_ON_res_3_2_distance_agents_13}
    \end{figure}
  \end{minipage}
\end{minipage}
}


\noindent\makebox[\linewidth][c]{%
\begin{minipage}{\linewidth}
  \begin{minipage}{0.45\linewidth}
    \begin{figure}[H]
      \scalebox{0.7}{% This file was created by matlab2tikz.
%
%The latest updates can be retrieved from
%  http://www.mathworks.com/matlabcentral/fileexchange/22022-matlab2tikz-matlab2tikz
%where you can also make suggestions and rate matlab2tikz.
%
\definecolor{mycolor1}{rgb}{0.00000,0.44700,0.74100}%
\definecolor{mycolor2}{rgb}{0.85000,0.32500,0.09800}%
\definecolor{mycolor3}{rgb}{0.00000,1.00000,1.00000}%
%
\begin{tikzpicture}

\begin{axis}[%
width=4.133in,
height=3.26in,
at={(0.693in,0.44in)},
scale only axis,
xmin=0,
xmax=100,
xmajorgrids,
ymin=1.2,
ymax=7,
ymajorgrids,
xlabel={time [iterations]},
axis background/.style={fill=white},
legend style={at={(0.705,0.572)},anchor=south west,legend cell align=left,align=left,draw=white!15!black}
]
\addplot [color=mycolor1,solid]
  table[row sep=crcr]{%
1	6.4621977685614\\
2	5.77342365648139\\
3	5.13953125073559\\
4	4.45577518834934\\
5	3.67653432780378\\
6	2.92019357211574\\
7	2.35706002276644\\
8	1.93663782306074\\
9	1.82215494547734\\
10	2.17588134962582\\
11	2.7758627527564\\
12	3.44636525171151\\
13	4.15176564977888\\
14	4.870918478485\\
15	5.55685670784799\\
16	6.184512139916\\
17	6.38506713895202\\
18	6.44608518232096\\
19	6.4622858640847\\
20	6.46437540229293\\
21	6.46196456118633\\
22	6.45817383469753\\
23	6.45406853535106\\
24	6.45010074050291\\
25	6.44653462047598\\
26	6.44356702793008\\
27	6.44133994438276\\
28	6.43998547676853\\
29	6.43957445678675\\
30	6.44013003284419\\
31	6.44162963075208\\
32	6.44398500024544\\
33	6.45301658069248\\
34	6.45811082569937\\
35	6.45738582698246\\
36	6.46001327021509\\
37	6.4638109949674\\
38	6.46784656679855\\
39	6.47169837865593\\
40	6.47514312284882\\
41	6.47802562096869\\
42	6.48024856383819\\
43	6.48174578184368\\
44	6.48246465547821\\
45	6.48245479503116\\
46	6.48164199289328\\
47	6.48011042662648\\
48	6.47790195921122\\
49	6.47511714823907\\
50	6.47167313755442\\
51	6.46776070097008\\
52	6.46371968039077\\
53	6.45945329390456\\
54	6.45515685151227\\
55	6.45101387252769\\
56	6.44724499361798\\
57	6.44400905600602\\
58	6.44145892862497\\
59	6.43974158260337\\
60	6.43894536169521\\
61	6.43911275892206\\
62	6.44023896050537\\
63	6.44014387030188\\
64	6.44770577199336\\
65	6.45838130956439\\
66	6.46561163096152\\
67	6.46117431717493\\
68	6.46259623298884\\
69	6.46593794904962\\
70	6.46968950737199\\
71	6.47325909485781\\
72	6.47636463789235\\
73	6.47886005437408\\
74	6.48065747172313\\
75	6.4816868492063\\
76	6.481955263064\\
77	6.48146717326543\\
78	6.48023082711676\\
79	6.47829354658849\\
80	6.47577772177869\\
81	6.47249066178186\\
82	6.46877240306506\\
83	6.46467163233192\\
84	6.46054756538046\\
85	6.45625916516831\\
86	6.45203720782322\\
87	6.44810008301778\\
88	6.44462299068274\\
89	6.44177054969541\\
90	6.43970393105542\\
91	6.4385281815893\\
92	6.43830500740896\\
93	6.4390486503228\\
94	6.44064197815333\\
95	6.44678006646274\\
96	6.45985674861404\\
97	6.47531006239829\\
98	6.47659839146644\\
99	6.46600757613679\\
100	6.46550038484404\\
};
\addlegendentry{$\text{d}_{\text{1,o}_\text{1}}$};

\addplot [color=mycolor2,solid]
  table[row sep=crcr]{%
1	6.06712452484701\\
2	5.30039011238323\\
3	4.32139025573568\\
4	4.24628707766915\\
5	3.47758594970692\\
6	2.75071295342263\\
7	2.0304232122775\\
8	1.62598115043089\\
9	1.52118298776769\\
10	1.53476993563286\\
11	1.74681187187881\\
12	2.26790050461024\\
13	2.90235452290828\\
14	3.588956183504\\
15	4.29748744649038\\
16	5.01273709105748\\
17	5.73771774539431\\
18	6.00528743889623\\
19	6.05524498245576\\
20	6.06269772612364\\
21	6.06093652007169\\
22	6.05799569105404\\
23	6.05542349611715\\
24	6.05343909752514\\
25	6.05207955773985\\
26	6.05135261418431\\
27	6.05127958654341\\
28	6.05190719369877\\
29	6.05322041795152\\
30	6.05517114123554\\
31	6.05768176669241\\
32	6.06062315334293\\
33	6.06379482310594\\
34	6.06727968639287\\
35	6.07059780740217\\
36	6.07376229818487\\
37	6.07661484436792\\
38	6.0790302854235\\
39	6.08091337027148\\
40	6.08220577832947\\
41	6.08286518370089\\
42	6.0828912067785\\
43	6.08228581220915\\
44	6.0811536896598\\
45	6.07951892250805\\
46	6.07744970684583\\
47	6.07504460608421\\
48	6.07239396264282\\
49	6.06961999228542\\
50	6.06662723762277\\
51	6.06359750916885\\
52	6.06085396817951\\
53	6.05822895069672\\
54	6.05585570084415\\
55	6.05384358470338\\
56	6.05231072619261\\
57	6.05136330715808\\
58	6.0510562747382\\
59	6.05140664911602\\
60	6.05245156946402\\
61	6.05415871517598\\
62	6.05646143958602\\
63	6.05925078328695\\
64	6.06028142419677\\
65	6.0688927384116\\
66	6.07715935395142\\
67	6.07519140222731\\
68	6.07633831583656\\
69	6.07833461096116\\
70	6.08024952204565\\
71	6.08173105041705\\
72	6.08263146255242\\
73	6.08290798308096\\
74	6.08256431781182\\
75	6.08161725699269\\
76	6.08014981066819\\
77	6.07826005356934\\
78	6.07597595321165\\
79	6.07341322566092\\
80	6.07072109659625\\
81	6.06769444544534\\
82	6.06468309800724\\
83	6.06170406879047\\
84	6.05909074475946\\
85	6.0566183462154\\
86	6.05445021752989\\
87	6.05271075730646\\
88	6.05151940118446\\
89	6.05094449206611\\
90	6.0510191259672\\
91	6.05179109385355\\
92	6.05324310820677\\
93	6.05532198815587\\
94	6.05793369936167\\
95	6.06426022734065\\
96	6.07122804140718\\
97	6.08149064661975\\
98	6.08531350559172\\
99	6.07881594200097\\
100	6.07519140222731\\
};
\addlegendentry{$\text{d}_{\text{2,o}_\text{1}}$};

\addplot [color=mycolor3,solid]
  table[row sep=crcr]{%
0	1.51\\
100	1.51\\
};
\addlegendentry{$\text{d}_{\text{min}}$};

\end{axis}
\end{tikzpicture}%
}
      \caption{The distance between each agent and obstacle 1 over time. The
        minimum allowed distance has a value of $1.51$.}
      \label{fig:d_ON_res_3_2_distance_obstacle_1_agents}
    \end{figure}
  \end{minipage}
  \hfill
  \begin{minipage}{0.45\linewidth}
    \begin{figure}[H]
      \scalebox{0.7}{% This file was created by matlab2tikz.
%
%The latest updates can be retrieved from
%  http://www.mathworks.com/matlabcentral/fileexchange/22022-matlab2tikz-matlab2tikz
%where you can also make suggestions and rate matlab2tikz.
%
\definecolor{mycolor1}{rgb}{0.00000,1.00000,1.00000}%
\definecolor{mycolor2}{rgb}{0.00000,0.44700,0.74100}%
\definecolor{mycolor3}{rgb}{0.85000,0.32500,0.09800}%
%
\begin{tikzpicture}

\begin{axis}[%
width=4.133in,
height=3.26in,
at={(0.693in,0.44in)},
scale only axis,
xmin=0,
xmax=30,
xmajorgrids,
ymin=-11,
ymax=11,
ymajorgrids,
xlabel={time [iterations]},
axis background/.style={fill=white},
axis x line*=bottom,
axis y line*=left,
legend style={at={(0.701,0.649)},anchor=south west,legend cell align=left,align=left,draw=white!15!black}
]
\addplot [color=mycolor1,solid]
  table[row sep=crcr]{%
0	-10\\
100	-10\\
};
\addlegendentry{$\text{u}_{\text{max}}$};

\addplot [color=mycolor1,solid]
  table[row sep=crcr]{%
0	10\\
100	10\\
};
\addlegendentry{$\text{u}_{\text{min}}$};

\addplot [color=mycolor2,solid]
  table[row sep=crcr]{%
1	10\\
2	-1.11698496783172\\
3	10\\
4	-10\\
5	-10\\
6	-10\\
7	-10\\
8	8.53187332196045\\
9	10\\
10	10\\
11	10\\
12	10\\
13	10\\
14	10\\
15	10\\
16	5.35405695040881\\
17	2.03803908644785\\
18	0.773118734135391\\
19	0.309600337835671\\
20	0.150792271409178\\
21	0.104863926480301\\
22	0.0951384055027615\\
23	0.0972825753980916\\
24	0.102835243163086\\
25	0.105913095954429\\
26	0.105756484479186\\
27	0.101923023068663\\
28	0.0946652282718316\\
29	0.0837953987351282\\
30	0.0693070055058716\\
31	0.0525375276163475\\
32	0.0120590311151963\\
33	0.0936520830543511\\
34	-0.0437216850707346\\
35	-0.0455769414720293\\
36	-0.0578011316334126\\
37	-0.0725871620054869\\
38	-0.0868845847932372\\
39	-0.0986306309554369\\
40	-0.106682545543668\\
41	-0.110407617150959\\
42	-0.109774529716338\\
43	-0.103958361270894\\
44	-0.094058043682\\
45	-0.0804546403914296\\
46	-0.0633811507876395\\
47	-0.0426316765796642\\
48	-0.0229769587677308\\
49	-0.00179213593139588\\
50	0.0196803034848826\\
51	0.0399821577304565\\
52	0.0586234899209811\\
53	0.0767077436425035\\
54	0.0889523085926679\\
55	0.0979502338734428\\
56	0.103893305650278\\
57	0.10590599252423\\
58	0.103865141519754\\
59	0.0982208839035373\\
60	0.0888503428664486\\
61	0.0756780723868683\\
62	0.0184476240778992\\
63	0.134694257432473\\
64	0.13889535862263\\
65	0.0544060228994914\\
66	-0.159223374512263\\
67	-0.0962367034978447\\
68	-0.0828255091129714\\
69	-0.0873062029862128\\
70	-0.0964307040062912\\
71	-0.104694178100559\\
72	-0.109749344555067\\
73	-0.110615237739052\\
74	-0.107184012312053\\
75	-0.0989832765152654\\
76	-0.0865786121858721\\
77	-0.070834921713688\\
78	-0.0520933328463304\\
79	-0.0310807596151177\\
80	-0.0110055176486294\\
81	0.0109717044301393\\
82	0.0317297727479661\\
83	0.0510732807499682\\
84	0.0699935716049536\\
85	0.0839921086967478\\
86	0.0945141836157436\\
87	0.101899369020163\\
88	0.10544063724556\\
89	0.105081569681172\\
90	0.101089894259007\\
91	0.0932743248177505\\
92	0.0819799815034226\\
93	0.0552567422249215\\
94	0.145982102154696\\
95	0.183847400890766\\
96	0.124306090893118\\
97	-0.0577246390431225\\
98	-0.227467916160482\\
99	-0.127309278220662\\
100	-0.0998393576815686\\
};
\addlegendentry{$\text{u}_{\text{2,1}}$};

\addplot [color=mycolor3,solid]
  table[row sep=crcr]{%
1	3.38023866112825\\
2	-5.29847634183979\\
3	-10\\
4	-10\\
5	-6.98017202260544\\
6	0.76192353097952\\
7	10\\
8	10\\
9	10\\
10	3.85239160585693\\
11	-4.97442652576898\\
12	0.155865503777596\\
13	-0.513942213115992\\
14	-0.646772308632892\\
15	-0.667270068544616\\
16	-0.300626637084903\\
17	-0.000293537479333174\\
18	0.100349772363056\\
19	0.0733323249977434\\
20	0.0681694732753595\\
21	0.0758859510728698\\
22	0.0869566033808804\\
23	0.0983350272618868\\
24	0.107010592829422\\
25	0.111957132516482\\
26	0.112551831948598\\
27	0.108530392796933\\
28	0.0998162433054866\\
29	0.0869303456876543\\
30	0.0686483918628535\\
31	0.0511629254893694\\
32	0.116783251966676\\
33	0.0357732656923007\\
34	-0.0892245552253088\\
35	-0.0601801201598081\\
36	-0.0649321877732574\\
37	-0.0759460004338628\\
38	-0.088035182464507\\
39	-0.0980876910301824\\
40	-0.104842769319085\\
41	-0.107793065803301\\
42	-0.106493139652892\\
43	-0.102109381808291\\
44	-0.0932642564003293\\
45	-0.0809043563681191\\
46	-0.065414100085624\\
47	-0.0467158584203504\\
48	-0.0273693133026658\\
49	-0.00595986679860676\\
50	0.0157919387018909\\
51	0.0374329011708818\\
52	0.0576354999593171\\
53	0.0776626924123473\\
54	0.0920801919830516\\
55	0.103161266791214\\
56	0.110241327216895\\
57	0.112861728227164\\
58	0.110763899734663\\
59	0.103927135206697\\
60	0.0927079100299223\\
61	0.0755919763011571\\
62	0.160788983955144\\
63	0.110775632127244\\
64	-0.0107989884161842\\
65	-0.112923376223662\\
66	-0.0649726489154527\\
67	-0.0522893683798504\\
68	-0.0643379456054456\\
69	-0.0797026543453072\\
70	-0.0926697532470037\\
71	-0.101792232970624\\
72	-0.106763381659852\\
73	-0.107561970539694\\
74	-0.104171711677569\\
75	-0.0971063781092495\\
76	-0.086508765583983\\
77	-0.0721741182000863\\
78	-0.0550792866640416\\
79	-0.0351475446670153\\
80	-0.0151111813567247\\
81	0.0069668140091695\\
82	0.0287051947007019\\
83	0.049546996057424\\
84	0.0703733349321608\\
85	0.0861287893910111\\
86	0.0988327214703555\\
87	0.107869268744345\\
88	0.112418834504281\\
89	0.112280593875127\\
90	0.10732865880877\\
91	0.0978512778078907\\
92	0.0841299947922857\\
93	0.175831019336828\\
94	0.137252987871735\\
95	0.0440316120827406\\
96	-0.127215452213623\\
97	-0.202435083246638\\
98	0.00106309230299226\\
99	-0.030096181734152\\
100	-0.0601982550661684\\
};
\addlegendentry{$\text{u}_{\text{2,2}}$};

\end{axis}
\end{tikzpicture}%
}
      \caption{The inputs signals directing agent 2 over time. Their value is
        constrained between $-10$ and $10$.}
      \label{fig:d_ON_res_3_2_inputs_agent_2}
    \end{figure}
  \end{minipage}
\end{minipage}
}



\noindent\makebox[\linewidth][c]{%
\begin{minipage}{\linewidth}
  \begin{minipage}{0.45\linewidth}
    \begin{figure}[H]
      \scalebox{0.7}{% This file was created by matlab2tikz.
%
%The latest updates can be retrieved from
%  http://www.mathworks.com/matlabcentral/fileexchange/22022-matlab2tikz-matlab2tikz
%where you can also make suggestions and rate matlab2tikz.
%
\definecolor{mycolor1}{rgb}{0.00000,0.44700,0.74100}%
\definecolor{mycolor2}{rgb}{0.85000,0.32500,0.09800}%
\definecolor{mycolor3}{rgb}{0.92900,0.69400,0.12500}%
%
\begin{tikzpicture}

\begin{axis}[%
width=4.133in,
height=3.26in,
at={(0.693in,0.44in)},
scale only axis,
xmin=0,
xmax=100,
xmajorgrids,
ymin=0,
ymax=140,
ymajorgrids,
axis background/.style={fill=white},
legend style={legend cell align=left,align=left,draw=white!15!black}
]
\addplot [color=mycolor1,solid]
  table[row sep=crcr]{%
1	139.046509148017\\
2	115.543369587391\\
3	96.1917888775145\\
4	78.5229688016341\\
5	62.7380136439174\\
6	48.7795384302997\\
7	36.6314612377145\\
8	26.2704545051121\\
9	17.7980213323107\\
10	10.8584702054142\\
11	5.72654181273514\\
12	2.33702551522838\\
13	0.6334629492279\\
14	0.26143956924958\\
15	0.179906961128308\\
16	0.0800320266420552\\
17	0.0211307136312828\\
18	0.00575472950546586\\
19	0.00215269688049926\\
20	0.00155474872756838\\
21	0.00119323157730092\\
22	0.000900888533624879\\
23	0.00069714997017499\\
24	0.00060379700888714\\
25	0.000627556107182872\\
26	0.000759632675259094\\
27	0.000973535689167284\\
28	0.00123381378489017\\
29	0.00149799065653719\\
30	0.00172460345486413\\
31	0.00187925002333944\\
32	0.0019385679040424\\
33	0.00189430424052218\\
34	0.00175316622075224\\
35	0.00153707381420003\\
36	0.00127965332348026\\
37	0.00102079044073947\\
38	0.000800384440013917\\
39	0.00065295277838146\\
40	0.000602159761910665\\
41	0.000657080761429113\\
42	0.00081078365733564\\
43	0.00103640865718687\\
44	0.00130089396786257\\
45	0.00156000369428559\\
46	0.0017714367502421\\
47	0.00189956951234831\\
48	0.00192219691289059\\
49	0.00183394112498763\\
50	0.00164782359528109\\
51	0.00139695081693536\\
52	0.00112223006002857\\
53	0.000868784154320816\\
54	0.000679441330464789\\
55	0.000586118988586509\\
56	0.000604787556762009\\
57	0.000734305387201642\\
58	0.000955195711635978\\
59	0.0012345491799484\\
60	0.00153124896453597\\
61	0.00180158488504578\\
62	0.00200700039148652\\
63	0.00211824742198783\\
64	0.00220554664777174\\
65	0.00205980790374476\\
66	0.00168555151807043\\
67	0.00142169040204742\\
68	0.00119125531318633\\
69	0.000957007052827939\\
70	0.00076017213297261\\
71	0.000639081586608334\\
72	0.000615872597301928\\
73	0.000694334833583704\\
74	0.00085912596611053\\
75	0.00108033772587288\\
76	0.00131965417401409\\
77	0.00153429048962994\\
78	0.00168580443374441\\
79	0.00174657273912316\\
80	0.00170436267363107\\
81	0.00156461347061094\\
82	0.00135172937867271\\
83	0.00110308764433573\\
84	0.000862528331222993\\
85	0.000673855002076011\\
86	0.000572521498745632\\
87	0.00058016043065781\\
88	0.000701766985038741\\
89	0.000923282821978993\\
90	0.00121559329418651\\
91	0.00153954661826606\\
92	0.0018503441674981\\
93	0.00210560149966953\\
94	0.00227145126286184\\
95	0.0026069174618839\\
96	0.00277583089462351\\
97	0.00249256461734574\\
98	0.00195234096428093\\
99	0.00140108105712062\\
100	0.00111689790934691\\
};
\addlegendentry{$\text{V}_\text{1}$};

\addplot [color=mycolor2,solid]
  table[row sep=crcr]{%
1	139.046509148017\\
2	115.545628044301\\
3	96.5572265192315\\
4	78.9966010648\\
5	63.7152212827868\\
6	53.2826464876119\\
7	47.401053513029\\
8	38.1780251329108\\
9	27.7345442921337\\
10	18.9804675309216\\
11	11.9724897274334\\
12	6.64283330248357\\
13	2.98672028224046\\
14	1.20245810882726\\
15	0.233683975999015\\
16	0.102756354410276\\
17	0.0446477577663407\\
18	0.0123685462168742\\
19	0.00272624726224999\\
20	0.00133799731722673\\
21	0.00101336088935077\\
22	0.000811145974403537\\
23	0.000660078552556493\\
24	0.000589039648436629\\
25	0.000618836935531007\\
26	0.000749363942354362\\
27	0.00095878763017568\\
28	0.0012139424152153\\
29	0.00147346812425855\\
30	0.00169632738522654\\
31	0.00184835604040387\\
32	0.00190630286400299\\
33	0.0018619569593814\\
34	0.00172202278825907\\
35	0.00150836892436864\\
36	0.00125456442182532\\
37	0.00100034818345703\\
38	0.000785446044996242\\
39	0.00064416705672511\\
40	0.000599935208474515\\
41	0.000661585434544688\\
42	0.000821723086191519\\
43	0.00105356212106675\\
44	0.00132353024943205\\
45	0.00158717792685642\\
46	0.00180201003048248\\
47	0.00193226069327617\\
48	0.00195563636317421\\
49	0.0018667123221348\\
50	0.00167853830389102\\
51	0.00142437095101854\\
52	0.00114524934036948\\
53	0.000886484955491419\\
54	0.000691152012019859\\
55	0.000591385669873847\\
56	0.000603460548484993\\
57	0.000726348301797391\\
58	0.000941167963619319\\
59	0.00121486060146566\\
60	0.00150667511468868\\
61	0.00177302411058009\\
62	0.00197549678374297\\
63	0.00208495854489344\\
64	0.00212932122992616\\
65	0.00197320882825502\\
66	0.00164968945794487\\
67	0.00142532475900993\\
68	0.00118701693068753\\
69	0.000947926846707983\\
70	0.00075163880519645\\
71	0.000634299963239556\\
72	0.000616341598822077\\
73	0.000700607747712061\\
74	0.000871007402965737\\
75	0.00109749196278588\\
76	0.00134133285399187\\
77	0.00155953079704834\\
78	0.00171348995789269\\
79	0.00177547780151995\\
80	0.00173320054842258\\
81	0.0015921123854328\\
82	0.00137670011852346\\
83	0.00112444679638555\\
84	0.000879362309192492\\
85	0.000685449385265375\\
86	0.000578394387208875\\
87	0.000580054714989585\\
88	0.000695624161547261\\
89	0.00091138839391301\\
90	0.0011982920989665\\
91	0.0015174649706289\\
92	0.00182426938923925\\
93	0.00207642847268203\\
94	0.00224019468753892\\
95	0.00253416009079726\\
96	0.00259324883857054\\
97	0.00229697056252799\\
98	0.00181912708437081\\
99	0.00137953845196017\\
100	0.0011106214128137\\
};
\addlegendentry{$\text{V}_\text{2}$};

\addplot [color=mycolor3,solid]
  table[row sep=crcr]{%
1	139.046509148017\\
2	111.27749298451\\
3	94.5051162129047\\
4	76.7987404210116\\
5	61.3152084827165\\
6	47.5775872598259\\
7	35.640189909278\\
8	25.4724889094006\\
9	17.0590024516658\\
10	10.3823604758105\\
11	5.42854977060507\\
12	2.19197243712592\\
13	0.603930438627824\\
14	0.277825051482059\\
15	0.18948797087439\\
16	0.0798968212099667\\
17	0.0202684804614153\\
18	0.00568289642539551\\
19	0.00221175958275725\\
20	0.00158162226192024\\
21	0.00120232581341343\\
22	0.000903242459878595\\
23	0.000697423162827658\\
24	0.000603865242811059\\
25	0.000628169355541706\\
26	0.000761081528727146\\
27	0.000975833967894341\\
28	0.00123692581458402\\
29	0.00150179751977476\\
30	0.00172896191069794\\
31	0.00188399128218418\\
32	0.00194350779583913\\
33	0.00189925107644166\\
34	0.001757927347448\\
35	0.00154146294826521\\
36	0.00128349229886088\\
37	0.0010239229568361\\
38	0.000802680406738424\\
39	0.000654313670920475\\
40	0.000602523547777067\\
41	0.000656422059800787\\
42	0.000809147519743647\\
43	0.00103382792040191\\
44	0.00129747949843074\\
45	0.00155589929210621\\
46	0.00176681561328354\\
47	0.0018946263181288\\
48	0.00191714080054936\\
49	0.0018289877432572\\
50	0.00164318139016654\\
51	0.00139280859788809\\
52	0.00111875626264624\\
53	0.00086611838188393\\
54	0.000677686001765479\\
55	0.000585342749913548\\
56	0.000605013371111731\\
57	0.000735537266534946\\
58	0.000957351719694805\\
59	0.00123756583337605\\
60	0.00153500816434174\\
61	0.00180594997240199\\
62	0.00201181279109545\\
63	0.00212333129058954\\
64	0.00221689405538721\\
65	0.00207287315903559\\
66	0.00169378638406845\\
67	0.00142238592309581\\
68	0.0011924354489811\\
69	0.000958606676674779\\
70	0.000761559578167363\\
71	0.000639851977522223\\
72	0.000615830932546865\\
73	0.000693409707733168\\
74	0.000857351345814492\\
75	0.00107776516619025\\
76	0.00131639746285047\\
77	0.00153049543573523\\
78	0.00168163976128261\\
79	0.00174222367759115\\
80	0.00170002371865014\\
81	0.0015604768476515\\
82	0.00134797484538783\\
83	0.00109987898776272\\
84	0.000860003648927228\\
85	0.000672122381019746\\
86	0.000571653944093266\\
87	0.000580196456868947\\
88	0.000702086354968701\\
89	0.000925000089993856\\
90	0.00121822835300389\\
91	0.00154292849677921\\
92	0.00185433194156142\\
93	0.00211033981175929\\
94	0.00227635239979708\\
95	0.00261779404745254\\
96	0.0027685268089296\\
97	0.00249025414306838\\
98	0.00194571301845387\\
99	0.00140236578935758\\
100	0.0011197194584705\\
};
\addlegendentry{$\text{V}_\text{3}$};

\end{axis}
\end{tikzpicture}%}
      \caption{The $\mat{P}-$norms of the errors of the three agents through time.}
      \label{fig:d_ON_res_3_2_V}
    \end{figure}
  \end{minipage}
  \hfill
  \begin{minipage}{0.45\linewidth}
    \begin{figure}[H]
      \scalebox{0.7}{% This file was created by matlab2tikz.
%
%The latest updates can be retrieved from
%  http://www.mathworks.com/matlabcentral/fileexchange/22022-matlab2tikz-matlab2tikz
%where you can also make suggestions and rate matlab2tikz.
%
\definecolor{mycolor1}{rgb}{0.00000,0.44700,0.74100}%
\definecolor{mycolor2}{rgb}{0.85000,0.32500,0.09800}%
\definecolor{mycolor3}{rgb}{0.92900,0.69400,0.12500}%
\definecolor{mycolor4}{rgb}{1.00000,0.00000,1.00000}%
\definecolor{mycolor5}{rgb}{0.00000,1.00000,1.00000}%
%
\begin{tikzpicture}

\begin{axis}[%
width=4.133in,
height=3.26in,
at={(0.693in,0.44in)},
scale only axis,
xmin=1,
xmax=100,
xmajorgrids,
ymin=0,
ymax=0.075,
restrict y to domain=0:1,
ymajorgrids,
xlabel={time [iterations]},
axis background/.style={fill=white},
legend style={at={(0.691,0.511)},anchor=south west,legend cell align=left,align=left,draw=white!15!black}
]
\addplot [color=mycolor1,solid]
  table[row sep=crcr]{%
1	52.9128\\
2	44.4413598839013\\
3	41.6745180421666\\
4	33.9640090751648\\
5	31.0593970058162\\
6	25.5715220314916\\
7	20.6319043552943\\
8	17.2385813621321\\
9	13.1650095982034\\
10	11.4921051756317\\
11	8.18725275076584\\
12	5.19212791743271\\
13	3.04720585725706\\
14	1.40842192534422\\
15	0.443698384288822\\
16	0.0759503636727551\\
17	0.0144214746142258\\
18	0.0035926055740935\\
19	0.00123710725335536\\
20	0.000733702454757136\\
21	0.000514952416607438\\
22	0.000367791373651735\\
23	0.000271399319539843\\
24	0.000229200985064239\\
25	0.000244561816227874\\
26	0.000312886729232865\\
27	0.000423427158022945\\
28	0.000558179474558724\\
29	0.000695344358193898\\
30	0.000813243583147557\\
31	0.000894911456159609\\
32	0.000924717074604925\\
33	0.000946389561941187\\
34	0.000845632741235276\\
35	0.000719899480980441\\
36	0.000584873612683953\\
37	0.000449820265621871\\
38	0.000335791387873015\\
39	0.000259229380583564\\
40	0.000230767453192554\\
41	0.000253646832848859\\
42	0.000323245423311109\\
43	0.000427730694999289\\
44	0.000549878422978745\\
45	0.000669878565569208\\
46	0.000768146118282285\\
47	0.000828782092879421\\
48	0.000841936589241644\\
49	0.000803335449300603\\
50	0.000720164534252415\\
51	0.000605952325839716\\
52	0.000479494197926574\\
53	0.000359782878058089\\
54	0.000268425414710837\\
55	0.000223342450905076\\
56	0.000233572425050663\\
57	0.000298900641781438\\
58	0.000411078760308694\\
59	0.000554023736844824\\
60	0.000706456891736737\\
61	0.000845811513507443\\
62	0.000953331872814088\\
63	0.00100958981881427\\
64	0.00102159601005515\\
65	0.00106059234693717\\
66	0.0010310270237847\\
67	0.000769258906347425\\
68	0.000586448604694919\\
69	0.00043535781826679\\
70	0.000321211228889622\\
71	0.00025335049027926\\
72	0.000237023777677409\\
73	0.000270828472399666\\
74	0.000346222772685551\\
75	0.000448387052460131\\
76	0.000558966213467547\\
77	0.000658628867514487\\
78	0.00072943482452567\\
79	0.000759150287196417\\
80	0.000741918033116033\\
81	0.000679573674855217\\
82	0.000583152843379203\\
83	0.000469293803916347\\
84	0.000358387531001687\\
85	0.000267319639746587\\
86	0.000217793260355931\\
87	0.00022211139397744\\
88	0.000282953726691224\\
89	0.000394583540579935\\
90	0.000543300723767179\\
91	0.000708840306220322\\
92	0.00086828206787475\\
93	0.000999365114661427\\
94	0.00112028345377191\\
95	0.00130714175302668\\
96	0.00145214994392889\\
97	0.00138162249224267\\
98	0.00100419868249869\\
99	0.00070369350745032\\
100	0.000532306348855202\\
};
\addlegendentry{$\text{V}_\text{1}$};

\addplot [color=mycolor2,solid]
  table[row sep=crcr]{%
1	57.572352\\
2	48.171279245189\\
3	49.7947544530147\\
4	47.3145748994151\\
5	46.2921764339438\\
6	40.9156060282624\\
7	33.3153843652366\\
8	25.6504354616224\\
9	23.2909595741337\\
10	17.5174904459273\\
11	12.6045464374563\\
12	8.8181171770105\\
13	5.5115505283895\\
14	3.0052699807577\\
15	1.26413811798729\\
16	0.270804423765105\\
17	0.0401261702468183\\
18	0.0066883994654579\\
19	0.0017030505020553\\
20	0.000822316784537832\\
21	0.000535147844769747\\
22	0.00036850588902887\\
23	0.000268039295280204\\
24	0.000229335679894953\\
25	0.000250465555078682\\
26	0.00032655842527787\\
27	0.000444900564307091\\
28	0.000586823943784439\\
29	0.000730105167346825\\
30	0.000852767506762753\\
31	0.000937623300374297\\
32	0.000969030905453581\\
33	0.000965340131965908\\
34	0.000960614513072175\\
35	0.000787779341853085\\
36	0.000632716500656409\\
37	0.000483915814688716\\
38	0.000359038880261255\\
39	0.000272796256892236\\
40	0.000235057556876056\\
41	0.000248891411578046\\
42	0.000309808398056537\\
43	0.000406320167724128\\
44	0.000521171082979954\\
45	0.000635287576463257\\
46	0.000729116835760699\\
47	0.000786843487872771\\
48	0.000798857053131917\\
49	0.000760934833988786\\
50	0.000680245591771834\\
51	0.00057022690871321\\
52	0.000449511616831595\\
53	0.000339207090269581\\
54	0.000254811340089415\\
55	0.000217731573261764\\
56	0.000236809623804479\\
57	0.00031115619567377\\
58	0.000431851442802973\\
59	0.000582634475318867\\
60	0.000741876057229694\\
61	0.000886735251270507\\
62	0.000998241831369988\\
63	0.00106902696114706\\
64	0.00112565609919827\\
65	0.00120278115189607\\
66	0.00115564451115705\\
67	0.000818444004515037\\
68	0.000617346559676949\\
69	0.000457429096396688\\
70	0.000336107706061064\\
71	0.000261197379085472\\
72	0.000237747312834089\\
73	0.000264515481456334\\
74	0.000333219392256845\\
75	0.000429351120992285\\
76	0.000534748095149323\\
77	0.000630037923091954\\
78	0.000698100715275197\\
79	0.000726322336345718\\
80	0.000709035177782059\\
81	0.000648095772361021\\
82	0.000554495725688948\\
83	0.000444766956445951\\
84	0.000339135320648681\\
85	0.00025421226620857\\
86	0.000211603039375113\\
87	0.0002231701475621\\
88	0.000291246611645329\\
89	0.000410076300068827\\
90	0.000565472861100239\\
91	0.00073691445002954\\
92	0.000901265444701774\\
93	0.00103609005594982\\
94	0.00110891525154309\\
95	0.0012333254728907\\
96	0.00149466658260435\\
97	0.00160356015054543\\
98	0.00131917728179443\\
99	0.000805108779019294\\
100	0.00057142779522082\\
};
\addlegendentry{$\text{V}_\text{2}$};

\addplot [color=mycolor3,solid]
  table[row sep=crcr]{%
1	52.486272\\
2	44.3435567972138\\
3	36.7218313149414\\
4	29.8737971383113\\
5	23.7530878314778\\
6	18.2797598730565\\
7	13.5438908816977\\
8	9.53522287695978\\
9	6.3523512370318\\
10	5.16793545224434\\
11	3.11668773081895\\
12	1.55039043706743\\
13	0.655019777015209\\
14	0.150612117987905\\
15	0.0345509016674889\\
16	0.0107861644524041\\
17	0.00483949069583048\\
18	0.00259953894011636\\
19	0.00104470903724956\\
20	0.000616175721011867\\
21	0.000476510344469001\\
22	0.000364800061605478\\
23	0.000277169117946632\\
24	0.000231923632836642\\
25	0.000239597579635107\\
26	0.000298836916288409\\
27	0.000400388659086588\\
28	0.000526961009947964\\
29	0.000657178798532775\\
30	0.00076966471889553\\
31	0.00084767592951687\\
32	0.00087570130902585\\
33	0.00085140807602185\\
34	0.000779543346957451\\
35	0.000671988968258168\\
36	0.000544597482658061\\
37	0.000418527464455005\\
38	0.00031368747100156\\
39	0.000246523751516229\\
40	0.000227583096289655\\
41	0.000259921216844398\\
42	0.000338647139286787\\
43	0.000451536794952511\\
44	0.000581191079623629\\
45	0.000707330318480821\\
46	0.000810509535120737\\
47	0.000874179964984611\\
48	0.000888533786664781\\
49	0.000849218460655188\\
50	0.000763434221437657\\
51	0.000644805362695757\\
52	0.000512297589319758\\
53	0.000384148949172229\\
54	0.000284851549443644\\
55	0.000230863384332629\\
56	0.000231642562591393\\
57	0.000287473620422982\\
58	0.000390538925871489\\
59	0.000525176237792974\\
60	0.000670419201734084\\
61	0.000803974780115802\\
62	0.000907298336353696\\
63	0.000961109217424731\\
64	0.000969330363379791\\
65	0.000979207338861891\\
66	0.000920395236040233\\
67	0.000714186664684369\\
68	0.000547835865253144\\
69	0.000407335837054409\\
70	0.00030263173016037\\
71	0.000244088856360809\\
72	0.000237084043667419\\
73	0.000280029490741836\\
74	0.000364080223116133\\
75	0.000474045668000247\\
76	0.000591362733507762\\
77	0.000696233265462483\\
78	0.000770892713202399\\
79	0.000802565480912774\\
80	0.000785415877702403\\
81	0.000721262811524065\\
82	0.000621203705387746\\
83	0.000502016765777613\\
84	0.000384307655399966\\
85	0.000285319654975102\\
86	0.000226877462025422\\
87	0.000221986193160167\\
88	0.000273317701185412\\
89	0.000375648413855495\\
90	0.000515772566718268\\
91	0.000673725108519754\\
92	0.000826843745504656\\
93	0.000953103832687054\\
94	0.0010356985471515\\
95	0.00106581501342765\\
96	0.00114465855988911\\
97	0.00116993214558422\\
98	0.00103521241240618\\
99	0.000714300491158593\\
100	0.000522620635482958\\
};
\addlegendentry{$\text{V}_\text{3}$};

\addplot [color=mycolor4,solid]
  table[row sep=crcr]{%
0	0.0654\\
100	0.0654\\
};
\addlegendentry{$\varepsilon_{\Psi}$};

\addplot [color=mycolor5,solid]
  table[row sep=crcr]{%
0	0.0035\\
100	0.0035\\
};
\addlegendentry{$\varepsilon_{\Omega}$};

\end{axis}
\end{tikzpicture}%
}
      \caption{The $\mat{P}-$norms of the errors of the three agents through time,
        focused. The colour magenta is used to illustrate the threshold
        $\mathcal{E}_{\Psi}$, while cyan is used for $\mathcal{E}_{\Omega}$.}
      \label{fig:d_ON_res_3_2_V_zoom}
    \end{figure}
  \end{minipage}
\end{minipage}
}
