\section{Simulation results}

In this case the initial configurations of the three agents are
$\vect{z}_1$ $=$ $[-6, 3.5, 0]^{\top}$,
$\vect{z}_2$ $=$ $[-6, 2.3, 0]^{\top}$ and
$\vect{z}_3$ $=$ $[-6, 4.7, 0]^{\top}$.
Their desired configurations in steady-state are
$\vect{z}_{1,des}$ $=$ $[6, 3.5, 0]^{\top}$,
$\vect{z}_{2,des}$ $=$ $[6, 2.3, 0]^{\top}$ and
$\vect{z}_{3,des}$ $=$ $[6, 4.7, 0]^{\top}$.
Obstacles $o_1$ and $o_2$ are placed between the two at $[0, 2.0]^{\top}$
and $[0, 5.5]^{\top}$ respectively. The penalty
matrices $\mat{Q}$, $\mat{R}$, $\mat{P}$ were set to
$\mat{Q} = 0.7 (I_3 + 0.5\dagger_3)$, $\mat{R} = 0.005 I_2$ and
$\mat{P} = 0.5 (I_3 + 0.5\dagger_3)$, where $\dagger_N$ is a $N \times N$
matrix whose elements are chosen at random between the values $0.0$ and $1.0$.
The sampling time is $h = 0.1$ sec, the time-horizon is $T_p = 0.5$ sec, and
the total execution time given was $10$ sec.

For compatibility with real situations, we assume that the disturbance signals
affecting the agents are of the same nature (consider for instance the case
of UAV's affected by wind); the disturbance signal considered was
$\delta_i(t) = 0.1 * \sin 2t$ for all $i \in \mathcal{V} = \{1,2,3\}$.
Therefore, $\overline{\delta}_i = 0.1$.

The evolution of the trajectories of the agents in the
$x-y$ plane is omitted; they are (with minor variations) equivalent to those
in the case where disturbances are absent. Figures
\eqref{fig:d_ON_res_3_2_distance_agents_13} and
\eqref{fig:d_ON_res_3_2_distance_obstacle_1_agents} show the evolution of the
distance between agents 1 and 3 through time, and the evolution of the
distance between all agents and obstacle $o_1$ respectively.
Figure \eqref{fig:d_ON_res_3_2_inputs_agent_2} shows the input signals
directing agent 2 through time. Just as in the case of absent disturbances,
the compound system manages to clear the narrow between the two obstacles without
agents colliding with each other or the obstacles, and in general, without
violating any constraint present.

Figure \eqref{fig:d_ON_res_3_2_errors_agent_1} depicts the evolution of the
error states of agent 1 through time. In contrast to the disturbance-free
case, the error states do not converge to 0 at steady-state; rather, they
oscillate periodically in accordance with the periodic nature of the disturbance,
and in this case, with different amplitudes. Component-wise,
the $y-$component exhibits the largest amplitude (the compound system oscillates
in the vertical direction), however the largest effect of the disturbance
is still being attenuated by a factor of 2: the peak-to-peak values of the
$y-$component are approximately 0.1, which is half of the peak-to-peak value
of the disturbance signal. For the $x-$ and $\theta-$components, the
disturbance is attenuated by a factor of approximately 4.
Boundary values for the inputs and distances are portrayed in the colour
\textcolor{cyan}{cyan}.


\noindent\makebox[\linewidth][c]{%
\begin{minipage}{\linewidth}
  \begin{minipage}{0.45\linewidth}
    \begin{figure}[H]
      \scalebox{0.6}{% This file was created by matlab2tikz.
%
%The latest updates can be retrieved from
%  http://www.mathworks.com/matlabcentral/fileexchange/22022-matlab2tikz-matlab2tikz
%where you can also make suggestions and rate matlab2tikz.
%
\definecolor{mycolor1}{rgb}{0.00000,0.44700,0.74100}%
\definecolor{mycolor2}{rgb}{0.85000,0.32500,0.09800}%
\definecolor{mycolor3}{rgb}{0.92900,0.69400,0.12500}%
%
\begin{tikzpicture}

\begin{axis}[%
width=4.133in,
height=3.26in,
at={(0.693in,0.44in)},
scale only axis,
xmin=1,
xmax=100,
xmajorgrids,
ymin=-1,
ymax=1,
ymajorgrids,
axis background/.style={fill=white},
legend style={legend cell align=left,align=left,draw=white!15!black}
]
\addplot [color=mycolor1,solid]
  table[row sep=crcr]{%
1	-12\\
2	-10.9990104620416\\
3	-10.635534400582\\
4	-9.63270696839852\\
5	-8.99270758131905\\
6	-8.28862830118856\\
7	-7.48386692274089\\
8	-6.84904698185161\\
9	-6.05316197449761\\
10	-5.64032549932534\\
11	-4.73932987487637\\
12	-3.7884183031066\\
13	-2.90575208359328\\
14	-1.95729094930657\\
15	-1.06440430305293\\
16	-0.377071633092927\\
17	-0.101588908968895\\
18	-0.00483464929410599\\
19	-0.00208219748894829\\
20	-0.00642752154634344\\
21	-0.0105827176000995\\
22	-0.0133155100885203\\
23	-0.014982864925944\\
24	-0.0157658775626898\\
25	-0.015796568700575\\
26	-0.0151593773199552\\
27	-0.0138864430513286\\
28	-0.0120287241244376\\
29	-0.00961939348590755\\
30	-0.006734249319389\\
31	-0.00355621412817398\\
32	-0.000128588667252325\\
33	-0.00587287675534814\\
34	0.00327090209092106\\
35	0.00864258415810156\\
36	0.0121637913708792\\
37	0.0145511687316726\\
38	0.0161247153277542\\
39	0.0169306672685455\\
40	0.0169823592240332\\
41	0.0163065880703763\\
42	0.0149575783751155\\
43	0.0129926447356598\\
44	0.0105703045271912\\
45	0.00780370311390972\\
46	0.00475305775928922\\
47	0.00160606640972528\\
48	-0.00140705252978039\\
49	-0.00444730221693533\\
50	-0.00733281466868337\\
51	-0.00992554388125053\\
52	-0.0121617694764022\\
53	-0.0138599246843339\\
54	-0.0150109506252375\\
55	-0.0156935010201434\\
56	-0.0158340786669363\\
57	-0.0153774908744665\\
58	-0.0143260498157908\\
59	-0.0126915705451875\\
60	-0.0104849981349864\\
61	-0.0077781340377214\\
62	-0.00470744162769902\\
63	-0.0013373423885311\\
64	0.0031415945551691\\
65	0.015077322240925\\
66	0.0251678320339809\\
67	0.0181293752500491\\
68	0.0166085206926557\\
69	0.0168762538459417\\
70	0.0173845096381947\\
71	0.0175178391720392\\
72	0.0170529883003925\\
73	0.0159414536409698\\
74	0.0142196478917142\\
75	0.0119505096649797\\
76	0.00927527382529297\\
77	0.00635898232741606\\
78	0.00321515931600653\\
79	6.53920372764409e-05\\
80	-0.00296004610566215\\
81	-0.00596138152979996\\
82	-0.00867335613900523\\
83	-0.0110667093274282\\
84	-0.0130611743439602\\
85	-0.0144205310487673\\
86	-0.0152925029110052\\
87	-0.0156755086218221\\
88	-0.0154728621954603\\
89	-0.0146644046320107\\
90	-0.0132629130555262\\
91	-0.0112728494665317\\
92	-0.00875056294627912\\
93	-0.00577158470385801\\
94	-0.00853271079258988\\
95	-0.0165801636463534\\
96	-0.0243776143717684\\
97	-0.0279082199310151\\
98	-0.0102000015405083\\
99	0.00570862908026934\\
100	0.0126656366158944\\
};
\addlegendentry{$\text{e}_{\text{1,1}}$};

\addplot [color=mycolor2,solid]
  table[row sep=crcr]{%
1	0\\
2	0.00422478621589708\\
3	-0.00754097901045493\\
4	0.0136492650587786\\
5	-0.22981441499876\\
6	-0.490879603849974\\
7	-0.319647459403459\\
8	-0.173002421826505\\
9	0.12627611703062\\
10	0.28489921967395\\
11	0.338609120931272\\
12	0.330255459711086\\
13	0.369360597381008\\
14	0.358812010042025\\
15	0.248222667063667\\
16	0.114258913738937\\
17	0.063349570308242\\
18	0.0494872586280444\\
19	0.0455636534151399\\
20	0.0402421950774835\\
21	0.033315592003909\\
22	0.0250439941414477\\
23	0.0157626963503455\\
24	0.00584426146560644\\
25	-0.00430976025881203\\
26	-0.0142864431969605\\
27	-0.0236807627373258\\
28	-0.0321130570691089\\
29	-0.0392457619777872\\
30	-0.0447968905186755\\
31	-0.0485517176798007\\
32	-0.0503703520096941\\
33	-0.0502383827206217\\
34	-0.0480052612998947\\
35	-0.0439456137934442\\
36	-0.0382010506112421\\
37	-0.0309938508083041\\
38	-0.0226103279316515\\
39	-0.0133847049982085\\
40	-0.00368193939858427\\
41	0.00611664985399249\\
42	0.0156274648554623\\
43	0.0244778982488503\\
44	0.0323212935211752\\
45	0.0388472376750679\\
46	0.0437928307277777\\
47	0.0469545715749049\\
48	0.0481953834689016\\
49	0.0474521604037717\\
50	0.0447401988895661\\
51	0.0401534065907277\\
52	0.0338634045052349\\
53	0.02611238561295\\
54	0.0172076757499682\\
55	0.00750813436356463\\
56	-0.00259488161772877\\
57	-0.0126918351759632\\
58	-0.022372445935012\\
59	-0.0312447409092841\\
60	-0.0389521818561483\\
61	-0.0451880402897691\\
62	-0.0497089695108613\\
63	-0.0523431865689977\\
64	-0.0529677761033934\\
65	-0.0514589666751307\\
66	-0.0480693518769526\\
67	-0.0431659805494844\\
68	-0.0365938980405839\\
69	-0.0286841659903726\\
70	-0.0197755727442214\\
71	-0.0102305113806745\\
72	-0.000428415489800259\\
73	0.00924528769933999\\
74	0.0184117484661285\\
75	0.0267113271236245\\
76	0.0338172270837279\\
77	0.0394464363145932\\
78	0.0433688067000501\\
79	0.0454192529930137\\
80	0.0455030002448727\\
81	0.0436028547094317\\
82	0.0397795967153731\\
83	0.0341733799513598\\
84	0.026998571229175\\
85	0.0185345284303778\\
86	0.00912217962109849\\
87	-0.000858655292052482\\
88	-0.0110045576462651\\
89	-0.0209034944957436\\
90	-0.0301538426751595\\
91	-0.0383826233680015\\
92	-0.0452607097980726\\
93	-0.0505169668563767\\
94	-0.0539362864049164\\
95	-0.0554345111903496\\
96	-0.054928799661407\\
97	-0.0524788366198617\\
98	-0.0479060086516621\\
99	-0.0416394398320769\\
100	-0.0341466433448372\\
};
\addlegendentry{$\text{e}_{\text{1,2}}$};

\addplot [color=mycolor3,solid]
  table[row sep=crcr]{%
1	0\\
2	0.00678759060302288\\
3	-0.0880606934698573\\
4	0.121222524313982\\
5	-0.872346030407891\\
6	0.135474189766058\\
7	0.26710282131364\\
8	0.164504257444214\\
9	0.540980864121299\\
10	0.166148619508435\\
11	-0.0670439194505661\\
12	0.0308582660040746\\
13	0.0412368833846525\\
14	-0.0765634297943572\\
15	-0.180816664010419\\
16	-0.212510207924936\\
17	-0.156757076689918\\
18	-0.092208595665088\\
19	-0.0427210511422329\\
20	-0.0251542172835926\\
21	-0.0196381208880462\\
22	-0.0182424822887025\\
23	-0.01796126757098\\
24	-0.0176260606291354\\
25	-0.0168296804193534\\
26	-0.0153729398866123\\
27	-0.0133314844461801\\
28	-0.0107749969111019\\
29	-0.00783008798352415\\
30	-0.00464323581281703\\
31	-0.00149849203360504\\
32	0.0018092185618809\\
33	0.0106645640710641\\
34	0.0141028461393729\\
35	0.0131074012329122\\
36	0.0139635069117685\\
37	0.0150333206807907\\
38	0.0159856883217509\\
39	0.0165514089060658\\
40	0.0166051175939844\\
41	0.0160916293591164\\
42	0.014994115848581\\
43	0.0133440191681161\\
44	0.0111176240519031\\
45	0.00838772785698593\\
46	0.00529946076610281\\
47	0.00194165143924513\\
48	-0.00149231766965602\\
49	-0.00498494115447198\\
50	-0.00829885580174794\\
51	-0.011292920551001\\
52	-0.0137945973282836\\
53	-0.0155653794010023\\
54	-0.0166704439943025\\
55	-0.0170565520378425\\
56	-0.0167032401801025\\
57	-0.0156109724456839\\
58	-0.013847282853105\\
59	-0.0115063895556343\\
60	-0.00871772186585971\\
61	-0.00561473791507784\\
62	-0.00253918146659207\\
63	0.000765035153348467\\
64	0.0141650103950762\\
65	0.0198625163317325\\
66	0.0157282451570693\\
67	0.0134044653628089\\
68	0.0142526681905273\\
69	0.0155234373849049\\
70	0.0164469340867602\\
71	0.0168179004929354\\
72	0.0165878243112637\\
73	0.0157518651078692\\
74	0.0143245277749973\\
75	0.0123434444148015\\
76	0.0098419107015773\\
77	0.0068808555203219\\
78	0.00363064170215617\\
79	0.000184427711571035\\
80	-0.00325374615006438\\
81	-0.00667384888757787\\
82	-0.00979590673951757\\
83	-0.0125007170695977\\
84	-0.0146559554783473\\
85	-0.0159807232152935\\
86	-0.0166454366413983\\
87	-0.0165795876883218\\
88	-0.0157783248374758\\
89	-0.014279444579807\\
90	-0.0121665941169727\\
91	-0.00955292387293068\\
92	-0.00656883423293793\\
93	-0.00336805263113991\\
94	0.00301669243445002\\
95	0.000579872221323097\\
96	-0.000807385865740775\\
97	0.00920697743430325\\
98	0.0249811864353229\\
99	0.0201409928877399\\
100	0.0182196973175084\\
};
\addlegendentry{$\text{e}_{\text{1,3}}$};

\end{axis}
\end{tikzpicture}%}
      \caption{The evolution of the error states of agent 1 over time.}
      \label{fig:d_ON_res_3_2_errors_agent_1}
    \end{figure}
  \end{minipage}
  \hfill
  \begin{minipage}{0.45\linewidth}
    \begin{figure}[H]
      \scalebox{0.6}{% This file was created by matlab2tikz.
%
%The latest updates can be retrieved from
%  http://www.mathworks.com/matlabcentral/fileexchange/22022-matlab2tikz-matlab2tikz
%where you can also make suggestions and rate matlab2tikz.
%
\definecolor{mycolor1}{rgb}{0.00000,1.00000,1.00000}%
\definecolor{mycolor2}{rgb}{0.00000,0.44700,0.74100}%

%
\begin{tikzpicture}

\begin{axis}[%
width=4.133in,
height=3.26in,
at={(0.693in,0.44in)},
scale only axis,
xmin=0,
xmax=100,
xmajorgrids,
ymin=0.8,
ymax=2.2,
ymajorgrids,
xlabel={time [iterations]},
axis background/.style={fill=white},
axis x line*=bottom,
axis y line*=left,
legend style={at={(0.699,0.582)},anchor=south west,legend cell align=left,align=left,draw=white!15!black}
]
\addplot [color=mycolor1,solid]
  table[row sep=crcr]{%
0	1.01\\
100	1.01\\
};
\addlegendentry{$\text{d}_{\text{max}}$};

\addplot [color=mycolor1,solid]
  table[row sep=crcr]{%
0	2.01\\
100	2.01\\
};
\addlegendentry{$\text{d}_{\text{min}}$};

\addplot [color=mycolor2,solid]
  table[row sep=crcr]{%
1	1.2\\
2	1.13028748544624\\
3	1.20843567591419\\
4	1.09973251897809\\
5	1.40168431045553\\
6	1.68380298292713\\
7	1.66084436659695\\
8	1.88360008816731\\
9	1.98829687135464\\
10	1.98999051923358\\
11	1.9937575649837\\
12	1.99767488367426\\
13	2.00314221004763\\
14	2.0088428549773\\
15	1.68157856599251\\
16	1.33841237603773\\
17	1.22725239728932\\
18	1.20035658505603\\
19	1.19875891080645\\
20	1.19870178112698\\
21	1.19864519288565\\
22	1.1986146256676\\
23	1.19860457277368\\
24	1.19860255567031\\
25	1.19860340923153\\
26	1.19860513598442\\
27	1.1986070880808\\
28	1.1986090613695\\
29	1.19861091871173\\
30	1.19861256499516\\
31	1.19861390515348\\
32	1.19861491810557\\
33	1.19860317370551\\
34	1.19865097355751\\
35	1.19867294467597\\
36	1.19867842171633\\
37	1.19867772557985\\
38	1.19867610143894\\
39	1.19867446205803\\
40	1.1986729007508\\
41	1.19867137677415\\
42	1.19866987128414\\
43	1.19866834186544\\
44	1.19866718226884\\
45	1.1986662971627\\
46	1.19866505699853\\
47	1.1986641399038\\
48	1.19866350809043\\
49	1.19866318250216\\
50	1.19866320613601\\
51	1.19866361179797\\
52	1.19866438752901\\
53	1.19866697029939\\
54	1.19866764423678\\
55	1.19866899996185\\
56	1.19867094217547\\
57	1.1986727382721\\
58	1.19867473832284\\
59	1.19867675109324\\
60	1.19867863493054\\
61	1.19868032091887\\
62	1.19868172997167\\
63	1.19868283745241\\
64	1.19870246965061\\
65	1.19874360193547\\
66	1.19876449341507\\
67	1.19871554247192\\
68	1.1987007854248\\
69	1.19869480384684\\
70	1.19869189478297\\
71	1.19869002999072\\
72	1.19868846858383\\
73	1.19868697581124\\
74	1.19868551005639\\
75	1.19868402231687\\
76	1.19868240698117\\
77	1.19868150274237\\
78	1.19868041868908\\
79	1.19867965330894\\
80	1.19867920066431\\
81	1.19867906962599\\
82	1.19867931143717\\
83	1.19867992329953\\
84	1.19868089297335\\
85	1.19868213669356\\
86	1.19868379829671\\
87	1.19868526792606\\
88	1.19868718652181\\
89	1.19868948809198\\
90	1.19869152288882\\
91	1.19869344698404\\
92	1.19869520778262\\
93	1.19869673976627\\
94	1.19872529146706\\
95	1.19879776261745\\
96	1.19907764632389\\
97	1.19947516133022\\
98	1.19916886360199\\
99	1.19892234186361\\
100	1.19895242470188\\
};
\addlegendentry{$\text{d}_{\text{13,a}}$};

\end{axis}
\end{tikzpicture}%
}
      \caption{The distance between agents 1 and 3 over time. The maximum allowed
        distance has a value of $2.01$ and the minimum allowed distance a value
        of $1.01$.}
      \label{fig:d_ON_res_3_2_distance_agents_13}
    \end{figure}
  \end{minipage}
\end{minipage}
} \\[2.5ex]

\noindent\makebox[\linewidth][c]{%
\begin{minipage}{\linewidth}
  \begin{minipage}{0.45\linewidth}
    \begin{figure}[H]
      \scalebox{0.6}{% This file was created by matlab2tikz.
%
%The latest updates can be retrieved from
%  http://www.mathworks.com/matlabcentral/fileexchange/22022-matlab2tikz-matlab2tikz
%where you can also make suggestions and rate matlab2tikz.
%
\definecolor{mycolor1}{rgb}{0.00000,0.44700,0.74100}%
\definecolor{mycolor2}{rgb}{0.85000,0.32500,0.09800}%
\definecolor{mycolor3}{rgb}{0.92900,0.69400,0.12500}%
%
\begin{tikzpicture}

\begin{axis}[%
width=4.133in,
height=3.26in,
at={(0.693in,0.44in)},
scale only axis,
xmin=0,
xmax=30,
xmajorgrids,
ymin=1,
ymax=7,
ymajorgrids,
axis background/.style={fill=white},
axis x line*=bottom,
axis y line*=left,
legend style={at={(0.709,0.471)},anchor=south west,legend cell align=left,align=left,draw=white!15!black}
]
\addplot [color=mycolor1,solid]
  table[row sep=crcr]{%
1	6.18465843842649\\
2	5.22457911435046\\
3	4.70952348201006\\
4	4.25357505173589\\
5	3.29142937658314\\
6	2.29743667279803\\
7	1.60000000000095\\
8	1.60205041774449\\
9	1.6\\
10	1.59999999999789\\
11	1.88875925135479\\
12	1.88875925135478\\
13	2.35791444602961\\
14	3.11684318147756\\
15	3.98065811698705\\
16	4.86257637805357\\
17	5.67771113269129\\
18	5.99113188223004\\
19	6.11068932683868\\
20	6.15631806490324\\
21	6.17370610804389\\
22	6.18038044230932\\
23	6.1829130848151\\
24	6.18381573578702\\
25	6.18410659049785\\
26	6.18421697373347\\
27	6.18427158480716\\
28	6.18431514217932\\
29	6.18433708512913\\
30	6.18435841208529\\
};
\addlegendentry{$\text{d}_{\text{1,o}_\text{1}}$};

\addplot [color=mycolor2,solid]
  table[row sep=crcr]{%
1	6.00749531835024\\
2	5.52445563655816\\
3	5.24007197102491\\
4	4.39963668051028\\
5	3.4851248553256\\
6	2.5942018585821\\
7	2.63381415343535\\
8	2.56465394335752\\
9	1.90799672575356\\
10	1.60000000000737\\
11	2.39725683438621\\
12	1.60000000000845\\
13	1.6\\
14	2.09241868486076\\
15	2.79678301732133\\
16	3.64198294403795\\
17	4.54393179288929\\
18	5.45800051581825\\
19	5.81940590415996\\
20	5.95153682059535\\
21	5.99456439159845\\
22	6.00653901189964\\
23	6.00877092918798\\
24	6.00863831511779\\
25	6.00823012336834\\
26	6.00799108174315\\
27	6.00784057670739\\
28	6.00775182731596\\
29	6.00769237996213\\
30	6.0076371248806\\
};
\addlegendentry{$\text{d}_{\text{2,o}_\text{1}}$};

\addplot [color=mycolor3,solid]
  table[row sep=crcr]{%
1	6.57951365983839\\
2	5.68962577780603\\
3	4.841670289039\\
4	3.97820416983991\\
5	3.14167826254574\\
6	2.40270351731317\\
7	1.9053376397957\\
8	1.89352833032692\\
9	2.07589050900767\\
10	2.77601891646894\\
11	3.62882851324919\\
12	3.62882851345129\\
13	4.3356654974551\\
14	5.21106995081784\\
15	6.05989018953154\\
16	6.39886166455981\\
17	6.52112670416013\\
18	6.56240645596922\\
19	6.57520118613744\\
20	6.57875833882697\\
21	6.57944329274666\\
22	6.57941602851822\\
23	6.5793444944852\\
24	6.57929361073565\\
25	6.57925902802557\\
26	6.57920740024087\\
27	6.57917532500663\\
28	6.57915621019736\\
29	6.57913493594217\\
30	6.57912675522103\\
};
\addlegendentry{$\text{d}_{\text{3,o}_\text{1}}$};

\end{axis}
\end{tikzpicture}%}
      \caption{The distance between each agent and obstacle 1 over time. The
        minimum allowed distance has a value of $1.51$.}
      \label{fig:d_ON_res_3_2_distance_obstacle_1_agents}
    \end{figure}
  \end{minipage}
  \hfill
  \begin{minipage}{0.45\linewidth}
    \begin{figure}[H]
      \scalebox{0.6}{% This file was created by matlab2tikz.
%
%The latest updates can be retrieved from
%  http://www.mathworks.com/matlabcentral/fileexchange/22022-matlab2tikz-matlab2tikz
%where you can also make suggestions and rate matlab2tikz.
%
\definecolor{mycolor1}{rgb}{0.00000,1.00000,1.00000}%
\definecolor{mycolor2}{rgb}{0.00000,0.44700,0.74100}%
\definecolor{mycolor3}{rgb}{0.85000,0.32500,0.09800}%
%
\begin{tikzpicture}

\begin{axis}[%
width=4.133in,
height=3.26in,
at={(0.693in,0.44in)},
scale only axis,
xmin=0,
xmax=30,
xmajorgrids,
ymin=-11,
ymax=11,
ymajorgrids,
axis background/.style={fill=white},
axis x line*=bottom,
axis y line*=left,
legend style={at={(0.701,0.649)},anchor=south west,legend cell align=left,align=left,draw=white!15!black}
]
\addplot [color=mycolor1,solid]
  table[row sep=crcr]{%
0	-10\\
100	-10\\
};
\addlegendentry{$\text{u}_{\text{max}}$};

\addplot [color=mycolor1,solid]
  table[row sep=crcr]{%
0	10\\
100	10\\
};
\addlegendentry{$\text{u}_{\text{min}}$};

\addplot [color=mycolor2,solid]
  table[row sep=crcr]{%
1	10\\
2	-1.11698496783172\\
3	10\\
4	-10\\
5	-10\\
6	-10\\
7	-10\\
8	8.53187332196045\\
9	10\\
10	10\\
11	10\\
12	10\\
13	10\\
14	10\\
15	10\\
16	5.35405695040881\\
17	2.03803908644785\\
18	0.773118734135391\\
19	0.309600337835671\\
20	0.150792271409178\\
21	0.104863926480301\\
22	0.0951384055027615\\
23	0.0972825753980916\\
24	0.102835243163086\\
25	0.105913095954429\\
26	0.105756484479186\\
27	0.101923023068663\\
28	0.0946652282718316\\
29	0.0837953987351282\\
30	0.0693070055058716\\
31	0.0525375276163475\\
32	0.0120590311151963\\
33	0.0936520830543511\\
34	-0.0437216850707346\\
35	-0.0455769414720293\\
36	-0.0578011316334126\\
37	-0.0725871620054869\\
38	-0.0868845847932372\\
39	-0.0986306309554369\\
40	-0.106682545543668\\
41	-0.110407617150959\\
42	-0.109774529716338\\
43	-0.103958361270894\\
44	-0.094058043682\\
45	-0.0804546403914296\\
46	-0.0633811507876395\\
47	-0.0426316765796642\\
48	-0.0229769587677308\\
49	-0.00179213593139588\\
50	0.0196803034848826\\
51	0.0399821577304565\\
52	0.0586234899209811\\
53	0.0767077436425035\\
54	0.0889523085926679\\
55	0.0979502338734428\\
56	0.103893305650278\\
57	0.10590599252423\\
58	0.103865141519754\\
59	0.0982208839035373\\
60	0.0888503428664486\\
61	0.0756780723868683\\
62	0.0184476240778992\\
63	0.134694257432473\\
64	0.13889535862263\\
65	0.0544060228994914\\
66	-0.159223374512263\\
67	-0.0962367034978447\\
68	-0.0828255091129714\\
69	-0.0873062029862128\\
70	-0.0964307040062912\\
71	-0.104694178100559\\
72	-0.109749344555067\\
73	-0.110615237739052\\
74	-0.107184012312053\\
75	-0.0989832765152654\\
76	-0.0865786121858721\\
77	-0.070834921713688\\
78	-0.0520933328463304\\
79	-0.0310807596151177\\
80	-0.0110055176486294\\
81	0.0109717044301393\\
82	0.0317297727479661\\
83	0.0510732807499682\\
84	0.0699935716049536\\
85	0.0839921086967478\\
86	0.0945141836157436\\
87	0.101899369020163\\
88	0.10544063724556\\
89	0.105081569681172\\
90	0.101089894259007\\
91	0.0932743248177505\\
92	0.0819799815034226\\
93	0.0552567422249215\\
94	0.145982102154696\\
95	0.183847400890766\\
96	0.124306090893118\\
97	-0.0577246390431225\\
98	-0.227467916160482\\
99	-0.127309278220662\\
100	-0.0998393576815686\\
};
\addlegendentry{$\text{u}_{\text{2,1}}$};

\addplot [color=mycolor3,solid]
  table[row sep=crcr]{%
1	3.38023866112825\\
2	-5.29847634183979\\
3	-10\\
4	-10\\
5	-6.98017202260544\\
6	0.76192353097952\\
7	10\\
8	10\\
9	10\\
10	3.85239160585693\\
11	-4.97442652576898\\
12	0.155865503777596\\
13	-0.513942213115992\\
14	-0.646772308632892\\
15	-0.667270068544616\\
16	-0.300626637084903\\
17	-0.000293537479333174\\
18	0.100349772363056\\
19	0.0733323249977434\\
20	0.0681694732753595\\
21	0.0758859510728698\\
22	0.0869566033808804\\
23	0.0983350272618868\\
24	0.107010592829422\\
25	0.111957132516482\\
26	0.112551831948598\\
27	0.108530392796933\\
28	0.0998162433054866\\
29	0.0869303456876543\\
30	0.0686483918628535\\
31	0.0511629254893694\\
32	0.116783251966676\\
33	0.0357732656923007\\
34	-0.0892245552253088\\
35	-0.0601801201598081\\
36	-0.0649321877732574\\
37	-0.0759460004338628\\
38	-0.088035182464507\\
39	-0.0980876910301824\\
40	-0.104842769319085\\
41	-0.107793065803301\\
42	-0.106493139652892\\
43	-0.102109381808291\\
44	-0.0932642564003293\\
45	-0.0809043563681191\\
46	-0.065414100085624\\
47	-0.0467158584203504\\
48	-0.0273693133026658\\
49	-0.00595986679860676\\
50	0.0157919387018909\\
51	0.0374329011708818\\
52	0.0576354999593171\\
53	0.0776626924123473\\
54	0.0920801919830516\\
55	0.103161266791214\\
56	0.110241327216895\\
57	0.112861728227164\\
58	0.110763899734663\\
59	0.103927135206697\\
60	0.0927079100299223\\
61	0.0755919763011571\\
62	0.160788983955144\\
63	0.110775632127244\\
64	-0.0107989884161842\\
65	-0.112923376223662\\
66	-0.0649726489154527\\
67	-0.0522893683798504\\
68	-0.0643379456054456\\
69	-0.0797026543453072\\
70	-0.0926697532470037\\
71	-0.101792232970624\\
72	-0.106763381659852\\
73	-0.107561970539694\\
74	-0.104171711677569\\
75	-0.0971063781092495\\
76	-0.086508765583983\\
77	-0.0721741182000863\\
78	-0.0550792866640416\\
79	-0.0351475446670153\\
80	-0.0151111813567247\\
81	0.0069668140091695\\
82	0.0287051947007019\\
83	0.049546996057424\\
84	0.0703733349321608\\
85	0.0861287893910111\\
86	0.0988327214703555\\
87	0.107869268744345\\
88	0.112418834504281\\
89	0.112280593875127\\
90	0.10732865880877\\
91	0.0978512778078907\\
92	0.0841299947922857\\
93	0.175831019336828\\
94	0.137252987871735\\
95	0.0440316120827406\\
96	-0.127215452213623\\
97	-0.202435083246638\\
98	0.00106309230299226\\
99	-0.030096181734152\\
100	-0.0601982550661684\\
};
\addlegendentry{$\text{u}_{\text{2,2}}$};

\end{axis}
\end{tikzpicture}%}
      \caption{The inputs signals directing agent 2 over time. Their value is
        constrained between $-10$ and $10$.}
      \label{fig:d_ON_res_3_2_inputs_agent_2}
    \end{figure}
  \end{minipage}
\end{minipage}
}\\[2.5ex]

Last but not at all least, figures
\eqref{fig:d_ON_res_3_2_V}, \eqref{fig:d_ON_res_3_2_V_zoom} and
\eqref{fig:d_ON_res_3_2_V_zoom_zoom} depict
the evolution of the quadratic Lyapunov function
$\vect{e}^{\top} \mat{P} \vect{e}$ through time for all three agents.
The related constants concerned with the execution
of this simulation are as follows: $L_{g_i} = 10.7354$, $L_{V_i} = 0.0471$,
$\varepsilon_{\Psi_i} = 0.0654$ and $\varepsilon_{\Omega_i} = 0.0035$ for
all $i \in \mathcal{V}$. The zoom-in'ed version of figure
\eqref{fig:d_ON_res_3_2_V}, figure \eqref{fig:d_ON_res_3_2_V_zoom},
illustrates that once the energy measure of each agent (as measured by
the Lyapunov function $V = \vect{e}^{\top} \mat{P} \vect{e}$) becomes lower than
$\varepsilon_{\Psi}$ (alternatively $-$ when each system's trajectory enters
set $\Psi$), it gets trapped below the value $\varepsilon_{\Omega}$ (and
hence each system's trajectory is in turn trapped inside the terminal
set $\Omega$) in finite time, and does not exit it.


\noindent\makebox[\linewidth][c]{%
\begin{minipage}{\linewidth}
  \begin{minipage}{0.45\linewidth}
    \begin{figure}[H]
      \scalebox{0.6}{% This file was created by matlab2tikz.
%
%The latest updates can be retrieved from
%  http://www.mathworks.com/matlabcentral/fileexchange/22022-matlab2tikz-matlab2tikz
%where you can also make suggestions and rate matlab2tikz.
%
\definecolor{mycolor1}{rgb}{0.00000,0.44700,0.74100}%
\definecolor{mycolor2}{rgb}{0.85000,0.32500,0.09800}%
\definecolor{mycolor3}{rgb}{0.92900,0.69400,0.12500}%
%
\begin{tikzpicture}

\begin{axis}[%
width=4.133in,
height=3.26in,
at={(0.693in,0.44in)},
scale only axis,
xmin=0,
xmax=100,
xmajorgrids,
ymin=0,
ymax=60,
ymajorgrids,
axis background/.style={fill=white},
legend style={legend cell align=left,align=left,draw=white!15!black}
]
\addplot [color=mycolor1,solid]
  table[row sep=crcr]{%
1	52.9128\\
2	44.4413598839013\\
3	41.6745180421666\\
4	33.9640090751648\\
5	31.0593970058162\\
6	25.5715220314916\\
7	20.6319043552943\\
8	17.2385813621321\\
9	13.1650095982034\\
10	11.4921051756317\\
11	8.18725275076584\\
12	5.19212791743271\\
13	3.04720585725706\\
14	1.40842192534422\\
15	0.443698384288822\\
16	0.0759503636727551\\
17	0.0144214746142258\\
18	0.0035926055740935\\
19	0.00123710725335536\\
20	0.000733702454757136\\
21	0.000514952416607438\\
22	0.000367791373651735\\
23	0.000271399319539843\\
24	0.000229200985064239\\
25	0.000244561816227874\\
26	0.000312886729232865\\
27	0.000423427158022945\\
28	0.000558179474558724\\
29	0.000695344358193898\\
30	0.000813243583147557\\
31	0.000894911456159609\\
32	0.000924717074604925\\
33	0.000946389561941187\\
34	0.000845632741235276\\
35	0.000719899480980441\\
36	0.000584873612683953\\
37	0.000449820265621871\\
38	0.000335791387873015\\
39	0.000259229380583564\\
40	0.000230767453192554\\
41	0.000253646832848859\\
42	0.000323245423311109\\
43	0.000427730694999289\\
44	0.000549878422978745\\
45	0.000669878565569208\\
46	0.000768146118282285\\
47	0.000828782092879421\\
48	0.000841936589241644\\
49	0.000803335449300603\\
50	0.000720164534252415\\
51	0.000605952325839716\\
52	0.000479494197926574\\
53	0.000359782878058089\\
54	0.000268425414710837\\
55	0.000223342450905076\\
56	0.000233572425050663\\
57	0.000298900641781438\\
58	0.000411078760308694\\
59	0.000554023736844824\\
60	0.000706456891736737\\
61	0.000845811513507443\\
62	0.000953331872814088\\
63	0.00100958981881427\\
64	0.00102159601005515\\
65	0.00106059234693717\\
66	0.0010310270237847\\
67	0.000769258906347425\\
68	0.000586448604694919\\
69	0.00043535781826679\\
70	0.000321211228889622\\
71	0.00025335049027926\\
72	0.000237023777677409\\
73	0.000270828472399666\\
74	0.000346222772685551\\
75	0.000448387052460131\\
76	0.000558966213467547\\
77	0.000658628867514487\\
78	0.00072943482452567\\
79	0.000759150287196417\\
80	0.000741918033116033\\
81	0.000679573674855217\\
82	0.000583152843379203\\
83	0.000469293803916347\\
84	0.000358387531001687\\
85	0.000267319639746587\\
86	0.000217793260355931\\
87	0.00022211139397744\\
88	0.000282953726691224\\
89	0.000394583540579935\\
90	0.000543300723767179\\
91	0.000708840306220322\\
92	0.00086828206787475\\
93	0.000999365114661427\\
94	0.00112028345377191\\
95	0.00130714175302668\\
96	0.00145214994392889\\
97	0.00138162249224267\\
98	0.00100419868249869\\
99	0.00070369350745032\\
100	0.000532306348855202\\
};
\addlegendentry{$\text{V}_\text{1}$};

\addplot [color=mycolor2,solid]
  table[row sep=crcr]{%
1	57.572352\\
2	48.171279245189\\
3	49.7947544530147\\
4	47.3145748994151\\
5	46.2921764339438\\
6	40.9156060282624\\
7	33.3153843652366\\
8	25.6504354616224\\
9	23.2909595741337\\
10	17.5174904459273\\
11	12.6045464374563\\
12	8.8181171770105\\
13	5.5115505283895\\
14	3.0052699807577\\
15	1.26413811798729\\
16	0.270804423765105\\
17	0.0401261702468183\\
18	0.0066883994654579\\
19	0.0017030505020553\\
20	0.000822316784537832\\
21	0.000535147844769747\\
22	0.00036850588902887\\
23	0.000268039295280204\\
24	0.000229335679894953\\
25	0.000250465555078682\\
26	0.00032655842527787\\
27	0.000444900564307091\\
28	0.000586823943784439\\
29	0.000730105167346825\\
30	0.000852767506762753\\
31	0.000937623300374297\\
32	0.000969030905453581\\
33	0.000965340131965908\\
34	0.000960614513072175\\
35	0.000787779341853085\\
36	0.000632716500656409\\
37	0.000483915814688716\\
38	0.000359038880261255\\
39	0.000272796256892236\\
40	0.000235057556876056\\
41	0.000248891411578046\\
42	0.000309808398056537\\
43	0.000406320167724128\\
44	0.000521171082979954\\
45	0.000635287576463257\\
46	0.000729116835760699\\
47	0.000786843487872771\\
48	0.000798857053131917\\
49	0.000760934833988786\\
50	0.000680245591771834\\
51	0.00057022690871321\\
52	0.000449511616831595\\
53	0.000339207090269581\\
54	0.000254811340089415\\
55	0.000217731573261764\\
56	0.000236809623804479\\
57	0.00031115619567377\\
58	0.000431851442802973\\
59	0.000582634475318867\\
60	0.000741876057229694\\
61	0.000886735251270507\\
62	0.000998241831369988\\
63	0.00106902696114706\\
64	0.00112565609919827\\
65	0.00120278115189607\\
66	0.00115564451115705\\
67	0.000818444004515037\\
68	0.000617346559676949\\
69	0.000457429096396688\\
70	0.000336107706061064\\
71	0.000261197379085472\\
72	0.000237747312834089\\
73	0.000264515481456334\\
74	0.000333219392256845\\
75	0.000429351120992285\\
76	0.000534748095149323\\
77	0.000630037923091954\\
78	0.000698100715275197\\
79	0.000726322336345718\\
80	0.000709035177782059\\
81	0.000648095772361021\\
82	0.000554495725688948\\
83	0.000444766956445951\\
84	0.000339135320648681\\
85	0.00025421226620857\\
86	0.000211603039375113\\
87	0.0002231701475621\\
88	0.000291246611645329\\
89	0.000410076300068827\\
90	0.000565472861100239\\
91	0.00073691445002954\\
92	0.000901265444701774\\
93	0.00103609005594982\\
94	0.00110891525154309\\
95	0.0012333254728907\\
96	0.00149466658260435\\
97	0.00160356015054543\\
98	0.00131917728179443\\
99	0.000805108779019294\\
100	0.00057142779522082\\
};
\addlegendentry{$\text{V}_\text{2}$};

\addplot [color=mycolor3,solid]
  table[row sep=crcr]{%
1	52.486272\\
2	44.3435567972138\\
3	36.7218313149414\\
4	29.8737971383113\\
5	23.7530878314778\\
6	18.2797598730565\\
7	13.5438908816977\\
8	9.53522287695978\\
9	6.3523512370318\\
10	5.16793545224434\\
11	3.11668773081895\\
12	1.55039043706743\\
13	0.655019777015209\\
14	0.150612117987905\\
15	0.0345509016674889\\
16	0.0107861644524041\\
17	0.00483949069583048\\
18	0.00259953894011636\\
19	0.00104470903724956\\
20	0.000616175721011867\\
21	0.000476510344469001\\
22	0.000364800061605478\\
23	0.000277169117946632\\
24	0.000231923632836642\\
25	0.000239597579635107\\
26	0.000298836916288409\\
27	0.000400388659086588\\
28	0.000526961009947964\\
29	0.000657178798532775\\
30	0.00076966471889553\\
31	0.00084767592951687\\
32	0.00087570130902585\\
33	0.00085140807602185\\
34	0.000779543346957451\\
35	0.000671988968258168\\
36	0.000544597482658061\\
37	0.000418527464455005\\
38	0.00031368747100156\\
39	0.000246523751516229\\
40	0.000227583096289655\\
41	0.000259921216844398\\
42	0.000338647139286787\\
43	0.000451536794952511\\
44	0.000581191079623629\\
45	0.000707330318480821\\
46	0.000810509535120737\\
47	0.000874179964984611\\
48	0.000888533786664781\\
49	0.000849218460655188\\
50	0.000763434221437657\\
51	0.000644805362695757\\
52	0.000512297589319758\\
53	0.000384148949172229\\
54	0.000284851549443644\\
55	0.000230863384332629\\
56	0.000231642562591393\\
57	0.000287473620422982\\
58	0.000390538925871489\\
59	0.000525176237792974\\
60	0.000670419201734084\\
61	0.000803974780115802\\
62	0.000907298336353696\\
63	0.000961109217424731\\
64	0.000969330363379791\\
65	0.000979207338861891\\
66	0.000920395236040233\\
67	0.000714186664684369\\
68	0.000547835865253144\\
69	0.000407335837054409\\
70	0.00030263173016037\\
71	0.000244088856360809\\
72	0.000237084043667419\\
73	0.000280029490741836\\
74	0.000364080223116133\\
75	0.000474045668000247\\
76	0.000591362733507762\\
77	0.000696233265462483\\
78	0.000770892713202399\\
79	0.000802565480912774\\
80	0.000785415877702403\\
81	0.000721262811524065\\
82	0.000621203705387746\\
83	0.000502016765777613\\
84	0.000384307655399966\\
85	0.000285319654975102\\
86	0.000226877462025422\\
87	0.000221986193160167\\
88	0.000273317701185412\\
89	0.000375648413855495\\
90	0.000515772566718268\\
91	0.000673725108519754\\
92	0.000826843745504656\\
93	0.000953103832687054\\
94	0.0010356985471515\\
95	0.00106581501342765\\
96	0.00114465855988911\\
97	0.00116993214558422\\
98	0.00103521241240618\\
99	0.000714300491158593\\
100	0.000522620635482958\\
};
\addlegendentry{$\text{V}_\text{3}$};

\end{axis}
\end{tikzpicture}%}
      \caption{The $\mat{P}-$norms of the errors of the three agents through time.}
      \label{fig:d_ON_res_3_2_V}
    \end{figure}
  \end{minipage}
  \hfill
  \begin{minipage}{0.45\linewidth}
    \begin{figure}[H]
      \scalebox{0.6}{% This file was created by matlab2tikz.
%
%The latest updates can be retrieved from
%  http://www.mathworks.com/matlabcentral/fileexchange/22022-matlab2tikz-matlab2tikz
%where you can also make suggestions and rate matlab2tikz.
%
\definecolor{mycolor1}{rgb}{0.00000,0.44700,0.74100}%
\definecolor{mycolor2}{rgb}{0.85000,0.32500,0.09800}%
\definecolor{mycolor3}{rgb}{0.92900,0.69400,0.12500}%
\definecolor{mycolor4}{rgb}{1.00000,0.00000,1.00000}%
\definecolor{mycolor5}{rgb}{0.00000,1.00000,1.00000}%
%
\begin{tikzpicture}

\begin{axis}[%
width=4.133in,
height=3.26in,
at={(0.693in,0.44in)},
scale only axis,
xmin=1,
xmax=100,
xmajorgrids,
ymin=0,
ymax=0.075,
restrict y to domain=0:1,
ymajorgrids,
xlabel={time [iterations]},
axis background/.style={fill=white},
legend style={at={(0.691,0.511)},anchor=south west,legend cell align=left,align=left,draw=white!15!black}
]
\addplot [color=mycolor1,solid]
  table[row sep=crcr]{%
1	52.9128\\
2	44.4413598839013\\
3	41.6745180421666\\
4	33.9640090751648\\
5	31.0593970058162\\
6	25.5715220314916\\
7	20.6319043552943\\
8	17.2385813621321\\
9	13.1650095982034\\
10	11.4921051756317\\
11	8.18725275076584\\
12	5.19212791743271\\
13	3.04720585725706\\
14	1.40842192534422\\
15	0.443698384288822\\
16	0.0759503636727551\\
17	0.0144214746142258\\
18	0.0035926055740935\\
19	0.00123710725335536\\
20	0.000733702454757136\\
21	0.000514952416607438\\
22	0.000367791373651735\\
23	0.000271399319539843\\
24	0.000229200985064239\\
25	0.000244561816227874\\
26	0.000312886729232865\\
27	0.000423427158022945\\
28	0.000558179474558724\\
29	0.000695344358193898\\
30	0.000813243583147557\\
31	0.000894911456159609\\
32	0.000924717074604925\\
33	0.000946389561941187\\
34	0.000845632741235276\\
35	0.000719899480980441\\
36	0.000584873612683953\\
37	0.000449820265621871\\
38	0.000335791387873015\\
39	0.000259229380583564\\
40	0.000230767453192554\\
41	0.000253646832848859\\
42	0.000323245423311109\\
43	0.000427730694999289\\
44	0.000549878422978745\\
45	0.000669878565569208\\
46	0.000768146118282285\\
47	0.000828782092879421\\
48	0.000841936589241644\\
49	0.000803335449300603\\
50	0.000720164534252415\\
51	0.000605952325839716\\
52	0.000479494197926574\\
53	0.000359782878058089\\
54	0.000268425414710837\\
55	0.000223342450905076\\
56	0.000233572425050663\\
57	0.000298900641781438\\
58	0.000411078760308694\\
59	0.000554023736844824\\
60	0.000706456891736737\\
61	0.000845811513507443\\
62	0.000953331872814088\\
63	0.00100958981881427\\
64	0.00102159601005515\\
65	0.00106059234693717\\
66	0.0010310270237847\\
67	0.000769258906347425\\
68	0.000586448604694919\\
69	0.00043535781826679\\
70	0.000321211228889622\\
71	0.00025335049027926\\
72	0.000237023777677409\\
73	0.000270828472399666\\
74	0.000346222772685551\\
75	0.000448387052460131\\
76	0.000558966213467547\\
77	0.000658628867514487\\
78	0.00072943482452567\\
79	0.000759150287196417\\
80	0.000741918033116033\\
81	0.000679573674855217\\
82	0.000583152843379203\\
83	0.000469293803916347\\
84	0.000358387531001687\\
85	0.000267319639746587\\
86	0.000217793260355931\\
87	0.00022211139397744\\
88	0.000282953726691224\\
89	0.000394583540579935\\
90	0.000543300723767179\\
91	0.000708840306220322\\
92	0.00086828206787475\\
93	0.000999365114661427\\
94	0.00112028345377191\\
95	0.00130714175302668\\
96	0.00145214994392889\\
97	0.00138162249224267\\
98	0.00100419868249869\\
99	0.00070369350745032\\
100	0.000532306348855202\\
};
\addlegendentry{$\text{V}_\text{1}$};

\addplot [color=mycolor2,solid]
  table[row sep=crcr]{%
1	57.572352\\
2	48.171279245189\\
3	49.7947544530147\\
4	47.3145748994151\\
5	46.2921764339438\\
6	40.9156060282624\\
7	33.3153843652366\\
8	25.6504354616224\\
9	23.2909595741337\\
10	17.5174904459273\\
11	12.6045464374563\\
12	8.8181171770105\\
13	5.5115505283895\\
14	3.0052699807577\\
15	1.26413811798729\\
16	0.270804423765105\\
17	0.0401261702468183\\
18	0.0066883994654579\\
19	0.0017030505020553\\
20	0.000822316784537832\\
21	0.000535147844769747\\
22	0.00036850588902887\\
23	0.000268039295280204\\
24	0.000229335679894953\\
25	0.000250465555078682\\
26	0.00032655842527787\\
27	0.000444900564307091\\
28	0.000586823943784439\\
29	0.000730105167346825\\
30	0.000852767506762753\\
31	0.000937623300374297\\
32	0.000969030905453581\\
33	0.000965340131965908\\
34	0.000960614513072175\\
35	0.000787779341853085\\
36	0.000632716500656409\\
37	0.000483915814688716\\
38	0.000359038880261255\\
39	0.000272796256892236\\
40	0.000235057556876056\\
41	0.000248891411578046\\
42	0.000309808398056537\\
43	0.000406320167724128\\
44	0.000521171082979954\\
45	0.000635287576463257\\
46	0.000729116835760699\\
47	0.000786843487872771\\
48	0.000798857053131917\\
49	0.000760934833988786\\
50	0.000680245591771834\\
51	0.00057022690871321\\
52	0.000449511616831595\\
53	0.000339207090269581\\
54	0.000254811340089415\\
55	0.000217731573261764\\
56	0.000236809623804479\\
57	0.00031115619567377\\
58	0.000431851442802973\\
59	0.000582634475318867\\
60	0.000741876057229694\\
61	0.000886735251270507\\
62	0.000998241831369988\\
63	0.00106902696114706\\
64	0.00112565609919827\\
65	0.00120278115189607\\
66	0.00115564451115705\\
67	0.000818444004515037\\
68	0.000617346559676949\\
69	0.000457429096396688\\
70	0.000336107706061064\\
71	0.000261197379085472\\
72	0.000237747312834089\\
73	0.000264515481456334\\
74	0.000333219392256845\\
75	0.000429351120992285\\
76	0.000534748095149323\\
77	0.000630037923091954\\
78	0.000698100715275197\\
79	0.000726322336345718\\
80	0.000709035177782059\\
81	0.000648095772361021\\
82	0.000554495725688948\\
83	0.000444766956445951\\
84	0.000339135320648681\\
85	0.00025421226620857\\
86	0.000211603039375113\\
87	0.0002231701475621\\
88	0.000291246611645329\\
89	0.000410076300068827\\
90	0.000565472861100239\\
91	0.00073691445002954\\
92	0.000901265444701774\\
93	0.00103609005594982\\
94	0.00110891525154309\\
95	0.0012333254728907\\
96	0.00149466658260435\\
97	0.00160356015054543\\
98	0.00131917728179443\\
99	0.000805108779019294\\
100	0.00057142779522082\\
};
\addlegendentry{$\text{V}_\text{2}$};

\addplot [color=mycolor3,solid]
  table[row sep=crcr]{%
1	52.486272\\
2	44.3435567972138\\
3	36.7218313149414\\
4	29.8737971383113\\
5	23.7530878314778\\
6	18.2797598730565\\
7	13.5438908816977\\
8	9.53522287695978\\
9	6.3523512370318\\
10	5.16793545224434\\
11	3.11668773081895\\
12	1.55039043706743\\
13	0.655019777015209\\
14	0.150612117987905\\
15	0.0345509016674889\\
16	0.0107861644524041\\
17	0.00483949069583048\\
18	0.00259953894011636\\
19	0.00104470903724956\\
20	0.000616175721011867\\
21	0.000476510344469001\\
22	0.000364800061605478\\
23	0.000277169117946632\\
24	0.000231923632836642\\
25	0.000239597579635107\\
26	0.000298836916288409\\
27	0.000400388659086588\\
28	0.000526961009947964\\
29	0.000657178798532775\\
30	0.00076966471889553\\
31	0.00084767592951687\\
32	0.00087570130902585\\
33	0.00085140807602185\\
34	0.000779543346957451\\
35	0.000671988968258168\\
36	0.000544597482658061\\
37	0.000418527464455005\\
38	0.00031368747100156\\
39	0.000246523751516229\\
40	0.000227583096289655\\
41	0.000259921216844398\\
42	0.000338647139286787\\
43	0.000451536794952511\\
44	0.000581191079623629\\
45	0.000707330318480821\\
46	0.000810509535120737\\
47	0.000874179964984611\\
48	0.000888533786664781\\
49	0.000849218460655188\\
50	0.000763434221437657\\
51	0.000644805362695757\\
52	0.000512297589319758\\
53	0.000384148949172229\\
54	0.000284851549443644\\
55	0.000230863384332629\\
56	0.000231642562591393\\
57	0.000287473620422982\\
58	0.000390538925871489\\
59	0.000525176237792974\\
60	0.000670419201734084\\
61	0.000803974780115802\\
62	0.000907298336353696\\
63	0.000961109217424731\\
64	0.000969330363379791\\
65	0.000979207338861891\\
66	0.000920395236040233\\
67	0.000714186664684369\\
68	0.000547835865253144\\
69	0.000407335837054409\\
70	0.00030263173016037\\
71	0.000244088856360809\\
72	0.000237084043667419\\
73	0.000280029490741836\\
74	0.000364080223116133\\
75	0.000474045668000247\\
76	0.000591362733507762\\
77	0.000696233265462483\\
78	0.000770892713202399\\
79	0.000802565480912774\\
80	0.000785415877702403\\
81	0.000721262811524065\\
82	0.000621203705387746\\
83	0.000502016765777613\\
84	0.000384307655399966\\
85	0.000285319654975102\\
86	0.000226877462025422\\
87	0.000221986193160167\\
88	0.000273317701185412\\
89	0.000375648413855495\\
90	0.000515772566718268\\
91	0.000673725108519754\\
92	0.000826843745504656\\
93	0.000953103832687054\\
94	0.0010356985471515\\
95	0.00106581501342765\\
96	0.00114465855988911\\
97	0.00116993214558422\\
98	0.00103521241240618\\
99	0.000714300491158593\\
100	0.000522620635482958\\
};
\addlegendentry{$\text{V}_\text{3}$};

\addplot [color=mycolor4,solid]
  table[row sep=crcr]{%
0	0.0654\\
100	0.0654\\
};
\addlegendentry{$\varepsilon_{\Psi}$};

\addplot [color=mycolor5,solid]
  table[row sep=crcr]{%
0	0.0035\\
100	0.0035\\
};
\addlegendentry{$\varepsilon_{\Omega}$};

\end{axis}
\end{tikzpicture}%
}
      \caption{The $\mat{P}-$norms of the errors of the three agents through time,
        focused. The colour magenta is used to illustrate the threshold
        $\varepsilon_{\Psi}$, while cyan is used for $\varepsilon_{\Omega}$.}
      \label{fig:d_ON_res_3_2_V_zoom}
    \end{figure}
  \end{minipage}
\end{minipage}
}\\[2.5ex]

\begin{figure}[H]\centering
  \scalebox{0.7}{% This file was created by matlab2tikz.
%
%The latest updates can be retrieved from
%  http://www.mathworks.com/matlabcentral/fileexchange/22022-matlab2tikz-matlab2tikz
%where you can also make suggestions and rate matlab2tikz.
%
\definecolor{mycolor1}{rgb}{0.00000,0.44700,0.74100}%
\definecolor{mycolor2}{rgb}{0.85000,0.32500,0.09800}%
\definecolor{mycolor3}{rgb}{0.92900,0.69400,0.12500}%
\definecolor{mycolor4}{rgb}{0.00000,1.00000,1.00000}%
%
\begin{tikzpicture}

\begin{axis}[%
width=6.902in,
height=3.26in,
at={(1.158in,0.44in)},
scale only axis,
xmin=1,
xmax=2000,
xmajorgrids,
ymin=0,
ymax=0.007,
ymajorgrids,
restrict y to domain=0:0.1,
axis background/.style={fill=white},
xtick={10, 100, 500, 1000, 1500, 2000},
xticklabels={{10},{100},{500},{1000},{1500},{2000}},
ytick={0.001, 0.002, 0.0035, 0.005, 0.006},
scaled y ticks = false,
yticklabels={{0.001}, {0.002}, {0.0035}, {0.005}, {0.006}},
legend style={legend cell align=left,align=left,draw=white!15!black}
]
\addplot [color=mycolor1,solid]
  table[row sep=crcr]{%
1	52.9128\\
2	44.1878902435301\\
3	36.5683317369223\\
4	29.5391017584424\\
5	23.2838475877326\\
6	17.7728617858696\\
7	13.0131837062046\\
8	9.00365244872423\\
9	5.74212826580214\\
10	3.22415731117927\\
11	1.44337735438535\\
12	0.38042484068588\\
13	0.0626581552023496\\
14	0.0139075319172458\\
15	0.00569601787849718\\
16	0.00477581777807223\\
17	0.00515861626329256\\
18	0.00241920119311108\\
19	0.00131037567095927\\
20	0.000788951631299665\\
21	0.000581826225302849\\
22	0.000428503273783082\\
23	0.000313646026577403\\
24	0.000247776775254623\\
25	0.000237948898530337\\
26	0.000282642020316208\\
27	0.000369779118204131\\
28	0.00048371018746809\\
29	0.000603585016993466\\
30	0.000708067494132003\\
31	0.000780680902306305\\
32	0.000806082649212572\\
33	0.000782024374991919\\
34	0.000713239091363601\\
35	0.000611471586001833\\
36	0.000492343070782437\\
37	0.00037668654539629\\
38	0.000284005963075339\\
39	0.000230268097871972\\
40	0.000225493821944444\\
41	0.000272184718680903\\
42	0.0003648811696391\\
43	0.000490704301055572\\
44	0.000631822167231428\\
45	0.000767829560699608\\
46	0.000878269094753603\\
47	0.000946632354727926\\
48	0.000966827144602058\\
49	0.000939418529701851\\
50	0.000862445773703481\\
51	0.000719661429370547\\
52	0.000570679997701269\\
53	0.00043076511244599\\
54	0.000314913774986482\\
55	0.000245524824453138\\
56	0.000231431092790791\\
57	0.000273087091949398\\
58	0.000362663650021833\\
59	0.000484941184551494\\
60	0.000619663895912039\\
61	0.000744711951161677\\
62	0.000841867006074828\\
63	0.000892080699028093\\
64	0.000889723900796612\\
65	0.000835641460264556\\
66	0.000739310340818316\\
67	0.000616184912859741\\
68	0.000484179865107109\\
69	0.000364766779255893\\
70	0.000276323004236924\\
71	0.000232265475161756\\
72	0.000239055869440661\\
73	0.000295258685459608\\
74	0.000391764406400101\\
75	0.000512941853581346\\
76	0.000640041021215198\\
77	0.000752419039610273\\
78	0.000832434370620658\\
79	0.000866842858600493\\
80	0.000849847197959685\\
81	0.000783123948470293\\
82	0.000677855945409035\\
83	0.000551034753589979\\
84	0.000423576537241512\\
85	0.000313278746389814\\
86	0.000242055335774779\\
87	0.000224163789215118\\
88	0.000261463932827661\\
89	0.000350603615607928\\
90	0.00047847081658056\\
91	0.000625557579364118\\
92	0.000769619618657864\\
93	0.000888965655156321\\
94	0.000967057945308507\\
95	0.00105482372740576\\
96	0.00105946324190347\\
97	0.00105432551435817\\
98	0.000802619005603137\\
99	0.000626374567358635\\
100	0.000472478899092665\\
101	0.00034862041680812\\
102	0.000266745277293947\\
103	0.000234881421277568\\
104	0.000254570130104885\\
105	0.000320021542130553\\
106	0.000418662720661514\\
107	0.00053297779186247\\
108	0.000643326867851428\\
109	0.00073082553484468\\
110	0.000780587048785131\\
111	0.000783991493560703\\
112	0.000738542059850059\\
113	0.000652235266219446\\
114	0.000540357061207653\\
115	0.000421702898366152\\
116	0.000314291261365735\\
117	0.000240279570803884\\
118	0.000215487561821151\\
119	0.000247525198422673\\
120	0.000332723395937796\\
121	0.000462040132140895\\
122	0.000617942171904794\\
123	0.000778180114041157\\
124	0.000919933122959033\\
125	0.00102468374396193\\
126	0.00111670319023557\\
127	0.00117629458996986\\
128	0.00125791931706901\\
129	0.00118092833010798\\
130	0.000808588540379797\\
131	0.000599782120492732\\
132	0.000440544325279085\\
133	0.000323766571667866\\
134	0.000255358700852161\\
135	0.000238928997665909\\
136	0.000271921072512663\\
137	0.00034492310958331\\
138	0.000442585235352979\\
139	0.000546470074943492\\
140	0.000637573943990014\\
141	0.000698988414251816\\
142	0.000720157388283077\\
143	0.000694484368906334\\
144	0.000627716221161782\\
145	0.000530780568221174\\
146	0.000421133860765188\\
147	0.00031936115378817\\
148	0.000241587523299862\\
149	0.000208755979130626\\
150	0.000231517199148379\\
151	0.000310298226462534\\
152	0.000437978156080448\\
153	0.000599058386263038\\
154	0.00077222785123955\\
155	0.000934159293524467\\
156	0.00106284466590482\\
157	0.00119762878054796\\
158	0.00131204335699116\\
159	0.00158351242347379\\
160	0.00160879542931045\\
161	0.00128425541292159\\
162	0.000769800308513955\\
163	0.000540343167356842\\
164	0.00038982191590713\\
165	0.000290979274679503\\
166	0.000243682364885185\\
167	0.000247564376954691\\
168	0.000296363757053829\\
169	0.000377627950292346\\
170	0.000473762386476303\\
171	0.000566410009701368\\
172	0.000637476366458505\\
173	0.000673366375221249\\
174	0.000666973015370709\\
175	0.00061750089709192\\
176	0.000534396673132111\\
177	0.000433037499409158\\
178	0.000332487527371514\\
179	0.000249607643746482\\
180	0.000206086860296192\\
181	0.000215228288246952\\
182	0.000281224321110515\\
183	0.000399954372746537\\
184	0.000558766004578155\\
185	0.000737956388634144\\
186	0.000914409878445428\\
187	0.00106485642506724\\
188	0.00117976712323839\\
189	0.00134713068049623\\
190	0.00174915971964303\\
191	0.00205470807787016\\
192	0.00171071826150401\\
193	0.000957613535069461\\
194	0.000618295109238277\\
195	0.000445674399946666\\
196	0.000329091846738402\\
197	0.000259304078358441\\
198	0.000238954237385397\\
199	0.000265932051848133\\
200	0.000331044221863165\\
201	0.000419110454924945\\
202	0.000512015246819276\\
203	0.000591298556766056\\
204	0.000641391637829732\\
205	0.000652736517246935\\
206	0.000620526945806005\\
207	0.00055215100980448\\
208	0.00045851835271527\\
209	0.000358029718273251\\
210	0.000271238521956644\\
211	0.000212942957376794\\
212	0.000203504554671518\\
213	0.000251504861371075\\
214	0.000354383404262769\\
215	0.000503398148840404\\
216	0.00068144010762235\\
217	0.000865999465265184\\
218	0.00103325377410136\\
219	0.00121045000864446\\
220	0.00159223238801197\\
221	0.00226863751043641\\
222	0.00287235735455886\\
223	0.00309893718785911\\
224	0.00196657480030562\\
225	0.000913397900635389\\
226	0.000573963332593062\\
227	0.000393710477805748\\
228	0.000288570737423818\\
229	0.000245009607124263\\
230	0.000256504555927471\\
231	0.000313088264541502\\
232	0.000399981963713646\\
233	0.00049889679865843\\
234	0.000590580829206727\\
235	0.000657645319123172\\
236	0.000687395231868666\\
237	0.000673584644517674\\
238	0.000617235165365209\\
239	0.000528915224967734\\
240	0.000424647274208036\\
241	0.000324194651235091\\
242	0.00024405939767348\\
243	0.000205896223560489\\
244	0.000221738331480509\\
245	0.000294249145954026\\
246	0.000418038770126329\\
247	0.000579226517008001\\
248	0.000757350774482868\\
249	0.000929101250262873\\
250	0.00107161256744316\\
251	0.00115901047856205\\
252	0.00134571675529221\\
253	0.00172948706429368\\
254	0.00195388927763516\\
255	0.0015444958661004\\
256	0.000873614373146586\\
257	0.000608575310388687\\
258	0.000439016120972093\\
259	0.000321998444441492\\
260	0.000255778414071609\\
261	0.000241246351343995\\
262	0.000273992267502105\\
263	0.000343155052984468\\
264	0.000432549935842095\\
265	0.000523794432614698\\
266	0.000598798244448736\\
267	0.000642526231462428\\
268	0.000646491837834654\\
269	0.000607682415083927\\
270	0.000534046893964651\\
271	0.000438392714200405\\
272	0.000339452735878904\\
273	0.000257670695386569\\
274	0.000207056215331288\\
275	0.000207566147962323\\
276	0.000265468469932259\\
277	0.000378080917801597\\
278	0.00053442082670879\\
279	0.000715933224202697\\
280	0.000899874220683099\\
281	0.00106248359173253\\
282	0.001164124330998\\
283	0.00135258315647008\\
284	0.00181938809150789\\
285	0.00228510683714819\\
286	0.00218711044035299\\
287	0.00150505762906777\\
288	0.00074630125062533\\
289	0.000495950435711966\\
290	0.000357630098252134\\
291	0.000273571642156918\\
292	0.000239521185421839\\
293	0.000253981547561947\\
294	0.00030945034374525\\
295	0.000392201287366115\\
296	0.000484570461064596\\
297	0.000567897293266055\\
298	0.000625720121419719\\
299	0.000647305665136639\\
300	0.000625358164499527\\
301	0.000564927301922065\\
302	0.000476626616328449\\
303	0.000377146343387826\\
304	0.000286465989493579\\
305	0.00022033583482054\\
306	0.000199640348958394\\
307	0.000234678590996498\\
308	0.000326223592840575\\
309	0.000466978817058865\\
310	0.000641372065924258\\
311	0.000827862709445843\\
312	0.00100266298329693\\
313	0.00114312080741176\\
314	0.00134661023360029\\
315	0.00162380593995812\\
316	0.00211830898835293\\
317	0.00228111570083527\\
318	0.00167943315338743\\
319	0.000873330788368067\\
320	0.000588144705584063\\
321	0.000420295875627723\\
322	0.00030957307119433\\
323	0.000250655982438698\\
324	0.000243046874096164\\
325	0.000281265581764897\\
326	0.000353570233483636\\
327	0.000443258386234279\\
328	0.000531886796464453\\
329	0.00060165802820172\\
330	0.000638600185955191\\
331	0.000635163952694967\\
332	0.000589860084898047\\
333	0.000511436764775251\\
334	0.000414968997316655\\
335	0.000319014361810816\\
336	0.000240312295957777\\
337	0.000200504865073828\\
338	0.00021307881775069\\
339	0.000282580207069446\\
340	0.000405416387474376\\
341	0.000569178140639427\\
342	0.00075430267212187\\
343	0.000937586292416847\\
344	0.00109542374472909\\
345	0.00119573375588789\\
346	0.00144452188431902\\
347	0.00196776855230353\\
348	0.00240458923165254\\
349	0.00214328355250198\\
350	0.00116770607653034\\
351	0.00064612356946325\\
352	0.000452941078480759\\
353	0.000334716679497161\\
354	0.000262511911631846\\
355	0.000238018176839034\\
356	0.000260420885371003\\
357	0.000321635153278018\\
358	0.00040720269004915\\
359	0.000499326697047347\\
360	0.000579335603694768\\
361	0.000631569127553819\\
362	0.000645736227537445\\
363	0.000617166174019668\\
364	0.000551511877674917\\
365	0.00046006618770242\\
366	0.000360548224147129\\
367	0.000273345237745118\\
368	0.00021339129231376\\
369	0.000201612181083348\\
370	0.000246326065528075\\
371	0.000347070522659091\\
372	0.000494821487125623\\
373	0.000672962695291563\\
374	0.000859264839645412\\
375	0.00102987016851584\\
376	0.00119882831429402\\
377	0.00159071899059712\\
378	0.00230522342222089\\
379	0.00296790609928315\\
380	0.00327126303715722\\
381	0.00211063580170442\\
382	0.00094753453564868\\
383	0.000589942311042572\\
384	0.000403000202294661\\
385	0.000293346845067864\\
386	0.000245967503811626\\
387	0.000254293404821804\\
388	0.000308653128129316\\
389	0.000394612720495413\\
390	0.000493859878268442\\
391	0.000587445726124921\\
392	0.000657451225349532\\
393	0.00069065518205034\\
394	0.000680484270294022\\
395	0.000627008074276902\\
396	0.000540326827183579\\
397	0.000436163239686726\\
398	0.000334014670748786\\
399	0.000250672854595042\\
400	0.000207788046695854\\
401	0.000218160253051424\\
402	0.000285304749464631\\
403	0.000404511345352332\\
404	0.000562541778659588\\
405	0.000739365311821746\\
406	0.000911826407417264\\
407	0.00105691703421178\\
408	0.00116478926547029\\
409	0.00131364320623149\\
410	0.00167016248427633\\
411	0.00190422040182237\\
412	0.00153716356317646\\
413	0.000883545770483848\\
414	0.000620204999612762\\
415	0.000448624740837668\\
416	0.000328440785317803\\
417	0.000258647314702235\\
418	0.000240646165887751\\
419	0.0002705368955494\\
420	0.00033790658648919\\
421	0.000426875035560666\\
422	0.000519097053465727\\
423	0.000596490737945778\\
424	0.000643196742208467\\
425	0.000650671237053549\\
426	0.00061486914444572\\
427	0.000542996750454669\\
428	0.000448207980567676\\
429	0.000348568477189522\\
430	0.00026432703540471\\
431	0.000209717280760302\\
432	0.000205663566570304\\
433	0.000258405686454474\\
434	0.000366169466538951\\
435	0.000518798482243177\\
436	0.000698565369968342\\
437	0.000882711049299884\\
438	0.0010474503036896\\
439	0.0011609675218743\\
440	0.00131885436729024\\
441	0.00175184091302943\\
442	0.00222269290569764\\
443	0.00218969503166559\\
444	0.00151192152157938\\
445	0.000768023985512963\\
446	0.000513633721106584\\
447	0.000369552759770022\\
448	0.000280051590483575\\
449	0.000241009012200819\\
450	0.000251142445260324\\
451	0.000303317397402401\\
452	0.000384194043946131\\
453	0.000476165597768424\\
454	0.000560630799294605\\
455	0.000620846263818981\\
456	0.000645672941788811\\
457	0.000627040180942983\\
458	0.000569281343613536\\
459	0.000482773416489757\\
460	0.00038377173018693\\
461	0.000291964879317325\\
462	0.000223558528705522\\
463	0.000199119716480765\\
464	0.000230127797294271\\
465	0.000317577749776861\\
466	0.000455232914896457\\
467	0.000628019595996277\\
468	0.000814705630338002\\
469	0.000991556684786668\\
470	0.00113568628849012\\
471	0.00122373392448842\\
472	0.00156165117709422\\
473	0.00211945862249727\\
474	0.0024108943014103\\
475	0.00177816262140163\\
476	0.000911169569465799\\
477	0.000613261544978663\\
478	0.000439029725517931\\
479	0.000321878726108959\\
480	0.000255983360767242\\
481	0.000241404644094136\\
482	0.000273375050100509\\
483	0.000340776883887304\\
484	0.000427312178436018\\
485	0.000514690652278816\\
486	0.000585137722360117\\
487	0.000624152151512783\\
488	0.000623844041425888\\
489	0.000582001194725251\\
490	0.000506596186942705\\
491	0.000412520086583197\\
492	0.000317901314319858\\
493	0.000242768886968553\\
494	0.000200458310025986\\
495	0.000210948520289327\\
496	0.000278945316349063\\
497	0.000400719803544542\\
498	0.00056473903334159\\
499	0.000751643178390047\\
500	0.000938257982719644\\
501	0.00110077027919663\\
502	0.00120566155813868\\
503	0.00146779579171802\\
504	0.00202884522069138\\
505	0.00252849422969934\\
506	0.00234076618244284\\
507	0.00123348435054243\\
508	0.00066207711856146\\
509	0.000463024396045318\\
510	0.000342542611030047\\
511	0.000266902314446661\\
512	0.000238182120390949\\
513	0.000256472427249\\
514	0.000314434634478389\\
515	0.000398140121999335\\
516	0.000489957961412614\\
517	0.000571196730326857\\
518	0.000625753037456588\\
519	0.000643459359627514\\
520	0.000617879785475938\\
521	0.000554845434092315\\
522	0.000465165803158134\\
523	0.000366063874718619\\
524	0.000277726310233789\\
525	0.000215530565934402\\
526	0.000200173542037291\\
527	0.00024123577671905\\
528	0.000338332623372096\\
529	0.000483593852053646\\
530	0.000660742406521246\\
531	0.000847839325088392\\
532	0.00102102178253115\\
533	0.00115241130668299\\
534	0.0015845528265048\\
535	0.00233740169867505\\
536	0.0030181734036715\\
537	0.00332121364644983\\
538	0.00217059755384359\\
539	0.000964834065808168\\
540	0.000600344893566935\\
541	0.000410419701163729\\
542	0.000297614665338063\\
543	0.000246865641220043\\
544	0.00025209817892234\\
545	0.000304155955345296\\
546	0.00038904796340426\\
547	0.000488556737503872\\
548	0.000583833004433835\\
549	0.000656657387406356\\
550	0.000693364252744701\\
551	0.00068683860060388\\
552	0.000636350457502473\\
553	0.000551436441625728\\
554	0.000447559573229298\\
555	0.000343917941646304\\
556	0.000257519383682522\\
557	0.000210104930276916\\
558	0.000215192832391341\\
559	0.000277085171012218\\
560	0.000391615183776367\\
561	0.000546366246528387\\
562	0.000721760692245918\\
563	0.00089480574841752\\
564	0.00104236338167377\\
565	0.00114514073744113\\
566	0.00127710091314652\\
567	0.0016133374199045\\
568	0.0018674012355128\\
569	0.00154948541182918\\
570	0.000893133689948076\\
571	0.000628965630440379\\
572	0.000456781037767115\\
573	0.00033451250268472\\
574	0.000261560486051916\\
575	0.000240090263881158\\
576	0.000266918313532339\\
577	0.000332200702460717\\
578	0.000420437287081083\\
579	0.000513382069394917\\
580	0.000592574440372008\\
581	0.00064239525894357\\
582	0.000653381305654743\\
583	0.000620798143036351\\
584	0.000552123008420628\\
585	0.000458259538543609\\
586	0.000357681536569774\\
587	0.000270959874096314\\
588	0.000212858208592644\\
589	0.000203720671681092\\
590	0.000252075899002549\\
591	0.000355223004201977\\
592	0.000504404866616505\\
593	0.000682451199145434\\
594	0.000866826933457339\\
595	0.00103371476274109\\
596	0.00121142036721483\\
597	0.00159227608135593\\
598	0.00226546070050473\\
599	0.00286254037163667\\
600	0.00308159612775239\\
601	0.00194544949101929\\
602	0.000908050995550277\\
603	0.000571772423813917\\
604	0.000392518906881011\\
605	0.000287974797132232\\
606	0.000244889583324699\\
607	0.00025677068261195\\
608	0.000313625643463298\\
609	0.000400640706467613\\
610	0.000499521510874434\\
611	0.000591014024683136\\
612	0.000657760782851913\\
613	0.000687120543655762\\
614	0.000672910957447135\\
615	0.000616246379004249\\
616	0.000527741072265196\\
617	0.000423455590244371\\
618	0.000323182835872585\\
619	0.000243396305046291\\
620	0.000205737050892863\\
621	0.000222167305298349\\
622	0.000295238201491396\\
623	0.000419510571408303\\
624	0.000581020149010437\\
625	0.000759259038199305\\
626	0.000930902325890076\\
627	0.00107312161395287\\
628	0.00115962479669963\\
629	0.00134945741404159\\
630	0.00173550881315578\\
631	0.00195760746835968\\
632	0.0015442629828869\\
633	0.000872249097258503\\
634	0.000607121295115114\\
635	0.000437835458312351\\
636	0.000321213624608832\\
637	0.000255435052165562\\
638	0.000241331692975438\\
639	0.000274434716541285\\
640	0.000343831556555626\\
641	0.000433302727101934\\
642	0.000524462734277096\\
643	0.000599230042260994\\
644	0.000642624742865715\\
645	0.000646204152936297\\
646	0.000607033589153764\\
647	0.000533204482542099\\
648	0.000437437905933426\\
649	0.000338561392959814\\
650	0.000257033141359313\\
651	0.000206815491300035\\
652	0.000207837515983671\\
653	0.000266267947576008\\
654	0.000379355543229217\\
655	0.000536037080014888\\
656	0.000717701073333958\\
657	0.000901577598436974\\
658	0.00106391394776967\\
659	0.00116908737943719\\
660	0.00135642392043497\\
661	0.00182282489774584\\
662	0.00228499523455883\\
663	0.00217926656967732\\
664	0.00150203803168609\\
665	0.000743138782773617\\
666	0.000493659765924708\\
667	0.000356144924556874\\
668	0.000272783371283413\\
669	0.000239356645191999\\
670	0.000254356520242873\\
671	0.000310242184607793\\
672	0.000393244095634179\\
673	0.000485691420535699\\
674	0.000568917895099014\\
675	0.000626490815331163\\
676	0.000647723561618639\\
677	0.000625406180352989\\
678	0.000564668273389864\\
679	0.000476147481200409\\
680	0.000376587502024459\\
681	0.000285995817633579\\
682	0.000220096432495028\\
683	0.000199752090855433\\
684	0.000235194357245288\\
685	0.000327117883393786\\
686	0.000468152766741731\\
687	0.0006426614349997\\
688	0.000829073080168497\\
689	0.00100359718329694\\
690	0.0011436083497577\\
691	0.00134264297483498\\
692	0.00162385178917156\\
693	0.00211888335698246\\
694	0.00227779200455809\\
695	0.00167519226721917\\
696	0.000870932066998288\\
697	0.000586346617113167\\
698	0.000418979118964541\\
699	0.000308747637454799\\
700	0.000250332339893014\\
701	0.000243194989083916\\
702	0.000281801086510652\\
703	0.000354359298107481\\
704	0.000444134974590045\\
705	0.00053269105272194\\
706	0.000602227212066892\\
707	0.000638838103308949\\
708	0.000635022422046671\\
709	0.000589374005774487\\
710	0.000510720810636611\\
711	0.00041415736975595\\
712	0.000318280660999376\\
713	0.000239815983658946\\
714	0.000200397690369556\\
715	0.000213447369891102\\
716	0.000283425763438148\\
717	0.000406670580000745\\
718	0.00057069698431481\\
719	0.00075589449382043\\
720	0.000939039371695263\\
721	0.0010965406638063\\
722	0.00119599479627978\\
723	0.00144719552211787\\
724	0.00197185846423727\\
725	0.00240594836569239\\
726	0.00213597324226334\\
727	0.00116160273043528\\
728	0.000644231710865894\\
729	0.00045181594894504\\
730	0.000333942420184603\\
731	0.000262112241344299\\
732	0.000238020054808576\\
733	0.000260786089124005\\
734	0.00032226187168914\\
735	0.000407942191682098\\
736	0.000500010713603689\\
737	0.000579815291425693\\
738	0.000631776992598465\\
739	0.000645560277675769\\
740	0.000616606515849285\\
741	0.000550698513271653\\
742	0.000459106969265607\\
743	0.000359616648846483\\
744	0.000272628938613389\\
745	0.000213045222823181\\
746	0.000201756587887138\\
747	0.000247005338445994\\
748	0.000348246883758277\\
749	0.000496379966994561\\
750	0.000674722344204202\\
751	0.000861013199895179\\
752	0.0010313933602463\\
753	0.0012041995620129\\
754	0.0015977006081803\\
755	0.00231368623972398\\
756	0.00297729152813238\\
757	0.00327291510052915\\
758	0.00209210783352747\\
759	0.000943239590038796\\
760	0.000588060664374926\\
761	0.000401844757203125\\
762	0.00029271321765346\\
763	0.000245823785216672\\
764	0.000254585405694099\\
765	0.000309278276719587\\
766	0.000395421359976681\\
767	0.000494780162036525\\
768	0.000588127552296059\\
769	0.000657819960676385\\
770	0.000690739198545556\\
771	0.000680135352553433\\
772	0.000626299826666075\\
773	0.000539395329980984\\
774	0.000435166225745045\\
775	0.000333139290929773\\
776	0.000250035698385302\\
777	0.000207616697360141\\
778	0.000218514183899882\\
779	0.000286149119921955\\
780	0.000405795706894647\\
781	0.000564110592735568\\
782	0.000741017779708303\\
783	0.000913347090110759\\
784	0.00105810965491773\\
785	0.00116851774635813\\
786	0.00131740967733233\\
787	0.00167428252372054\\
788	0.00190377702472061\\
789	0.00153454549996408\\
790	0.00088130316532932\\
791	0.000618264278398449\\
792	0.000447087013725676\\
793	0.000327402063966932\\
794	0.00025816236407122\\
795	0.000240711420211938\\
796	0.000271084007143742\\
797	0.00033880885931446\\
798	0.000427963250683782\\
799	0.000520191539812346\\
800	0.000597291515343004\\
801	0.000643821339136231\\
802	0.000650929224623094\\
803	0.000614767575206762\\
804	0.000542599630010633\\
805	0.00044763850395423\\
806	0.000347985204319187\\
807	0.000263895861527695\\
808	0.000209567123131112\\
809	0.00020588895883881\\
810	0.000259023377937757\\
811	0.000367143048474955\\
812	0.000520034680861229\\
813	0.000699868398477006\\
814	0.000883877339241212\\
815	0.00104828582436066\\
816	0.00115810756515666\\
817	0.00131858877446655\\
818	0.00175516701593408\\
819	0.00222481797906318\\
820	0.00218564312048349\\
821	0.00150905146323199\\
822	0.00076530331040434\\
823	0.000511666294866227\\
824	0.000368229479486106\\
825	0.000279317925033735\\
826	0.000240829806893469\\
827	0.000251454818890988\\
828	0.00030401183749477\\
829	0.00038511380709319\\
830	0.000477146305137068\\
831	0.000561497823399956\\
832	0.00062145626783738\\
833	0.000645926453674572\\
834	0.000626923746471509\\
835	0.000568874837578098\\
836	0.000482162317836659\\
837	0.00038308613337872\\
838	0.000291382439500403\\
839	0.000223224569701125\\
840	0.000199178643555323\\
841	0.000230652388143134\\
842	0.000318521817202876\\
843	0.000456507341715699\\
844	0.000629465041073638\\
845	0.000816125156414811\\
846	0.000992745287921147\\
847	0.00113646447561773\\
848	0.00122373075713742\\
849	0.00156631145965543\\
850	0.00212374184419892\\
851	0.00240859861454261\\
852	0.00176486333357161\\
853	0.000906930635375353\\
854	0.000610945393266355\\
855	0.000437484960182159\\
856	0.000320910930578301\\
857	0.000255561276933503\\
858	0.000241492953024493\\
859	0.000273891590631342\\
860	0.000341589758426699\\
861	0.000428296062157071\\
862	0.000515600897522596\\
863	0.000585850956269706\\
864	0.000624542317160963\\
865	0.000623861805723626\\
866	0.000581667507766378\\
867	0.00050602096911442\\
868	0.000411826199957773\\
869	0.000317255560051472\\
870	0.000242340280181074\\
871	0.000200371625615077\\
872	0.000211282962855007\\
873	0.000279801876880124\\
874	0.000401933452748234\\
875	0.000566183408154279\\
876	0.000753142751393603\\
877	0.000939607760198233\\
878	0.00110177516727806\\
879	0.00120572595672144\\
880	0.00147003052071107\\
881	0.002031931694878\\
882	0.00252914575969915\\
883	0.00233297081533829\\
884	0.00126413675351689\\
885	0.00066807170018076\\
886	0.000463532018559295\\
887	0.000342030856581675\\
888	0.000266486233393715\\
889	0.000238150345173934\\
890	0.000256849719028467\\
891	0.000315132361928796\\
892	0.000399007941223429\\
893	0.000490823192913941\\
894	0.000571895899664025\\
895	0.000626177477056086\\
896	0.000643527979657455\\
897	0.000617580239727287\\
898	0.000554265951361426\\
899	0.000464408367215365\\
900	0.000365288507782651\\
901	0.0002771116106572\\
902	0.000215234656601149\\
903	0.000200304511362267\\
904	0.000241839226049361\\
905	0.000339387563866567\\
906	0.000484987186710803\\
907	0.000662302380739935\\
908	0.00084936096130919\\
909	0.00102229829143598\\
910	0.0011613789771757\\
911	0.00158139238585061\\
912	0.00232624410197159\\
913	0.003013169112208\\
914	0.0033254139893737\\
915	0.00217848418995653\\
916	0.000966150172487457\\
917	0.000600318078908047\\
918	0.000410120642467558\\
919	0.000297392282969715\\
920	0.00024684665006515\\
921	0.000252292923744979\\
922	0.000304497460202072\\
923	0.000389414927507485\\
924	0.000488814181362097\\
925	0.000583838572155576\\
926	0.000656302503594581\\
927	0.000692591241265441\\
928	0.000685651160984775\\
929	0.000634840347674909\\
930	0.000549762744563806\\
931	0.000445905990820644\\
932	0.00034250066031921\\
933	0.000256528330294193\\
934	0.000209719117782622\\
935	0.000215501609335754\\
936	0.000278096075043443\\
937	0.000393264047373379\\
938	0.000548492492238943\\
939	0.000724148148394823\\
940	0.000897215591326609\\
941	0.00104456769727309\\
942	0.0011532442088287\\
943	0.00128440326245484\\
944	0.00162229992667291\\
945	0.00187359805093436\\
946	0.0015492220740766\\
947	0.00089700128335385\\
948	0.000632865609361162\\
949	0.000458788821375911\\
950	0.000335327881818368\\
951	0.000261893111300188\\
952	0.000240307780485507\\
953	0.00026713704775816\\
954	0.000332392735783785\\
955	0.00042050257252645\\
956	0.000513199444364461\\
957	0.000592049684742961\\
958	0.000641451083406446\\
959	0.000652012572431126\\
960	0.000619085675946173\\
961	0.000550236690581787\\
962	0.000456358065352367\\
963	0.000355987427330555\\
964	0.000269701856010129\\
965	0.000212212546324169\\
966	0.000203837534819009\\
967	0.000253056183160497\\
968	0.000356923342147364\\
969	0.000506718135037239\\
970	0.000685180863714136\\
971	0.000869731243091373\\
972	0.00103654458004204\\
973	0.0012189121516838\\
974	0.00160695009496845\\
975	0.00229271667218913\\
976	0.00289502307952588\\
977	0.00303566691099189\\
978	0.00194984511489458\\
979	0.000909131791269871\\
980	0.000572044832605003\\
981	0.000392544881099998\\
982	0.000287977644055566\\
983	0.000244949266955759\\
984	0.000256877861694357\\
985	0.000313708996056211\\
986	0.00040058865958968\\
987	0.00049921365930993\\
988	0.000590339088536509\\
989	0.000656645918762809\\
990	0.000685549755281258\\
991	0.000670935269042344\\
992	0.00061401120518091\\
993	0.000525434426839345\\
994	0.0004212993612935\\
995	0.000321421077386252\\
996	0.000242235181385655\\
997	0.000205364399241816\\
998	0.000222686394957615\\
999	0.000296612747950613\\
1000	0.00042166066739356\\
1001	0.000583764705362883\\
1002	0.000762359888297753\\
1003	0.00093409825217965\\
1004	0.00107616073155198\\
1005	0.00116170352045589\\
1006	0.00135799120167816\\
1007	0.00175105400491714\\
1008	0.0019741311815799\\
1009	0.00155128662197028\\
1010	0.000872596718935708\\
1011	0.000606393885438196\\
1012	0.000437114307280124\\
1013	0.000320726820019644\\
1014	0.000255233386490676\\
1015	0.000241397730066217\\
1016	0.000274698326840316\\
1017	0.000344179178252343\\
1018	0.000433593935760156\\
1019	0.000524558381515391\\
1020	0.000599007436594982\\
1021	0.000642016041083742\\
1022	0.000645191462383723\\
1023	0.000605677184752804\\
1024	0.000531706368785543\\
1025	0.000435910487063503\\
1026	0.000337210531402041\\
1027	0.000256072981467432\\
1028	0.000206405595872775\\
1029	0.000208108588922281\\
1030	0.000267237644998869\\
1031	0.000380969630069713\\
1032	0.000538153369587256\\
1033	0.000720114278614487\\
1034	0.000904050806109682\\
1035	0.00106621288348477\\
1036	0.00117802412826874\\
1037	0.00136575249183317\\
1038	0.00183492072415053\\
1039	0.00229320016431021\\
1040	0.00217364498387622\\
1041	0.00152727478360749\\
1042	0.00074302840125351\\
1043	0.000490751196836961\\
1044	0.000353882927295792\\
1045	0.000271530102399851\\
1046	0.000239063543058397\\
1047	0.000254909148259234\\
1048	0.000311485468599184\\
1049	0.000394976335279702\\
1050	0.000487704667103902\\
1051	0.000570989462047339\\
1052	0.000628419187985707\\
1053	0.000649352023570346\\
1054	0.000626660770284426\\
1055	0.000565554618032518\\
1056	0.000476696693468658\\
1057	0.000376890108305166\\
1058	0.00028617546879159\\
1059	0.000220261638628779\\
1060	0.000199995252157482\\
1061	0.000235553297105286\\
1062	0.00032756404400039\\
1063	0.000468594030899513\\
1064	0.000642957620910315\\
1065	0.000829060120268296\\
1066	0.00100311680731939\\
1067	0.001142537487413\\
1068	0.00134193535433403\\
1069	0.00161914614467605\\
1070	0.00210789845948194\\
1071	0.00225821555350636\\
1072	0.00166301312221147\\
1073	0.000866780066434474\\
1074	0.000583847902375618\\
1075	0.000417206077067203\\
1076	0.000307625062395418\\
1077	0.000249885481061375\\
1078	0.000243407026431446\\
1079	0.000282587376310557\\
1080	0.000355664214417219\\
1081	0.000445713067127559\\
1082	0.000534313629753691\\
1083	0.000603597230945544\\
1084	0.000641158413584404\\
1085	0.000635881278506771\\
1086	0.000589691740624899\\
1087	0.000510713050903164\\
1088	0.000413974314709938\\
1089	0.000318047534062044\\
1090	0.000239732500384032\\
1091	0.000200476121596963\\
1092	0.000213785455272556\\
1093	0.000284056351792884\\
1094	0.000407502433005484\\
1095	0.000571599757552919\\
1096	0.000756696292679835\\
1097	0.000939556204389924\\
1098	0.00109660626504763\\
1099	0.00119547924978243\\
1100	0.00144657990725886\\
1101	0.00196919360997414\\
1102	0.00239768059402799\\
1103	0.0021193242582503\\
1104	0.00115134317490741\\
1105	0.000641846177158746\\
1106	0.000450647799830423\\
1107	0.000333148949911205\\
1108	0.000261698830607948\\
1109	0.000238034509423062\\
1110	0.000261202408501781\\
1111	0.000322981255544905\\
1112	0.000408813138474747\\
1113	0.000500859069423884\\
1114	0.000580486205522845\\
1115	0.000632204446472706\\
1116	0.000645599287602679\\
1117	0.000616260935168216\\
1118	0.000550087681329965\\
1119	0.000458328908255438\\
1120	0.000358834790806763\\
1121	0.000272024873679886\\
1122	0.000212767461787925\\
1123	0.000201919977771292\\
1124	0.000247651574530426\\
1125	0.000349337220127541\\
1126	0.000497800076838717\\
1127	0.000676296014731768\\
1128	0.000862533822946046\\
1129	0.00103265396077037\\
1130	0.00120809424000385\\
1131	0.00160272219300991\\
1132	0.00231918731399387\\
1133	0.00298128312342378\\
1134	0.00326968695138362\\
1135	0.00207377251392209\\
1136	0.000938906117485789\\
1137	0.000586162603464457\\
1138	0.000400706418803239\\
1139	0.000292101288462807\\
1140	0.000245688710953543\\
1141	0.000254866109151053\\
1142	0.0003098706006691\\
1143	0.00039617679408658\\
1144	0.000495541777773641\\
1145	0.000588729785029914\\
1146	0.000658127148412617\\
1147	0.000690667059912253\\
1148	0.000679661570145672\\
1149	0.000625492118727728\\
1150	0.000538376924526442\\
1151	0.000434094652283297\\
1152	0.0003322041662275\\
1153	0.000249403635530986\\
1154	0.000207441938569632\\
1155	0.000218879032831676\\
1156	0.000287044352713517\\
1157	0.000407143345653301\\
1158	0.000565756114887778\\
1159	0.000742761344009393\\
1160	0.000914972342654364\\
1161	0.00105941722708318\\
1162	0.00115850876124512\\
1163	0.00131723098716364\\
1164	0.00167850943186026\\
1165	0.00191077651668773\\
1166	0.00153668495533025\\
1167	0.000880961252774245\\
1168	0.000617540566686521\\
1169	0.000446449232120114\\
1170	0.000326974015300766\\
1171	0.000257981613159742\\
1172	0.000240764100237042\\
1173	0.000271306222827433\\
1174	0.000339093061042718\\
1175	0.000428173884411478\\
1176	0.000520191688470763\\
1177	0.00059696488188956\\
1178	0.000643100055218469\\
1179	0.000649796078124439\\
1180	0.00061329621823137\\
1181	0.000540924780425776\\
1182	0.000445946195670681\\
1183	0.000346490962785856\\
1184	0.000262812486321423\\
1185	0.000209059227564224\\
1186	0.000206059072762196\\
1187	0.000259904462557967\\
1188	0.000368723423962637\\
1189	0.000522300406696598\\
1190	0.000702421542232547\\
1191	0.000886510204599976\\
1192	0.00105077685389721\\
1193	0.00116364493325725\\
1194	0.00132629003955796\\
1195	0.00176588784400414\\
1196	0.0022351988760315\\
1197	0.0021872070499285\\
1198	0.00150798158847263\\
1199	0.000762982664797728\\
1200	0.000509708544224099\\
1201	0.0003668872751813\\
1202	0.000278577018744995\\
1203	0.000240648908625943\\
1204	0.000251763944629297\\
1205	0.000304697796189452\\
1206	0.000386018703030244\\
1207	0.000478105635364078\\
1208	0.000562338212116112\\
1209	0.000622035787014512\\
1210	0.000646147445516569\\
1211	0.000626774804574922\\
1212	0.000568439220270892\\
1213	0.000481524785749985\\
1214	0.000382378609518056\\
1215	0.000290784827288481\\
1216	0.00022288236295327\\
1217	0.000199237288225997\\
1218	0.000231184860024833\\
1219	0.00031947990413194\\
1220	0.000457801464012876\\
1221	0.000630934712999059\\
1222	0.00081757245973989\\
1223	0.000993964388830384\\
1224	0.00113727318213536\\
1225	0.00122450718784391\\
1226	0.00156918155364289\\
1227	0.00212633018833732\\
1228	0.00240609849012632\\
1229	0.00175697216840546\\
1230	0.000903743774316639\\
1231	0.000608888890505149\\
1232	0.000436018016976031\\
1233	0.000319971254923421\\
1234	0.000255151691068207\\
1235	0.000241587451622254\\
1236	0.000274410401416964\\
1237	0.000342401492530615\\
1238	0.000429232955715747\\
1239	0.000516494368724066\\
1240	0.00058654023286804\\
1241	0.00062492306168592\\
1242	0.000623872106157148\\
1243	0.000581327465229871\\
1244	0.000505439606010642\\
1245	0.000411128307172698\\
1246	0.000316607776606009\\
1247	0.000238619050562685\\
1248	0.000198799082530297\\
1249	0.000210937995382478\\
1250	0.000280021607126818\\
1251	0.000402883750721933\\
1252	0.00056750237217573\\
1253	0.000754550945468743\\
1254	0.000940870091306134\\
1255	0.00110268817663054\\
1256	0.00120590385991057\\
1257	0.00147192877124729\\
1258	0.00203492536513927\\
1259	0.0025296508864609\\
1260	0.00232419599433433\\
1261	0.00125736188548906\\
1262	0.000665768816555331\\
1263	0.00046219517849261\\
1264	0.000341120968716019\\
1265	0.000265998343636579\\
1266	0.000238106729206383\\
1267	0.00025721314657674\\
1268	0.00031580188778628\\
1269	0.000399832679363838\\
1270	0.000491632527386266\\
1271	0.000572529625007096\\
1272	0.000626533742066769\\
1273	0.00064352073338862\\
1274	0.000617204368867324\\
1275	0.000553614115689143\\
1276	0.000463587089505875\\
1277	0.000364461355506103\\
1278	0.000276460440732199\\
1279	0.000214918169285295\\
1280	0.000200432902378673\\
1281	0.00024245888284527\\
1282	0.000340475414965721\\
1283	0.00048642919674534\\
1284	0.000663925383612782\\
1285	0.000850957817977138\\
1286	0.00102365918970057\\
1287	0.00116890723065757\\
1288	0.00157709207510846\\
1289	0.0023155521085864\\
1290	0.00300669997088613\\
1291	0.00332491028719503\\
1292	0.00217895931048478\\
1293	0.000965965248158276\\
1294	0.000599891638839458\\
1295	0.000409719806809649\\
1296	0.000297153250990113\\
1297	0.000246820701674138\\
1298	0.000252462247831354\\
1299	0.000304785839313113\\
1300	0.000389700207210986\\
1301	0.000488963351714628\\
1302	0.000583714283823723\\
1303	0.000655802244038595\\
1304	0.000691664443730145\\
1305	0.000684308799536892\\
1306	0.000633182273367995\\
1307	0.000547954396730088\\
1308	0.000444138766127098\\
1309	0.000340995629092839\\
1310	0.000255480858284518\\
1311	0.00020930976117831\\
1312	0.000215822820853813\\
1313	0.000279151795085627\\
1314	0.000394991748600708\\
1315	0.000550728744889404\\
1316	0.000726672644789614\\
1317	0.000899784938451235\\
1318	0.00104694816668103\\
1319	0.00116243611113979\\
1320	0.00129259034602174\\
1321	0.00163332059215451\\
1322	0.00188122170392192\\
1323	0.00154774760136011\\
1324	0.000894652000597813\\
1325	0.000630799759806985\\
1326	0.000457209145537194\\
1327	0.000334278425815287\\
1328	0.000261389240637245\\
1329	0.000240327868974075\\
1330	0.00026760835412528\\
1331	0.000333190938515929\\
1332	0.000421463162398689\\
1333	0.000514145239711374\\
1334	0.000592846627051581\\
1335	0.00064193455151569\\
1336	0.000652118354085886\\
1337	0.00061882619323314\\
1338	0.000549713537637938\\
1339	0.000455663479883068\\
1340	0.000355284387687825\\
1341	0.000269168034566978\\
1342	0.000211984811772864\\
1343	0.000204034939017555\\
1344	0.000253744362419647\\
1345	0.00035799767679717\\
1346	0.000508074634083463\\
1347	0.000686647718422157\\
1348	0.000871106052308827\\
1349	0.00103762734740568\\
1350	0.00123031459670975\\
1351	0.00165217080496157\\
1352	0.00234457927043266\\
1353	0.0029299600647951\\
1354	0.00314901032310334\\
1355	0.00182512870334607\\
1356	0.000881889958205812\\
1357	0.000562412969524679\\
1358	0.000387386764185281\\
1359	0.000285255278504515\\
1360	0.000244233696265734\\
1361	0.000257943371272748\\
1362	0.000316319686691777\\
1363	0.000404445843349133\\
1364	0.000504011955969097\\
1365	0.000595720626917088\\
1366	0.000662247508939978\\
1367	0.000691051860925502\\
1368	0.000676065003187653\\
1369	0.000618586135549066\\
1370	0.000529322893876902\\
1371	0.000424429190233343\\
1372	0.000323776947445282\\
1373	0.000243823794733868\\
1374	0.000206194492811698\\
1375	0.000222754579567037\\
1376	0.000295919228789461\\
1377	0.000420133547912094\\
1378	0.000581329963189821\\
1379	0.000758946892141002\\
1380	0.000929663595964691\\
1381	0.00107071619680319\\
1382	0.00115634912583225\\
1383	0.00134019282396421\\
1384	0.00171418594580327\\
1385	0.00191980692439142\\
1386	0.00151873366941074\\
1387	0.000863436494361357\\
1388	0.000601632445670541\\
1389	0.000433849877454491\\
1390	0.00031860242806205\\
1391	0.000254270704959278\\
1392	0.000241602802399897\\
1393	0.00027598961518372\\
1394	0.000346387230187547\\
1395	0.000436480051896551\\
1396	0.000527864884322433\\
1397	0.000602444303367815\\
1398	0.000645356437448566\\
1399	0.000648227074501236\\
1400	0.000608276823657595\\
1401	0.00053388235597434\\
1402	0.000437595085425637\\
1403	0.000338448906398446\\
1404	0.000256942671539533\\
1405	0.000206968562221466\\
1406	0.000208408785349381\\
1407	0.000267281888381481\\
1408	0.00038070294864416\\
1409	0.000537479248057783\\
1410	0.000718911278915903\\
1411	0.000902198687541238\\
1412	0.00106361890303872\\
1413	0.00116950607382806\\
1414	0.00135479659670646\\
1415	0.00181540213419441\\
1416	0.00226521081910566\\
1417	0.00214047144492198\\
1418	0.0015069279046061\\
1419	0.000737986977563575\\
1420	0.000488579683675483\\
1421	0.000352573129664493\\
1422	0.000270828412639878\\
1423	0.000238957853221715\\
1424	0.000255360983270291\\
1425	0.00031239125922209\\
1426	0.000396173481350534\\
1427	0.000489016323495785\\
1428	0.000572230430518426\\
1429	0.000629429524735284\\
1430	0.000650018643682068\\
1431	0.000626955906602059\\
1432	0.000565529900412295\\
1433	0.000476428311677653\\
1434	0.000376509656785451\\
1435	0.000285841139397335\\
1436	0.000220106833585108\\
1437	0.000200135640686712\\
1438	0.000236038500720864\\
1439	0.000328367354313053\\
1440	0.000469725659788702\\
1441	0.000644078015133571\\
1442	0.000830016138550769\\
1443	0.00100374310423211\\
1444	0.0011426822378551\\
1445	0.00133785449026577\\
1446	0.00161764702080113\\
1447	0.00210583872349549\\
1448	0.00225496276850994\\
1449	0.00166060389205896\\
1450	0.000865267613271664\\
1451	0.000582511155647887\\
1452	0.000416113981851711\\
1453	0.000306905693820869\\
1454	0.000249610172078859\\
1455	0.000243572568659381\\
1456	0.000283120247903799\\
1457	0.000356434263172995\\
1458	0.000446556088926289\\
1459	0.000535058454646045\\
1460	0.000604105075425888\\
1461	0.000641331331695234\\
1462	0.000635680511321306\\
1463	0.000589152304776975\\
1464	0.000509949358533261\\
1465	0.000413123681972045\\
1466	0.000317286083515915\\
1467	0.000239227369581088\\
1468	0.000200367467828092\\
1469	0.00021416182377317\\
1470	0.000284920195177012\\
1471	0.000408785109163428\\
1472	0.000573155619457868\\
1473	0.000758331099410388\\
1474	0.000941057113984702\\
1475	0.00109777464600463\\
1476	0.00119355757007162\\
1477	0.00146721769460207\\
1478	0.00198247705213199\\
1479	0.00248070726118269\\
1480	0.00231148236772397\\
1481	0.00123545406636851\\
1482	0.000651449292262723\\
1483	0.000453227256831749\\
1484	0.00033566102695282\\
1485	0.00026340228098923\\
1486	0.000237996239428343\\
1487	0.000258964181906911\\
1488	0.00031840996887783\\
1489	0.00040198366950639\\
1490	0.000491937475254386\\
1491	0.000569793760423367\\
1492	0.000620152999191365\\
1493	0.000632713333110214\\
1494	0.000603160908505904\\
1495	0.000537486941645352\\
1496	0.000446915458058073\\
1497	0.000349307760323917\\
1498	0.000265024397805967\\
1499	0.000208784542758771\\
1500	0.000201372168490669\\
1501	0.000250643967738114\\
1502	0.000355858423758092\\
1503	0.000507621864037323\\
1504	0.000688992164832322\\
1505	0.000877548950526178\\
1506	0.00104934447472041\\
1507	0.00120421724571518\\
1508	0.0013499706447827\\
1509	0.00181657805769489\\
1510	0.00235793421854249\\
1511	0.00243427451005533\\
1512	0.00166299728439712\\
1513	0.00082332102836234\\
1514	0.000546457785462779\\
1515	0.000391714620835301\\
1516	0.000292941864508532\\
1517	0.00024469304662103\\
1518	0.000246260280230366\\
1519	0.000291443313226025\\
1520	0.000367770796223352\\
1521	0.000457887990180332\\
1522	0.000543406393429504\\
1523	0.000607320857986903\\
1524	0.000636363842200094\\
1525	0.000624546110298472\\
1526	0.000572186494849119\\
1527	0.000489492939181877\\
1528	0.000392112079106262\\
1529	0.000299362867649256\\
1530	0.00022750570913786\\
1531	0.000197961630032705\\
1532	0.000222732233162665\\
1533	0.000304846060829558\\
1534	0.00043768275578529\\
1535	0.000608163614707456\\
1536	0.000795694520493317\\
1537	0.000976602603448061\\
1538	0.00112756475542642\\
1539	0.00121761981274953\\
1540	0.00153748928282013\\
1541	0.00211272860848267\\
1542	0.00248408652662988\\
1543	0.0020153996923192\\
1544	0.000999793178385569\\
1545	0.000613451649482297\\
1546	0.000437668906279529\\
1547	0.000323829381327447\\
1548	0.000257076862208991\\
1549	0.000239186959731738\\
1550	0.000267629136640664\\
1551	0.000332805243053288\\
1552	0.000419256793110009\\
1553	0.00050887080067384\\
1554	0.000583643584077912\\
1555	0.000628269768062005\\
1556	0.000634163203390605\\
1557	0.000597804732833486\\
1558	0.000526291921645465\\
1559	0.000432890469981311\\
1560	0.000335728425337637\\
1561	0.000255067974645675\\
1562	0.000205065185111768\\
1563	0.000205792467079716\\
1564	0.000263843093695151\\
1565	0.000376919891016168\\
1566	0.000534247318523946\\
1567	0.000717773642526636\\
1568	0.000904576078856888\\
1569	0.00107079012218315\\
1570	0.00118965730676438\\
1571	0.00138722273712447\\
1572	0.00188607549794052\\
1573	0.0023973634199144\\
1574	0.00233710030077512\\
1575	0.00157915283069667\\
1576	0.000770119969176074\\
1577	0.000509164281856089\\
1578	0.000366359825212435\\
1579	0.000278454056996291\\
1580	0.000240558880029647\\
1581	0.000251427495119929\\
1582	0.000303942392202254\\
1583	0.000384731235682834\\
1584	0.000476204381818337\\
1585	0.000559788773045602\\
1586	0.000618860496087217\\
1587	0.000642424810059388\\
1588	0.000622681134700282\\
1589	0.000564248743897302\\
1590	0.000477503211796264\\
1591	0.000378827122261532\\
1592	0.000288029875182016\\
1593	0.000221178388588216\\
1594	0.000198832471454685\\
1595	0.000232221973325924\\
1596	0.000321872932686983\\
1597	0.000461442566021672\\
1598	0.000635599894655185\\
1599	0.00082296721936053\\
1600	0.000999756701800378\\
1601	0.00114313414348592\\
1602	0.00123206643537237\\
1603	0.0015926856037757\\
1604	0.00216504938229791\\
1605	0.00244821579781466\\
1606	0.00177630462431904\\
1607	0.000906055186148329\\
1608	0.000608579907455274\\
1609	0.000435530642268781\\
1610	0.000319660792745252\\
1611	0.000255033996288487\\
1612	0.000241613289989167\\
1613	0.000274496826964466\\
1614	0.000342437258957426\\
1615	0.000429091027563417\\
1616	0.000516057108794785\\
1617	0.000585715395391425\\
1618	0.000623674076558095\\
1619	0.000622218061578217\\
1620	0.000579370458175352\\
1621	0.0005033681786066\\
1622	0.000409140812554525\\
1623	0.000314937780632342\\
1624	0.000237474363942952\\
1625	0.000198376639227808\\
1626	0.000211352884093599\\
1627	0.000281282842209252\\
1628	0.000404927069468574\\
1629	0.000570166740705787\\
1630	0.000757612027514516\\
1631	0.000944069775194202\\
1632	0.0011057706836471\\
1633	0.00120804817855181\\
1634	0.00148111941107772\\
1635	0.00205197020103562\\
1636	0.00254959219927192\\
1637	0.00233535355194974\\
1638	0.00125894224239479\\
1639	0.000664334541878545\\
1640	0.000460880744233186\\
1641	0.000340227833291899\\
1642	0.000265539197991932\\
1643	0.000238052455420125\\
1644	0.000257497046283361\\
1645	0.000316313821820373\\
1646	0.000400425237644911\\
1647	0.000492145509612711\\
1648	0.000572817472998951\\
1649	0.00062651047996019\\
1650	0.00064266303365224\\
1651	0.000616436500949438\\
1652	0.000552560953981751\\
1653	0.000462416890698351\\
1654	0.000363344297861738\\
1655	0.000275608386737953\\
1656	0.000214313233365608\\
1657	0.000200535674066495\\
1658	0.000243772030750441\\
1659	0.000342063383761089\\
1660	0.000488251620708206\\
1661	0.000665943523379774\\
1662	0.000852999432245828\\
1663	0.00102551872923166\\
1664	0.00117998475652272\\
1665	0.00158167187725852\\
1666	0.00231357029673817\\
1667	0.00300995809423198\\
1668	0.00333252748953004\\
1669	0.00217825812077954\\
1670	0.000965711615991377\\
1671	0.000599435223347505\\
1672	0.000409228464882452\\
1673	0.000296832676352396\\
1674	0.00024676827274477\\
1675	0.00025267684524313\\
1676	0.000305193100697745\\
1677	0.000390172215418246\\
1678	0.000489359712398908\\
1679	0.00058388558373845\\
1680	0.000655632854383688\\
1681	0.000691089005996652\\
1682	0.00068332319765921\\
1683	0.000631870540988931\\
1684	0.000546466098912451\\
1685	0.000442649491518865\\
1686	0.000339715213845713\\
1687	0.00025459067464187\\
1688	0.00020898382708218\\
1689	0.000216148195599862\\
1690	0.000280127672117874\\
1691	0.000396556809149096\\
1692	0.000552723742596889\\
1693	0.000728885178720066\\
1694	0.00090198193051033\\
1695	0.00104890879117452\\
1696	0.00113614950301601\\
1697	0.00128703286520355\\
1698	0.00163893752455386\\
1699	0.00190081892874911\\
1700	0.00156489036438162\\
1701	0.000896764657461859\\
1702	0.00063050064697636\\
1703	0.000456976687957932\\
1704	0.000334270075887701\\
1705	0.000261447138051458\\
1706	0.000240315046783098\\
1707	0.000267414722488575\\
1708	0.000332711848481377\\
1709	0.000420596800827419\\
1710	0.000512803901974615\\
1711	0.000590970576187808\\
1712	0.000639531927933234\\
1713	0.000649246078954368\\
1714	0.000615636548276282\\
1715	0.000546448457034381\\
1716	0.000452547894050153\\
1717	0.000352602367933772\\
1718	0.0002671999879061\\
1719	0.00021094389682808\\
1720	0.000204104295795888\\
1721	0.000255060323911379\\
1722	0.000360367090480128\\
1723	0.000511386115149056\\
1724	0.000690678088582621\\
1725	0.000875574442426093\\
1726	0.00104223393563979\\
1727	0.00118485584689041\\
1728	0.00131908201709903\\
1729	0.00174525487040925\\
1730	0.0022202010052402\\
1731	0.00221366141323642\\
1732	0.00153659485561405\\
1733	0.000782954599267169\\
1734	0.000524026252923069\\
1735	0.000376561110376941\\
1736	0.000283972471618852\\
1737	0.000241998134634673\\
1738	0.0002495328522847\\
1739	0.000299724401234442\\
1740	0.00037951106055277\\
1741	0.000471247050832018\\
1742	0.000556663895173592\\
1743	0.000618540761632948\\
1744	0.000645522802482488\\
1745	0.000629096482830757\\
1746	0.000573190653540715\\
1747	0.000487832944254753\\
1748	0.000389013434368426\\
1749	0.000296322244776867\\
1750	0.000225959006510249\\
1751	0.000198991925330229\\
1752	0.000227041231296062\\
1753	0.000311566418117005\\
1754	0.000446879975619649\\
1755	0.000618298110708479\\
1756	0.000804828278986379\\
1757	0.000982793693535687\\
1758	0.00112917395132017\\
1759	0.00124386669908952\\
1760	0.00154996400874204\\
1761	0.00210077819329411\\
1762	0.00238907808217024\\
1763	0.00179045967185979\\
1764	0.000921118686451109\\
1765	0.000621427590448203\\
1766	0.000445100501345789\\
1767	0.000325758847500351\\
1768	0.000257660760313869\\
1769	0.00024105813789877\\
1770	0.000271456068955407\\
1771	0.000337996267856809\\
1772	0.000424550685709673\\
1773	0.000512828081072363\\
1774	0.000584956318841326\\
1775	0.000626111912639214\\
1776	0.000628116594753644\\
1777	0.000588278756894699\\
1778	0.000515101157553003\\
1779	0.000420994030718998\\
1780	0.000324996132025773\\
1781	0.000247497481167621\\
1782	0.000202155683111969\\
1783	0.000209468141188359\\
1784	0.000273159942588018\\
1785	0.000391578734024595\\
1786	0.000553062928320524\\
1787	0.000738651012340655\\
1788	0.000925330465790831\\
1789	0.00108924218325111\\
1790	0.00125717066787455\\
1791	0.00146695031033017\\
1792	0.00198400493610928\\
1793	0.00244945147348942\\
1794	0.00227737033478\\
1795	0.00165266601157096\\
1796	0.000742991719506694\\
1797	0.000479773113507521\\
1798	0.0003461937633638\\
1799	0.000267565538940851\\
1800	0.00023799314497862\\
1801	0.000256093087289254\\
1802	0.000314419966580695\\
1803	0.000399159390165042\\
1804	0.000492712033505709\\
1805	0.00057627022729049\\
1806	0.000633456381959139\\
1807	0.000653807284000544\\
1808	0.000630369329038957\\
1809	0.000568465416540773\\
1810	0.000478815509303661\\
1811	0.000378347625978754\\
1812	0.000287177034309914\\
1813	0.000220967440284884\\
1814	0.000200593043519339\\
1815	0.000236105390822826\\
1816	0.000328068256101618\\
1817	0.000468757570212381\\
1818	0.000642559328649088\\
1819	0.000827808786664342\\
1820	0.00100073977331572\\
1821	0.0011388372541135\\
1822	0.00134237007677737\\
1823	0.00160382586341154\\
1824	0.00207402085882257\\
1825	0.00220497719711271\\
1826	0.00163079715168712\\
1827	0.000857005739186758\\
1828	0.0005783881946023\\
1829	0.000413372367350435\\
1830	0.000305193575616094\\
1831	0.000248930535556785\\
1832	0.000243912604943649\\
1833	0.000284386423685187\\
1834	0.000358452878121312\\
1835	0.000449095392171957\\
1836	0.000537859774035013\\
1837	0.000606941369984365\\
1838	0.000642925373342819\\
1839	0.000637938415523177\\
1840	0.00059107241142335\\
1841	0.000511455580665456\\
1842	0.000414219965719087\\
1843	0.000318070917958123\\
1844	0.000239570007571803\\
1845	0.000200675581959191\\
1846	0.00021463469257428\\
1847	0.000285320916343841\\
1848	0.000408866777819063\\
1849	0.000572873203444093\\
1850	0.000757589319929107\\
1851	0.000939739129436656\\
1852	0.00109577997153304\\
1853	0.0011936508832756\\
1854	0.00144223818296503\\
1855	0.00195762901851064\\
1856	0.00237290325548939\\
1857	0.0020767565912015\\
1858	0.00112824324090823\\
1859	0.000637202555670648\\
1860	0.000448538708552094\\
1861	0.00033166911749113\\
1862	0.000260918171692421\\
1863	0.000238098080001233\\
1864	0.000262080689358564\\
1865	0.000324483810338713\\
1866	0.000410637125485509\\
1867	0.000502654250884447\\
1868	0.000581944412551364\\
1869	0.000633200166809471\\
1870	0.000645851077769972\\
1871	0.000615687217697484\\
1872	0.000548990771682967\\
1873	0.000456884676713353\\
1874	0.00035736742405509\\
1875	0.000270889938145313\\
1876	0.000212274993525049\\
1877	0.000202259428382669\\
1878	0.000248936808514266\\
1879	0.000351476353078161\\
1880	0.000500563237438627\\
1881	0.000679332892366594\\
1882	0.000865434253723243\\
1883	0.00103501062839413\\
1884	0.00121508096473691\\
1885	0.00161225513574766\\
1886	0.00232922080543049\\
1887	0.00298255081678398\\
1888	0.00326612549267047\\
1889	0.00203997512764244\\
1890	0.000934843434942947\\
1891	0.000583507250657416\\
1892	0.000398821515501883\\
1893	0.000291036513801598\\
1894	0.000245484546219555\\
1895	0.000255450273784231\\
1896	0.000311054353521352\\
1897	0.000397671228925088\\
1898	0.000497036224560459\\
1899	0.000589896146936288\\
1900	0.000658696899409005\\
1901	0.000690475123647259\\
1902	0.000678666723539423\\
1903	0.000623834985155189\\
1904	0.000536305432866018\\
1905	0.00043192626008471\\
1906	0.000330320293791635\\
1907	0.000248136495970973\\
1908	0.000207100931339983\\
1909	0.00021962824994658\\
1910	0.000288866261149343\\
1911	0.000409872746829479\\
1912	0.000569082077095865\\
1913	0.000746283389058832\\
1914	0.000918257271038466\\
1915	0.00106206690330449\\
1916	0.00117565047251634\\
1917	0.00132923805761954\\
1918	0.00168930115101074\\
1919	0.00190748600631082\\
1920	0.00152880900119892\\
1921	0.000875372768697409\\
1922	0.000612930842048588\\
1923	0.000442851639490387\\
1924	0.000324559857051247\\
1925	0.000256860964475324\\
1926	0.000240928589626108\\
1927	0.00027261149349843\\
1928	0.000341271246309563\\
1929	0.000430865642591321\\
1930	0.000523015164283081\\
1931	0.000599547116455108\\
1932	0.000645163789937172\\
1933	0.000651137952438749\\
1934	0.000614072647974733\\
1935	0.000541203202405415\\
1936	0.000445759329507487\\
1937	0.000346032149250096\\
1938	0.00026240644728828\\
1939	0.00020915365015506\\
1940	0.000206371945716147\\
1941	0.00026088049187527\\
1942	0.000370350024024074\\
1943	0.000524120966623482\\
1944	0.000704261397564201\\
1945	0.000887947481156636\\
1946	0.00105143432505272\\
1947	0.00116500989461609\\
1948	0.00132737749822991\\
1949	0.00176803016916226\\
1950	0.00222636065737261\\
1951	0.00215856959002856\\
1952	0.0014914401796016\\
1953	0.000755453813738239\\
1954	0.000505008043515939\\
1955	0.000363782395180659\\
1956	0.000276857909964657\\
1957	0.000240245939988044\\
1958	0.000252553672576693\\
1959	0.00030645232243108\\
1960	0.0003884038269618\\
1961	0.000480767690554422\\
1962	0.000564894762765195\\
1963	0.000624161207629392\\
1964	0.000647602407161392\\
1965	0.000627489845816245\\
1966	0.000568538222626438\\
1967	0.000481124286692122\\
1968	0.000381701860791068\\
1969	0.000290153754304461\\
1970	0.000222560551275264\\
1971	0.000199486679200848\\
1972	0.000232133450284562\\
1973	0.000321037036092817\\
1974	0.000459789459526713\\
1975	0.000633048528570833\\
1976	0.000819441065063089\\
1977	0.000995209729429408\\
1978	0.0011375754761709\\
1979	0.00122433166620911\\
1980	0.00156950032366694\\
1981	0.00212196560968648\\
1982	0.00238925959302409\\
1983	0.00174287520399429\\
1984	0.000897384720128855\\
1985	0.000604545250160976\\
1986	0.000432829894270495\\
1987	0.000317904158437888\\
1988	0.000254251215500051\\
1989	0.000241815691136375\\
1990	0.000275607689252107\\
1991	0.000344296411220519\\
1992	0.000431475493020997\\
1993	0.000518731874820114\\
1994	0.000588425069625254\\
1995	0.000626226274746326\\
1996	0.000624444911386361\\
1997	0.00058118651350545\\
1998	0.000504776509496712\\
1999	0.000410167772940728\\
2000	0.000315660922415848\\
};
\addlegendentry{$\text{V}_\text{1}$};

\addplot [color=mycolor2,solid]
  table[row sep=crcr]{%
1	52.9128\\
2	44.1607627952315\\
3	36.5930264011252\\
4	29.5555350566623\\
5	23.3016073117634\\
6	17.7896857667196\\
7	13.0289499852718\\
8	9.01787708202573\\
9	5.75438875003877\\
10	3.23404604507042\\
11	1.45048059549153\\
12	0.384413123642817\\
13	0.0634029397296949\\
14	0.0140576162440732\\
15	0.00573937033182431\\
16	0.00481003186815674\\
17	0.00514909475670998\\
18	0.00240303254655293\\
19	0.00130394228624197\\
20	0.000786969514975597\\
21	0.00058056463032608\\
22	0.000427535714352421\\
23	0.00031298495099423\\
24	0.000247459277650156\\
25	0.000237980795497334\\
26	0.000282333857402175\\
27	0.000370171943659088\\
28	0.00048460495839755\\
29	0.000604788639881769\\
30	0.000709480566677719\\
31	0.00078222963363357\\
32	0.000807695929583739\\
33	0.000783632819206238\\
34	0.000714774642081201\\
35	0.000612870036618863\\
36	0.000493545926742565\\
37	0.000377642849263883\\
38	0.000284674274970189\\
39	0.000230618463489882\\
40	0.000225508717629508\\
41	0.000271859730016886\\
42	0.000364225377608496\\
43	0.00048974194121275\\
44	0.000630585802200471\\
45	0.000766367184015341\\
46	0.0008766307974707\\
47	0.000944882445039245\\
48	0.000964489042149037\\
49	0.000937263978523133\\
50	0.000859799063939418\\
51	0.00071779025985079\\
52	0.000569262164962255\\
53	0.000429717995145866\\
54	0.000314211086552087\\
55	0.000245163297097205\\
56	0.000231407842811248\\
57	0.000273391744437568\\
58	0.000363284806866511\\
59	0.000485850997426757\\
60	0.000620822025989974\\
61	0.000746069755050587\\
62	0.000843369137851147\\
63	0.00089366761154517\\
64	0.000891332329984999\\
65	0.000837207473607781\\
66	0.000740771937660065\\
67	0.000617484901931216\\
68	0.00048526717337758\\
69	0.000365598315962096\\
70	0.000276866071777478\\
71	0.000232498618290913\\
72	0.000238969793253761\\
73	0.000294856582186154\\
74	0.000391062248536199\\
75	0.000511968243920592\\
76	0.000638828444630726\\
77	0.000751025069685352\\
78	0.000830912008649037\\
79	0.000865253025783842\\
80	0.000848254473604584\\
81	0.000781594164204642\\
82	0.000676452774701075\\
83	0.000549816449873719\\
84	0.000422595022524152\\
85	0.000312571973830533\\
86	0.000241657739896366\\
87	0.000224080953317019\\
88	0.00026172180644787\\
89	0.000351182129419297\\
90	0.000479345796755606\\
91	0.000626696416793219\\
92	0.000770978184864564\\
93	0.000890493526664671\\
94	0.000968694297136461\\
95	0.00103264453664329\\
96	0.0010482069717561\\
97	0.001046184094589\\
98	0.000802559769299776\\
99	0.000627662918947254\\
100	0.000473962695853918\\
101	0.000349877305245416\\
102	0.000267539828052452\\
103	0.000235088426507112\\
104	0.00025413971400195\\
105	0.000318955864576277\\
106	0.000417007645668602\\
107	0.000530792927911801\\
108	0.000640705750813203\\
109	0.000727884085267042\\
110	0.000777448885823436\\
111	0.000780790116993379\\
112	0.000735415757739675\\
113	0.000649319823200459\\
114	0.000537780675834109\\
115	0.000419580310878316\\
116	0.000312713956407602\\
117	0.000239328859285272\\
118	0.000215212645463718\\
119	0.000247967407183259\\
120	0.000333830989340315\\
121	0.000463789746126197\\
122	0.000620275741295296\\
123	0.000781015428668668\\
124	0.000923167539296729\\
125	0.00102819986306076\\
126	0.00110358743491935\\
127	0.00117679088787495\\
128	0.00126535073498991\\
129	0.00118632509598504\\
130	0.000812213658709129\\
131	0.000602781817178459\\
132	0.000442971610381248\\
133	0.000325480968387421\\
134	0.000256243240783334\\
135	0.000238925702229137\\
136	0.000271026795462476\\
137	0.000343181201602494\\
138	0.000440082737002234\\
139	0.000543321114945621\\
140	0.000634008223540384\\
141	0.000694992848116357\\
142	0.000716000516264828\\
143	0.000690347740579099\\
144	0.000623790755808887\\
145	0.000527247438631347\\
146	0.000418158355719059\\
147	0.000317086344975963\\
148	0.000240118927144746\\
149	0.000208189608436242\\
150	0.000231909879216845\\
151	0.000311616063342499\\
152	0.000440204534168345\\
153	0.000602123251645519\\
154	0.000776029537124852\\
155	0.000938567997441515\\
156	0.00106770781729395\\
157	0.00118344203123508\\
158	0.00132014813644599\\
159	0.00161130750456988\\
160	0.00166278706997433\\
161	0.00131598401381662\\
162	0.000777756484284172\\
163	0.000543768942691647\\
164	0.00039190908372042\\
165	0.000292223092984179\\
166	0.000244123470171209\\
167	0.000247196338200077\\
168	0.000295207845843752\\
169	0.000375715162072282\\
170	0.000471288198460456\\
171	0.000563262684657572\\
172	0.000634191946912785\\
173	0.000669748437505629\\
174	0.000663297268189915\\
175	0.000613942841875857\\
176	0.000531128573981861\\
177	0.000430217037007002\\
178	0.000330253830127989\\
179	0.000248069075900536\\
180	0.000205342697641883\\
181	0.000215325949877163\\
182	0.000282166842296458\\
183	0.000401749382790775\\
184	0.000561361514428015\\
185	0.000741267966955705\\
186	0.000918327258208629\\
187	0.00106924745548533\\
188	0.0012009383788109\\
189	0.00136716605066449\\
190	0.00177715683876513\\
191	0.00208322307962166\\
192	0.00172791197847244\\
193	0.000961788786868249\\
194	0.00061850762308258\\
195	0.000445494166296073\\
196	0.000328979974109609\\
197	0.000259255001199561\\
198	0.000238927457059505\\
199	0.000265904005785592\\
200	0.00033100514567872\\
201	0.000419057938729893\\
202	0.000511950314833446\\
203	0.000591223003877321\\
204	0.000641309808123015\\
205	0.000652651891371547\\
206	0.000620443571507809\\
207	0.000552072610861824\\
208	0.000458448898666696\\
209	0.000357973245729686\\
210	0.000271197416052802\\
211	0.000212918750656111\\
212	0.000203499990795327\\
213	0.000251521660560628\\
214	0.000354419249774398\\
215	0.000503452926162893\\
216	0.00068151218672128\\
217	0.000866087043541625\\
218	0.00103335329238734\\
219	0.00121841491995879\\
220	0.00159879893050667\\
221	0.00227428273426711\\
222	0.00288119129796782\\
223	0.0031171388546059\\
224	0.00196187885023464\\
225	0.000912808088388629\\
226	0.000573820277490701\\
227	0.000393591091394083\\
228	0.000288484983837614\\
229	0.000244986995715144\\
230	0.000256555893404112\\
231	0.000313213132068293\\
232	0.000400173769077441\\
233	0.000499147455998565\\
234	0.000590878408958391\\
235	0.000657975742247722\\
236	0.000687743489022733\\
237	0.000673935007881997\\
238	0.000617571198238087\\
239	0.000529221227435541\\
240	0.000424908699959735\\
241	0.00032439870662223\\
242	0.000244195788327731\\
243	0.000205956779677227\\
244	0.000221717776751825\\
245	0.000294148964914124\\
246	0.000417859287016381\\
247	0.000578972953365851\\
248	0.000757031293981931\\
249	0.000928726578553276\\
250	0.00107119548444728\\
251	0.00115861110406034\\
252	0.00134443424153279\\
253	0.00172693423112291\\
254	0.00195043278563677\\
255	0.00154253524874611\\
256	0.000873148705037355\\
257	0.000608377312162105\\
258	0.000438891266650113\\
259	0.000321917746414882\\
260	0.000255740234029471\\
261	0.000241251772792461\\
262	0.000274040750167774\\
263	0.000343243938446566\\
264	0.000432674505890223\\
265	0.000523948897300284\\
266	0.000598974631559518\\
267	0.000642716831787219\\
268	0.000646687507702604\\
269	0.000607873843001936\\
270	0.000534224837920526\\
271	0.000438548280532914\\
272	0.00033957857450715\\
273	0.000257759730548842\\
274	0.000207103422731552\\
275	0.000207568133278403\\
276	0.00026542451647823\\
277	0.000377990792528822\\
278	0.000534286870131351\\
279	0.000715759441574523\\
280	0.000899666052084103\\
281	0.00106224783932383\\
282	0.00116343625423617\\
283	0.00135174865459649\\
284	0.00181797533638871\\
285	0.00228329672536179\\
286	0.00218556462972855\\
287	0.00150415300748456\\
288	0.000746192582493237\\
289	0.000495952391050969\\
290	0.000357638820958261\\
291	0.000273576353089313\\
292	0.000239523417022279\\
293	0.000253983078587266\\
294	0.00030945206523686\\
295	0.000392203523429499\\
296	0.000484573257585518\\
297	0.000567900567823037\\
298	0.000625723767772589\\
299	0.000647309447550334\\
300	0.000625361978822928\\
301	0.000564930822828684\\
302	0.00047662983571426\\
303	0.000377149232070459\\
304	0.000286468356458168\\
305	0.000220337049530558\\
306	0.000199640717771474\\
307	0.000234678064172352\\
308	0.000326222043346715\\
309	0.000466976420943425\\
310	0.000641368951800357\\
311	0.000827859009992701\\
312	0.00100265897634319\\
313	0.00114311410081389\\
314	0.00134654376620883\\
315	0.00162376743879243\\
316	0.00211828689275947\\
317	0.002281115542312\\
318	0.00167942898514802\\
319	0.000873330816772076\\
320	0.000588145598272176\\
321	0.000420296897908664\\
322	0.000309573858763377\\
323	0.000250656326122786\\
324	0.000243046728198273\\
325	0.00028126494717317\\
326	0.000353569139959611\\
327	0.000443256886387924\\
328	0.000531884933907417\\
329	0.000601655933298471\\
330	0.000638597975215986\\
331	0.000635161621843676\\
332	0.000589857893165501\\
333	0.0005114350567845\\
334	0.00041496722909868\\
335	0.000319012348862736\\
336	0.000240311101582048\\
337	0.000200504394856229\\
338	0.000213078802812065\\
339	0.000282580799334389\\
340	0.000405417588468684\\
341	0.000569179685671553\\
342	0.000754304818053224\\
343	0.000937588553256032\\
344	0.00109542660256092\\
345	0.00119573564926053\\
346	0.00144452992875164\\
347	0.00196778588432147\\
348	0.00240461426339484\\
349	0.00214330930121422\\
350	0.00116771608562994\\
351	0.000646124897394289\\
352	0.000452941322960222\\
353	0.00033471683366484\\
354	0.000262512003097785\\
355	0.000238018175301944\\
356	0.000260420746435615\\
357	0.000321634872684245\\
358	0.000407202280095366\\
359	0.000499326198599241\\
360	0.000579334989260745\\
361	0.000631568420492203\\
362	0.000645735589964253\\
363	0.000617165494792897\\
364	0.000551511238974247\\
365	0.000460065597486224\\
366	0.00036054795191071\\
367	0.000273345113410626\\
368	0.000213391405132135\\
369	0.000201612127092154\\
370	0.000246326128644404\\
371	0.000347070796053097\\
372	0.000494822072674404\\
373	0.000672963387827577\\
374	0.000859265389789456\\
375	0.0010298706942766\\
376	0.00119883871405529\\
377	0.00159072217895106\\
378	0.00230521991650426\\
379	0.00296791323608902\\
380	0.00327128114556971\\
381	0.0021106474352131\\
382	0.000947537530970778\\
383	0.000589943457142247\\
384	0.00040300076143322\\
385	0.000293347106209329\\
386	0.000245967586172848\\
387	0.000254293371203184\\
388	0.000308652998363746\\
389	0.000394612517701542\\
390	0.000493859605297601\\
391	0.000587445413006844\\
392	0.000657450860302223\\
393	0.000690654759107637\\
394	0.000680483745348382\\
395	0.000627007736976646\\
396	0.000540326646785164\\
397	0.000436163319123678\\
398	0.000334014363044146\\
399	0.000250672586300148\\
400	0.000207788334212569\\
401	0.000218160392308838\\
402	0.000285304816708924\\
403	0.000404511341538971\\
404	0.00056254196125535\\
405	0.000739365573245835\\
406	0.00091182686370825\\
407	0.00105691758337388\\
408	0.00116476605815344\\
409	0.0013136414501111\\
410	0.00167016910260938\\
411	0.00190423655322425\\
412	0.00153717051867057\\
413	0.000883548067834404\\
414	0.000620206448062178\\
415	0.000448625920823071\\
416	0.0003284416326199\\
417	0.000258647740488445\\
418	0.000240646120746775\\
419	0.000270536362664246\\
420	0.000337905592067302\\
421	0.000426873634067383\\
422	0.00051909529765824\\
423	0.000596488698655081\\
424	0.000643194486728546\\
425	0.000650669064189291\\
426	0.000614866940010815\\
427	0.000542994669697726\\
428	0.000448205921589561\\
429	0.000348566174182692\\
430	0.000264325824969168\\
431	0.00020971662095049\\
432	0.000205663481329139\\
433	0.000258405875297757\\
434	0.000366170214946623\\
435	0.000518800136860122\\
436	0.000698567447028067\\
437	0.000882713687942636\\
438	0.00104745326139759\\
439	0.00116095700029553\\
440	0.00132614657456473\\
441	0.00175842732587002\\
442	0.00222814346699139\\
443	0.00220330489056327\\
444	0.00151722422905166\\
445	0.000768717940693663\\
446	0.000514035269533981\\
447	0.000370014732995656\\
448	0.000280393234751898\\
449	0.000241093748927649\\
450	0.000250911522893652\\
451	0.000302762189095463\\
452	0.00038333885079527\\
453	0.000475045155754233\\
454	0.000559300734654209\\
455	0.00061937097411512\\
456	0.000644121606784763\\
457	0.000625486400950392\\
458	0.000567799547121351\\
459	0.000481435400692684\\
460	0.000382644447983169\\
461	0.000291106285640861\\
462	0.0002230115680193\\
463	0.000198925185370546\\
464	0.000230312432938251\\
465	0.00031812937354237\\
466	0.00045614557776417\\
467	0.000629268300600784\\
468	0.000816252708341618\\
469	0.0009933522972502\\
470	0.00113767391196908\\
471	0.00122556198984557\\
472	0.00157178983770731\\
473	0.00213572382297068\\
474	0.00242998788552489\\
475	0.00179349861342737\\
476	0.000913923629330896\\
477	0.000613930509992319\\
478	0.000439337228647765\\
479	0.000322071488069071\\
480	0.000256072995844293\\
481	0.000241378582821053\\
482	0.000273228610972524\\
483	0.000340515088503879\\
484	0.000426947939343559\\
485	0.000514240801002898\\
486	0.000584625526317741\\
487	0.000623600608799575\\
488	0.000623280062577521\\
489	0.00058145225359998\\
490	0.000506089408995839\\
491	0.000412080244973083\\
492	0.000317551302298956\\
493	0.00024252737619463\\
494	0.000200339656694462\\
495	0.000210958010139605\\
496	0.000279108799181105\\
497	0.000401011362079264\\
498	0.000565154887742852\\
499	0.000752173932831184\\
500	0.000938887331727044\\
501	0.00110147825907944\\
502	0.00120622209047266\\
503	0.00146989115462469\\
504	0.00203328922811541\\
505	0.00253811134783056\\
506	0.00234868891596606\\
507	0.00130195253544474\\
508	0.000686766588648469\\
509	0.000473053752099663\\
510	0.000346830946602969\\
511	0.000268701475471911\\
512	0.00023869348458774\\
513	0.00025614243335098\\
514	0.000313449736453871\\
515	0.000396614248024598\\
516	0.000487991929948427\\
517	0.000568896201918816\\
518	0.000623214216486717\\
519	0.000640791239728487\\
520	0.000615230705825777\\
521	0.000552340508376159\\
522	0.000462928169282134\\
523	0.000364205068481152\\
524	0.00027634412295929\\
525	0.00021468409309199\\
526	0.000199946974518168\\
527	0.000241706393479327\\
528	0.000339404909409176\\
529	0.00048527615996226\\
530	0.000662988610723482\\
531	0.000850583430469494\\
532	0.00102417666482481\\
533	0.00116907606527291\\
534	0.00158480535647953\\
535	0.00233853895963575\\
536	0.00304844995871708\\
537	0.00336226695391959\\
538	0.00220439666764534\\
539	0.000973299181083455\\
540	0.000603765198378278\\
541	0.000412176365222642\\
542	0.000298514525405714\\
543	0.000247174876638096\\
544	0.000251923588128467\\
545	0.000303554654705661\\
546	0.000388071768360385\\
547	0.000487251803012177\\
548	0.000582266238078923\\
549	0.000654900789226641\\
550	0.00069149785521988\\
551	0.000684947765361318\\
552	0.000634522040069482\\
553	0.000549755524050347\\
554	0.000446105794325633\\
555	0.000342761410394776\\
556	0.000256715542065581\\
557	0.000209702711246614\\
558	0.000215213206722111\\
559	0.000277536782762425\\
560	0.000392498264411348\\
561	0.000547653922801196\\
562	0.000723410191434348\\
563	0.000896761415922141\\
564	0.00104455822768761\\
565	0.00115279778513317\\
566	0.00128466823409825\\
567	0.00162626671456445\\
568	0.00188322063713138\\
569	0.0015602668909539\\
570	0.000894133824722319\\
571	0.000628970525215407\\
572	0.000456430742090654\\
573	0.00033429732057767\\
574	0.000261461660437634\\
575	0.000240044179136808\\
576	0.00026688658087377\\
577	0.000332166063960866\\
578	0.00042039388209329\\
579	0.000513329276206237\\
580	0.000592513219640273\\
581	0.00064232902841599\\
582	0.000653312894329105\\
583	0.000620730749361397\\
584	0.00055205978097095\\
585	0.000458203830529679\\
586	0.000357635977743599\\
587	0.000270926535929137\\
588	0.000212838336952577\\
589	0.000203716820555106\\
590	0.000252089261598901\\
591	0.00035525197801235\\
592	0.000504448963314669\\
593	0.000682509435205321\\
594	0.000866897384207694\\
595	0.00103379545607234\\
596	0.00121191558368665\\
597	0.00159356397952381\\
598	0.0022671965324644\\
599	0.00286456757471868\\
600	0.00308387959862343\\
601	0.00194457193995905\\
602	0.00090792982379295\\
603	0.000571739891991869\\
604	0.000392496449094798\\
605	0.000287960092072652\\
606	0.00024488566288575\\
607	0.000256778630957166\\
608	0.000313645186294025\\
609	0.000400670737227035\\
610	0.000499560725739305\\
611	0.000591060557361869\\
612	0.00065781244204386\\
613	0.000687175018395169\\
614	0.000672965649364114\\
615	0.000616298876984326\\
616	0.000527789130842414\\
617	0.000423496266906416\\
618	0.000323214103983794\\
619	0.000243416900908542\\
620	0.000205746280943063\\
621	0.000222163952083639\\
622	0.000295222390550641\\
623	0.000419482446969747\\
624	0.000580980447412023\\
625	0.000759209054004294\\
626	0.00093084369912295\\
627	0.00107305513527677\\
628	0.00115960253917135\\
629	0.00134925840328995\\
630	0.0017350990722826\\
631	0.00195703828697306\\
632	0.0015439485410531\\
633	0.000872173960033735\\
634	0.000607088186265426\\
635	0.000437813803821428\\
636	0.000321199373597495\\
637	0.000255428306248585\\
638	0.000241332762402954\\
639	0.000274443538358975\\
640	0.000343847657256618\\
641	0.000433325253166422\\
642	0.000524490649795178\\
643	0.000599261919871608\\
644	0.00064265918943273\\
645	0.000646239353229549\\
646	0.000607068100865015\\
647	0.000533236604604932\\
648	0.000437465958025695\\
649	0.000338584480695568\\
650	0.000257049374667466\\
651	0.000206823512635109\\
652	0.000207837462778549\\
653	0.00026625976535857\\
654	0.000379339178431226\\
655	0.000536012779481462\\
656	0.000717669776325393\\
657	0.000901540263880234\\
658	0.00106387155023285\\
659	0.00116886167890319\\
660	0.00135614752783505\\
661	0.00182277600316852\\
662	0.0022849307266873\\
663	0.00217910914440653\\
664	0.00150210200263153\\
665	0.000743172687637733\\
666	0.000493671082626507\\
667	0.000356146088743496\\
668	0.000272782197626415\\
669	0.000239356844622404\\
670	0.000254359528729883\\
671	0.000310248384442252\\
672	0.000393253335283579\\
673	0.000485703331113312\\
674	0.000568931931873147\\
675	0.000626506334535336\\
676	0.000647739790017594\\
677	0.000625422398470631\\
678	0.000564683606112626\\
679	0.000476161195095901\\
680	0.000376598996404515\\
681	0.000286004389968143\\
682	0.000220101769595733\\
683	0.000199753943882176\\
684	0.000235192278649396\\
685	0.000327112005115095\\
686	0.000468142675406273\\
687	0.000642648077760433\\
688	0.000829056963418393\\
689	0.00100357827556905\\
690	0.00114358766166597\\
691	0.00134263250288705\\
692	0.00162377355041922\\
693	0.00211873874340339\\
694	0.00227761539592951\\
695	0.00167510457785333\\
696	0.000870913098870041\\
697	0.000586339425669662\\
698	0.000418974559184575\\
699	0.000308744694683055\\
700	0.000250331098220323\\
701	0.000243195534550981\\
702	0.000281803420562378\\
703	0.000354363314870553\\
704	0.000444140424977806\\
705	0.000532697770149971\\
706	0.0006022348499955\\
707	0.000638846211882907\\
708	0.000635030743044352\\
709	0.000589381999821994\\
710	0.000510728014414221\\
711	0.000414163622024016\\
712	0.000318285682524621\\
713	0.000239819309829719\\
714	0.00020039901709267\\
715	0.000213446910532961\\
716	0.000283423403381468\\
717	0.000406666366296287\\
718	0.000570690974175584\\
719	0.00075588669579727\\
720	0.000939030013396941\\
721	0.00109653014694521\\
722	0.00119598610886111\\
723	0.00144716250887428\\
724	0.00197179231299288\\
725	0.00240585726646751\\
726	0.00213587968058009\\
727	0.00116156832492703\\
728	0.000644227801607283\\
729	0.000451815563607757\\
730	0.000333942151673319\\
731	0.000262112034341817\\
732	0.000238020132798163\\
733	0.00026078657426307\\
734	0.000322262797103079\\
735	0.000407943529440742\\
736	0.000500012444457871\\
737	0.000579817256907028\\
738	0.000631779162352971\\
739	0.000645562494767397\\
740	0.00061660876420565\\
741	0.000550700525800569\\
742	0.000459109116951371\\
743	0.000359617989857933\\
744	0.000272629974023786\\
745	0.000213046039451134\\
746	0.000201756678577471\\
747	0.000247004877908767\\
748	0.000348245884345873\\
749	0.000496378127559452\\
750	0.000674720278683122\\
751	0.000861010904434996\\
752	0.00103139050873785\\
753	0.00120405030310169\\
754	0.00159719795869181\\
755	0.00231315361299532\\
756	0.0029768614535483\\
757	0.00327259234085164\\
758	0.00209252840877831\\
759	0.000943320912262569\\
760	0.000588088349059241\\
761	0.000401861509716305\\
762	0.000292723232426615\\
763	0.000245826914491254\\
764	0.000254581665360724\\
765	0.000309268034916138\\
766	0.000395405278047875\\
767	0.000494758929867101\\
768	0.000588102255099051\\
769	0.000657791788125492\\
770	0.000690709265868057\\
771	0.000680105224094764\\
772	0.000626270839043885\\
773	0.000539368764701232\\
774	0.00043514310861961\\
775	0.000333120748371776\\
776	0.000250023392812002\\
777	0.000207610610758698\\
778	0.000218515077850142\\
779	0.000286157115357672\\
780	0.000405810502532481\\
781	0.000564131665048776\\
782	0.000741044554033893\\
783	0.000913378656088704\\
784	0.00105814501174483\\
785	0.00116783362526496\\
786	0.00131865887584239\\
787	0.00167569181149974\\
788	0.00190466176342928\\
789	0.00153499984701814\\
790	0.000881314822524623\\
791	0.000618210874461485\\
792	0.000447033786098219\\
793	0.000327363701840624\\
794	0.000258143049486594\\
795	0.000240712223611578\\
796	0.000271104604834789\\
797	0.000338847954817746\\
798	0.000428018666266077\\
799	0.000520260597282042\\
800	0.000597370906055731\\
801	0.000643907523626155\\
802	0.000651017861605553\\
803	0.000614854580030185\\
804	0.000542680920641923\\
805	0.000447709694268416\\
806	0.000348043322408859\\
807	0.000263937719734508\\
808	0.00020958991984628\\
809	0.000205892362965993\\
810	0.000259007387225801\\
811	0.000367106469000634\\
812	0.000519976354286456\\
813	0.000699791375447935\\
814	0.000883784045796174\\
815	0.00104817967101452\\
816	0.00115797078783207\\
817	0.00132730290657528\\
818	0.00176123370858364\\
819	0.00222860196790538\\
820	0.00219525127318854\\
821	0.00150806297511659\\
822	0.000764568756787312\\
823	0.000511534024674159\\
824	0.000368408917353261\\
825	0.000279507409012316\\
826	0.000240869306794307\\
827	0.000251279671857133\\
828	0.000303607916946722\\
829	0.000384495872991643\\
830	0.000476338500395289\\
831	0.000560539796228074\\
832	0.000620394348436453\\
833	0.000644810303254036\\
834	0.00062580632359298\\
835	0.000567809622129954\\
836	0.000481200888019833\\
837	0.000382276604410677\\
838	0.000290766436901123\\
839	0.000222832786503704\\
840	0.000199040557486134\\
841	0.000230787566134081\\
842	0.000318920648680047\\
843	0.000457165653734113\\
844	0.000630364647888773\\
845	0.000817238732099639\\
846	0.000994036940479135\\
847	0.0011378913282895\\
848	0.00122561205837936\\
849	0.00157238830634177\\
850	0.00213420895034689\\
851	0.0024214123452664\\
852	0.00177557957974627\\
853	0.000908964051817804\\
854	0.000611497428867552\\
855	0.000437752574592087\\
856	0.00032107518180588\\
857	0.000255636600269704\\
858	0.00024147389432601\\
859	0.000273776744564879\\
860	0.00034138404516145\\
861	0.000428010063197178\\
862	0.000515248114772419\\
863	0.000585449365325682\\
864	0.000624109895416057\\
865	0.000623419959007007\\
866	0.000581237489566347\\
867	0.000505623550107288\\
868	0.00041148166566939\\
869	0.000316981489575686\\
870	0.000242151600725544\\
871	0.000200279247197027\\
872	0.000211291452551988\\
873	0.000279934367472756\\
874	0.000402165090849272\\
875	0.00056651086285254\\
876	0.000753559477799009\\
877	0.000940101493606156\\
878	0.00110233015296807\\
879	0.00120629184233665\\
880	0.00147182686385602\\
881	0.00203551921688797\\
882	0.00253508893133891\\
883	0.00233837012426113\\
884	0.00126591682193011\\
885	0.000668242490796214\\
886	0.00046352394420932\\
887	0.000342025705549134\\
888	0.000266487431232515\\
889	0.000238146566314544\\
890	0.000256834476527598\\
891	0.000315103522933563\\
892	0.000398965953472723\\
893	0.000490769842350147\\
894	0.000571833676649774\\
895	0.000626108668216158\\
896	0.000643455865123275\\
897	0.000617508723360727\\
898	0.000554198525557443\\
899	0.000464347922067264\\
900	0.000365238055441353\\
901	0.000277074276353403\\
902	0.000215211179410454\\
903	0.000200298124227429\\
904	0.000241851386994851\\
905	0.000339416138512273\\
906	0.000485032513170212\\
907	0.000662362915934407\\
908	0.000849435037509595\\
909	0.0010223832498485\\
910	0.00116237414709897\\
911	0.00158177354298736\\
912	0.00232636462766004\\
913	0.00301423979615152\\
914	0.00332734095176486\\
915	0.00217938283792807\\
916	0.000966395993895315\\
917	0.000600420597195392\\
918	0.000410169847547051\\
919	0.000297415623030026\\
920	0.000246854601433757\\
921	0.000252289640557104\\
922	0.000304484809289165\\
923	0.000389394215908273\\
924	0.000488786487840943\\
925	0.000583805307305983\\
926	0.00065626520723646\\
927	0.000692551599869534\\
928	0.000685611126356192\\
929	0.000634801565352204\\
930	0.000549726868892817\\
931	0.0004458750353329\\
932	0.000342476411475031\\
933	0.000256512055102129\\
934	0.000209710945752245\\
935	0.00021550201639443\\
936	0.000278105542526017\\
937	0.00039328285966288\\
938	0.000548519875509688\\
939	0.000724183279349255\\
940	0.000897257342051243\\
941	0.00104461441497116\\
942	0.00115341926158371\\
943	0.00128430118284883\\
944	0.00162258125702062\\
945	0.00187399020261135\\
946	0.00154948956032774\\
947	0.000896939070996717\\
948	0.000632790913455883\\
949	0.000458746387279202\\
950	0.000335308468793473\\
951	0.000261885113239076\\
952	0.00024030460584022\\
953	0.00026713563815108\\
954	0.000332391824125689\\
955	0.000420501697308487\\
956	0.000513198432028147\\
957	0.000592048574718143\\
958	0.000641449750398416\\
959	0.000652011299876166\\
960	0.000619084428225832\\
961	0.000550235604816635\\
962	0.000456357044706392\\
963	0.000355986499874795\\
964	0.000269701260518257\\
965	0.000212211519371217\\
966	0.000203837252624679\\
967	0.000253056997290416\\
968	0.000356924054253651\\
969	0.00050671897520905\\
970	0.000685181868205665\\
971	0.000869732529410447\\
972	0.00103654614169031\\
973	0.00121891508055111\\
974	0.00160695858052637\\
975	0.00229273405787935\\
976	0.00289505068593462\\
977	0.00303569074149626\\
978	0.00194984696584804\\
979	0.000909132399314776\\
980	0.000572045039176945\\
981	0.000392544907268879\\
982	0.000287977635114303\\
983	0.000244949260217404\\
984	0.000256877903151606\\
985	0.000313709087215537\\
986	0.000400588798631512\\
987	0.000499213828962429\\
988	0.000590339273291448\\
989	0.000656646186124849\\
990	0.000685550011185814\\
991	0.000670935602857173\\
992	0.00061401141268136\\
993	0.000525434412282452\\
994	0.000421299645263406\\
995	0.00032142162681086\\
996	0.000242235739489052\\
997	0.000205364899153538\\
998	0.000222686635281199\\
999	0.000296612723406529\\
1000	0.000421660651151039\\
1001	0.000583764594995548\\
1002	0.000762359695902939\\
1003	0.000934097869473848\\
1004	0.00107616201504324\\
1005	0.00116170388762981\\
1006	0.00135799068739524\\
1007	0.00175105167248451\\
1008	0.00197412858170403\\
1009	0.00155128422945754\\
1010	0.00087259620812894\\
1011	0.000606393755984411\\
1012	0.000437114221572416\\
1013	0.000320726766111841\\
1014	0.000255233358050808\\
1015	0.000241397731648801\\
1016	0.000274698370675298\\
1017	0.000344179259556563\\
1018	0.000433594052933839\\
1019	0.000524558541651492\\
1020	0.000599007594041852\\
1021	0.000642016296439343\\
1022	0.000645191536473871\\
1023	0.000605677343535871\\
1024	0.000531706569901984\\
1025	0.000435911125226784\\
1026	0.000337211266534731\\
1027	0.000256073448111062\\
1028	0.000206406300691999\\
1029	0.0002081086535299\\
1030	0.000267237614423312\\
1031	0.000380969723194368\\
1032	0.000538153439789092\\
1033	0.000720114121218642\\
1034	0.00090405101329526\\
1035	0.00106621275832434\\
1036	0.00117801857651679\\
1037	0.00136570846059004\\
1038	0.00183484869654344\\
1039	0.00229316001828683\\
1040	0.00217366317419667\\
1041	0.00152729682165406\\
1042	0.000743042217568738\\
1043	0.000490759326409912\\
1044	0.00035388761967778\\
1045	0.00027153249425255\\
1046	0.000239064182961364\\
1047	0.000254908273291532\\
1048	0.000311483236978041\\
1049	0.000394972911580146\\
1050	0.000487700220245686\\
1051	0.000570984227309316\\
1052	0.000628413450144533\\
1053	0.000649345978463301\\
1054	0.000626654734401279\\
1055	0.000565548923161503\\
1056	0.000476691878525465\\
1057	0.000376886343169847\\
1058	0.000286172454928979\\
1059	0.000220259879715612\\
1060	0.000199994405753768\\
1061	0.000235553910165617\\
1062	0.000327566025165431\\
1063	0.000468597990257319\\
1064	0.000642962594172205\\
1065	0.0008290663361603\\
1066	0.00100312374285753\\
1067	0.00114254719918482\\
1068	0.00134192463696532\\
1069	0.00161917194881539\\
1070	0.00210795585180708\\
1071	0.00225829435940729\\
1072	0.00166304050235353\\
1073	0.00086678649995024\\
1074	0.000583851088637706\\
1075	0.000417208247455866\\
1076	0.000307626489369163\\
1077	0.00024988608417235\\
1078	0.000243406748743645\\
1079	0.000282586202958768\\
1080	0.000355662206237651\\
1081	0.000445710333553681\\
1082	0.000534310324378575\\
1083	0.000603593504688376\\
1084	0.000641154403206222\\
1085	0.000635876976510524\\
1086	0.000589687756653272\\
1087	0.000510709661983099\\
1088	0.00041397160898586\\
1089	0.000318045307794981\\
1090	0.000239731014718491\\
1091	0.000200475560335303\\
1092	0.000213785817016679\\
1093	0.000284057282064739\\
1094	0.000407504459653519\\
1095	0.000571602821416402\\
1096	0.00075670001364271\\
1097	0.000939560680951076\\
1098	0.00109661138867762\\
1099	0.00119548372084528\\
1100	0.00144659767822664\\
1101	0.00196922841531893\\
1102	0.00239772830261408\\
1103	0.00211937177748391\\
1104	0.00115136071374889\\
1105	0.00064184816020663\\
1106	0.000450647936846819\\
1107	0.00033314903132925\\
1108	0.000261698918612371\\
1109	0.000238034464264517\\
1110	0.000261202179023024\\
1111	0.000322980825737343\\
1112	0.000408812511909322\\
1113	0.00050085829428816\\
1114	0.000580485298111244\\
1115	0.000632203429718067\\
1116	0.00064559828094937\\
1117	0.000616259904967721\\
1118	0.000550086722173332\\
1119	0.000458327868811968\\
1120	0.00035883354307004\\
1121	0.00027202418132117\\
1122	0.000212766683033807\\
1123	0.000201919642803672\\
1124	0.000247651627543501\\
1125	0.00034933757596473\\
1126	0.000497800193489855\\
1127	0.000676296678545576\\
1128	0.000862534687570811\\
1129	0.0010326553448476\\
1130	0.00120811286591102\\
1131	0.00160274515745296\\
1132	0.00231920582234812\\
1133	0.00298130815579366\\
1134	0.00326972278144266\\
1135	0.00207378990106364\\
1136	0.00093891037205993\\
1137	0.000586164056768482\\
1138	0.000400706919050341\\
1139	0.00029210145114355\\
1140	0.000245688771422786\\
1141	0.000254866157825894\\
1142	0.000309870679452983\\
1143	0.000396176902136133\\
1144	0.000495541927640001\\
1145	0.000588729919487379\\
1146	0.000658127315371707\\
1147	0.000690667176340745\\
1148	0.000679661685335293\\
1149	0.000625492332451226\\
1150	0.000538377103994103\\
1151	0.000434094532739183\\
1152	0.000332204225464779\\
1153	0.000249403453841172\\
1154	0.000207441965016773\\
1155	0.000218878973182628\\
1156	0.000287044309020251\\
1157	0.000407143230495552\\
1158	0.00056575602342353\\
1159	0.000742761071029741\\
1160	0.000914971907349174\\
1161	0.00105941709925039\\
1162	0.00115881083388789\\
1163	0.00131653411850664\\
1164	0.00167787613843955\\
1165	0.00191051769932166\\
1166	0.00153667925006383\\
1167	0.000881020372888924\\
1168	0.000617594078188324\\
1169	0.000446488385743397\\
1170	0.000326999142658343\\
1171	0.000257993777681332\\
1172	0.000240764083606168\\
1173	0.000271294762373231\\
1174	0.000339071096006719\\
1175	0.000428142719059965\\
1176	0.000520152849706351\\
1177	0.000596920245394186\\
1178	0.000643051644212187\\
1179	0.000649746135730377\\
1180	0.000613247334648985\\
1181	0.000540879509096978\\
1182	0.000445906467046014\\
1183	0.000346458340810165\\
1184	0.000262789211432898\\
1185	0.000209046377004174\\
1186	0.000206057432699259\\
1187	0.00025991417750769\\
1188	0.000368745343783613\\
1189	0.00052233367697949\\
1190	0.000702465228697605\\
1191	0.00088656281407194\\
1192	0.00105083628756252\\
1193	0.00116372525446402\\
1194	0.00132646527660327\\
1195	0.00176635921395948\\
1196	0.00223579677615272\\
1197	0.00218774081717147\\
1198	0.00150821510343054\\
1199	0.000763000238705878\\
1200	0.000509703730419161\\
1201	0.000366884607470055\\
1202	0.000278576367354436\\
1203	0.000240648333179145\\
1204	0.000251762300498068\\
1205	0.000304694656614525\\
1206	0.000386014047366843\\
1207	0.000478099593978935\\
1208	0.000562331022067993\\
1209	0.000622027837733649\\
1210	0.000646139100835095\\
1211	0.000626766437286134\\
1212	0.000568431322118495\\
1213	0.000481517550962793\\
1214	0.000382372670502463\\
1215	0.000290780182451597\\
1216	0.000222879869154764\\
1217	0.00019923647848529\\
1218	0.000231185929946067\\
1219	0.000319482901845473\\
1220	0.000457806371715299\\
1221	0.000630941288129968\\
1222	0.000817580596863607\\
1223	0.000993974132119781\\
1224	0.00113728536259776\\
1225	0.00122452167564756\\
1226	0.00156923012687673\\
1227	0.00212641024729858\\
1228	0.00240619456713176\\
1229	0.00175702112688183\\
1230	0.00090375342371807\\
1231	0.000608891866272723\\
1232	0.000436019767522431\\
1233	0.000319972432380436\\
1234	0.000255152232353985\\
1235	0.000241587296922095\\
1236	0.000274409525767496\\
1237	0.000342399931776047\\
1238	0.000429230833954174\\
1239	0.000516491729274047\\
1240	0.000586537158676846\\
1241	0.000624919839016724\\
1242	0.000623868641250393\\
1243	0.000581324215823075\\
1244	0.000505436726354483\\
1245	0.000411125344830394\\
1246	0.00031660536566948\\
1247	0.000238617984954126\\
1248	0.000198798500045914\\
1249	0.000210938103748164\\
1250	0.000280022499989488\\
1251	0.000402885576525845\\
1252	0.00056750487991008\\
1253	0.00075455410660896\\
1254	0.000940873675873854\\
1255	0.00110269238010206\\
1256	0.00120590724097997\\
1257	0.00147194041265659\\
1258	0.00203495170432127\\
1259	0.00252929912234633\\
1260	0.00232404762655933\\
1261	0.00125734060037996\\
1262	0.000665735721094814\\
1263	0.000462177998520248\\
1264	0.000341113924998634\\
1265	0.000265995705711857\\
1266	0.000238105703796858\\
1267	0.000257212545595569\\
1268	0.00031580127100581\\
1269	0.000399831916996915\\
1270	0.000491631599453821\\
1271	0.000572528606320538\\
1272	0.000626532540069244\\
1273	0.00064351948565548\\
1274	0.000617203170553089\\
1275	0.000553612889401351\\
1276	0.000463585995215927\\
1277	0.000364460549976688\\
1278	0.000276459971454446\\
1279	0.000214917469722108\\
1280	0.00020043246687241\\
1281	0.000242458966677368\\
1282	0.0003404758042402\\
1283	0.000486429846078211\\
1284	0.000663926386549082\\
1285	0.000850958963766669\\
1286	0.00102366069090377\\
1287	0.00116891732770049\\
1288	0.00157709958339186\\
1289	0.00231556309146411\\
1290	0.00300671820188674\\
1291	0.00332493128447743\\
1292	0.00217896740243481\\
1293	0.000965967305645999\\
1294	0.000599892459008634\\
1295	0.000409720187392751\\
1296	0.000297153400697304\\
1297	0.000246820757997703\\
1298	0.000252462239061135\\
1299	0.000304785781317644\\
1300	0.000389700113853881\\
1301	0.000488963219873187\\
1302	0.000583714120021204\\
1303	0.000655802073517983\\
1304	0.000691664195419569\\
1305	0.000684308660676279\\
1306	0.000633182078060559\\
1307	0.000547954343888503\\
1308	0.000444138413268933\\
1309	0.000340996407922252\\
1310	0.000255479903730317\\
1311	0.00020930976532408\\
1312	0.000215822919478647\\
1313	0.000279152031025637\\
1314	0.000394991928123923\\
1315	0.000550728791427258\\
1316	0.000726672765861436\\
1317	0.000899785177117896\\
1318	0.00104694842294101\\
1319	0.00116243433244013\\
1320	0.00129258962717377\\
1321	0.00163332064861308\\
1322	0.00188122394081613\\
1323	0.00154774893400445\\
1324	0.000894652580782318\\
1325	0.000630800211530088\\
1326	0.000457209417829692\\
1327	0.000334278617927587\\
1328	0.000261389332999631\\
1329	0.00024032787708869\\
1330	0.00026760828562284\\
1331	0.000333190799301169\\
1332	0.000421462948897433\\
1333	0.000514144970167151\\
1334	0.000592846360814333\\
1335	0.000641934256102955\\
1336	0.000652118029765364\\
1337	0.000618825846032092\\
1338	0.000549713322463042\\
1339	0.000455663294325345\\
1340	0.000355284714511634\\
1341	0.000269167928920681\\
1342	0.000211984900829448\\
1343	0.000204035079519469\\
1344	0.000253744600482355\\
1345	0.000357997908617297\\
1346	0.000508074986820436\\
1347	0.000686648226897655\\
1348	0.000871106433250004\\
1349	0.00103762774586251\\
1350	0.00123032464849896\\
1351	0.00165220917094559\\
1352	0.00234462767800509\\
1353	0.00293000131116416\\
1354	0.00314903532509289\\
1355	0.00182507417194476\\
1356	0.000881878422086491\\
1357	0.000562409181196982\\
1358	0.000387384887888694\\
1359	0.000285254311255141\\
1360	0.000244233415316124\\
1361	0.000257943674524141\\
1362	0.000316320507803041\\
1363	0.000404447108176159\\
1364	0.000504013620387388\\
1365	0.000595722570163262\\
1366	0.000662249623595168\\
1367	0.000691054145860641\\
1368	0.000676067325966349\\
1369	0.000618588339694812\\
1370	0.000529324980374227\\
1371	0.00042443046417897\\
1372	0.000323777849749805\\
1373	0.000243823466466141\\
1374	0.000206194276993425\\
1375	0.000222753994413849\\
1376	0.00029591855054623\\
1377	0.00042013232639718\\
1378	0.000581328288599061\\
1379	0.000758944484991721\\
1380	0.000929660963131117\\
1381	0.00107071161227774\\
1382	0.00115633849679364\\
1383	0.00134018074536076\\
1384	0.00171416775630386\\
1385	0.00191978903855421\\
1386	0.00151872347258135\\
1387	0.000863434483819687\\
1388	0.000601631994216249\\
1389	0.000433849655819172\\
1390	0.000318602302445791\\
1391	0.000254270653150129\\
1392	0.00024160282048596\\
1393	0.000275989714358439\\
1394	0.000346387404238841\\
1395	0.000436480298272879\\
1396	0.000527865172869894\\
1397	0.000602444654018022\\
1398	0.000645356770784447\\
1399	0.000648227554718371\\
1400	0.000608277209373081\\
1401	0.000533882675069059\\
1402	0.000437595802390009\\
1403	0.000338448994172081\\
1404	0.000256942823700889\\
1405	0.000206968143270638\\
1406	0.000208408826962981\\
1407	0.000267281920814124\\
1408	0.000380702941929572\\
1409	0.000537479130779269\\
1410	0.00071891099526232\\
1411	0.000902198194707932\\
1412	0.0010636186358982\\
1413	0.00116950500771925\\
1414	0.00135479429312346\\
1415	0.00181540049322024\\
1416	0.00226520780197931\\
1417	0.00214046880347406\\
1418	0.00150692662010299\\
1419	0.000737986667006765\\
1420	0.000488579630886566\\
1421	0.000352573084543865\\
1422	0.000270828394142067\\
1423	0.000238957857684306\\
1424	0.000255361009787355\\
1425	0.000312391314846199\\
1426	0.000396173565486883\\
1427	0.000489016436457843\\
1428	0.000572230544528736\\
1429	0.000629429580579883\\
1430	0.000650018768785964\\
1431	0.000626955937544441\\
1432	0.000565530137920268\\
1433	0.000476428534380482\\
1434	0.000376509411798393\\
1435	0.000285840927409501\\
1436	0.000220107552438686\\
1437	0.000200136098494181\\
1438	0.0002360385492654\\
1439	0.000328367476599671\\
1440	0.000469725624020717\\
1441	0.000644077955814783\\
1442	0.000830016270369366\\
1443	0.00100374328235271\\
1444	0.00114268359542704\\
1445	0.00133870782867809\\
1446	0.00161809873710859\\
1447	0.00210585704682005\\
1448	0.00225446106729804\\
1449	0.00166030725889904\\
1450	0.000865169613164581\\
1451	0.000582454714117512\\
1452	0.000416074410359454\\
1453	0.000306880379195422\\
1454	0.000249599505778402\\
1455	0.000243576817759148\\
1456	0.000283139002017215\\
1457	0.000356466418033243\\
1458	0.00044659991156671\\
1459	0.00053511167178948\\
1460	0.000604165265922387\\
1461	0.000641396032917777\\
1462	0.000635746138490981\\
1463	0.000589216010096351\\
1464	0.000510008198080438\\
1465	0.000413174229884544\\
1466	0.000317325670964971\\
1467	0.000239255003345891\\
1468	0.000200380108622163\\
1469	0.000214159058781161\\
1470	0.000284902024475812\\
1471	0.000408751252501343\\
1472	0.000573107136865291\\
1473	0.000758269579252544\\
1474	0.000940984579055403\\
1475	0.00109769322846906\\
1476	0.00119568629012188\\
1477	0.00144885230545853\\
1478	0.0019728219785047\\
1479	0.00239851357956882\\
1480	0.00211125652039504\\
1481	0.00114493892338796\\
1482	0.0006400670408076\\
1483	0.000449621316131049\\
1484	0.000332432604526391\\
1485	0.000261330148252159\\
1486	0.000238049804283003\\
1487	0.000261567500585378\\
1488	0.000323596007518768\\
1489	0.000409530109248437\\
1490	0.000501512067335787\\
1491	0.000580930787511714\\
1492	0.000632381788585074\\
1493	0.000645372238270841\\
1494	0.000615656573160337\\
1495	0.000549234299667102\\
1496	0.000457336312321686\\
1497	0.000357878390628954\\
1498	0.000271294500078663\\
1499	0.000212416810692602\\
1500	0.000202071874508172\\
1501	0.00024834696815704\\
1502	0.000350537805928818\\
1503	0.000499388228526619\\
1504	0.00067809058630281\\
1505	0.000864320500360445\\
1506	0.00103421823791998\\
1507	0.00121278311353397\\
1508	0.0016079348761953\\
1509	0.00232589007226247\\
1510	0.00298500029250163\\
1511	0.00327435435635126\\
1512	0.00205964914382649\\
1513	0.000935639745855794\\
1514	0.000584560345093781\\
1515	0.000399682508058517\\
1516	0.000291540206968094\\
1517	0.000245573678343862\\
1518	0.000255147805028244\\
1519	0.000310449724314796\\
1520	0.000396907436914007\\
1521	0.000496268610102024\\
1522	0.000589289290721141\\
1523	0.000658386060739637\\
1524	0.000690543708911468\\
1525	0.000679136413985614\\
1526	0.000624636262694265\\
1527	0.000537315992388716\\
1528	0.00043298807733272\\
1529	0.000331243793631132\\
1530	0.000248756324336158\\
1531	0.000207263492226037\\
1532	0.000219251877325123\\
1533	0.000287959802924042\\
1534	0.000408520157606543\\
1535	0.000567438611059163\\
1536	0.000744548420153208\\
1537	0.00091664658225868\\
1538	0.00106077848390783\\
1539	0.00117795041043064\\
1540	0.00132768128192977\\
1541	0.001685481924073\\
1542	0.00190466816022843\\
1543	0.00152997228557616\\
1544	0.000876810200967378\\
1545	0.000614292143361586\\
1546	0.000443944646861929\\
1547	0.000325290411718484\\
1548	0.000257185979982996\\
1549	0.000240858019377539\\
1550	0.000272212691485791\\
1551	0.000340662615919049\\
1552	0.000430200481440259\\
1553	0.000522451495735585\\
1554	0.000599230176319453\\
1555	0.000645188353292632\\
1556	0.000651554358634022\\
1557	0.000614854796456\\
1558	0.000542179609966822\\
1559	0.00044683915904381\\
1560	0.000347035684728552\\
1561	0.000263139007261366\\
1562	0.000209465965042142\\
1563	0.000206142308589939\\
1564	0.000260086850358103\\
1565	0.000369041022538995\\
1566	0.000522426410909383\\
1567	0.000702373504298815\\
1568	0.000886085643069442\\
1569	0.00104981317715751\\
1570	0.001161461822231\\
1571	0.00132243771947707\\
1572	0.00176183599998494\\
1573	0.00222235814395532\\
1574	0.00216346434246534\\
1575	0.00149789030822474\\
1576	0.000758859475121061\\
1577	0.000507255713442506\\
1578	0.000365243616398023\\
1579	0.000277652356745755\\
1580	0.000240430284535455\\
1581	0.000252199971340634\\
1582	0.000305676212943974\\
1583	0.000387368070559266\\
1584	0.000479642837548615\\
1585	0.000563861678625976\\
1586	0.000623370697716268\\
1587	0.000647161241887215\\
1588	0.000627418940554861\\
1589	0.000568760975483238\\
1590	0.000481570163240165\\
1591	0.000382246799736545\\
1592	0.000290625495264619\\
1593	0.000222819939306012\\
1594	0.000199395384637087\\
1595	0.000231622508104019\\
1596	0.000320154963596467\\
1597	0.000458626415058884\\
1598	0.000631761953253485\\
1599	0.000818223610398913\\
1600	0.000994258939665198\\
1601	0.00113706284766434\\
1602	0.00122389262551255\\
1603	0.00156781616299501\\
1604	0.00212155786043152\\
1605	0.00239452649369936\\
1606	0.00174841759896228\\
1607	0.000900237155653367\\
1608	0.00060660182027162\\
1609	0.000434351899165468\\
1610	0.000318888557001392\\
1611	0.000254676931900348\\
1612	0.000241704255887948\\
1613	0.000275039545961652\\
1614	0.0003434059769466\\
1615	0.000430434794727788\\
1616	0.000517713109067279\\
1617	0.000587597347074628\\
1618	0.00062569862330417\\
1619	0.000624285237046572\\
1620	0.000581379937996947\\
1621	0.000505221261642052\\
1622	0.000410745670604952\\
1623	0.000316211440340084\\
1624	0.000238351337026681\\
1625	0.000198798106776466\\
1626	0.000211284827700513\\
1627	0.000280718853350678\\
1628	0.000403868798904468\\
1629	0.000568640795375575\\
1630	0.000755663417551899\\
1631	0.000941760058455296\\
1632	0.00110317509654929\\
1633	0.00119831902691407\\
1634	0.00148820596514753\\
1635	0.00204754795194381\\
1636	0.00261519750498124\\
1637	0.00253798957470042\\
1638	0.00174367842917131\\
1639	0.000749098480785675\\
1640	0.000477168976118285\\
1641	0.000344734170019519\\
1642	0.00026728582091363\\
1643	0.00023807011799042\\
1644	0.000255711636293427\\
1645	0.000312684437888008\\
1646	0.000395081228351081\\
1647	0.00048533958971253\\
1648	0.000564879728435035\\
1649	0.000617740669775552\\
1650	0.000633930371900984\\
1651	0.000607333526845413\\
1652	0.000544005862455656\\
1653	0.000454782287169064\\
1654	0.000357022036342292\\
1655	0.00027091866521992\\
1656	0.000211641437769054\\
1657	0.000199837399887347\\
1658	0.000244903660203507\\
1659	0.000345540037610053\\
1660	0.000494112923462056\\
1661	0.000673738018775336\\
1662	0.000862490739513376\\
1663	0.00103641074611601\\
1664	0.00121810790518802\\
1665	0.00163975892667063\\
1666	0.00241988935034796\\
1667	0.00317323921723361\\
1668	0.00346506311455382\\
1669	0.00221303417529993\\
1670	0.000975113389715869\\
1671	0.000603043273180597\\
1672	0.000410633451498377\\
1673	0.000297350398938048\\
1674	0.000246945513788565\\
1675	0.0002527224538147\\
1676	0.000305180986044762\\
1677	0.000390128374035183\\
1678	0.000489294788621371\\
1679	0.000583805591226704\\
1680	0.000655542499409462\\
1681	0.000690992871509783\\
1682	0.000683225707350689\\
1683	0.000631776538524011\\
1684	0.00054637978494162\\
1685	0.000442575301300023\\
1686	0.000339656719839442\\
1687	0.000254549963525655\\
1688	0.000208963691618569\\
1689	0.000216150001957083\\
1690	0.000280151826457398\\
1691	0.000396603056570276\\
1692	0.000552790872140649\\
1693	0.000728970973228161\\
1694	0.000902083266294471\\
1695	0.00104902232677758\\
1696	0.00113642022476258\\
1697	0.00128692446302881\\
1698	0.00163915685826165\\
1699	0.00190156594033229\\
1700	0.0015657391098849\\
1701	0.000896727049971044\\
1702	0.000630452512013353\\
1703	0.000456915907251013\\
1704	0.000334235219373367\\
1705	0.000261431289633181\\
1706	0.000240308627371549\\
1707	0.000267412260758301\\
1708	0.000332710827458155\\
1709	0.000420596220459821\\
1710	0.000512803368073086\\
1711	0.000590970057264392\\
1712	0.000639531317984398\\
1713	0.000649245477802815\\
1714	0.000615635998943987\\
1715	0.000546447806161595\\
1716	0.000452547158830751\\
1717	0.000352601983243336\\
1718	0.000267199728071143\\
1719	0.000210943662550887\\
1720	0.000204104199501417\\
1721	0.000255060319533223\\
1722	0.00036036728909719\\
1723	0.00051138623374883\\
1724	0.000690678514292109\\
1725	0.000875574901943247\\
1726	0.00104223436804835\\
1727	0.00118485438304055\\
1728	0.00131907710633353\\
1729	0.00174525083321915\\
1730	0.00222020382563488\\
1731	0.00221367101358931\\
1732	0.00153660142748422\\
1733	0.000782956882039042\\
1734	0.000524027465263752\\
1735	0.000376561792997263\\
1736	0.000283972849189961\\
1737	0.000241998266087261\\
1738	0.000249532758078396\\
1739	0.000299724103610831\\
1740	0.000379510584525859\\
1741	0.000471246413343977\\
1742	0.000556663169281944\\
1743	0.000618539952918034\\
1744	0.000645521904639168\\
1745	0.00062909562244235\\
1746	0.000573189783736153\\
1747	0.000487832333846903\\
1748	0.000389013154488657\\
1749	0.000296322016632493\\
1750	0.000225958504963511\\
1751	0.000198991674109629\\
1752	0.000227041396419122\\
1753	0.000311566831013961\\
1754	0.000446880442572487\\
1755	0.000618298822394499\\
1756	0.000804829255230006\\
1757	0.000982794937420374\\
1758	0.00112917328532566\\
1759	0.00124436325457425\\
1760	0.00155014347410478\\
1761	0.00210069301107627\\
1762	0.00238862658599202\\
1763	0.00178972988386556\\
1764	0.000920968051597334\\
1765	0.000621377055018901\\
1766	0.000445071843140848\\
1767	0.000325740475164907\\
1768	0.000257651953458202\\
1769	0.000241059332656747\\
1770	0.000271467275438554\\
1771	0.000338016913354892\\
1772	0.000424579680559568\\
1773	0.000512864022478427\\
1774	0.00058499742407488\\
1775	0.000626156323378906\\
1776	0.000628161945844365\\
1777	0.000588323132467144\\
1778	0.000515142363426473\\
1779	0.000421030036612545\\
1780	0.000325024932962318\\
1781	0.000247517525232595\\
1782	0.000202165846908283\\
1783	0.000209468364863767\\
1784	0.000273148492121818\\
1785	0.000391556597837501\\
1786	0.000553030340543143\\
1787	0.00073860908031448\\
1788	0.000925280743086977\\
1789	0.0010891859931918\\
1790	0.00125696420183063\\
1791	0.00146319456282159\\
1792	0.00197712549070297\\
1793	0.00244542921926542\\
1794	0.00227772782683283\\
1795	0.00166368545276831\\
1796	0.000747324731040125\\
1797	0.000480559464296446\\
1798	0.000346299524565622\\
1799	0.000267558062340002\\
1800	0.000237987308457936\\
1801	0.000256109467188998\\
1802	0.000314459765772558\\
1803	0.000399219846343206\\
1804	0.000492789590136131\\
1805	0.000576360865001991\\
1806	0.00063355651284596\\
1807	0.000653912920147124\\
1808	0.00063047455472184\\
1809	0.000568565115349894\\
1810	0.000478905073992445\\
1811	0.000378422069860903\\
1812	0.000287233148258407\\
1813	0.000221001735220392\\
1814	0.000200604101051854\\
1815	0.000236091200976568\\
1816	0.00032803036557452\\
1817	0.000468694322025069\\
1818	0.000642473392243827\\
1819	0.000827703094297653\\
1820	0.00100061769138316\\
1821	0.0011387052557722\\
1822	0.00134258504735293\\
1823	0.0016033402047451\\
1824	0.00207299444792193\\
1825	0.00220370685966327\\
1826	0.00163019624973782\\
1827	0.0008568822733272\\
1828	0.000578340785530724\\
1829	0.000413339914860118\\
1830	0.000305171612521643\\
1831	0.000248921433897469\\
1832	0.000243918004472812\\
1833	0.000284406558819683\\
1834	0.000358486867970359\\
1835	0.000449141513631604\\
1836	0.000537915705497941\\
1837	0.000607004722879463\\
1838	0.000642994470760408\\
1839	0.000638007629783861\\
1840	0.00059113921716208\\
1841	0.000511516884478223\\
1842	0.000414272591603372\\
1843	0.00031811210664775\\
1844	0.000239598695231651\\
1845	0.000200688889752325\\
1846	0.000214632346022001\\
1847	0.00028530181268507\\
1848	0.000408831663766892\\
1849	0.000572822607486168\\
1850	0.000757524831064161\\
1851	0.000939662757564501\\
1852	0.0010956947718565\\
1853	0.00119350931469897\\
1854	0.0014419668145046\\
1855	0.00195710391860111\\
1856	0.00237220688329198\\
1857	0.00207611671876465\\
1858	0.00112801127886555\\
1859	0.000637175351707525\\
1860	0.000448536257753875\\
1861	0.000331667774413397\\
1862	0.000260917143607555\\
1863	0.00023809867180649\\
1864	0.000262083647406438\\
1865	0.000324489337195748\\
1866	0.000410645065574763\\
1867	0.000502664265145433\\
1868	0.000581956013814059\\
1869	0.0006332131799347\\
1870	0.000645864285364433\\
1871	0.000615700289603468\\
1872	0.000549002970511754\\
1873	0.000456895282126347\\
1874	0.000357375880904208\\
1875	0.000270896156178672\\
1876	0.000212279103514246\\
1877	0.000202260282727003\\
1878	0.000248934485499637\\
1879	0.000351470980145011\\
1880	0.000500554445428351\\
1881	0.000679321570374192\\
1882	0.00086542048073454\\
1883	0.00103499504492401\\
1884	0.0012149600774683\\
1885	0.00161013983027456\\
1886	0.00232697094240337\\
1887	0.00298081753713909\\
1888	0.00325892922218447\\
1889	0.00203989160776449\\
1890	0.000935004918328918\\
1891	0.000583628949248329\\
1892	0.000398903752165658\\
1893	0.000291083219840042\\
1894	0.000245498011431211\\
1895	0.000255432020539591\\
1896	0.000311006231045113\\
1897	0.000397596236587383\\
1898	0.000496937629055232\\
1899	0.000589778716707301\\
1900	0.00065856617750931\\
1901	0.000690337133804823\\
1902	0.000678527613941703\\
1903	0.000623701161184623\\
1904	0.000536183485431059\\
1905	0.000431821145292674\\
1906	0.000330237478911091\\
1907	0.000248079631039478\\
1908	0.000207074818065803\\
1909	0.000219634097259504\\
1910	0.000288904504770613\\
1911	0.000409942327913666\\
1912	0.000569181051305964\\
1913	0.000746408573668738\\
1914	0.000918404450085253\\
1915	0.0010622312558754\\
1916	0.00114586478891152\\
1917	0.00131900395085841\\
1918	0.00168784583909423\\
1919	0.00192389638404012\\
1920	0.00153992699640794\\
1921	0.000879810530931257\\
1922	0.000615802778119016\\
1923	0.000444979140846057\\
1924	0.000325996472960758\\
1925	0.00025756336612005\\
1926	0.000240873031637855\\
1927	0.000271810318439906\\
1928	0.000339771970246424\\
1929	0.000428749851322479\\
1930	0.000520383748194989\\
1931	0.000596530596592237\\
1932	0.000641895759296684\\
1933	0.000647776755987625\\
1934	0.000610778093531883\\
1935	0.000538133199812542\\
1936	0.000443063386823292\\
1937	0.000343845202229903\\
1938	0.000260842775902354\\
1939	0.000208295969707479\\
1940	0.000206293760879752\\
1941	0.000261591924250396\\
1942	0.000371854086287969\\
1943	0.000526382544132317\\
1944	0.000707210313814982\\
1945	0.000891490086039028\\
1946	0.0010554537225304\\
1947	0.00117615656611307\\
1948	0.00134324503139995\\
1949	0.00179399361135244\\
1950	0.00225801915775879\\
1951	0.00218662779952262\\
1952	0.00150481363538512\\
1953	0.000756634667227184\\
1954	0.000504773787419458\\
1955	0.000363601869701048\\
1956	0.000276787591870847\\
1957	0.00024020595166946\\
1958	0.000252493228122271\\
1959	0.000306350410402169\\
1960	0.000388256448847627\\
1961	0.000480577776894427\\
1962	0.000564670561062696\\
1963	0.000623913171968503\\
1964	0.000647342149277096\\
1965	0.000627229711094613\\
1966	0.000568290477297111\\
1967	0.00048090087375559\\
1968	0.000381514678837942\\
1969	0.000290011479016927\\
1970	0.000222470780183647\\
1971	0.000199456137365283\\
1972	0.000232166501883525\\
1973	0.000321131282255915\\
1974	0.000459943858876152\\
1975	0.000633258876190659\\
1976	0.00081970078113341\\
1977	0.000995510844793587\\
1978	0.00113790629521085\\
1979	0.00122470872970539\\
1980	0.00157107265512439\\
1981	0.00212461939237275\\
1982	0.00239236953241035\\
1983	0.00174441866401543\\
1984	0.000897680623322766\\
1985	0.00060463592661519\\
1986	0.000432884173757931\\
1987	0.000317940962157276\\
1988	0.000254268235256814\\
1989	0.000241810178221476\\
1990	0.000275578995671109\\
1991	0.000344245703720024\\
1992	0.000431405357169161\\
1993	0.000518645482635556\\
1994	0.00058832706711879\\
1995	0.000626120912428895\\
1996	0.000624337345880376\\
1997	0.000581081987996604\\
1998	0.000504680087902305\\
1999	0.000410084043651011\\
2000	0.0003155943535739\\
};
\addlegendentry{$\text{V}_\text{2}$};

\addplot [color=mycolor3,solid]
  table[row sep=crcr]{%
1	52.9128\\
2	44.1607664039228\\
3	36.5930180270703\\
4	29.5555335247108\\
5	23.3015994817039\\
6	17.7896860351103\\
7	13.0289505311703\\
8	9.01784370605211\\
9	5.75440505748554\\
10	3.23406137257205\\
11	1.45049462626911\\
12	0.384422848602529\\
13	0.0634053191909889\\
14	0.014058224115675\\
15	0.00573957458572181\\
16	0.00481013586751911\\
17	0.00514912822416434\\
18	0.00240314540366841\\
19	0.00130396425490313\\
20	0.000786967048213759\\
21	0.000580559191342764\\
22	0.00042752943612696\\
23	0.000312981291117092\\
24	0.000247457064214414\\
25	0.000237980435473734\\
26	0.000282335445046113\\
27	0.000370175529386868\\
28	0.000484610634491262\\
29	0.000604795454378568\\
30	0.000709488598253572\\
31	0.000782238277063973\\
32	0.000807704600559833\\
33	0.000783641584601306\\
34	0.0007147827215407\\
35	0.000612877530465403\\
36	0.000493552291561235\\
37	0.000377647943357334\\
38	0.000284677836796228\\
39	0.000230620338602581\\
40	0.00022550880748739\\
41	0.000271857996770233\\
42	0.000364221869644761\\
43	0.000489736796318847\\
44	0.000630579187744826\\
45	0.000766359299770871\\
46	0.000876621985053003\\
47	0.000944873066664627\\
48	0.000964506958575946\\
49	0.000937267157732185\\
50	0.000859806101635437\\
51	0.000717790950792084\\
52	0.000569257561794577\\
53	0.000429711751050399\\
54	0.00031420704080858\\
55	0.000245160646587338\\
56	0.000231407992058881\\
57	0.000273395353370343\\
58	0.000363290490002165\\
59	0.000485858497609441\\
60	0.000620831396973025\\
61	0.000746080970821211\\
62	0.000843381736980669\\
63	0.000893680856124653\\
64	0.000891345609737232\\
65	0.000837220399343934\\
66	0.000740783995384111\\
67	0.000617495543133778\\
68	0.000485275926757179\\
69	0.000365605062348058\\
70	0.000276870502887169\\
71	0.000232500533130686\\
72	0.000238969084821138\\
73	0.000294853272601157\\
74	0.000391056469870383\\
75	0.000511960228703833\\
76	0.000638818472902849\\
77	0.000751013602080758\\
78	0.000830899506831923\\
79	0.000865240004170424\\
80	0.000848241337927708\\
81	0.000781581578410606\\
82	0.000676440965726668\\
83	0.000549806282013807\\
84	0.000422586326729922\\
85	0.000312565209158943\\
86	0.0002416540419764\\
87	0.000224079265536892\\
88	0.000261723822325918\\
89	0.00035118681254217\\
90	0.000479353001701503\\
91	0.000626705648549015\\
92	0.000770989183771285\\
93	0.000890506345805623\\
94	0.000968707880164184\\
95	0.00103248445567061\\
96	0.00104558271325975\\
97	0.00104430137318507\\
98	0.000802252075079691\\
99	0.000627655585456312\\
100	0.000474018523972047\\
101	0.000349932469816419\\
102	0.000267575299814781\\
103	0.00023509940852357\\
104	0.000254125917601527\\
105	0.00031891848489569\\
106	0.00041694890744316\\
107	0.000530715189738262\\
108	0.000640612434874983\\
109	0.000727779386644537\\
110	0.000777337155474573\\
111	0.000780675888623604\\
112	0.000735304433922663\\
113	0.000649216457554384\\
114	0.000537689359558614\\
115	0.000419504900250631\\
116	0.000312659188932839\\
117	0.000239295886802784\\
118	0.000215203396229309\\
119	0.000247983707836626\\
120	0.000333870746030744\\
121	0.000463852198022913\\
122	0.000620359194901715\\
123	0.000781116638138585\\
124	0.000923283097531036\\
125	0.00102832568185854\\
126	0.00110314761912841\\
127	0.00117683501436347\\
128	0.00126562916458787\\
129	0.00118651275727675\\
130	0.000812340733190686\\
131	0.000602887926209589\\
132	0.00044305799000915\\
133	0.000325542236467016\\
134	0.000256275001920998\\
135	0.00023892575285044\\
136	0.000270994977831442\\
137	0.000343119046412549\\
138	0.000439993362183053\\
139	0.000543208592665331\\
140	0.00063387538628392\\
141	0.000694849091498069\\
142	0.000715851248460455\\
143	0.00069019950284163\\
144	0.000623649972343018\\
145	0.000527120998377653\\
146	0.000418052389387365\\
147	0.000317005405441821\\
148	0.000240066729290098\\
149	0.00020816955547528\\
150	0.000231924332177446\\
151	0.000311663653526156\\
152	0.000440284458764964\\
153	0.000602233506908043\\
154	0.000776166436330652\\
155	0.000938726580942165\\
156	0.0010678831193798\\
157	0.00118279246101079\\
158	0.00132053000715961\\
159	0.00161243180846751\\
160	0.0016644751162124\\
161	0.00131698076414281\\
162	0.000778024036897002\\
163	0.000543896646022139\\
164	0.000391991391180068\\
165	0.00029227311725225\\
166	0.000244141460092658\\
167	0.000247181820653063\\
168	0.000295161670047914\\
169	0.000375639759809512\\
170	0.000471187374472303\\
171	0.000563141212156112\\
172	0.000634053310192321\\
173	0.000669602282376267\\
174	0.000663149361337274\\
175	0.000613800036524294\\
176	0.000530997444324074\\
177	0.000430103275315714\\
178	0.000330163941282297\\
179	0.000248007893859075\\
180	0.000205313035170332\\
181	0.000215330130851489\\
182	0.000282205279497243\\
183	0.000401822135699886\\
184	0.000561466282123853\\
185	0.000741401468463969\\
186	0.000918485029860111\\
187	0.00106942430847594\\
188	0.00120118405768674\\
189	0.001367819332368\\
190	0.00177830175118884\\
191	0.00208466396243275\\
192	0.00172917255468481\\
193	0.000962324843240383\\
194	0.000618519962733734\\
195	0.000445463673906286\\
196	0.000328964007347721\\
197	0.000259248641113612\\
198	0.000238923141578247\\
199	0.000265897757398817\\
200	0.000330995441072191\\
201	0.000419044581599715\\
202	0.000511933683035682\\
203	0.000591203677223711\\
204	0.000641288920368202\\
205	0.000652630221033166\\
206	0.000620422105121955\\
207	0.000552052779710443\\
208	0.00045843134272008\\
209	0.000357958537364424\\
210	0.000271186583183978\\
211	0.000212912667463664\\
212	0.000203498560432883\\
213	0.000251525793303748\\
214	0.000354428391997146\\
215	0.000503466827116299\\
216	0.000681530718308831\\
217	0.000866109368266336\\
218	0.00103337880594522\\
219	0.00121078070208457\\
220	0.00159299610817332\\
221	0.00227372212132198\\
222	0.00288045608959358\\
223	0.00310269106039835\\
224	0.001962604920975\\
225	0.000912480596576031\\
226	0.000573632754529724\\
227	0.00039354137288347\\
228	0.000288483617540645\\
229	0.000244983741270129\\
230	0.000256531476627303\\
231	0.000313162425502765\\
232	0.000400097618404821\\
233	0.000499048423485065\\
234	0.000590761013585267\\
235	0.000657845470153938\\
236	0.000687606281232197\\
237	0.000673796928085306\\
238	0.000617438696079594\\
239	0.00052910033894673\\
240	0.000424805379418557\\
241	0.000324318103305102\\
242	0.000244141282876022\\
243	0.000205932525365863\\
244	0.00022172583288634\\
245	0.000294188357849636\\
246	0.00041792978511861\\
247	0.000579072746240848\\
248	0.000757157267145926\\
249	0.000928874346880393\\
250	0.00107135978932036\\
251	0.00115880916720403\\
252	0.00134494850253572\\
253	0.00172793807935463\\
254	0.00195177036839451\\
255	0.00154329672179671\\
256	0.000873328015344926\\
257	0.000608452000339722\\
258	0.000438937732223761\\
259	0.000321947623518993\\
260	0.000255754335993107\\
261	0.000241249778938751\\
262	0.000274022868556629\\
263	0.000343211148468656\\
264	0.000432628555440371\\
265	0.000523891933039822\\
266	0.000598909614837646\\
267	0.000642646544770444\\
268	0.000646615371714146\\
269	0.000607803233302697\\
270	0.000534159179997799\\
271	0.000438491149775734\\
272	0.000339531938273659\\
273	0.000257726893703093\\
274	0.000207086281032239\\
275	0.00020756767579084\\
276	0.000265440740707545\\
277	0.000378024082847075\\
278	0.00053433622485469\\
279	0.000715823445724849\\
280	0.000899742870459175\\
281	0.00106233501139347\\
282	0.00116370089244823\\
283	0.00135206158816512\\
284	0.00181832681770674\\
285	0.00228388639651152\\
286	0.00218612919320307\\
287	0.00150449241984896\\
288	0.000746250895704491\\
289	0.000495964069780037\\
290	0.000357642658156555\\
291	0.000273578163027322\\
292	0.000239523643795506\\
293	0.00025398156504908\\
294	0.000309448769635876\\
295	0.000392198564330106\\
296	0.000484566835543842\\
297	0.000567892996234773\\
298	0.000625715353480468\\
299	0.000647300700878353\\
300	0.000625353136823675\\
301	0.000564922625297198\\
302	0.000476622764227704\\
303	0.000377142890359701\\
304	0.000286463480608405\\
305	0.000220333842357715\\
306	0.000199639639549729\\
307	0.000234679117814883\\
308	0.000326225350379839\\
309	0.000466982024465543\\
310	0.000641376311168433\\
311	0.000827867774876139\\
312	0.00100266892532685\\
313	0.00114312592348485\\
314	0.00134650760147663\\
315	0.00162379762600777\\
316	0.00211837078943684\\
317	0.00228123578293627\\
318	0.00167949038375315\\
319	0.000873344373725897\\
320	0.000588151112374942\\
321	0.000420300572843731\\
322	0.000309576313920973\\
323	0.000250657397474162\\
324	0.000243046281305725\\
325	0.000281262966420921\\
326	0.000353565724165221\\
327	0.000443252204194679\\
328	0.00053187921224487\\
329	0.000601649451630748\\
330	0.000638591050484165\\
331	0.000635154678027786\\
332	0.000589851041384119\\
333	0.000511428353322233\\
334	0.00041496184038181\\
335	0.000319008318001795\\
336	0.000240308390610289\\
337	0.000200503033581269\\
338	0.000213079008910617\\
339	0.000282582861862103\\
340	0.000405421178570196\\
341	0.000569184944558313\\
342	0.000754311395051985\\
343	0.000937596530496335\\
344	0.00109543511571065\\
345	0.00119574238817177\\
346	0.00144455596021079\\
347	0.00196784085398223\\
348	0.00240469072572328\\
349	0.00214338835825326\\
350	0.00116774480448325\\
351	0.000646128223334329\\
352	0.000452941681168023\\
353	0.00033471708202343\\
354	0.000262512205601475\\
355	0.000238018113218141\\
356	0.000260420327091635\\
357	0.000321634070205686\\
358	0.000407201122583091\\
359	0.00049932471754277\\
360	0.000579333328983462\\
361	0.000631566555966099\\
362	0.000645733575427479\\
363	0.000617163515761687\\
364	0.000551509501757945\\
365	0.000460063955544319\\
366	0.000360546172823626\\
367	0.000273343826992723\\
368	0.000213390719645485\\
369	0.000201612149062155\\
370	0.000246326558501727\\
371	0.000347071582640342\\
372	0.000494823538116453\\
373	0.000672965025404136\\
374	0.000859267525055468\\
375	0.00102987278451284\\
376	0.00119886581287055\\
377	0.00158549433348672\\
378	0.00229897929016901\\
379	0.00296792496204297\\
380	0.00326559032760057\\
381	0.00210873161534116\\
382	0.000946981443330385\\
383	0.000589803058809983\\
384	0.00040303959603234\\
385	0.000293412092342872\\
386	0.000245985119931084\\
387	0.000254236507809387\\
388	0.000308516298552952\\
389	0.000394401045442011\\
390	0.000493581234189613\\
391	0.000587113669104653\\
392	0.000657080717179995\\
393	0.000690263077308699\\
394	0.00068008850573508\\
395	0.000626626885332904\\
396	0.000539978198385315\\
397	0.000435863730557074\\
398	0.000333778310116892\\
399	0.000250511700726108\\
400	0.000207711817472514\\
401	0.000218173912284496\\
402	0.000285408060429509\\
403	0.000404704693050121\\
404	0.000562819548008379\\
405	0.000739718311943159\\
406	0.000912242554722574\\
407	0.00105738228986009\\
408	0.00116668366084517\\
409	0.00131640573271826\\
410	0.00167384617963331\\
411	0.001907473237848\\
412	0.00153903492947275\\
413	0.000883844299541651\\
414	0.000620228310891898\\
415	0.000448599909553453\\
416	0.000328415531527355\\
417	0.000258633482254808\\
418	0.000240646695707712\\
419	0.00027055200603319\\
420	0.000337935398589261\\
421	0.000426915948455888\\
422	0.000519148022062977\\
423	0.000596550393674431\\
424	0.000643260687350738\\
425	0.000650737003979378\\
426	0.00061493355239428\\
427	0.000543056891771644\\
428	0.000448260815778739\\
429	0.000348610970071809\\
430	0.00026435787003906\\
431	0.000209733789245106\\
432	0.000205666145479211\\
433	0.000258393847688275\\
434	0.000366142565145364\\
435	0.000518755586511684\\
436	0.000698508347176497\\
437	0.000882642161806123\\
438	0.00104737211150106\\
439	0.00116069171909676\\
440	0.0013182504897467\\
441	0.00174954123968627\\
442	0.00222074042758807\\
443	0.00218888455074855\\
444	0.00151179045692388\\
445	0.000768200691754465\\
446	0.000513776571813999\\
447	0.000369639097361272\\
448	0.00028009655850404\\
449	0.000241023556241002\\
450	0.000251132281581247\\
451	0.00030328585489083\\
452	0.000384144033038565\\
453	0.000476099709212965\\
454	0.000560552497576964\\
455	0.000620759413356564\\
456	0.000645581577947137\\
457	0.000626948638119118\\
458	0.000569194079760325\\
459	0.000482694596410876\\
460	0.000383705663807853\\
461	0.00029191476700552\\
462	0.000223527294516064\\
463	0.000199108690677132\\
464	0.000230138762448122\\
465	0.000317610345243116\\
466	0.000455286523117942\\
467	0.000628093026193466\\
468	0.000814796895981536\\
469	0.000991662308712583\\
470	0.00113580390871594\\
471	0.00122317779875906\\
472	0.00156139473695148\\
473	0.00212000311917736\\
474	0.00241252227887605\\
475	0.00178040481829114\\
476	0.000911682784887758\\
477	0.000613449620901201\\
478	0.000439132380887575\\
479	0.000321940037942945\\
480	0.000256011357271299\\
481	0.000241400871084849\\
482	0.000273340723823312\\
483	0.000340714226561825\\
484	0.000427224685286564\\
485	0.00051458253965425\\
486	0.000585014573909227\\
487	0.000624019585083559\\
488	0.000623708386586469\\
489	0.000581869219481744\\
490	0.000506474305138732\\
491	0.00041241416234266\\
492	0.000317816919892038\\
493	0.000242710695636084\\
494	0.000200429903127754\\
495	0.000210950784363434\\
496	0.000278984622214445\\
497	0.000400789843719468\\
498	0.00056483895519674\\
499	0.000751770661170822\\
500	0.000938409114042153\\
501	0.00110094020175545\\
502	0.00120578907932825\\
503	0.00146830458058388\\
504	0.00202992047030356\\
505	0.00253005267435259\\
506	0.00234237099705994\\
507	0.00126881178650613\\
508	0.00066991775301126\\
509	0.000464766026512474\\
510	0.000342916387105524\\
511	0.000266969893844778\\
512	0.000238196634393111\\
513	0.000256492640452867\\
514	0.000314471768846417\\
515	0.000398194101879167\\
516	0.000490026244034561\\
517	0.000571276075203792\\
518	0.000625840605247721\\
519	0.000643551441327817\\
520	0.000617971060177609\\
521	0.000554931798747098\\
522	0.000465243113370262\\
523	0.000366128081105002\\
524	0.000277774033846801\\
525	0.0002155587533501\\
526	0.000200181492791299\\
527	0.000241219419380348\\
528	0.000338295892035095\\
529	0.000483535726820386\\
530	0.000660664964649258\\
531	0.000847744664963166\\
532	0.00102091364512068\\
533	0.0011539154775845\\
534	0.00124977667336788\\
535	0.00164900198613535\\
536	0.00221863014437538\\
537	0.00241538295789113\\
538	0.00171573810711353\\
539	0.000866850378172101\\
540	0.000578114665970806\\
541	0.00041296127233369\\
542	0.000305336912890646\\
543	0.000249174301212908\\
544	0.000243751185381551\\
545	0.000283277365503747\\
546	0.0003557870195219\\
547	0.000444257280387876\\
548	0.000530335695213633\\
549	0.000596445465296091\\
550	0.000630520912266645\\
551	0.000622107921944445\\
552	0.000573661035513845\\
553	0.000493903589402408\\
554	0.000398025731629035\\
555	0.000304839111858251\\
556	0.000230914104739509\\
557	0.000197562620232357\\
558	0.000217652917293187\\
559	0.000294732705396284\\
560	0.000424310099501773\\
561	0.000593205595300102\\
562	0.000781280724519935\\
563	0.000965062240323466\\
564	0.0011210147160743\\
565	0.00121384045214298\\
566	0.00151866830681107\\
567	0.00210314678355973\\
568	0.00254518551148786\\
569	0.00219431155258357\\
570	0.00114453194156474\\
571	0.00063211117194634\\
572	0.000442785575650222\\
573	0.00032783442368557\\
574	0.000259113149027995\\
575	0.000238082217337045\\
576	0.000263409533094517\\
577	0.000326495194276866\\
578	0.000412499024150747\\
579	0.000503493620683057\\
580	0.000581060754468991\\
581	0.000629917536799552\\
582	0.000640466873660932\\
583	0.00060806978956652\\
584	0.000540191308903356\\
585	0.000447708246631101\\
586	0.00034903776574709\\
587	0.000264674101917075\\
588	0.000209263038138182\\
589	0.000203105333751641\\
590	0.000254744059024338\\
591	0.000360810626816089\\
592	0.000512881144475025\\
593	0.000693597272267387\\
594	0.00088030245241508\\
595	0.00104908452094914\\
596	0.00120687053435175\\
597	0.00135611011022098\\
598	0.0018035632763268\\
599	0.00229824649427407\\
600	0.0023030672255469\\
601	0.00158285886478265\\
602	0.000794615961048704\\
603	0.000529543216849288\\
604	0.000380245802918487\\
605	0.000286157177102195\\
606	0.000242572862453393\\
607	0.000248540354810511\\
608	0.000297426355453144\\
609	0.000376361943539425\\
610	0.00046774296521302\\
611	0.000553363861535316\\
612	0.000615907150735547\\
613	0.000642938438140942\\
614	0.000628485789339171\\
615	0.000573676162459406\\
616	0.00048905738648157\\
617	0.000390541568703182\\
618	0.000297688822386171\\
619	0.000226602088868525\\
620	0.00019871619382295\\
621	0.000225458588531689\\
622	0.000309066528952262\\
623	0.000443548017813109\\
624	0.000614586404949353\\
625	0.000801303176968613\\
626	0.000980047560480723\\
627	0.00112770719456326\\
628	0.00121546137132579\\
629	0.00153444763696512\\
630	0.00209190491635268\\
631	0.00242771819812741\\
632	0.00188756843744698\\
633	0.000939627699694197\\
634	0.000626882998120456\\
635	0.000448948889263416\\
636	0.000328684974181902\\
637	0.000259082490793646\\
638	0.0002406539190489\\
639	0.000269283791069882\\
640	0.000334402914236639\\
641	0.000420069604179573\\
642	0.000508047027941534\\
643	0.00058055066083778\\
644	0.000622332963375919\\
645	0.000625383021712304\\
646	0.000586679452791862\\
647	0.000514307782904349\\
648	0.000420929138007148\\
649	0.000325278104846067\\
650	0.000247682654821096\\
651	0.000201910571517692\\
652	0.000208562898629703\\
653	0.000271498779977168\\
654	0.000389371625523797\\
655	0.000550670384871701\\
656	0.000736573398334169\\
657	0.000924115223023747\\
658	0.00108938915730641\\
659	0.00125641739403029\\
660	0.00146368340939386\\
661	0.00198607775338513\\
662	0.00247326410148685\\
663	0.00233431021690895\\
664	0.00161541457449227\\
665	0.000751999491126817\\
666	0.000490192426113692\\
667	0.000353286165357183\\
668	0.000271333835955881\\
669	0.000238827213472185\\
670	0.000254389700326123\\
671	0.000310654838693254\\
672	0.000393938758072792\\
673	0.000486610015593298\\
674	0.000570006222555759\\
675	0.00062769251107069\\
676	0.000648981625716926\\
677	0.000626660836253026\\
678	0.000565859687935566\\
679	0.00047721771825824\\
680	0.000377483593104243\\
681	0.000286671198069112\\
682	0.000220517278962681\\
683	0.000199887874431763\\
684	0.000235031322306594\\
685	0.000326652914930285\\
686	0.000467393656082073\\
687	0.000641633291998924\\
688	0.000827807330239114\\
689	0.00100213462780489\\
690	0.00114199533802433\\
691	0.00134581245006894\\
692	0.00161883701585679\\
693	0.00210695354897898\\
694	0.00226135962790531\\
695	0.00166667247634635\\
696	0.000869093457332014\\
697	0.000585634302484348\\
698	0.000418518132210904\\
699	0.000308444751684644\\
700	0.000250203056545869\\
701	0.000243254022296311\\
702	0.000282049210522587\\
703	0.000354784898883721\\
704	0.000444716312024202\\
705	0.000533403490703179\\
706	0.000603031019747294\\
707	0.000639697955795105\\
708	0.000635897369563023\\
709	0.000590222206248494\\
710	0.000511500985525552\\
711	0.00041483118922751\\
712	0.000318813597478755\\
713	0.000240181039786954\\
714	0.000200570701055724\\
715	0.000213415248947593\\
716	0.000283185651573098\\
717	0.000406224011034373\\
718	0.000570055795023093\\
719	0.000755077737104697\\
720	0.0009380732592033\\
721	0.00109545662661831\\
722	0.00119516512146338\\
723	0.00144369689698293\\
724	0.00196480494605105\\
725	0.00239617624706424\\
726	0.00212596748536845\\
727	0.00115793659807432\\
728	0.000643828241576306\\
729	0.000451781891265207\\
730	0.0003339158394984\\
731	0.000262090519988491\\
732	0.0002380284706679\\
733	0.000260838242310747\\
734	0.000322361155321021\\
735	0.000408085591014806\\
736	0.000500191431074708\\
737	0.00058002510337959\\
738	0.000632011950698954\\
739	0.000645802762963247\\
740	0.000616844356729283\\
741	0.000550922528146102\\
742	0.000459305699948549\\
743	0.000359780624787064\\
744	0.000272749436143182\\
745	0.000213115842992952\\
746	0.000201771726198276\\
747	0.000246964102408434\\
748	0.00034814847905581\\
749	0.000496225709460159\\
750	0.000674517773355487\\
751	0.000860764831596085\\
752	0.00103110884136249\\
753	0.00120312208891283\\
754	0.00159539513827338\\
755	0.00230995605634674\\
756	0.00297395470529608\\
757	0.00326848376468754\\
758	0.00209187888054664\\
759	0.000943068562303867\\
760	0.000587978282291508\\
761	0.000401824040430758\\
762	0.000292713462561237\\
763	0.000245823298351782\\
764	0.000254575969587253\\
765	0.000309257725936979\\
766	0.000395390046333536\\
767	0.000494739099732525\\
768	0.000588078640758542\\
769	0.000657765511116284\\
770	0.000690681584511018\\
771	0.00068007727763266\\
772	0.000626243865149104\\
773	0.000539344011344856\\
774	0.000435122066631379\\
775	0.000333104459608913\\
776	0.000250012136588391\\
777	0.000207605535802831\\
778	0.000218516358403254\\
779	0.000286164519687344\\
780	0.000405824285397887\\
781	0.000564151478804423\\
782	0.000741069604933807\\
783	0.000913408352348542\\
784	0.00105817799850884\\
785	0.00116863170610041\\
786	0.00131926009184296\\
787	0.0016761614578073\\
788	0.00190462363170195\\
789	0.0015347344735254\\
790	0.000881179159586846\\
791	0.000618124904362169\\
792	0.000446978082408779\\
793	0.00032732978428585\\
794	0.000258126897505925\\
795	0.000240711914126091\\
796	0.000271118865262671\\
797	0.000338875498668906\\
798	0.000428057826094463\\
799	0.000520309427781468\\
800	0.000597426982765801\\
801	0.000643968377614195\\
802	0.000651080748648966\\
803	0.000614916107044198\\
804	0.000542738291652681\\
805	0.000447760392059123\\
806	0.000348084402662175\\
807	0.000263967355321594\\
808	0.000209606508837944\\
809	0.000205894815210447\\
810	0.000258995547913527\\
811	0.000367079854485518\\
812	0.000519934807888593\\
813	0.000699736822608702\\
814	0.000883718418871231\\
815	0.00104810526402933\\
816	0.0011578980575424\\
817	0.0013180445572079\\
818	0.00175083913720902\\
819	0.0022207978241508\\
820	0.00218367959713026\\
821	0.00150839575266793\\
822	0.000765482396027574\\
823	0.000511882332933578\\
824	0.000368371579914211\\
825	0.000279394226489938\\
826	0.000240854189788485\\
827	0.000251435678304083\\
828	0.000303954535843525\\
829	0.00038502338784057\\
830	0.000477027347652405\\
831	0.000561356567037472\\
832	0.000621299615788332\\
833	0.000645761813936688\\
834	0.000626758939748908\\
835	0.00056871770318775\\
836	0.00048202037185815\\
837	0.000382966360260039\\
838	0.000291291365784421\\
839	0.000223166991459894\\
840	0.000199158276089719\\
841	0.000230672229668285\\
842	0.00031858053683201\\
843	0.000456604263525853\\
844	0.00062959756268058\\
845	0.000816289191934858\\
846	0.000992935470800827\\
847	0.00113667516964645\\
848	0.00122400812588878\\
849	0.00156720254878935\\
850	0.00212528041555597\\
851	0.00241048331105039\\
852	0.0017664386282265\\
853	0.000907230737811351\\
854	0.000611027358399405\\
855	0.000437524843886714\\
856	0.00032093541349539\\
857	0.000255572491789192\\
858	0.000241490132577915\\
859	0.000273874554909922\\
860	0.000341559238215165\\
861	0.000428253645073146\\
862	0.000515548601725965\\
863	0.000585791411618032\\
864	0.000624478190919291\\
865	0.000623796376679542\\
866	0.000581603715209895\\
867	0.000505962181325328\\
868	0.000411775339525476\\
869	0.00031721488994687\\
870	0.000242312312525372\\
871	0.000200358417355251\\
872	0.000211284575342702\\
873	0.000279821576076866\\
874	0.000401967817663946\\
875	0.000566231942281935\\
876	0.000753204567166795\\
877	0.000939680812121886\\
878	0.00110185732973137\\
879	0.00120576575225367\\
880	0.00147028056121615\\
881	0.00203246383525313\\
882	0.00252976445685917\\
883	0.00233370790180146\\
884	0.00126438609571799\\
885	0.000668097142266684\\
886	0.000463534775167525\\
887	0.000342034227888722\\
888	0.000266489047017678\\
889	0.000238150211803084\\
890	0.00025684548213474\\
891	0.000315123764140794\\
892	0.000398995289381007\\
893	0.000490807081186322\\
894	0.000571877118078249\\
895	0.000626156677235461\\
896	0.00064350616506894\\
897	0.000617558589422018\\
898	0.000554245579027575\\
899	0.000464390153483032\\
900	0.000365272999675664\\
901	0.000277100238914887\\
902	0.000215227735458418\\
903	0.000200302661190932\\
904	0.00024184272710661\\
905	0.000339396122044628\\
906	0.000485000721089237\\
907	0.00066232072245005\\
908	0.000849383376267563\\
909	0.00102232394641537\\
910	0.00116191808602857\\
911	0.00158150155463886\\
912	0.00232607463318769\\
913	0.00301345163751797\\
914	0.0033262541725712\\
915	0.00217891485465316\\
916	0.000966272216766809\\
917	0.000600371006176841\\
918	0.000410146896527795\\
919	0.000297405037980621\\
920	0.000246850976233443\\
921	0.000252290843656254\\
922	0.000304489926812354\\
923	0.000389402655100489\\
924	0.000488797781177101\\
925	0.000583818870103045\\
926	0.000656280400519135\\
927	0.000692567765284095\\
928	0.000685627455829717\\
929	0.000634817406687606\\
930	0.000549741766050458\\
931	0.000445887968125921\\
932	0.000342486088534239\\
933	0.000256517952710564\\
934	0.000209713833260185\\
935	0.000215501665815944\\
936	0.000278101691068424\\
937	0.000393275185491997\\
938	0.000548508616288951\\
939	0.000724168873211001\\
940	0.000897240304520245\\
941	0.0010445952330083\\
942	0.00115327526217741\\
943	0.00128413245587129\\
944	0.00162241533004763\\
945	0.00187389881934073\\
946	0.00154949425303238\\
947	0.000896910302071825\\
948	0.000632768673609916\\
949	0.000458734615233535\\
950	0.000335303218066762\\
951	0.000261882977690101\\
952	0.000240303763544838\\
953	0.000267135247077576\\
954	0.00033239154812475\\
955	0.000420501410464972\\
956	0.000513198121913602\\
957	0.000592048223550478\\
958	0.000641449355261602\\
959	0.000652010882785715\\
960	0.000619084049956267\\
961	0.000550235135903859\\
962	0.000456356747589941\\
963	0.000355986147052401\\
964	0.000269700797468077\\
965	0.000212210451638068\\
966	0.000203836893614483\\
967	0.000253055847466223\\
968	0.000356923798040565\\
969	0.000506719069871664\\
970	0.000685182230087579\\
971	0.000869733175499153\\
972	0.00103654665759384\\
973	0.00121891648257509\\
974	0.00160696107072774\\
975	0.00229273810173102\\
976	0.00289505234311178\\
977	0.00303569519066348\\
978	0.00194985069803879\\
979	0.000909133297359539\\
980	0.00057204533581867\\
981	0.000392545027587713\\
982	0.000287977680457206\\
983	0.000244949282223118\\
984	0.000256877888625773\\
985	0.000313709045297841\\
986	0.000400588732294259\\
987	0.000499213748002348\\
988	0.000590339171929181\\
989	0.000656646096617832\\
990	0.000685549773286403\\
991	0.000670935470721236\\
992	0.000614011280661909\\
993	0.000525434585469896\\
994	0.000421299703059959\\
995	0.000321421830855098\\
996	0.000242235861663222\\
997	0.00020536465574804\\
998	0.000222686129862495\\
999	0.000296612746573577\\
1000	0.000421660806908181\\
1001	0.000583764562776151\\
1002	0.000762359594784662\\
1003	0.000934097788088544\\
1004	0.0010761619348449\\
1005	0.0011617054896475\\
1006	0.00135799211271383\\
1007	0.00175105381288386\\
1008	0.00197412959139005\\
1009	0.0015512838732877\\
1010	0.000872595909178086\\
1011	0.000606393362597329\\
1012	0.000437114005727391\\
1013	0.000320726654784797\\
1014	0.000255233304181492\\
1015	0.000241397733273577\\
1016	0.000274698416116517\\
1017	0.000344179342230294\\
1018	0.00043359416022494\\
1019	0.000524558688128491\\
1020	0.000599007786612566\\
1021	0.000642016452343396\\
1022	0.000645191684981571\\
1023	0.000605677524365579\\
1024	0.000531706761494323\\
1025	0.000435911130475557\\
1026	0.000337211044351228\\
1027	0.000256073131395008\\
1028	0.00020640545911354\\
1029	0.000208108571849441\\
1030	0.000267237505851257\\
1031	0.000380969504389233\\
1032	0.000538153153343809\\
1033	0.000720113945474983\\
1034	0.000904050614507953\\
1035	0.00106621239508596\\
1036	0.00117802341262082\\
1037	0.00136575098583515\\
1038	0.00183491886001263\\
1039	0.00229319691959038\\
1040	0.00217364222376914\\
1041	0.00152727249745055\\
1042	0.000743027662424473\\
1043	0.000490750890407715\\
1044	0.000353882844160604\\
1045	0.000271530067204306\\
1046	0.000239063536619067\\
1047	0.000254909147541053\\
1048	0.000311485467124196\\
1049	0.000394976326556503\\
1050	0.000487704660706848\\
1051	0.000570989452611642\\
1052	0.000628419100561567\\
1053	0.000649351992803704\\
1054	0.000626660770963973\\
1055	0.000565554617194687\\
1056	0.000476696809316041\\
1057	0.000376890766257361\\
1058	0.000286175592674358\\
1059	0.000220261981903717\\
1060	0.000199995119057705\\
1061	0.000235553207659594\\
1062	0.000327563738644883\\
1063	0.000468594097452482\\
1064	0.000642957487940567\\
1065	0.000829060169718713\\
1066	0.00100311695710956\\
1067	0.00114253909814045\\
1068	0.00134197480952454\\
1069	0.00161915728278854\\
1070	0.00210789134597225\\
1071	0.0022581907358971\\
1072	0.00166301094336654\\
1073	0.000866778647041666\\
1074	0.000583846581164283\\
1075	0.000417204794627448\\
1076	0.000307624166721897\\
1077	0.000249885085516118\\
1078	0.000243407224434555\\
1079	0.000282588170263507\\
1080	0.000355665567300025\\
1081	0.000445714915078033\\
1082	0.000534315871552613\\
1083	0.000603599736787515\\
1084	0.000641161165582579\\
1085	0.00063588405329805\\
1086	0.000589694427588893\\
1087	0.000510715420329002\\
1088	0.000413976362693755\\
1089	0.000318049071398065\\
1090	0.000239733806693997\\
1091	0.00020047672303613\\
1092	0.000213785514554248\\
1093	0.000284055634364214\\
1094	0.000407500966585762\\
1095	0.000571597791288958\\
1096	0.000756693633707782\\
1097	0.000939553126499669\\
1098	0.00109660277639823\\
1099	0.0011954765921088\\
1100	0.00144656993292081\\
1101	0.00196917267789813\\
1102	0.00239765334899934\\
1103	0.00211929158704622\\
1104	0.00115133018618452\\
1105	0.000641844517525992\\
1106	0.000450647523035944\\
1107	0.000333148781192954\\
1108	0.0002616987290618\\
1109	0.000238034536625282\\
1110	0.00026120260124052\\
1111	0.000322981626578262\\
1112	0.000408813658356167\\
1113	0.000500859716248804\\
1114	0.000580486968182455\\
1115	0.000632205352658809\\
1116	0.000645600223629089\\
1117	0.000616261815030287\\
1118	0.000550088516402881\\
1119	0.000458329380295451\\
1120	0.000358835141307988\\
1121	0.000272025282399381\\
1122	0.000212767510324628\\
1123	0.000201919850348592\\
1124	0.000247651359188158\\
1125	0.000349336803375021\\
1126	0.000497798796695077\\
1127	0.000676294984234035\\
1128	0.000862532694054269\\
1129	0.00103265295387622\\
1130	0.00120807661794736\\
1131	0.00160269086964956\\
1132	0.0023191609600358\\
1133	0.00298126121323646\\
1134	0.00326965801125916\\
1135	0.00207375761058689\\
1136	0.000938902394568637\\
1137	0.000586161237320612\\
1138	0.000400705867825309\\
1139	0.00029210107728532\\
1140	0.000245688641761529\\
1141	0.000254866078555145\\
1142	0.000309870584796587\\
1143	0.000396176773950968\\
1144	0.000495541768828156\\
1145	0.000588729724984289\\
1146	0.000658127126740857\\
1147	0.000690666995933109\\
1148	0.000679661460774088\\
1149	0.000625492120494997\\
1150	0.00053837700164336\\
1151	0.000434094776043111\\
1152	0.000332204457980897\\
1153	0.000249403542015783\\
1154	0.000207442063943438\\
1155	0.000218879013296877\\
1156	0.000287044513500312\\
1157	0.000407143360745555\\
1158	0.000565756142585026\\
1159	0.000742761346746391\\
1160	0.000914972013043347\\
1161	0.00105941725524066\\
1162	0.00115868090948025\\
1163	0.00131648878356385\\
1164	0.00167786665456488\\
1165	0.00191058301244403\\
1166	0.00153672061814424\\
1167	0.000881038696712416\\
1168	0.000617606656510247\\
1169	0.000446497578561787\\
1170	0.000327005339567425\\
1171	0.000257996798175465\\
1172	0.000240763903933413\\
1173	0.00027129144820176\\
1174	0.000339064844902796\\
1175	0.000428133875294256\\
1176	0.000520141843372154\\
1177	0.000596907583315141\\
1178	0.000643037933770635\\
1179	0.000649731824608101\\
1180	0.000613233485283416\\
1181	0.000540866226716224\\
1182	0.000445894683796159\\
1183	0.00034644884178599\\
1184	0.000262782360531475\\
1185	0.000209042756173868\\
1186	0.000206056820776584\\
1187	0.000259916792548178\\
1188	0.000368751549737791\\
1189	0.000522343042914895\\
1190	0.000702477451700254\\
1191	0.000886577550772318\\
1192	0.00105085338362246\\
1193	0.0011637530817348\\
1194	0.00132652464568795\\
1195	0.00177467042273533\\
1196	0.00224124970601029\\
1197	0.0021886547351369\\
1198	0.00150646898783177\\
1199	0.000761590977867421\\
1200	0.000508884064999968\\
1201	0.000366452350837252\\
1202	0.000278367883538542\\
1203	0.000240580807717256\\
1204	0.000251798405367188\\
1205	0.000304815947569978\\
1206	0.000386207306062731\\
1207	0.000478354096205927\\
1208	0.000562633247344098\\
1209	0.000622362745669202\\
1210	0.00064649104024601\\
1211	0.000627118566450609\\
1212	0.000568766994250751\\
1213	0.000481820654562611\\
1214	0.000382627524586821\\
1215	0.000290973928776196\\
1216	0.000223002909853607\\
1217	0.000199279532905436\\
1218	0.000231142876190548\\
1219	0.000319356776252668\\
1220	0.000457598686681378\\
1221	0.000630657611605662\\
1222	0.000817229572511659\\
1223	0.000993566789007536\\
1224	0.00113683495897867\\
1225	0.00122392781461498\\
1226	0.00156729600689688\\
1227	0.00212309507500948\\
1228	0.00240215452166154\\
1229	0.00175495749563505\\
1230	0.00090333604003019\\
1231	0.00060875491889792\\
1232	0.000435937005538387\\
1233	0.000319916985293747\\
1234	0.000255126308406143\\
1235	0.00024159411970681\\
1236	0.000274449722964469\\
1237	0.00034247172727789\\
1238	0.000429330498400257\\
1239	0.000516614568170066\\
1240	0.000586676937489596\\
1241	0.000625070212519361\\
1242	0.000624022370960579\\
1243	0.000581473632712893\\
1244	0.000505574609891578\\
1245	0.000411245469329416\\
1246	0.00031670100866449\\
1247	0.000238683667884194\\
1248	0.000198830434927038\\
1249	0.000210933608310588\\
1250	0.000279981389105723\\
1251	0.000402807474126155\\
1252	0.000567391953558336\\
1253	0.000754409722906168\\
1254	0.000940702465694495\\
1255	0.00110249947524046\\
1256	0.00120575482071066\\
1257	0.00147137042994254\\
1258	0.00203374374648469\\
1259	0.00252637764334384\\
1260	0.00232169229877028\\
1261	0.0012565773023242\\
1262	0.000665744918053056\\
1263	0.000462238962896577\\
1264	0.000341149807610099\\
1265	0.000266012101719054\\
1266	0.000238111436421968\\
1267	0.000257212394173068\\
1268	0.000315797417645105\\
1269	0.000399825417523246\\
1270	0.00049162309743902\\
1271	0.000572518566670587\\
1272	0.000626521477913625\\
1273	0.000643507938753361\\
1274	0.000617191735093778\\
1275	0.000553602086530217\\
1276	0.000463576005360587\\
1277	0.000364452455739458\\
1278	0.000276453733475048\\
1279	0.000214914635073553\\
1280	0.000200432048394612\\
1281	0.000242461102762942\\
1282	0.000340480506912677\\
1283	0.000486437354459587\\
1284	0.000663936156467235\\
1285	0.000850970837706551\\
1286	0.00102367416652478\\
1287	0.00116900891949512\\
1288	0.00157628204699106\\
1289	0.0023149615024242\\
1290	0.00300617146052907\\
1291	0.0033242237890203\\
1292	0.00217922372608659\\
1293	0.000965998000761786\\
1294	0.000599903480850405\\
1295	0.00040973547696968\\
1296	0.000297165875671708\\
1297	0.000246824943866386\\
1298	0.000252456097285615\\
1299	0.000304769135075608\\
1300	0.000389673783683699\\
1301	0.000488928255010851\\
1302	0.000583672255685699\\
1303	0.000655755156348356\\
1304	0.00069161446213432\\
1305	0.000684258180045248\\
1306	0.000633133428935175\\
1307	0.000547909864869623\\
1308	0.000444100197683433\\
1309	0.000340965749928948\\
1310	0.000255459061036585\\
1311	0.000209299061653616\\
1312	0.000215823660003386\\
1313	0.000279163978354025\\
1314	0.000395015661266506\\
1315	0.000550763401198116\\
1316	0.000726717059139508\\
1317	0.00089983770597601\\
1318	0.00104700711935916\\
1319	0.00116414546885117\\
1320	0.00129619934985226\\
1321	0.00163651062503482\\
1322	0.00188131669566716\\
1323	0.00154534782901893\\
1324	0.000893604452051588\\
1325	0.000630217724475974\\
1326	0.000456866770619925\\
1327	0.000334082526856116\\
1328	0.000261296042918161\\
1329	0.00024031726645711\\
1330	0.000267669961539449\\
1331	0.000333317030692719\\
1332	0.000421645277953746\\
1333	0.000514373681284381\\
1334	0.000593113920853668\\
1335	0.000642223106499746\\
1336	0.000652416120851712\\
1337	0.000619119245976349\\
1338	0.000549988236605623\\
1339	0.000455906128993931\\
1340	0.000355483079273943\\
1341	0.000269312243022495\\
1342	0.000212066989107687\\
1343	0.000204048861443456\\
1344	0.000253682590874876\\
1345	0.000357869351510549\\
1346	0.000507880116051246\\
1347	0.000686392204047376\\
1348	0.000870797208464771\\
1349	0.00103727534983084\\
1350	0.00122252789112584\\
1351	0.00155473099291027\\
1352	0.00222010846585894\\
1353	0.00283097890025056\\
1354	0.00291724712524294\\
1355	0.00183333093681045\\
1356	0.00088348885400505\\
1357	0.000567070649680046\\
1358	0.00039311864203908\\
1359	0.000289317264472875\\
1360	0.000245211752911228\\
1361	0.000255362098679304\\
1362	0.000310189286073561\\
1363	0.000395071449609467\\
1364	0.000491783976481755\\
1365	0.00058121842974805\\
1366	0.000646161700526449\\
1367	0.000674111666371631\\
1368	0.00065905049931259\\
1369	0.000602287434897611\\
1370	0.000514506670883493\\
1371	0.000411805471966598\\
1372	0.000313969549389241\\
1373	0.000237311025851627\\
1374	0.000203401394141133\\
1375	0.000223945362381001\\
1376	0.000300996217245221\\
1377	0.000429067792416522\\
1378	0.000593864222418142\\
1379	0.000774680892285765\\
1380	0.000948073334632319\\
1381	0.00109117795545366\\
1382	0.00117444032294313\\
1383	0.00140461930685163\\
1384	0.00184216286437339\\
1385	0.00209245076995682\\
1386	0.00161564258679803\\
1387	0.000886441385699157\\
1388	0.000611229974409446\\
1389	0.000439873545956426\\
1390	0.000322489411207036\\
1391	0.000256089803945241\\
1392	0.000241293728186119\\
1393	0.000273576168517605\\
1394	0.000342000003456413\\
1395	0.000430351401956264\\
1396	0.000520277702777339\\
1397	0.000593792085293828\\
1398	0.000636017599829179\\
1399	0.000638653955974957\\
1400	0.000598926486970405\\
1401	0.000525205014937689\\
1402	0.00043001639536672\\
1403	0.000332349766663451\\
1404	0.000252645636814812\\
1405	0.000204705713893079\\
1406	0.000208381417297986\\
1407	0.000269514954464418\\
1408	0.00038519601534999\\
1409	0.000544121973006008\\
1410	0.000727503638459031\\
1411	0.000912465959001813\\
1412	0.00107522411726591\\
1413	0.00120898952312019\\
1414	0.00140235665426962\\
1415	0.00188962913636283\\
1416	0.00235116057404162\\
1417	0.00221010279950565\\
1418	0.00154298179562961\\
1419	0.000739924931136477\\
1420	0.000487014172226688\\
1421	0.000351494257373262\\
1422	0.000270318175367723\\
1423	0.000238763353890464\\
1424	0.000255326247442294\\
1425	0.000312443284135705\\
1426	0.000396279383423025\\
1427	0.000489160519990193\\
1428	0.000572402261043001\\
1429	0.000629619263310708\\
1430	0.000650217173821945\\
1431	0.000627153818465778\\
1432	0.000565717523829814\\
1433	0.000476596478232416\\
1434	0.000376649962849411\\
1435	0.00028594664611238\\
1436	0.000220173070735449\\
1437	0.000200156868259966\\
1438	0.000236012162293286\\
1439	0.000328293150183452\\
1440	0.000469605965273303\\
1441	0.000643915812828179\\
1442	0.000829816405320168\\
1443	0.00100351267101249\\
1444	0.00114242935822202\\
1445	0.0013392338158025\\
1446	0.00161731790652364\\
1447	0.00210397249573501\\
1448	0.0022519052261158\\
1449	0.00165904261322591\\
1450	0.000864892097007114\\
1451	0.000582343075968644\\
1452	0.00041600047862666\\
1453	0.000306831649575968\\
1454	0.000249579030096574\\
1455	0.000243587081338607\\
1456	0.000283180096981219\\
1457	0.000356536387847879\\
1458	0.000446695120027979\\
1459	0.000535227208416738\\
1460	0.000604295996771936\\
1461	0.000641536517530854\\
1462	0.000635888922627025\\
1463	0.000589354362196814\\
1464	0.000510135423202601\\
1465	0.000413284394697694\\
1466	0.000317412465219391\\
1467	0.000239315196165106\\
1468	0.000200408338107805\\
1469	0.000214153470285847\\
1470	0.000284862679857108\\
1471	0.000408678067194663\\
1472	0.000573002102763155\\
1473	0.000758135906155667\\
1474	0.000940826382907164\\
1475	0.00109751598753727\\
1476	0.00119555803544358\\
1477	0.00144831247186506\\
1478	0.00197169435891149\\
1479	0.00239692200327065\\
1480	0.00210962036798191\\
1481	0.0011443238230954\\
1482	0.000639997430448365\\
1483	0.000449614692148569\\
1484	0.000332427882427543\\
1485	0.000261326523404698\\
1486	0.000238051306630284\\
1487	0.000261576340063025\\
1488	0.000323612726110885\\
1489	0.000409554190655024\\
1490	0.000501542355936325\\
1491	0.000580965956874739\\
1492	0.000632421331034445\\
1493	0.000645412718231771\\
1494	0.000615696321545271\\
1495	0.000549271560788447\\
1496	0.000457369491747785\\
1497	0.000357905592754647\\
1498	0.000271314407079313\\
1499	0.000212428167265996\\
1500	0.000202074090950792\\
1501	0.000248339858907692\\
1502	0.000350521191785838\\
1503	0.00049936247394867\\
1504	0.000678056341009666\\
1505	0.000864278947648238\\
1506	0.00103417062339067\\
1507	0.00121264400702647\\
1508	0.00160908791397437\\
1509	0.00232751185970966\\
1510	0.00298609431956116\\
1511	0.00327421992151101\\
1512	0.00205799897119824\\
1513	0.000935303934945996\\
1514	0.000584446391299458\\
1515	0.00039961974479321\\
1516	0.000291505180411853\\
1517	0.000245562840336313\\
1518	0.000255159577616499\\
1519	0.000310482407203311\\
1520	0.000396958716925688\\
1521	0.00049633615297087\\
1522	0.000589369788034069\\
1523	0.000658475703828669\\
1524	0.000690638606161223\\
1525	0.00067923186846969\\
1526	0.000624728193033563\\
1527	0.000537399794947427\\
1528	0.000433059750690747\\
1529	0.000331299688534075\\
1530	0.000248794261014831\\
1531	0.000207281256853853\\
1532	0.000219247863045554\\
1533	0.000287933739015148\\
1534	0.000408472572128931\\
1535	0.000567370909584391\\
1536	0.000744462741018279\\
1537	0.000916545695806085\\
1538	0.00106066591427531\\
1539	0.00117770931232126\\
1540	0.00132730626249211\\
1541	0.00168482118039109\\
1542	0.0019038324153321\\
1543	0.00152949783796595\\
1544	0.000876697044839789\\
1545	0.000614246520291929\\
1546	0.000443917011471897\\
1547	0.000325272753035105\\
1548	0.000257177481622935\\
1549	0.00024085879025879\\
1550	0.000272222625260938\\
1551	0.000340681159195053\\
1552	0.000430226658636189\\
1553	0.000522484049609356\\
1554	0.000599267499919035\\
1555	0.000645228897833792\\
1556	0.00065159592296614\\
1557	0.000614895612383104\\
1558	0.000542217720159914\\
1559	0.000446872503304387\\
1560	0.000347062986088138\\
1561	0.000263158487087937\\
1562	0.000209476916430699\\
1563	0.000206143605837205\\
1564	0.000260078305144257\\
1565	0.000369022645129413\\
1566	0.000522398977736257\\
1567	0.00070233726186832\\
1568	0.000886042110740721\\
1569	0.00104976375347778\\
1570	0.00116187284450668\\
1571	0.00132272560754893\\
1572	0.00176188611779303\\
1573	0.00222169588884102\\
1574	0.00216232389649539\\
1575	0.00149686782617415\\
1576	0.00075856542579931\\
1577	0.000507136041955225\\
1578	0.00036518460711797\\
1579	0.000277623457392905\\
1580	0.000240421853988441\\
1581	0.000252208510996479\\
1582	0.000305699841196771\\
1583	0.000387404935197547\\
1584	0.000479691129301718\\
1585	0.00056391895463386\\
1586	0.000623434153948916\\
1587	0.000647227857776019\\
1588	0.000627485521625533\\
1589	0.000568824461975531\\
1590	0.000481627352890674\\
1591	0.000382295145892179\\
1592	0.000290662356182103\\
1593	0.000222843005049496\\
1594	0.000199403411547268\\
1595	0.000231614231579162\\
1596	0.000320130989487119\\
1597	0.000458587077893109\\
1598	0.000631708281882578\\
1599	0.000818157069761191\\
1600	0.000994182158364137\\
1601	0.00113697809233451\\
1602	0.00122378013920195\\
1603	0.00156743479289506\\
1604	0.00212091432961183\\
1605	0.00239375796467635\\
1606	0.00174803501699049\\
1607	0.000900161506176022\\
1608	0.000606577712675614\\
1609	0.000434337398237287\\
1610	0.000318878822593469\\
1611	0.000254672411804594\\
1612	0.00024170558381338\\
1613	0.000275046859407169\\
1614	0.000343418966109099\\
1615	0.000430452783591828\\
1616	0.000517735285678202\\
1617	0.000587622492138628\\
1618	0.000625725753871297\\
1619	0.000624312785254427\\
1620	0.000581406840717692\\
1621	0.00050524637798113\\
1622	0.000410767314758959\\
1623	0.000316228508112409\\
1624	0.000238363089381779\\
1625	0.000198803823276374\\
1626	0.000211284100092354\\
1627	0.000280711448037921\\
1628	0.000403854596325759\\
1629	0.000568620343647415\\
1630	0.000755637205705387\\
1631	0.00094172909568451\\
1632	0.00110313995488089\\
1633	0.00119819880146642\\
1634	0.00148800258556933\\
1635	0.00204726186805457\\
1636	0.00261481940259689\\
1637	0.00253755979927685\\
1638	0.00174340646236679\\
1639	0.000749080497676306\\
1640	0.000477176569586432\\
1641	0.000344738938209948\\
1642	0.000267287715909356\\
1643	0.00023807112051462\\
1644	0.000255712843393433\\
1645	0.000312686302058686\\
1646	0.000395083833982223\\
1647	0.000485342865605524\\
1648	0.000564883533563479\\
1649	0.000617744860596066\\
1650	0.00063393478244763\\
1651	0.000607337940727453\\
1652	0.000544009959736575\\
1653	0.000454785484235104\\
1654	0.000357024713685218\\
1655	0.000270920733248019\\
1656	0.000211643246396258\\
1657	0.00019983787315629\\
1658	0.000244902864147129\\
1659	0.000345538250074784\\
1660	0.000494110113151698\\
1661	0.000673734268385335\\
1662	0.000862486070195323\\
1663	0.00103640500669669\\
1664	0.00121808965836728\\
1665	0.00163972036453142\\
1666	0.00241982401736397\\
1667	0.00317315062560406\\
1668	0.00346499471728566\\
1669	0.00221302229603809\\
1670	0.000975110090548279\\
1671	0.000603042122818673\\
1672	0.000410633141429397\\
1673	0.00029735041335486\\
1674	0.00024694551803829\\
1675	0.000252722339957251\\
1676	0.000305180725984983\\
1677	0.000390127974295766\\
1678	0.000489294267295651\\
1679	0.000583804968159886\\
1680	0.00065554178878775\\
1681	0.00069099217326936\\
1682	0.000683225033908105\\
1683	0.000631775793264349\\
1684	0.000546379045380973\\
1685	0.000442574351843991\\
1686	0.000339655788782836\\
1687	0.000254549531916665\\
1688	0.000208963516580916\\
1689	0.000216150059103751\\
1690	0.000280151768504911\\
1691	0.000396603376930794\\
1692	0.000552791332788872\\
1693	0.00072897153018407\\
1694	0.000902084200748462\\
1695	0.00104902326098456\\
1696	0.00113641273445128\\
1697	0.00128692313306556\\
1698	0.00163916058193116\\
1699	0.00190157651576525\\
1700	0.00156575068840798\\
1701	0.000896724681319945\\
1702	0.000630451220825897\\
1703	0.000456914710844085\\
1704	0.000334234531708295\\
1705	0.0002614309733636\\
1706	0.000240308505098397\\
1707	0.000267412219751656\\
1708	0.000332710814042501\\
1709	0.00042059621707499\\
1710	0.000512803403853989\\
1711	0.000590970065420545\\
1712	0.000639531308766146\\
1713	0.000649245478894317\\
1714	0.000615635953754953\\
1715	0.000546447935833628\\
1716	0.000452547322394876\\
1717	0.000352602048717491\\
1718	0.000267199867852914\\
1719	0.000210943305209027\\
1720	0.000204104193585679\\
1721	0.000255060431998802\\
1722	0.000360367325032592\\
1723	0.000511386294114009\\
1724	0.000690678499208456\\
1725	0.000875575048017843\\
1726	0.00104223458589864\\
1727	0.00118485568255323\\
1728	0.00131907857922815\\
1729	0.00174525193876984\\
1730	0.00222020347522522\\
1731	0.00221367118044671\\
1732	0.00153660199141845\\
1733	0.000782956666843288\\
1734	0.000524027164388148\\
1735	0.00037656161790151\\
1736	0.000283972750307236\\
1737	0.000241998235868562\\
1738	0.000249532787556095\\
1739	0.000299724187871059\\
1740	0.000379510717976372\\
1741	0.000471246590047697\\
1742	0.000556663358301714\\
1743	0.000618540117105776\\
1744	0.000645522148632456\\
1745	0.000629095944321531\\
1746	0.000573189980755279\\
1747	0.000487832444449322\\
1748	0.000389012912478236\\
1749	0.00029632189462068\\
1750	0.000225958923221844\\
1751	0.000198991830080584\\
1752	0.00022704126933961\\
1753	0.000311566654197299\\
1754	0.000446880436699765\\
1755	0.000618298653756583\\
1756	0.000804829003862387\\
1757	0.000982794535571403\\
1758	0.00112917428017345\\
1759	0.00124433029462433\\
1760	0.00155013049444371\\
1761	0.00210069636789643\\
1762	0.00238865185842426\\
1763	0.00178977524443141\\
1764	0.000920977603001728\\
1765	0.000621380178474878\\
1766	0.00044507360186765\\
1767	0.000325741637739826\\
1768	0.000257652511309594\\
1769	0.000241059251285692\\
1770	0.000271466544834015\\
1771	0.000338015577033546\\
1772	0.000424577795527032\\
1773	0.000512861711485471\\
1774	0.000584994734269865\\
1775	0.000626153417472545\\
1776	0.000628159155204064\\
1777	0.000588320268257659\\
1778	0.000515139607719121\\
1779	0.000421027630792743\\
1780	0.000325023317698295\\
1781	0.000247516519428468\\
1782	0.000202165247443777\\
1783	0.000209468465848991\\
1784	0.000273149368340361\\
1785	0.000391558038335737\\
1786	0.000553032504757827\\
1787	0.000738611870676448\\
1788	0.000925283844207235\\
1789	0.00108918965354231\\
1790	0.00125700045786242\\
1791	0.00146418937108081\\
1792	0.00197808691899348\\
1793	0.00244581048110569\\
1794	0.00227752652370622\\
1795	0.00166334385840549\\
1796	0.00074735295889624\\
1797	0.000480569474177378\\
1798	0.000346301131400795\\
1799	0.000267557455212987\\
1800	0.000237987714942115\\
1801	0.000256112159972349\\
1802	0.000314465092078963\\
1803	0.000399227694349288\\
1804	0.000492799606534647\\
1805	0.000576372573917678\\
1806	0.000633569447283647\\
1807	0.000653926527730567\\
1808	0.000630488103861705\\
1809	0.000568577992455539\\
1810	0.00047891663633793\\
1811	0.000378432164931034\\
1812	0.00028724063034204\\
1813	0.000221006868606215\\
1814	0.000200606163638295\\
1815	0.000236089583560569\\
1816	0.000328025452801789\\
1817	0.00046868628549427\\
1818	0.000642462444370816\\
1819	0.000827689194478219\\
1820	0.00100060212504752\\
1821	0.00113868774900741\\
1822	0.00134326372496013\\
1823	0.00160385455743174\\
1824	0.00207309267050779\\
1825	0.00220329078793436\\
1826	0.00162980787123123\\
1827	0.000856735772224378\\
1828	0.000578259506666438\\
1829	0.000413289412657745\\
1830	0.000305142020428226\\
1831	0.000248909296839432\\
1832	0.000243921817999658\\
1833	0.000284425193081701\\
1834	0.000358518952346387\\
1835	0.000449185201045256\\
1836	0.000537968718086116\\
1837	0.000607064736059028\\
1838	0.000643060078347744\\
1839	0.000638073547334895\\
1840	0.000591202541721754\\
1841	0.000511574864851868\\
1842	0.000414322984951311\\
1843	0.000318152420648054\\
1844	0.000239626259659674\\
1845	0.000200702791304218\\
1846	0.000214630843349472\\
1847	0.000285283967870421\\
1848	0.000408798650394918\\
1849	0.000572774785771863\\
1850	0.000757463774960676\\
1851	0.000939590699419232\\
1852	0.00109561391334769\\
1853	0.00119344413995352\\
1854	0.00144170782056491\\
1855	0.00195658047278585\\
1856	0.00237148498981821\\
1857	0.00207539860976295\\
1858	0.00112775067120764\\
1859	0.000637149288417413\\
1860	0.000448535302167307\\
1861	0.00033166654184604\\
1862	0.000260915898497232\\
1863	0.000238099478567221\\
1864	0.000262087593423787\\
1865	0.000324496699597569\\
1866	0.000410655636567008\\
1867	0.000502677531194548\\
1868	0.000581971483569581\\
1869	0.000633230557379315\\
1870	0.000645881966594472\\
1871	0.000615717641415366\\
1872	0.000549019332885648\\
1873	0.000456909797380911\\
1874	0.000357387965293251\\
1875	0.000270904869515662\\
1876	0.000212284072736915\\
1877	0.000202261147474378\\
1878	0.000248931341904528\\
1879	0.00035146359385073\\
1880	0.000500542906590376\\
1881	0.000679306368674483\\
1882	0.000865402192652676\\
1883	0.00103497426800295\\
1884	0.00121495449258309\\
1885	0.00160998684775147\\
1886	0.00232666321420261\\
1887	0.0029804706868801\\
1888	0.00325866671895767\\
1889	0.00203993657038635\\
1890	0.000935013165107737\\
1891	0.000583631905257202\\
1892	0.000398906374927193\\
1893	0.000291085061220547\\
1894	0.000245498534510148\\
1895	0.000255431045295409\\
1896	0.000311003775433598\\
1897	0.000397592435093086\\
1898	0.000496932633920926\\
1899	0.000589772775589765\\
1900	0.000658559578845797\\
1901	0.000690330054983032\\
1902	0.000678520449830695\\
1903	0.000623694441680737\\
1904	0.000536177244123105\\
1905	0.000431815777698246\\
1906	0.000330233029737173\\
1907	0.000248077306516745\\
1908	0.000207073687704575\\
1909	0.000219634586219585\\
1910	0.000288906473091399\\
1911	0.000409946023213954\\
1912	0.000569186102344926\\
1913	0.00074641484688082\\
1914	0.000918411735552105\\
1915	0.00106223936100829\\
1916	0.00113303976841077\\
1917	0.00131431224395317\\
1918	0.00168606448100876\\
1919	0.0019314246445799\\
1920	0.00154217761996426\\
1921	0.000881086492919712\\
1922	0.000616863063849224\\
1923	0.000445898133891403\\
1924	0.000326666306884683\\
1925	0.000257900182828714\\
1926	0.000240837955761776\\
1927	0.000271398877066292\\
1928	0.000339004412144744\\
1929	0.00042766658001495\\
1930	0.000519036326749173\\
1931	0.000594985095942964\\
1932	0.000640221154774726\\
1933	0.000646054216315499\\
1934	0.000609089925667296\\
1935	0.000536560414646593\\
1936	0.000441682592515739\\
1937	0.000342725570318096\\
1938	0.000260043222889405\\
1939	0.000207858396896977\\
1940	0.000206256824611771\\
1941	0.000261960485882382\\
1942	0.000372629609255475\\
1943	0.000527547332151903\\
1944	0.000708727990188216\\
1945	0.000893312274066486\\
1946	0.00105752082229908\\
1947	0.00117773902897578\\
1948	0.00134658196595411\\
1949	0.00180494698150848\\
1950	0.00227524434033656\\
1951	0.00220702303758331\\
1952	0.00151488599286227\\
1953	0.000758286110977787\\
1954	0.00050520268622649\\
1955	0.000363850468359541\\
1956	0.000276941920774811\\
1957	0.000240237221166133\\
1958	0.000252380842928945\\
1959	0.000306091243984554\\
1960	0.000387861522923176\\
1961	0.000480063170604704\\
1962	0.000564061401780403\\
1963	0.000623239106706367\\
1964	0.000646634637465471\\
1965	0.000626522230236996\\
1966	0.000567617018250089\\
1967	0.000480294230418731\\
1968	0.000381003959449225\\
1969	0.000289624368520136\\
1970	0.000222225717560077\\
1971	0.000199372125535596\\
1972	0.000232256078883165\\
1973	0.00032138770103306\\
1974	0.000460363721341512\\
1975	0.00063383099868678\\
1976	0.000820408098465862\\
1977	0.000996329975413535\\
1978	0.00113881018066043\\
1979	0.00122596344744193\\
1980	0.00157503480162035\\
1981	0.00213125642817672\\
1982	0.00240051936380729\\
1983	0.00174842786364176\\
1984	0.000898465972352176\\
1985	0.000604892857303721\\
1986	0.000433045804054726\\
1987	0.000318051842346953\\
1988	0.000254319679854981\\
1989	0.000241793969634543\\
1990	0.000275493347624507\\
1991	0.000344094171840095\\
1992	0.000431195681639242\\
1993	0.000518387331673228\\
1994	0.000588034012535548\\
1995	0.000625805853962845\\
1996	0.00062401569623747\\
1997	0.000580769553834514\\
1998	0.000504392103602566\\
1999	0.000409834866658022\\
2000	0.000315397306718559\\
};
\addlegendentry{$\text{V}_\text{3}$};

\addplot [color=mycolor4,solid]
  table[row sep=crcr]{%
0	0.0035\\
2000	0.0035\\
};
\addlegendentry{$\varepsilon_{\Omega}$};

\end{axis}
\end{tikzpicture}%
}
  \caption{The $\mat{P}-$norms of the errors of the three agents through time,
    in greater detail and for a longer time-period of execution. The colour
    cyan is used to depict their ceiling $\varepsilon_{\Omega}$}
  \label{fig:d_ON_res_3_2_V_zoom_zoom}
\end{figure}

Due to the interconstrained nature of the compound system, it is advisable that
once the trajectories of \textit{all} agents reach set $\Psi$ $-$ in other
words, once they all reach the vicinity of their desired configurations as
measured by $\varepsilon_{\Psi}$ $-$ it is advisable that they disable their
interconnectedness: since they have all reached the intended feasible
desired configurations it is already guaranteed that they will not collide with
each other or violate the connectivity constraints. If their interconnectedness
is not disabled, then the disturbance affecting each agent could propagate
from one agent to the rest due the fact that the trajectory of each agent
is regulated indirectly by the predicted states of its neighbours.
