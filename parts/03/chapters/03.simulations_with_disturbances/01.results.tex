\section{Simulation results}

For compatibility with real situations, we assume that in the case
of disturbances, the signals affecting the agents are of the same nature
(consider for instance the case of UAV's affected by wind); the disturbance
signal considered was $\delta(t) = 0.1 * \sin 2t$. Therefore,
$\overline{\delta} = 0.1$





In this case the initial configurations of the three agents are
$\vect{z}_1$ $=$ $[-6, 3.5, 0]^{\top}$,
$\vect{z}_2$ $=$ $[-6, 2.3, 0]^{\top}$ and
$\vect{z}_3$ $=$ $[-6, 4.7, 0]^{\top}$.
Their desired configurations in steady-state are
$\vect{z}_{1,des}$ $=$ $[6, 3.5, 0]^{\top}$,
$\vect{z}_{2,des}$ $=$ $[6, 2.3, 0]^{\top}$ and
$\vect{z}_{3,des}$ $=$ $[6, 4.7, 0]^{\top}$.
Obstacles $o_1$ and $o_2$ are placed between the two at $[0, 2.0]^{\top}$
and $[0, 5.5]^{\top}$ respectively. The penalty
matrices $\mat{Q}$, $\mat{R}$, $\mat{P}$ were set to
$\mat{Q} = 0.7 (I_3 + 0.5\dagger_3)$, $\mat{R} = 0.005 I_2$ and
$\mat{P} = 0.5 (I_3 + 0.5\dagger_3)$, where $\dagger_N$ is a $N \times N$
matrix whose elements are chosen at random between the values $0.0$ and $1.0$.
The sampling time is $h = 0.1$ sec, the time-horizon is $T_p = 0.5$ sec, and
the total execution time given was $10$ sec.

Figure \eqref{fig:d_ON_res_3_2_errors_agent_1} depicts the evolution of the
error states of agent 1 through time.
Figures \eqref{fig:d_ON_res_3_2_distance_agents_13} and
\eqref{fig:d_ON_res_3_2_distance_obstacle_1_agents} show the evolution of the
distance between agents 1 and 3 through time, and the evolution of the
distance between all agents and obstacle $o_1$ respectively.
Figure \eqref{fig:d_ON_res_3_2_inputs_agent_2} shows the input signals
directing agent 2 through time. Last but not at all least, figures
\eqref{fig:d_ON_res_3_2_V} and \eqref{fig:d_ON_res_3_2_V_zoom} depict
the evolution of the quadratic Lyapunov function
$\vect{e}^{\top} \mat{P} \vect{e}$ through time for all three agents.
Boundary values for the inputs and distances are portrayed in the colour
\textcolor{cyan}{cyan}. The evolution of the trajectories of the agents in the
$x-y$ plane are omitted; they are (with minor variations) equivalent to those
in the case where disturbances are absent.


\noindent\makebox[\linewidth][c]{%
\begin{minipage}{\linewidth}
  \begin{minipage}{0.45\linewidth}
    \begin{figure}[H]
      \scalebox{0.7}{% This file was created by matlab2tikz.
%
%The latest updates can be retrieved from
%  http://www.mathworks.com/matlabcentral/fileexchange/22022-matlab2tikz-matlab2tikz
%where you can also make suggestions and rate matlab2tikz.
%
\definecolor{mycolor1}{rgb}{0.00000,0.44700,0.74100}%
\definecolor{mycolor2}{rgb}{0.85000,0.32500,0.09800}%
\definecolor{mycolor3}{rgb}{0.92900,0.69400,0.12500}%
%
\begin{tikzpicture}

\begin{axis}[%
width=4.133in,
height=3.26in,
at={(0.693in,0.44in)},
scale only axis,
xmin=1,
xmax=100,
xmajorgrids,
ymin=-1,
ymax=1,
ymajorgrids,
axis background/.style={fill=white},
legend style={legend cell align=left,align=left,draw=white!15!black}
]
\addplot [color=mycolor1,solid]
  table[row sep=crcr]{%
1	-12\\
2	-10.9990104620416\\
3	-10.635534400582\\
4	-9.63270696839852\\
5	-8.99270758131905\\
6	-8.28862830118856\\
7	-7.48386692274089\\
8	-6.84904698185161\\
9	-6.05316197449761\\
10	-5.64032549932534\\
11	-4.73932987487637\\
12	-3.7884183031066\\
13	-2.90575208359328\\
14	-1.95729094930657\\
15	-1.06440430305293\\
16	-0.377071633092927\\
17	-0.101588908968895\\
18	-0.00483464929410599\\
19	-0.00208219748894829\\
20	-0.00642752154634344\\
21	-0.0105827176000995\\
22	-0.0133155100885203\\
23	-0.014982864925944\\
24	-0.0157658775626898\\
25	-0.015796568700575\\
26	-0.0151593773199552\\
27	-0.0138864430513286\\
28	-0.0120287241244376\\
29	-0.00961939348590755\\
30	-0.006734249319389\\
31	-0.00355621412817398\\
32	-0.000128588667252325\\
33	-0.00587287675534814\\
34	0.00327090209092106\\
35	0.00864258415810156\\
36	0.0121637913708792\\
37	0.0145511687316726\\
38	0.0161247153277542\\
39	0.0169306672685455\\
40	0.0169823592240332\\
41	0.0163065880703763\\
42	0.0149575783751155\\
43	0.0129926447356598\\
44	0.0105703045271912\\
45	0.00780370311390972\\
46	0.00475305775928922\\
47	0.00160606640972528\\
48	-0.00140705252978039\\
49	-0.00444730221693533\\
50	-0.00733281466868337\\
51	-0.00992554388125053\\
52	-0.0121617694764022\\
53	-0.0138599246843339\\
54	-0.0150109506252375\\
55	-0.0156935010201434\\
56	-0.0158340786669363\\
57	-0.0153774908744665\\
58	-0.0143260498157908\\
59	-0.0126915705451875\\
60	-0.0104849981349864\\
61	-0.0077781340377214\\
62	-0.00470744162769902\\
63	-0.0013373423885311\\
64	0.0031415945551691\\
65	0.015077322240925\\
66	0.0251678320339809\\
67	0.0181293752500491\\
68	0.0166085206926557\\
69	0.0168762538459417\\
70	0.0173845096381947\\
71	0.0175178391720392\\
72	0.0170529883003925\\
73	0.0159414536409698\\
74	0.0142196478917142\\
75	0.0119505096649797\\
76	0.00927527382529297\\
77	0.00635898232741606\\
78	0.00321515931600653\\
79	6.53920372764409e-05\\
80	-0.00296004610566215\\
81	-0.00596138152979996\\
82	-0.00867335613900523\\
83	-0.0110667093274282\\
84	-0.0130611743439602\\
85	-0.0144205310487673\\
86	-0.0152925029110052\\
87	-0.0156755086218221\\
88	-0.0154728621954603\\
89	-0.0146644046320107\\
90	-0.0132629130555262\\
91	-0.0112728494665317\\
92	-0.00875056294627912\\
93	-0.00577158470385801\\
94	-0.00853271079258988\\
95	-0.0165801636463534\\
96	-0.0243776143717684\\
97	-0.0279082199310151\\
98	-0.0102000015405083\\
99	0.00570862908026934\\
100	0.0126656366158944\\
};
\addlegendentry{$\text{e}_{\text{1,1}}$};

\addplot [color=mycolor2,solid]
  table[row sep=crcr]{%
1	0\\
2	0.00422478621589708\\
3	-0.00754097901045493\\
4	0.0136492650587786\\
5	-0.22981441499876\\
6	-0.490879603849974\\
7	-0.319647459403459\\
8	-0.173002421826505\\
9	0.12627611703062\\
10	0.28489921967395\\
11	0.338609120931272\\
12	0.330255459711086\\
13	0.369360597381008\\
14	0.358812010042025\\
15	0.248222667063667\\
16	0.114258913738937\\
17	0.063349570308242\\
18	0.0494872586280444\\
19	0.0455636534151399\\
20	0.0402421950774835\\
21	0.033315592003909\\
22	0.0250439941414477\\
23	0.0157626963503455\\
24	0.00584426146560644\\
25	-0.00430976025881203\\
26	-0.0142864431969605\\
27	-0.0236807627373258\\
28	-0.0321130570691089\\
29	-0.0392457619777872\\
30	-0.0447968905186755\\
31	-0.0485517176798007\\
32	-0.0503703520096941\\
33	-0.0502383827206217\\
34	-0.0480052612998947\\
35	-0.0439456137934442\\
36	-0.0382010506112421\\
37	-0.0309938508083041\\
38	-0.0226103279316515\\
39	-0.0133847049982085\\
40	-0.00368193939858427\\
41	0.00611664985399249\\
42	0.0156274648554623\\
43	0.0244778982488503\\
44	0.0323212935211752\\
45	0.0388472376750679\\
46	0.0437928307277777\\
47	0.0469545715749049\\
48	0.0481953834689016\\
49	0.0474521604037717\\
50	0.0447401988895661\\
51	0.0401534065907277\\
52	0.0338634045052349\\
53	0.02611238561295\\
54	0.0172076757499682\\
55	0.00750813436356463\\
56	-0.00259488161772877\\
57	-0.0126918351759632\\
58	-0.022372445935012\\
59	-0.0312447409092841\\
60	-0.0389521818561483\\
61	-0.0451880402897691\\
62	-0.0497089695108613\\
63	-0.0523431865689977\\
64	-0.0529677761033934\\
65	-0.0514589666751307\\
66	-0.0480693518769526\\
67	-0.0431659805494844\\
68	-0.0365938980405839\\
69	-0.0286841659903726\\
70	-0.0197755727442214\\
71	-0.0102305113806745\\
72	-0.000428415489800259\\
73	0.00924528769933999\\
74	0.0184117484661285\\
75	0.0267113271236245\\
76	0.0338172270837279\\
77	0.0394464363145932\\
78	0.0433688067000501\\
79	0.0454192529930137\\
80	0.0455030002448727\\
81	0.0436028547094317\\
82	0.0397795967153731\\
83	0.0341733799513598\\
84	0.026998571229175\\
85	0.0185345284303778\\
86	0.00912217962109849\\
87	-0.000858655292052482\\
88	-0.0110045576462651\\
89	-0.0209034944957436\\
90	-0.0301538426751595\\
91	-0.0383826233680015\\
92	-0.0452607097980726\\
93	-0.0505169668563767\\
94	-0.0539362864049164\\
95	-0.0554345111903496\\
96	-0.054928799661407\\
97	-0.0524788366198617\\
98	-0.0479060086516621\\
99	-0.0416394398320769\\
100	-0.0341466433448372\\
};
\addlegendentry{$\text{e}_{\text{1,2}}$};

\addplot [color=mycolor3,solid]
  table[row sep=crcr]{%
1	0\\
2	0.00678759060302288\\
3	-0.0880606934698573\\
4	0.121222524313982\\
5	-0.872346030407891\\
6	0.135474189766058\\
7	0.26710282131364\\
8	0.164504257444214\\
9	0.540980864121299\\
10	0.166148619508435\\
11	-0.0670439194505661\\
12	0.0308582660040746\\
13	0.0412368833846525\\
14	-0.0765634297943572\\
15	-0.180816664010419\\
16	-0.212510207924936\\
17	-0.156757076689918\\
18	-0.092208595665088\\
19	-0.0427210511422329\\
20	-0.0251542172835926\\
21	-0.0196381208880462\\
22	-0.0182424822887025\\
23	-0.01796126757098\\
24	-0.0176260606291354\\
25	-0.0168296804193534\\
26	-0.0153729398866123\\
27	-0.0133314844461801\\
28	-0.0107749969111019\\
29	-0.00783008798352415\\
30	-0.00464323581281703\\
31	-0.00149849203360504\\
32	0.0018092185618809\\
33	0.0106645640710641\\
34	0.0141028461393729\\
35	0.0131074012329122\\
36	0.0139635069117685\\
37	0.0150333206807907\\
38	0.0159856883217509\\
39	0.0165514089060658\\
40	0.0166051175939844\\
41	0.0160916293591164\\
42	0.014994115848581\\
43	0.0133440191681161\\
44	0.0111176240519031\\
45	0.00838772785698593\\
46	0.00529946076610281\\
47	0.00194165143924513\\
48	-0.00149231766965602\\
49	-0.00498494115447198\\
50	-0.00829885580174794\\
51	-0.011292920551001\\
52	-0.0137945973282836\\
53	-0.0155653794010023\\
54	-0.0166704439943025\\
55	-0.0170565520378425\\
56	-0.0167032401801025\\
57	-0.0156109724456839\\
58	-0.013847282853105\\
59	-0.0115063895556343\\
60	-0.00871772186585971\\
61	-0.00561473791507784\\
62	-0.00253918146659207\\
63	0.000765035153348467\\
64	0.0141650103950762\\
65	0.0198625163317325\\
66	0.0157282451570693\\
67	0.0134044653628089\\
68	0.0142526681905273\\
69	0.0155234373849049\\
70	0.0164469340867602\\
71	0.0168179004929354\\
72	0.0165878243112637\\
73	0.0157518651078692\\
74	0.0143245277749973\\
75	0.0123434444148015\\
76	0.0098419107015773\\
77	0.0068808555203219\\
78	0.00363064170215617\\
79	0.000184427711571035\\
80	-0.00325374615006438\\
81	-0.00667384888757787\\
82	-0.00979590673951757\\
83	-0.0125007170695977\\
84	-0.0146559554783473\\
85	-0.0159807232152935\\
86	-0.0166454366413983\\
87	-0.0165795876883218\\
88	-0.0157783248374758\\
89	-0.014279444579807\\
90	-0.0121665941169727\\
91	-0.00955292387293068\\
92	-0.00656883423293793\\
93	-0.00336805263113991\\
94	0.00301669243445002\\
95	0.000579872221323097\\
96	-0.000807385865740775\\
97	0.00920697743430325\\
98	0.0249811864353229\\
99	0.0201409928877399\\
100	0.0182196973175084\\
};
\addlegendentry{$\text{e}_{\text{1,3}}$};

\end{axis}
\end{tikzpicture}%}
      \caption{The evolution of the error states of agent 1 over time.}
      \label{fig:d_ON_res_3_2_errors_agent_1}
    \end{figure}
  \end{minipage}
  \hfill
  \begin{minipage}{0.45\linewidth}
    \begin{figure}[H]
      \scalebox{0.7}{% This file was created by matlab2tikz.
%
%The latest updates can be retrieved from
%  http://www.mathworks.com/matlabcentral/fileexchange/22022-matlab2tikz-matlab2tikz
%where you can also make suggestions and rate matlab2tikz.
%
\definecolor{mycolor1}{rgb}{0.00000,1.00000,1.00000}%
\definecolor{mycolor2}{rgb}{0.00000,0.44700,0.74100}%

%
\begin{tikzpicture}

\begin{axis}[%
width=4.133in,
height=3.26in,
at={(0.693in,0.44in)},
scale only axis,
xmin=0,
xmax=100,
xmajorgrids,
ymin=0.8,
ymax=2.2,
ymajorgrids,
xlabel={time [iterations]},
axis background/.style={fill=white},
axis x line*=bottom,
axis y line*=left,
legend style={at={(0.699,0.582)},anchor=south west,legend cell align=left,align=left,draw=white!15!black}
]
\addplot [color=mycolor1,solid]
  table[row sep=crcr]{%
0	1.01\\
100	1.01\\
};
\addlegendentry{$\text{d}_{\text{max}}$};

\addplot [color=mycolor1,solid]
  table[row sep=crcr]{%
0	2.01\\
100	2.01\\
};
\addlegendentry{$\text{d}_{\text{min}}$};

\addplot [color=mycolor2,solid]
  table[row sep=crcr]{%
1	1.2\\
2	1.13028748544624\\
3	1.20843567591419\\
4	1.09973251897809\\
5	1.40168431045553\\
6	1.68380298292713\\
7	1.66084436659695\\
8	1.88360008816731\\
9	1.98829687135464\\
10	1.98999051923358\\
11	1.9937575649837\\
12	1.99767488367426\\
13	2.00314221004763\\
14	2.0088428549773\\
15	1.68157856599251\\
16	1.33841237603773\\
17	1.22725239728932\\
18	1.20035658505603\\
19	1.19875891080645\\
20	1.19870178112698\\
21	1.19864519288565\\
22	1.1986146256676\\
23	1.19860457277368\\
24	1.19860255567031\\
25	1.19860340923153\\
26	1.19860513598442\\
27	1.1986070880808\\
28	1.1986090613695\\
29	1.19861091871173\\
30	1.19861256499516\\
31	1.19861390515348\\
32	1.19861491810557\\
33	1.19860317370551\\
34	1.19865097355751\\
35	1.19867294467597\\
36	1.19867842171633\\
37	1.19867772557985\\
38	1.19867610143894\\
39	1.19867446205803\\
40	1.1986729007508\\
41	1.19867137677415\\
42	1.19866987128414\\
43	1.19866834186544\\
44	1.19866718226884\\
45	1.1986662971627\\
46	1.19866505699853\\
47	1.1986641399038\\
48	1.19866350809043\\
49	1.19866318250216\\
50	1.19866320613601\\
51	1.19866361179797\\
52	1.19866438752901\\
53	1.19866697029939\\
54	1.19866764423678\\
55	1.19866899996185\\
56	1.19867094217547\\
57	1.1986727382721\\
58	1.19867473832284\\
59	1.19867675109324\\
60	1.19867863493054\\
61	1.19868032091887\\
62	1.19868172997167\\
63	1.19868283745241\\
64	1.19870246965061\\
65	1.19874360193547\\
66	1.19876449341507\\
67	1.19871554247192\\
68	1.1987007854248\\
69	1.19869480384684\\
70	1.19869189478297\\
71	1.19869002999072\\
72	1.19868846858383\\
73	1.19868697581124\\
74	1.19868551005639\\
75	1.19868402231687\\
76	1.19868240698117\\
77	1.19868150274237\\
78	1.19868041868908\\
79	1.19867965330894\\
80	1.19867920066431\\
81	1.19867906962599\\
82	1.19867931143717\\
83	1.19867992329953\\
84	1.19868089297335\\
85	1.19868213669356\\
86	1.19868379829671\\
87	1.19868526792606\\
88	1.19868718652181\\
89	1.19868948809198\\
90	1.19869152288882\\
91	1.19869344698404\\
92	1.19869520778262\\
93	1.19869673976627\\
94	1.19872529146706\\
95	1.19879776261745\\
96	1.19907764632389\\
97	1.19947516133022\\
98	1.19916886360199\\
99	1.19892234186361\\
100	1.19895242470188\\
};
\addlegendentry{$\text{d}_{\text{13,a}}$};

\end{axis}
\end{tikzpicture}%
}
      \caption{The distance between agents 1 and 3 over time. The maximum allowed
        distance has a value of $2.01$ and the minimum allowed distance a value
        of $1.01$.}
      \label{fig:d_ON_res_3_2_distance_agents_13}
    \end{figure}
  \end{minipage}
\end{minipage}
}


\noindent\makebox[\linewidth][c]{%
\begin{minipage}{\linewidth}
  \begin{minipage}{0.45\linewidth}
    \begin{figure}[H]
      \scalebox{0.7}{% This file was created by matlab2tikz.
%
%The latest updates can be retrieved from
%  http://www.mathworks.com/matlabcentral/fileexchange/22022-matlab2tikz-matlab2tikz
%where you can also make suggestions and rate matlab2tikz.
%
\definecolor{mycolor1}{rgb}{0.00000,0.44700,0.74100}%
\definecolor{mycolor2}{rgb}{0.85000,0.32500,0.09800}%
\definecolor{mycolor3}{rgb}{0.92900,0.69400,0.12500}%
%
\begin{tikzpicture}

\begin{axis}[%
width=4.133in,
height=3.26in,
at={(0.693in,0.44in)},
scale only axis,
xmin=0,
xmax=30,
xmajorgrids,
ymin=1,
ymax=7,
ymajorgrids,
axis background/.style={fill=white},
axis x line*=bottom,
axis y line*=left,
legend style={at={(0.709,0.471)},anchor=south west,legend cell align=left,align=left,draw=white!15!black}
]
\addplot [color=mycolor1,solid]
  table[row sep=crcr]{%
1	6.18465843842649\\
2	5.22457911435046\\
3	4.70952348201006\\
4	4.25357505173589\\
5	3.29142937658314\\
6	2.29743667279803\\
7	1.60000000000095\\
8	1.60205041774449\\
9	1.6\\
10	1.59999999999789\\
11	1.88875925135479\\
12	1.88875925135478\\
13	2.35791444602961\\
14	3.11684318147756\\
15	3.98065811698705\\
16	4.86257637805357\\
17	5.67771113269129\\
18	5.99113188223004\\
19	6.11068932683868\\
20	6.15631806490324\\
21	6.17370610804389\\
22	6.18038044230932\\
23	6.1829130848151\\
24	6.18381573578702\\
25	6.18410659049785\\
26	6.18421697373347\\
27	6.18427158480716\\
28	6.18431514217932\\
29	6.18433708512913\\
30	6.18435841208529\\
};
\addlegendentry{$\text{d}_{\text{1,o}_\text{1}}$};

\addplot [color=mycolor2,solid]
  table[row sep=crcr]{%
1	6.00749531835024\\
2	5.52445563655816\\
3	5.24007197102491\\
4	4.39963668051028\\
5	3.4851248553256\\
6	2.5942018585821\\
7	2.63381415343535\\
8	2.56465394335752\\
9	1.90799672575356\\
10	1.60000000000737\\
11	2.39725683438621\\
12	1.60000000000845\\
13	1.6\\
14	2.09241868486076\\
15	2.79678301732133\\
16	3.64198294403795\\
17	4.54393179288929\\
18	5.45800051581825\\
19	5.81940590415996\\
20	5.95153682059535\\
21	5.99456439159845\\
22	6.00653901189964\\
23	6.00877092918798\\
24	6.00863831511779\\
25	6.00823012336834\\
26	6.00799108174315\\
27	6.00784057670739\\
28	6.00775182731596\\
29	6.00769237996213\\
30	6.0076371248806\\
};
\addlegendentry{$\text{d}_{\text{2,o}_\text{1}}$};

\addplot [color=mycolor3,solid]
  table[row sep=crcr]{%
1	6.57951365983839\\
2	5.68962577780603\\
3	4.841670289039\\
4	3.97820416983991\\
5	3.14167826254574\\
6	2.40270351731317\\
7	1.9053376397957\\
8	1.89352833032692\\
9	2.07589050900767\\
10	2.77601891646894\\
11	3.62882851324919\\
12	3.62882851345129\\
13	4.3356654974551\\
14	5.21106995081784\\
15	6.05989018953154\\
16	6.39886166455981\\
17	6.52112670416013\\
18	6.56240645596922\\
19	6.57520118613744\\
20	6.57875833882697\\
21	6.57944329274666\\
22	6.57941602851822\\
23	6.5793444944852\\
24	6.57929361073565\\
25	6.57925902802557\\
26	6.57920740024087\\
27	6.57917532500663\\
28	6.57915621019736\\
29	6.57913493594217\\
30	6.57912675522103\\
};
\addlegendentry{$\text{d}_{\text{3,o}_\text{1}}$};

\end{axis}
\end{tikzpicture}%}
      \caption{The distance between each agent and obstacle 1 over time. The
        minimum allowed distance has a value of $1.51$.}
      \label{fig:d_ON_res_3_2_distance_obstacle_1_agents}
    \end{figure}
  \end{minipage}
  \hfill
  \begin{minipage}{0.45\linewidth}
    \begin{figure}[H]
      \scalebox{0.7}{% This file was created by matlab2tikz.
%
%The latest updates can be retrieved from
%  http://www.mathworks.com/matlabcentral/fileexchange/22022-matlab2tikz-matlab2tikz
%where you can also make suggestions and rate matlab2tikz.
%
\definecolor{mycolor1}{rgb}{0.00000,1.00000,1.00000}%
\definecolor{mycolor2}{rgb}{0.00000,0.44700,0.74100}%
\definecolor{mycolor3}{rgb}{0.85000,0.32500,0.09800}%
%
\begin{tikzpicture}

\begin{axis}[%
width=4.133in,
height=3.26in,
at={(0.693in,0.44in)},
scale only axis,
xmin=0,
xmax=30,
xmajorgrids,
ymin=-11,
ymax=11,
ymajorgrids,
axis background/.style={fill=white},
axis x line*=bottom,
axis y line*=left,
legend style={at={(0.701,0.649)},anchor=south west,legend cell align=left,align=left,draw=white!15!black}
]
\addplot [color=mycolor1,solid]
  table[row sep=crcr]{%
0	-10\\
100	-10\\
};
\addlegendentry{$\text{u}_{\text{max}}$};

\addplot [color=mycolor1,solid]
  table[row sep=crcr]{%
0	10\\
100	10\\
};
\addlegendentry{$\text{u}_{\text{min}}$};

\addplot [color=mycolor2,solid]
  table[row sep=crcr]{%
1	10\\
2	-1.11698496783172\\
3	10\\
4	-10\\
5	-10\\
6	-10\\
7	-10\\
8	8.53187332196045\\
9	10\\
10	10\\
11	10\\
12	10\\
13	10\\
14	10\\
15	10\\
16	5.35405695040881\\
17	2.03803908644785\\
18	0.773118734135391\\
19	0.309600337835671\\
20	0.150792271409178\\
21	0.104863926480301\\
22	0.0951384055027615\\
23	0.0972825753980916\\
24	0.102835243163086\\
25	0.105913095954429\\
26	0.105756484479186\\
27	0.101923023068663\\
28	0.0946652282718316\\
29	0.0837953987351282\\
30	0.0693070055058716\\
31	0.0525375276163475\\
32	0.0120590311151963\\
33	0.0936520830543511\\
34	-0.0437216850707346\\
35	-0.0455769414720293\\
36	-0.0578011316334126\\
37	-0.0725871620054869\\
38	-0.0868845847932372\\
39	-0.0986306309554369\\
40	-0.106682545543668\\
41	-0.110407617150959\\
42	-0.109774529716338\\
43	-0.103958361270894\\
44	-0.094058043682\\
45	-0.0804546403914296\\
46	-0.0633811507876395\\
47	-0.0426316765796642\\
48	-0.0229769587677308\\
49	-0.00179213593139588\\
50	0.0196803034848826\\
51	0.0399821577304565\\
52	0.0586234899209811\\
53	0.0767077436425035\\
54	0.0889523085926679\\
55	0.0979502338734428\\
56	0.103893305650278\\
57	0.10590599252423\\
58	0.103865141519754\\
59	0.0982208839035373\\
60	0.0888503428664486\\
61	0.0756780723868683\\
62	0.0184476240778992\\
63	0.134694257432473\\
64	0.13889535862263\\
65	0.0544060228994914\\
66	-0.159223374512263\\
67	-0.0962367034978447\\
68	-0.0828255091129714\\
69	-0.0873062029862128\\
70	-0.0964307040062912\\
71	-0.104694178100559\\
72	-0.109749344555067\\
73	-0.110615237739052\\
74	-0.107184012312053\\
75	-0.0989832765152654\\
76	-0.0865786121858721\\
77	-0.070834921713688\\
78	-0.0520933328463304\\
79	-0.0310807596151177\\
80	-0.0110055176486294\\
81	0.0109717044301393\\
82	0.0317297727479661\\
83	0.0510732807499682\\
84	0.0699935716049536\\
85	0.0839921086967478\\
86	0.0945141836157436\\
87	0.101899369020163\\
88	0.10544063724556\\
89	0.105081569681172\\
90	0.101089894259007\\
91	0.0932743248177505\\
92	0.0819799815034226\\
93	0.0552567422249215\\
94	0.145982102154696\\
95	0.183847400890766\\
96	0.124306090893118\\
97	-0.0577246390431225\\
98	-0.227467916160482\\
99	-0.127309278220662\\
100	-0.0998393576815686\\
};
\addlegendentry{$\text{u}_{\text{2,1}}$};

\addplot [color=mycolor3,solid]
  table[row sep=crcr]{%
1	3.38023866112825\\
2	-5.29847634183979\\
3	-10\\
4	-10\\
5	-6.98017202260544\\
6	0.76192353097952\\
7	10\\
8	10\\
9	10\\
10	3.85239160585693\\
11	-4.97442652576898\\
12	0.155865503777596\\
13	-0.513942213115992\\
14	-0.646772308632892\\
15	-0.667270068544616\\
16	-0.300626637084903\\
17	-0.000293537479333174\\
18	0.100349772363056\\
19	0.0733323249977434\\
20	0.0681694732753595\\
21	0.0758859510728698\\
22	0.0869566033808804\\
23	0.0983350272618868\\
24	0.107010592829422\\
25	0.111957132516482\\
26	0.112551831948598\\
27	0.108530392796933\\
28	0.0998162433054866\\
29	0.0869303456876543\\
30	0.0686483918628535\\
31	0.0511629254893694\\
32	0.116783251966676\\
33	0.0357732656923007\\
34	-0.0892245552253088\\
35	-0.0601801201598081\\
36	-0.0649321877732574\\
37	-0.0759460004338628\\
38	-0.088035182464507\\
39	-0.0980876910301824\\
40	-0.104842769319085\\
41	-0.107793065803301\\
42	-0.106493139652892\\
43	-0.102109381808291\\
44	-0.0932642564003293\\
45	-0.0809043563681191\\
46	-0.065414100085624\\
47	-0.0467158584203504\\
48	-0.0273693133026658\\
49	-0.00595986679860676\\
50	0.0157919387018909\\
51	0.0374329011708818\\
52	0.0576354999593171\\
53	0.0776626924123473\\
54	0.0920801919830516\\
55	0.103161266791214\\
56	0.110241327216895\\
57	0.112861728227164\\
58	0.110763899734663\\
59	0.103927135206697\\
60	0.0927079100299223\\
61	0.0755919763011571\\
62	0.160788983955144\\
63	0.110775632127244\\
64	-0.0107989884161842\\
65	-0.112923376223662\\
66	-0.0649726489154527\\
67	-0.0522893683798504\\
68	-0.0643379456054456\\
69	-0.0797026543453072\\
70	-0.0926697532470037\\
71	-0.101792232970624\\
72	-0.106763381659852\\
73	-0.107561970539694\\
74	-0.104171711677569\\
75	-0.0971063781092495\\
76	-0.086508765583983\\
77	-0.0721741182000863\\
78	-0.0550792866640416\\
79	-0.0351475446670153\\
80	-0.0151111813567247\\
81	0.0069668140091695\\
82	0.0287051947007019\\
83	0.049546996057424\\
84	0.0703733349321608\\
85	0.0861287893910111\\
86	0.0988327214703555\\
87	0.107869268744345\\
88	0.112418834504281\\
89	0.112280593875127\\
90	0.10732865880877\\
91	0.0978512778078907\\
92	0.0841299947922857\\
93	0.175831019336828\\
94	0.137252987871735\\
95	0.0440316120827406\\
96	-0.127215452213623\\
97	-0.202435083246638\\
98	0.00106309230299226\\
99	-0.030096181734152\\
100	-0.0601982550661684\\
};
\addlegendentry{$\text{u}_{\text{2,2}}$};

\end{axis}
\end{tikzpicture}%}
      \caption{The inputs signals directing agent 2 over time. Their value is
        constrained between $-10$ and $10$.}
      \label{fig:d_ON_res_3_2_inputs_agent_2}
    \end{figure}
  \end{minipage}
\end{minipage}
}



\noindent\makebox[\linewidth][c]{%
\begin{minipage}{\linewidth}
  \begin{minipage}{0.45\linewidth}
    \begin{figure}[H]
      \scalebox{0.7}{% This file was created by matlab2tikz.
%
%The latest updates can be retrieved from
%  http://www.mathworks.com/matlabcentral/fileexchange/22022-matlab2tikz-matlab2tikz
%where you can also make suggestions and rate matlab2tikz.
%
\definecolor{mycolor1}{rgb}{0.00000,0.44700,0.74100}%
\definecolor{mycolor2}{rgb}{0.85000,0.32500,0.09800}%
\definecolor{mycolor3}{rgb}{0.92900,0.69400,0.12500}%
%
\begin{tikzpicture}

\begin{axis}[%
width=4.133in,
height=3.26in,
at={(0.693in,0.44in)},
scale only axis,
xmin=0,
xmax=100,
xmajorgrids,
ymin=0,
ymax=60,
ymajorgrids,
axis background/.style={fill=white},
legend style={legend cell align=left,align=left,draw=white!15!black}
]
\addplot [color=mycolor1,solid]
  table[row sep=crcr]{%
1	52.9128\\
2	44.4413598839013\\
3	41.6745180421666\\
4	33.9640090751648\\
5	31.0593970058162\\
6	25.5715220314916\\
7	20.6319043552943\\
8	17.2385813621321\\
9	13.1650095982034\\
10	11.4921051756317\\
11	8.18725275076584\\
12	5.19212791743271\\
13	3.04720585725706\\
14	1.40842192534422\\
15	0.443698384288822\\
16	0.0759503636727551\\
17	0.0144214746142258\\
18	0.0035926055740935\\
19	0.00123710725335536\\
20	0.000733702454757136\\
21	0.000514952416607438\\
22	0.000367791373651735\\
23	0.000271399319539843\\
24	0.000229200985064239\\
25	0.000244561816227874\\
26	0.000312886729232865\\
27	0.000423427158022945\\
28	0.000558179474558724\\
29	0.000695344358193898\\
30	0.000813243583147557\\
31	0.000894911456159609\\
32	0.000924717074604925\\
33	0.000946389561941187\\
34	0.000845632741235276\\
35	0.000719899480980441\\
36	0.000584873612683953\\
37	0.000449820265621871\\
38	0.000335791387873015\\
39	0.000259229380583564\\
40	0.000230767453192554\\
41	0.000253646832848859\\
42	0.000323245423311109\\
43	0.000427730694999289\\
44	0.000549878422978745\\
45	0.000669878565569208\\
46	0.000768146118282285\\
47	0.000828782092879421\\
48	0.000841936589241644\\
49	0.000803335449300603\\
50	0.000720164534252415\\
51	0.000605952325839716\\
52	0.000479494197926574\\
53	0.000359782878058089\\
54	0.000268425414710837\\
55	0.000223342450905076\\
56	0.000233572425050663\\
57	0.000298900641781438\\
58	0.000411078760308694\\
59	0.000554023736844824\\
60	0.000706456891736737\\
61	0.000845811513507443\\
62	0.000953331872814088\\
63	0.00100958981881427\\
64	0.00102159601005515\\
65	0.00106059234693717\\
66	0.0010310270237847\\
67	0.000769258906347425\\
68	0.000586448604694919\\
69	0.00043535781826679\\
70	0.000321211228889622\\
71	0.00025335049027926\\
72	0.000237023777677409\\
73	0.000270828472399666\\
74	0.000346222772685551\\
75	0.000448387052460131\\
76	0.000558966213467547\\
77	0.000658628867514487\\
78	0.00072943482452567\\
79	0.000759150287196417\\
80	0.000741918033116033\\
81	0.000679573674855217\\
82	0.000583152843379203\\
83	0.000469293803916347\\
84	0.000358387531001687\\
85	0.000267319639746587\\
86	0.000217793260355931\\
87	0.00022211139397744\\
88	0.000282953726691224\\
89	0.000394583540579935\\
90	0.000543300723767179\\
91	0.000708840306220322\\
92	0.00086828206787475\\
93	0.000999365114661427\\
94	0.00112028345377191\\
95	0.00130714175302668\\
96	0.00145214994392889\\
97	0.00138162249224267\\
98	0.00100419868249869\\
99	0.00070369350745032\\
100	0.000532306348855202\\
};
\addlegendentry{$\text{V}_\text{1}$};

\addplot [color=mycolor2,solid]
  table[row sep=crcr]{%
1	57.572352\\
2	48.171279245189\\
3	49.7947544530147\\
4	47.3145748994151\\
5	46.2921764339438\\
6	40.9156060282624\\
7	33.3153843652366\\
8	25.6504354616224\\
9	23.2909595741337\\
10	17.5174904459273\\
11	12.6045464374563\\
12	8.8181171770105\\
13	5.5115505283895\\
14	3.0052699807577\\
15	1.26413811798729\\
16	0.270804423765105\\
17	0.0401261702468183\\
18	0.0066883994654579\\
19	0.0017030505020553\\
20	0.000822316784537832\\
21	0.000535147844769747\\
22	0.00036850588902887\\
23	0.000268039295280204\\
24	0.000229335679894953\\
25	0.000250465555078682\\
26	0.00032655842527787\\
27	0.000444900564307091\\
28	0.000586823943784439\\
29	0.000730105167346825\\
30	0.000852767506762753\\
31	0.000937623300374297\\
32	0.000969030905453581\\
33	0.000965340131965908\\
34	0.000960614513072175\\
35	0.000787779341853085\\
36	0.000632716500656409\\
37	0.000483915814688716\\
38	0.000359038880261255\\
39	0.000272796256892236\\
40	0.000235057556876056\\
41	0.000248891411578046\\
42	0.000309808398056537\\
43	0.000406320167724128\\
44	0.000521171082979954\\
45	0.000635287576463257\\
46	0.000729116835760699\\
47	0.000786843487872771\\
48	0.000798857053131917\\
49	0.000760934833988786\\
50	0.000680245591771834\\
51	0.00057022690871321\\
52	0.000449511616831595\\
53	0.000339207090269581\\
54	0.000254811340089415\\
55	0.000217731573261764\\
56	0.000236809623804479\\
57	0.00031115619567377\\
58	0.000431851442802973\\
59	0.000582634475318867\\
60	0.000741876057229694\\
61	0.000886735251270507\\
62	0.000998241831369988\\
63	0.00106902696114706\\
64	0.00112565609919827\\
65	0.00120278115189607\\
66	0.00115564451115705\\
67	0.000818444004515037\\
68	0.000617346559676949\\
69	0.000457429096396688\\
70	0.000336107706061064\\
71	0.000261197379085472\\
72	0.000237747312834089\\
73	0.000264515481456334\\
74	0.000333219392256845\\
75	0.000429351120992285\\
76	0.000534748095149323\\
77	0.000630037923091954\\
78	0.000698100715275197\\
79	0.000726322336345718\\
80	0.000709035177782059\\
81	0.000648095772361021\\
82	0.000554495725688948\\
83	0.000444766956445951\\
84	0.000339135320648681\\
85	0.00025421226620857\\
86	0.000211603039375113\\
87	0.0002231701475621\\
88	0.000291246611645329\\
89	0.000410076300068827\\
90	0.000565472861100239\\
91	0.00073691445002954\\
92	0.000901265444701774\\
93	0.00103609005594982\\
94	0.00110891525154309\\
95	0.0012333254728907\\
96	0.00149466658260435\\
97	0.00160356015054543\\
98	0.00131917728179443\\
99	0.000805108779019294\\
100	0.00057142779522082\\
};
\addlegendentry{$\text{V}_\text{2}$};

\addplot [color=mycolor3,solid]
  table[row sep=crcr]{%
1	52.486272\\
2	44.3435567972138\\
3	36.7218313149414\\
4	29.8737971383113\\
5	23.7530878314778\\
6	18.2797598730565\\
7	13.5438908816977\\
8	9.53522287695978\\
9	6.3523512370318\\
10	5.16793545224434\\
11	3.11668773081895\\
12	1.55039043706743\\
13	0.655019777015209\\
14	0.150612117987905\\
15	0.0345509016674889\\
16	0.0107861644524041\\
17	0.00483949069583048\\
18	0.00259953894011636\\
19	0.00104470903724956\\
20	0.000616175721011867\\
21	0.000476510344469001\\
22	0.000364800061605478\\
23	0.000277169117946632\\
24	0.000231923632836642\\
25	0.000239597579635107\\
26	0.000298836916288409\\
27	0.000400388659086588\\
28	0.000526961009947964\\
29	0.000657178798532775\\
30	0.00076966471889553\\
31	0.00084767592951687\\
32	0.00087570130902585\\
33	0.00085140807602185\\
34	0.000779543346957451\\
35	0.000671988968258168\\
36	0.000544597482658061\\
37	0.000418527464455005\\
38	0.00031368747100156\\
39	0.000246523751516229\\
40	0.000227583096289655\\
41	0.000259921216844398\\
42	0.000338647139286787\\
43	0.000451536794952511\\
44	0.000581191079623629\\
45	0.000707330318480821\\
46	0.000810509535120737\\
47	0.000874179964984611\\
48	0.000888533786664781\\
49	0.000849218460655188\\
50	0.000763434221437657\\
51	0.000644805362695757\\
52	0.000512297589319758\\
53	0.000384148949172229\\
54	0.000284851549443644\\
55	0.000230863384332629\\
56	0.000231642562591393\\
57	0.000287473620422982\\
58	0.000390538925871489\\
59	0.000525176237792974\\
60	0.000670419201734084\\
61	0.000803974780115802\\
62	0.000907298336353696\\
63	0.000961109217424731\\
64	0.000969330363379791\\
65	0.000979207338861891\\
66	0.000920395236040233\\
67	0.000714186664684369\\
68	0.000547835865253144\\
69	0.000407335837054409\\
70	0.00030263173016037\\
71	0.000244088856360809\\
72	0.000237084043667419\\
73	0.000280029490741836\\
74	0.000364080223116133\\
75	0.000474045668000247\\
76	0.000591362733507762\\
77	0.000696233265462483\\
78	0.000770892713202399\\
79	0.000802565480912774\\
80	0.000785415877702403\\
81	0.000721262811524065\\
82	0.000621203705387746\\
83	0.000502016765777613\\
84	0.000384307655399966\\
85	0.000285319654975102\\
86	0.000226877462025422\\
87	0.000221986193160167\\
88	0.000273317701185412\\
89	0.000375648413855495\\
90	0.000515772566718268\\
91	0.000673725108519754\\
92	0.000826843745504656\\
93	0.000953103832687054\\
94	0.0010356985471515\\
95	0.00106581501342765\\
96	0.00114465855988911\\
97	0.00116993214558422\\
98	0.00103521241240618\\
99	0.000714300491158593\\
100	0.000522620635482958\\
};
\addlegendentry{$\text{V}_\text{3}$};

\end{axis}
\end{tikzpicture}%}
      \caption{The $\mat{P}-$norms of the errors of the three agents through time.}
      \label{fig:d_ON_res_3_2_V}
    \end{figure}
  \end{minipage}
  \hfill
  \begin{minipage}{0.45\linewidth}
    \begin{figure}[H]
      \scalebox{0.7}{% This file was created by matlab2tikz.
%
%The latest updates can be retrieved from
%  http://www.mathworks.com/matlabcentral/fileexchange/22022-matlab2tikz-matlab2tikz
%where you can also make suggestions and rate matlab2tikz.
%
\definecolor{mycolor1}{rgb}{0.00000,0.44700,0.74100}%
\definecolor{mycolor2}{rgb}{0.85000,0.32500,0.09800}%
\definecolor{mycolor3}{rgb}{0.92900,0.69400,0.12500}%
\definecolor{mycolor4}{rgb}{1.00000,0.00000,1.00000}%
\definecolor{mycolor5}{rgb}{0.00000,1.00000,1.00000}%
%
\begin{tikzpicture}

\begin{axis}[%
width=4.133in,
height=3.26in,
at={(0.693in,0.44in)},
scale only axis,
xmin=1,
xmax=100,
xmajorgrids,
ymin=0,
ymax=0.075,
restrict y to domain=0:1,
ymajorgrids,
xlabel={time [iterations]},
axis background/.style={fill=white},
legend style={at={(0.691,0.511)},anchor=south west,legend cell align=left,align=left,draw=white!15!black}
]
\addplot [color=mycolor1,solid]
  table[row sep=crcr]{%
1	52.9128\\
2	44.4413598839013\\
3	41.6745180421666\\
4	33.9640090751648\\
5	31.0593970058162\\
6	25.5715220314916\\
7	20.6319043552943\\
8	17.2385813621321\\
9	13.1650095982034\\
10	11.4921051756317\\
11	8.18725275076584\\
12	5.19212791743271\\
13	3.04720585725706\\
14	1.40842192534422\\
15	0.443698384288822\\
16	0.0759503636727551\\
17	0.0144214746142258\\
18	0.0035926055740935\\
19	0.00123710725335536\\
20	0.000733702454757136\\
21	0.000514952416607438\\
22	0.000367791373651735\\
23	0.000271399319539843\\
24	0.000229200985064239\\
25	0.000244561816227874\\
26	0.000312886729232865\\
27	0.000423427158022945\\
28	0.000558179474558724\\
29	0.000695344358193898\\
30	0.000813243583147557\\
31	0.000894911456159609\\
32	0.000924717074604925\\
33	0.000946389561941187\\
34	0.000845632741235276\\
35	0.000719899480980441\\
36	0.000584873612683953\\
37	0.000449820265621871\\
38	0.000335791387873015\\
39	0.000259229380583564\\
40	0.000230767453192554\\
41	0.000253646832848859\\
42	0.000323245423311109\\
43	0.000427730694999289\\
44	0.000549878422978745\\
45	0.000669878565569208\\
46	0.000768146118282285\\
47	0.000828782092879421\\
48	0.000841936589241644\\
49	0.000803335449300603\\
50	0.000720164534252415\\
51	0.000605952325839716\\
52	0.000479494197926574\\
53	0.000359782878058089\\
54	0.000268425414710837\\
55	0.000223342450905076\\
56	0.000233572425050663\\
57	0.000298900641781438\\
58	0.000411078760308694\\
59	0.000554023736844824\\
60	0.000706456891736737\\
61	0.000845811513507443\\
62	0.000953331872814088\\
63	0.00100958981881427\\
64	0.00102159601005515\\
65	0.00106059234693717\\
66	0.0010310270237847\\
67	0.000769258906347425\\
68	0.000586448604694919\\
69	0.00043535781826679\\
70	0.000321211228889622\\
71	0.00025335049027926\\
72	0.000237023777677409\\
73	0.000270828472399666\\
74	0.000346222772685551\\
75	0.000448387052460131\\
76	0.000558966213467547\\
77	0.000658628867514487\\
78	0.00072943482452567\\
79	0.000759150287196417\\
80	0.000741918033116033\\
81	0.000679573674855217\\
82	0.000583152843379203\\
83	0.000469293803916347\\
84	0.000358387531001687\\
85	0.000267319639746587\\
86	0.000217793260355931\\
87	0.00022211139397744\\
88	0.000282953726691224\\
89	0.000394583540579935\\
90	0.000543300723767179\\
91	0.000708840306220322\\
92	0.00086828206787475\\
93	0.000999365114661427\\
94	0.00112028345377191\\
95	0.00130714175302668\\
96	0.00145214994392889\\
97	0.00138162249224267\\
98	0.00100419868249869\\
99	0.00070369350745032\\
100	0.000532306348855202\\
};
\addlegendentry{$\text{V}_\text{1}$};

\addplot [color=mycolor2,solid]
  table[row sep=crcr]{%
1	57.572352\\
2	48.171279245189\\
3	49.7947544530147\\
4	47.3145748994151\\
5	46.2921764339438\\
6	40.9156060282624\\
7	33.3153843652366\\
8	25.6504354616224\\
9	23.2909595741337\\
10	17.5174904459273\\
11	12.6045464374563\\
12	8.8181171770105\\
13	5.5115505283895\\
14	3.0052699807577\\
15	1.26413811798729\\
16	0.270804423765105\\
17	0.0401261702468183\\
18	0.0066883994654579\\
19	0.0017030505020553\\
20	0.000822316784537832\\
21	0.000535147844769747\\
22	0.00036850588902887\\
23	0.000268039295280204\\
24	0.000229335679894953\\
25	0.000250465555078682\\
26	0.00032655842527787\\
27	0.000444900564307091\\
28	0.000586823943784439\\
29	0.000730105167346825\\
30	0.000852767506762753\\
31	0.000937623300374297\\
32	0.000969030905453581\\
33	0.000965340131965908\\
34	0.000960614513072175\\
35	0.000787779341853085\\
36	0.000632716500656409\\
37	0.000483915814688716\\
38	0.000359038880261255\\
39	0.000272796256892236\\
40	0.000235057556876056\\
41	0.000248891411578046\\
42	0.000309808398056537\\
43	0.000406320167724128\\
44	0.000521171082979954\\
45	0.000635287576463257\\
46	0.000729116835760699\\
47	0.000786843487872771\\
48	0.000798857053131917\\
49	0.000760934833988786\\
50	0.000680245591771834\\
51	0.00057022690871321\\
52	0.000449511616831595\\
53	0.000339207090269581\\
54	0.000254811340089415\\
55	0.000217731573261764\\
56	0.000236809623804479\\
57	0.00031115619567377\\
58	0.000431851442802973\\
59	0.000582634475318867\\
60	0.000741876057229694\\
61	0.000886735251270507\\
62	0.000998241831369988\\
63	0.00106902696114706\\
64	0.00112565609919827\\
65	0.00120278115189607\\
66	0.00115564451115705\\
67	0.000818444004515037\\
68	0.000617346559676949\\
69	0.000457429096396688\\
70	0.000336107706061064\\
71	0.000261197379085472\\
72	0.000237747312834089\\
73	0.000264515481456334\\
74	0.000333219392256845\\
75	0.000429351120992285\\
76	0.000534748095149323\\
77	0.000630037923091954\\
78	0.000698100715275197\\
79	0.000726322336345718\\
80	0.000709035177782059\\
81	0.000648095772361021\\
82	0.000554495725688948\\
83	0.000444766956445951\\
84	0.000339135320648681\\
85	0.00025421226620857\\
86	0.000211603039375113\\
87	0.0002231701475621\\
88	0.000291246611645329\\
89	0.000410076300068827\\
90	0.000565472861100239\\
91	0.00073691445002954\\
92	0.000901265444701774\\
93	0.00103609005594982\\
94	0.00110891525154309\\
95	0.0012333254728907\\
96	0.00149466658260435\\
97	0.00160356015054543\\
98	0.00131917728179443\\
99	0.000805108779019294\\
100	0.00057142779522082\\
};
\addlegendentry{$\text{V}_\text{2}$};

\addplot [color=mycolor3,solid]
  table[row sep=crcr]{%
1	52.486272\\
2	44.3435567972138\\
3	36.7218313149414\\
4	29.8737971383113\\
5	23.7530878314778\\
6	18.2797598730565\\
7	13.5438908816977\\
8	9.53522287695978\\
9	6.3523512370318\\
10	5.16793545224434\\
11	3.11668773081895\\
12	1.55039043706743\\
13	0.655019777015209\\
14	0.150612117987905\\
15	0.0345509016674889\\
16	0.0107861644524041\\
17	0.00483949069583048\\
18	0.00259953894011636\\
19	0.00104470903724956\\
20	0.000616175721011867\\
21	0.000476510344469001\\
22	0.000364800061605478\\
23	0.000277169117946632\\
24	0.000231923632836642\\
25	0.000239597579635107\\
26	0.000298836916288409\\
27	0.000400388659086588\\
28	0.000526961009947964\\
29	0.000657178798532775\\
30	0.00076966471889553\\
31	0.00084767592951687\\
32	0.00087570130902585\\
33	0.00085140807602185\\
34	0.000779543346957451\\
35	0.000671988968258168\\
36	0.000544597482658061\\
37	0.000418527464455005\\
38	0.00031368747100156\\
39	0.000246523751516229\\
40	0.000227583096289655\\
41	0.000259921216844398\\
42	0.000338647139286787\\
43	0.000451536794952511\\
44	0.000581191079623629\\
45	0.000707330318480821\\
46	0.000810509535120737\\
47	0.000874179964984611\\
48	0.000888533786664781\\
49	0.000849218460655188\\
50	0.000763434221437657\\
51	0.000644805362695757\\
52	0.000512297589319758\\
53	0.000384148949172229\\
54	0.000284851549443644\\
55	0.000230863384332629\\
56	0.000231642562591393\\
57	0.000287473620422982\\
58	0.000390538925871489\\
59	0.000525176237792974\\
60	0.000670419201734084\\
61	0.000803974780115802\\
62	0.000907298336353696\\
63	0.000961109217424731\\
64	0.000969330363379791\\
65	0.000979207338861891\\
66	0.000920395236040233\\
67	0.000714186664684369\\
68	0.000547835865253144\\
69	0.000407335837054409\\
70	0.00030263173016037\\
71	0.000244088856360809\\
72	0.000237084043667419\\
73	0.000280029490741836\\
74	0.000364080223116133\\
75	0.000474045668000247\\
76	0.000591362733507762\\
77	0.000696233265462483\\
78	0.000770892713202399\\
79	0.000802565480912774\\
80	0.000785415877702403\\
81	0.000721262811524065\\
82	0.000621203705387746\\
83	0.000502016765777613\\
84	0.000384307655399966\\
85	0.000285319654975102\\
86	0.000226877462025422\\
87	0.000221986193160167\\
88	0.000273317701185412\\
89	0.000375648413855495\\
90	0.000515772566718268\\
91	0.000673725108519754\\
92	0.000826843745504656\\
93	0.000953103832687054\\
94	0.0010356985471515\\
95	0.00106581501342765\\
96	0.00114465855988911\\
97	0.00116993214558422\\
98	0.00103521241240618\\
99	0.000714300491158593\\
100	0.000522620635482958\\
};
\addlegendentry{$\text{V}_\text{3}$};

\addplot [color=mycolor4,solid]
  table[row sep=crcr]{%
0	0.0654\\
100	0.0654\\
};
\addlegendentry{$\varepsilon_{\Psi}$};

\addplot [color=mycolor5,solid]
  table[row sep=crcr]{%
0	0.0035\\
100	0.0035\\
};
\addlegendentry{$\varepsilon_{\Omega}$};

\end{axis}
\end{tikzpicture}%
}
      \caption{The $\mat{P}-$norms of the errors of the three agents through time,
        focused. The colour magenta is used to illustrate the threshold
        $\mathcal{E}_{\Psi}$, while cyan is used for $\mathcal{E}_{\Omega}$.}
      \label{fig:d_ON_res_3_2_V_zoom}
    \end{figure}
  \end{minipage}
\end{minipage}
}
