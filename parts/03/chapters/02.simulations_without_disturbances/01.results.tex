\section{Simulation results}

The initial configurations of the three agents are
$\vect{z}_1$ $=$ $[-6, 3.5, 0]^{\top}$,
$\vect{z}_2$ $=$ $[-6, 2.3, 0]^{\top}$ and
$\vect{z}_3$ $=$ $[-6, 4.7, 0]^{\top}$.
Their desired configurations in steady-state are
$\vect{z}_{1,des}$ $=$ $[6, 3.5, 0]^{\top}$,
$\vect{z}_{2,des}$ $=$ $[6, 2.3, 0]^{\top}$ and
$\vect{z}_{3,des}$ $=$ $[6, 4.7, 0]^{\top}$.
Obstacles $o_1$ and $o_2$ are placed between the two at $[0, 2.0]^{\top}$
and $[0, 5.5]^{\top}$ respectively. The penalty
matrices $\mat{Q}$, $\mat{R}$, $\mat{P}$ were set to
$\mat{Q} = 0.5 (I_3 + 0.05\dagger_3)$, $\mat{R} = 0.005 I_2$ and
$\mat{P} = 0.5 (I_3 + 0.05\dagger_3)$, where $\dagger_N$ is a $N \times N$
matrix whose elements are chosen at random between the values $0.0$ and $1.0$.
The sampling time is $h = 0.1$ sec, the time-horizon is $T_p = 0.5$ sec, and
the total execution time given was $3$ sec.

Frames of the evolution of the trajectories of the three agents in the $x-y$
plane are depicted in figure \eqref{fig:d_OFF_res_trajectory_3_2}. Here, the
compound system avoids the pitfall of coming to a dead-end by having agent 1
act as a kind of mediator between the agents at the extremes as regards their
trajectories: a too strict terminal penalty matrix $\mat{P}$, or an insufficient
time-horizon length in relation to the maximum allowed input values, or a
strongly diagonal structure for the penalty matrix $\mat{Q}$ could result in
a situation where agent 1 is trapped between the two obstacles, with agents 2
and 3 following behind him, halted by the insufficient space between the two
obstacles, and the geometry of the compound system resembling an isosceles
triangle.

Once agent 3 clears the narrow, it is held up by agent 1, since their
maximum allowed distance is tightly constrained. Agent 1 in turn has to make
sure that its distance to agent 2 is within the allowed bounds as well. Figure
\eqref{fig:d_OFF_res_3_2_distance_agents_13} shows the evolution of the
distance between agents 1 and 3 through time, and their abiding by the
maximum-allowed-distance-between-agents constraint. The minimum and maximum
allowed distances between the two agents are portrayed in the
colour \textcolor{cyan}{cyan}.

Figure \eqref{fig:d_OFF_res_3_2_distance_obstacle_2_agents} shows the
evolution of the distance between all agents and obstacle $o_2$ respectively,
and, most crucially, it illustrates the fact that all agents avoid collision
with it. Figure \eqref{fig:d_OFF_res_3_2_inputs_agent_1} shows the input signals
directing agent 1 through time. The minimum allowed distance between the agents
and the obstacle, as well as the minimum and maximum allowed input sizes are
portrayed in the colour \textcolor{cyan}{cyan}.

Figure \eqref{fig:d_OFF_res_3_2_errors_agent_1} depicts the evolution of the
error states of agent 1 through time. As the three error states converge to zero,
the agent is stabilized at its desired 3D configuration.


\begin{figure}[H]
  \input{./figures/06.without_disturbance/3_2/01.trajectories/trajectory_d_OFF_3_2.tex}
  \caption{The trajectories of the three agents in the $x-y$ plane. Agent 1 is with
    blue, agent 2 with red and agent 3 with yellow. The obstacles are black.
    Mark O denotes equilibrium configurations. Mark X marks desired configurations.}
  \label{fig:d_OFF_res_trajectory_3_2}
\end{figure}


\noindent\makebox[\linewidth][c]{%
\begin{minipage}{\linewidth}
  \begin{minipage}{0.45\linewidth}
    \begin{figure}[H]
      \scalebox{0.6}{% This file was created by matlab2tikz.
%
%The latest updates can be retrieved from
%  http://www.mathworks.com/matlabcentral/fileexchange/22022-matlab2tikz-matlab2tikz
%where you can also make suggestions and rate matlab2tikz.
%
\definecolor{mycolor1}{rgb}{0.00000,0.44700,0.74100}%
\definecolor{mycolor2}{rgb}{0.85000,0.32500,0.09800}%
\definecolor{mycolor3}{rgb}{0.92900,0.69400,0.12500}%
%
\begin{tikzpicture}

\begin{axis}[%
width=4.133in,
height=3.26in,
at={(0.693in,0.44in)},
scale only axis,
xmin=1,
xmax=30,
xmajorgrids,
ymin=-2,
ymax=1,
ymajorgrids,
xlabel={time [iterations]},
ylabel={component magnitude},
axis background/.style={fill=white},
legend style={legend cell align=left,align=left,draw=white!15!black}
]
\addplot [color=mycolor1,solid]
  table[row sep=crcr]{%
1	-12\\
2	-11.0001469447164\\
3	-10.4808387608309\\
4	-10.0065019668059\\
5	-9.0518190777714\\
6	-8.1429528224788\\
7	-7.32010620073896\\
8	-7.32563622749479\\
9	-7.00899613622404\\
10	-6.09531541589434\\
11	-5.09744679731788\\
12	-5.09744679731789\\
13	-4.29792567276087\\
14	-3.30067831532077\\
15	-2.31447961716055\\
16	-1.36978599329091\\
17	-0.521971137084024\\
18	-0.199030596012912\\
19	-0.0759440245990856\\
20	-0.0289514459189524\\
21	-0.0110371504893198\\
22	-0.00415926950700049\\
23	-0.00154905862353236\\
24	-0.000618710447507809\\
25	-0.000318924496327908\\
26	-0.000205150514504533\\
27	-0.000148861573939523\\
28	-0.000103965820983816\\
29	-8.13486002767745e-05\\
30	-5.9366284362472e-05\\
};
\addlegendentry{$\text{e}_\text{1,1}$};

\addplot [color=mycolor2,solid]
  table[row sep=crcr]{%
1	0\\
2	0.0148456863159429\\
3	-0.0502774654990725\\
4	-0.0714193369884602\\
5	-0.267154649769474\\
6	-0.671767866991735\\
7	-0.595942690548011\\
8	-0.60041446579624\\
9	-0.258256549409275\\
10	0.0971584052579576\\
11	0.159159192454583\\
12	0.159159192454582\\
13	0.131779249576538\\
14	0.0583240550487719\\
15	0.00418700699409481\\
16	-0.0148174238567389\\
17	-0.00748417651736422\\
18	-0.00253039954761965\\
19	-0.00130561839418295\\
20	-0.00105661178930735\\
21	-0.00100997478989251\\
22	-0.00100155228873798\\
23	-0.00100016068127349\\
24	-0.000999934378020436\\
25	-0.000999894969280664\\
26	-0.000999885823744575\\
27	-0.00099988279495086\\
28	-0.000999880976900566\\
29	-0.000999880223767495\\
30	-0.000999879580368125\\
};
\addlegendentry{$\text{e}_\text{1,2}$};

\addplot [color=mycolor3,solid]
  table[row sep=crcr]{%
1	0\\
2	0.0296935543194055\\
3	-0.279198422106711\\
4	0.190114513027465\\
5	-0.594562612119951\\
6	-0.243121764350111\\
7	0.426902400734066\\
8	0.933066156033846\\
9	0.715159290747244\\
10	0.0268032940088466\\
11	0.0973035972350047\\
12	0.00856149107769032\\
13	-0.0770255911312446\\
14	-0.070024749388066\\
15	-0.0396545007785583\\
16	-0.000574134892113367\\
17	0.0178728750925563\\
18	0.0128039080806008\\
19	0.00709656876859898\\
20	0.00350103116691437\\
21	0.00170563679718075\\
22	0.000743517809036174\\
23	0.000322761809562825\\
24	0.000163729709242579\\
25	9.91828081039968e-05\\
26	6.1583931768591e-05\\
27	4.60320113189769e-05\\
28	3.49578450792456e-05\\
29	3.16403528972278e-05\\
30	2.6897552973478e-05\\
};
\addlegendentry{$\text{e}_\text{1,3}$};

\end{axis}
\end{tikzpicture}%
}
      \caption{The evolution of the error states of agent 1 over time.}
      \label{fig:d_OFF_res_3_2_errors_agent_1}
    \end{figure}
  \end{minipage}
  \hfill
  \begin{minipage}{0.45\linewidth}
    \begin{figure}[H]
      \scalebox{0.6}{% This file was created by matlab2tikz.
%
%The latest updates can be retrieved from
%  http://www.mathworks.com/matlabcentral/fileexchange/22022-matlab2tikz-matlab2tikz
%where you can also make suggestions and rate matlab2tikz.
%
\definecolor{mycolor1}{rgb}{0.00000,1.00000,1.00000}%
%
\begin{tikzpicture}

\begin{axis}[%
width=4.133in,
height=3.26in,
at={(0.693in,0.44in)},
scale only axis,
xmin=0,
xmax=30,
xmajorgrids,
ymin=0.8,
ymax=2.2,
ymajorgrids,
axis background/.style={fill=white},
axis x line*=bottom,
axis y line*=left
]
\addplot [color=mycolor1,solid,forget plot]
  table[row sep=crcr]{%
0	1.1\\
30	1.1\\
};
\addplot [color=mycolor1,solid,forget plot]
  table[row sep=crcr]{%
0	2.1\\
30	2.1\\
};
\addplot [color=blue,solid,forget plot]
  table[row sep=crcr]{%
1	1.2\\
2	1.19999732077084\\
3	1.36521091151\\
4	1.53723482189595\\
5	1.54632534062289\\
6	1.71036961929341\\
7	1.57966816234267\\
8	2.09999999985556\\
9	2.0999999999955\\
10	2.09999999999102\\
11	2.09999999977532\\
12	2.1\\
13	2.09999999995514\\
14	2.09999999986519\\
15	2.09999999999797\\
16	1.68077678659047\\
17	1.28856266207741\\
18	1.21470405925992\\
19	1.20240233406648\\
20	1.20039387372556\\
21	1.20006432186432\\
22	1.20001005795673\\
23	1.20000154705776\\
24	1.20000022333303\\
25	1.19999999462576\\
26	1.19999992479106\\
27	1.19999989994857\\
28	1.19999988715586\\
29	1.19999988107988\\
30	1.19999987794025\\
};
\end{axis}
\end{tikzpicture}%}
      \caption{The distance between agents 1 and 3 over time. The maximum allowed
        distance has a value of $2.1$ and the minimum allowed distance a value
        of $1.1$.}
  \label{fig:d_OFF_res_3_2_distance_agents_13}
    \end{figure}
  \end{minipage}
\end{minipage}
}

\noindent\makebox[\linewidth][c]{%
\begin{minipage}{\linewidth}
  \begin{minipage}{0.45\linewidth}
    \begin{figure}[H]
      \scalebox{0.6}{% This file was created by matlab2tikz.
%
%The latest updates can be retrieved from
%  http://www.mathworks.com/matlabcentral/fileexchange/22022-matlab2tikz-matlab2tikz
%where you can also make suggestions and rate matlab2tikz.
%
\definecolor{mycolor1}{rgb}{0.00000,0.44700,0.74100}%
\definecolor{mycolor2}{rgb}{0.85000,0.32500,0.09800}%
\definecolor{mycolor3}{rgb}{0.92900,0.69400,0.12500}%
%
\begin{tikzpicture}

\begin{axis}[%
width=4.133in,
height=3.26in,
at={(0.693in,0.44in)},
scale only axis,
xmin=1,
xmax=30,
xmajorgrids,
ymin=1.5,
ymax=7,
ymajorgrids,
axis background/.style={fill=white},
axis x line*=bottom,
axis y line*=left,
legend style={at={(0.703,0.479)},anchor=south west,legend cell align=left,align=left,draw=white!15!black}
]
\addplot [color=mycolor1,solid]
  table[row sep=crcr]{%
1	6.32455532033676\\
2	5.37980549071204\\
3	4.92763165081336\\
4	4.51030332457465\\
5	3.80178772282473\\
6	3.4249949393334\\
7	2.91231846367102\\
8	2.91881256704999\\
9	2.47341784700138\\
10	1.90522732585578\\
11	2.05019442063337\\
12	2.05019442063337\\
13	2.52731196882367\\
14	3.3251230101429\\
15	4.19122058538753\\
16	5.04959117151099\\
17	5.83427741386267\\
18	6.13688634627583\\
19	6.2529723634364\\
20	6.29743188917348\\
21	6.31440550899511\\
22	6.32092661543391\\
23	6.32340219653524\\
24	6.32428467060215\\
25	6.32456904361269\\
26	6.32467697032657\\
27	6.32473036691825\\
28	6.32477295597452\\
29	6.32479441120973\\
30	6.32481526419489\\
};
\addlegendentry{$\text{d}_{\text{1,o}_\text{2}}$};

\addplot [color=mycolor2,solid]
  table[row sep=crcr]{%
1	6.8\\
2	6.4053632492008\\
3	6.2552251757312\\
4	5.8609810822034\\
5	5.40622455899826\\
6	5.00689533638028\\
7	5.01018244486427\\
8	4.47839241411677\\
9	3.53966570373938\\
10	2.62355069277046\\
11	1.80903430565789\\
12	1.98973871951536\\
13	2.10457417241241\\
14	2.66664801496521\\
15	3.49542704716424\\
16	4.39621016228712\\
17	5.33777091437137\\
18	6.27070065306331\\
19	6.62474535854054\\
20	6.74870510812654\\
21	6.78790010040939\\
22	6.79862933361955\\
23	6.80061297472504\\
24	6.80049553533332\\
25	6.80013457757052\\
26	6.79992331083544\\
27	6.79979031829651\\
28	6.79971190113048\\
29	6.79965937653708\\
30	6.7996105569016\\
};
\addlegendentry{$\text{d}_{\text{2,o}_\text{2}}$};

\addplot [color=mycolor3,solid]
  table[row sep=crcr]{%
1	6.05309838016862\\
2	5.06141684134664\\
3	4.07417165620072\\
4	3.14267168759638\\
5	2.31314449241178\\
6	1.71471478283139\\
7	1.6\\
8	1.78675250315056\\
9	1.96762715208123\\
10	2.50957758755285\\
11	3.26589672246329\\
12	3.26589672264825\\
13	3.91981984551029\\
14	4.73294699335171\\
15	5.53328341549778\\
16	5.86628630465754\\
17	5.99171607388954\\
18	6.03529344464427\\
19	6.04901351578096\\
20	6.05285729669339\\
21	6.05360002859138\\
22	6.05357042196635\\
23	6.0534926996183\\
24	6.05343740282609\\
25	6.05339981823955\\
26	6.05334370739466\\
27	6.05330884627764\\
28	6.05328807107311\\
29	6.05326494865147\\
30	6.05325605718366\\
};
\addlegendentry{$\text{d}_{\text{3,o}_\text{2}}$};

\end{axis}
\end{tikzpicture}%}
      \caption{The distance between each agent and obstacle 2 over time. The
        minimum allowed distance has a value of $1.6$.}
      \label{fig:d_OFF_res_3_2_distance_obstacle_2_agents}
    \end{figure}
  \end{minipage}
  \hfill
  \begin{minipage}{0.45\linewidth}
    \begin{figure}[H]
      \scalebox{0.6}{% This file was created by matlab2tikz.
%
%The latest updates can be retrieved from
%  http://www.mathworks.com/matlabcentral/fileexchange/22022-matlab2tikz-matlab2tikz
%where you can also make suggestions and rate matlab2tikz.
%
\definecolor{mycolor1}{rgb}{0.00000,1.00000,1.00000}%
\definecolor{mycolor2}{rgb}{0.00000,0.44700,0.74100}%
\definecolor{mycolor3}{rgb}{0.85000,0.32500,0.09800}%
%
\begin{tikzpicture}

\begin{axis}[%
width=4.133in,
height=3.26in,
at={(0.693in,0.44in)},
scale only axis,
xmin=1,
xmax=30,
xmajorgrids,
ymin=-11,
ymax=11,
ymajorgrids,
xlabel={time [iterations]},
axis background/.style={fill=white},
axis x line*=bottom,
axis y line*=left,
legend style={at={(0.717,0.639)},anchor=south west,legend cell align=left,align=left,draw=white!15!black}
]
\addplot [color=mycolor1,solid]
  table[row sep=crcr]{%
0	-10\\
30	-10\\
};
\addlegendentry{$\text{u}_{\text{max}}$};

\addplot [color=mycolor1,solid]
  table[row sep=crcr]{%
0	10\\
30	10\\
};
\addlegendentry{$\text{u}_{\text{min}}$};

\addplot [color=mycolor2,solid]
  table[row sep=crcr]{%
1	10\\
2	5.25462130162317\\
3	4.79193319738521\\
4	10\\
5	10\\
6	8.4199467228726\\
7	-0.0718830937370075\\
8	4.6711348570471\\
9	10\\
10	10\\
11	-7.94569359373262e-14\\
12	8.00234026799899\\
13	9.99951019516972\\
14	9.87721457016684\\
15	9.44944892693392\\
16	8.47858591990533\\
17	3.22978879082904\\
18	1.23092831962601\\
19	0.469932637132582\\
20	0.179143585415878\\
21	0.0687788640459975\\
22	0.0261021127368294\\
23	0.00930348204528589\\
24	0.00299785953822199\\
25	0.0011377398219765\\
26	0.00056288940647063\\
27	0.000448957529927465\\
28	0.000226172207195904\\
29	0.000219823159237385\\
30	0.000355769756634746\\
};
\addlegendentry{$\text{u}_{\text{1,1}}$};

\addplot [color=mycolor3,solid]
  table[row sep=crcr]{%
1	0.296935543194055\\
2	-3.08891976426117\\
3	4.69312935134176\\
4	-7.84677125147416\\
5	3.5144084776984\\
6	6.70024165084176\\
7	5.06163755299781\\
8	-2.17906865286603\\
9	-6.88355996738395\\
10	0.705003032261582\\
11	-0.887421061573144\\
12	-0.855870822089348\\
13	0.0700084174317851\\
14	0.303702486095077\\
15	0.390803658864449\\
16	0.184470099846696\\
17	-0.0506896701195546\\
18	-0.0570733931200179\\
19	-0.035955376016846\\
20	-0.0179539436973362\\
21	-0.0096211898814456\\
22	-0.0042075599947334\\
23	-0.00159032100320243\\
24	-0.000645469011385813\\
25	-0.000375988763354052\\
26	-0.000155519204496137\\
27	-0.000110741662397311\\
28	-3.31749218201766e-05\\
29	-4.74279992374978e-05\\
30	-8.91179333348645e-05\\
};
\addlegendentry{$\text{u}_{\text{1,2}}$};

\end{axis}
\end{tikzpicture}%
}
      \caption{The inputs signals directing agent 1 over time. Their value is
        constrained between $-10$ and $10$.}
      \label{fig:d_OFF_res_3_2_inputs_agent_1}
    \end{figure}
  \end{minipage}
\end{minipage}
}
