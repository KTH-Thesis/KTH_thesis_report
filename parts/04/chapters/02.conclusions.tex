\chapter{Conclusions}

This thesis addressed the problem of stabilizing a non-point multi-agent system
under constraints relating to the maintenance of connectivity between
agents, the aversion of collision among agents and between agents and
stationary obstacles within their working environment, and constraints
regarding their states and control inputs. In this thesis, two non-linear
model-predictive control schemes are designed, with the aim of stabilizing
all agents of the multi-agent system to predetermined desired configurations
while satisfying all pertinent constraints at all times. The first scheme
considers that the model of the system is perfectly known and that disturbances
affecting its states are absent; the second considers the presence of
bounded additive uncertainties.

After formulating and positing the motivating problem of this work
(chapter \ref{chapter:prob_formulation}), the control regimes and sufficient or
necessary conditions were documented (the conditions are mild in reality),
leading to the proofs that the compound multi-agent system can be stabilized
asymptotically in the disturbance-free case
(chapter \ref{chapter:stabilization_without_disturbance}), and in the case where
uncertainty is present (chapter \ref{chapter:stabilization_with_disturbance}),
that the trajectory of each agent can be made to get trapped in a neighbourhood
of its desired set point $-$ a neighbourhood smaller than if disturbances
were left unattenuated.

The efficacy of the two designed control regimes is verified through computer
simulations (chapters \ref{chapter:simulations_without_disturbances} and
\ref{chapter:simulations_with_disturbances}) on the equivalent problem to the
posed $12-D$ one, the problem of a multi-unicycle system having to avoid
collisions altogether, while agents are obliged to keep within certain distance
bounds between them, and the obstacles littered in their environment, $-$ and
state and input constraints.

In terms of pros over cons, the designed regime: (a) addresses in full the
objective sought (as captured in sections \eqref{sec:objective} and
\eqref{sec:problem_statement}) in terms of solution to the problem of
stabilization in the face of disturbances and in their absence; (b) is favored
over the equivalent centralized regime due to its reduced complexity and the
increased autonomy of the inner system; (c) is not difficult to be implemented;
(d) does not require the transmission and reception of more than the current and
predicted trajectories of each agent to others within its sensing range;
(e) is system-agnostic in terms of dynamics.
On the other hand, (x) it requires the continuous and unobstructed transmission
and reception of such a volume of information that could potentially require
much resources if the communication protocols cannot accommodate such a demand;
(y) it requires the tuning of penalty matrices, which, in practice,
requires carefulness, patience, and caution. Intuition helps. (z) If, for any
reason, an agent breaks down and becomes uncooperative, the control regime is
not designed to recover such a failure.

This is the conclusion of this work.
