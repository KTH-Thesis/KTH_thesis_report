\section{Test case two: two agents $-$ two obstacles}

In this case the initial configurations of the two agents are
$\vect{z}_1$ $=$ $[-6, 2.75, 0]^{\top}$ and
$\vect{z}_2$ $=$ $[-6, 4.25, 0]^{\top}$.
Their desired configurations in steady-state are
$\vect{z}_{1,des} = [6, 2.75, 0]^{\top}$ and
$\vect{z}_{2,des} = [6, 4.25, 0]^{\top}$.
Obstacles $o_1$, $0_2$ are placed between the two, at $[0, 1.85]^{\top}$
and $[0, 5.15]^{\top}$ respectively. The penalty
matrices $\mat{Q}$, $\mat{R}$, $\mat{P}$ were set to
$\mat{Q} = 0.5 (I_3 + 0.1 \dagger_3)$, $\mat{R} = 0.005 I_2$ and
$\mat{P} = 0.5 (I_3 + 0.1 \dagger_3)$, where $\dagger_N$ is a $N \times N$
matrix whose elements are chosen at random between the values $0.0$ and $1.0$.
The related constants concerned with the execution
of this simulation are as follows: $L_{g_i} = 10.265$, $L_{V_i} = 0.6685$,
$\varepsilon_{\Psi_i} = 0.7117$ and $\varepsilon_{\Omega_i} = 0.0035$ for
all $i \in \mathcal{V}$.

Subsections \ref{subsection:d_ON_errors_2_2} and
\ref{subsection:d_ON_inputs_2_2} illustrate the evolution of the error states
and the input signals of the two agents respectively. Subsection
\ref{subsection:d_ON_distances_2_2} features the figures relating to the
evolution of the distance between the two agents along with that between them
and the two obstacles. Subsection \ref{subsection:d_ON_V_2_2}
features the evolution of the quadratic function
$\vect{e}^{\top} \mat{P} \vect{e}$ through time for all three agents.


%-------------------------------------------------------------------------------
\subsection{State errors}
\label{subsection:d_ON_errors_2_2}

\noindent\makebox[\linewidth][c]{%
\begin{minipage}{\linewidth}
  \begin{minipage}{0.45\linewidth}
    \begin{figure}[H]
      \scalebox{0.6}{% This file was created by matlab2tikz.
%
%The latest updates can be retrieved from
%  http://www.mathworks.com/matlabcentral/fileexchange/22022-matlab2tikz-matlab2tikz
%where you can also make suggestions and rate matlab2tikz.
%
\definecolor{mycolor1}{rgb}{0.00000,0.44700,0.74100}%
\definecolor{mycolor2}{rgb}{0.85000,0.32500,0.09800}%
\definecolor{mycolor3}{rgb}{0.92900,0.69400,0.12500}%
%
\begin{tikzpicture}

\begin{axis}[%
width=4.133in,
height=3.26in,
at={(0.693in,0.44in)},
scale only axis,
xmin=1,
xmax=30,
xmajorgrids,
ymin=-2,
ymax=0.5,
ymajorgrids,
axis background/.style={fill=white},
axis x line*=bottom,
axis y line*=left
]
\addplot [color=mycolor1,solid,forget plot]
  table[row sep=crcr]{%
1	-12\\
2	-11.0022326190636\\
3	-10.0064833964413\\
4	-9.09075100540222\\
5	-8.25986536827002\\
6	-7.34649232247749\\
7	-6.36910253921803\\
8	-5.37372044898297\\
9	-4.38535987310174\\
10	-3.4295021807112\\
11	-2.46765265738195\\
12	-1.544733685219\\
13	-0.674946507753762\\
14	-0.25112620828626\\
15	-0.090794065246815\\
16	-0.0317853545107259\\
17	-0.0107851693103166\\
18	-0.00352136199166776\\
19	-0.00105761240924525\\
20	-0.000259664795541748\\
21	-2.80707815144702e-05\\
22	3.49610238669293e-05\\
23	4.38743795292881e-05\\
24	5.04631486268761e-05\\
25	5.63350081147557e-05\\
26	5.82191700408871e-05\\
27	5.53199780845883e-05\\
28	5.90303174640078e-05\\
29	5.91371910553997e-05\\
30	5.71445098028399e-05\\
};
\addplot [color=mycolor2,solid,forget plot]
  table[row sep=crcr]{%
1	0\\
2	0.057824688181757\\
3	0.0246723596067103\\
4	-0.357883516982828\\
5	-0.480562351160983\\
6	-0.459092963856734\\
7	-0.656844024153609\\
8	-0.628299189047184\\
9	-0.477627902952613\\
10	-0.3878256003521\\
11	-0.308497235160808\\
12	-0.199891188078674\\
13	-0.0714735801581983\\
14	-0.0170012029994771\\
15	-0.00399914878440709\\
16	-0.00149603400259468\\
17	-0.0010755108189131\\
18	-0.00101079992223427\\
19	-0.00100143328737144\\
20	-0.00100017826466349\\
21	-0.00100002153203774\\
22	-0.00100000002850996\\
23	-0.00099999806337853\\
24	-0.000999996940429051\\
25	-0.000999996108273977\\
26	-0.00099999586797717\\
27	-0.000999996158759532\\
28	-0.000999995846597495\\
29	-0.000999995837834379\\
30	-0.000999995977520773\\
};
\addplot [color=mycolor3,solid,forget plot]
  table[row sep=crcr]{%
1	0\\
2	0.115778650054471\\
3	-0.182341769343908\\
4	-0.609102036082591\\
5	0.315923663977634\\
6	-0.268921106433564\\
7	-0.13034048324429\\
8	0.18767929673635\\
9	0.114882572195712\\
10	0.0724664034808653\\
11	0.0921107521475161\\
12	0.14216518432427\\
13	0.151001999614394\\
14	0.10465054336164\\
15	0.0571843080728701\\
16	0.0276036832534965\\
17	0.0124404387050183\\
18	0.00537643941578293\\
19	0.00222708474353981\\
20	0.000918539484234357\\
21	0.000434972198537005\\
22	0.000247334989561869\\
23	0.000193605879797335\\
24	0.000147261876106492\\
25	0.000136176469093453\\
26	0.000118893758041787\\
27	8.1701694794513e-05\\
28	8.65643338692456e-05\\
29	7.74259655242939e-05\\
30	6.27734701031367e-05\\
};
\end{axis}
\end{tikzpicture}%}
      \caption{The evolution of the error states of agent 1 over time.}
      \label{fig:d_ON_2_2_errors_agent_1}
    \end{figure}
  \end{minipage}
  \hfill
  \begin{minipage}{0.45\linewidth}
    \begin{figure}[H]
      \scalebox{0.6}{% This file was created by matlab2tikz.
%
%The latest updates can be retrieved from
%  http://www.mathworks.com/matlabcentral/fileexchange/22022-matlab2tikz-matlab2tikz
%where you can also make suggestions and rate matlab2tikz.
%
\definecolor{mycolor1}{rgb}{0.00000,0.44700,0.74100}%
\definecolor{mycolor2}{rgb}{0.85000,0.32500,0.09800}%
\definecolor{mycolor3}{rgb}{0.92900,0.69400,0.12500}%
%
\begin{tikzpicture}

\begin{axis}[%
width=4.133in,
height=3.26in,
at={(0.693in,0.44in)},
scale only axis,
xmin=1,
xmax=100,
xmajorgrids,
ymin=-1,
ymax=1,
ymajorgrids,
axis background/.style={fill=white},
axis x line*=bottom,
axis y line*=left,
legend style={legend cell align=left,align=left,draw=white!15!black}
]
\addplot [color=mycolor1,solid]
  table[row sep=crcr]{%
1	-12\\
2	-11.2063029159317\\
3	-10.2594486854832\\
4	-10.1871147772015\\
5	-9.38575958508419\\
6	-8.58848620284489\\
7	-7.79907248369624\\
8	-7.25075517424531\\
9	-6.71313467098878\\
10	-5.942476332933\\
11	-5.13399766800277\\
12	-4.32548167484064\\
13	-3.51869369275716\\
14	-2.71544990997961\\
15	-1.91833338500396\\
16	-1.13118760851096\\
17	-0.351920385155077\\
18	-0.0712369008298993\\
19	-0.0192222025526496\\
20	-0.0107545554060087\\
21	-0.0114401446347571\\
22	-0.013129494078575\\
23	-0.0143039915846219\\
24	-0.014801942364958\\
25	-0.0146491752747064\\
26	-0.0139004618503406\\
27	-0.0125929633893919\\
28	-0.0107318842219341\\
29	-0.00837701762296028\\
30	-0.00561258143136186\\
31	-0.00254294913394437\\
32	0.000685256219284745\\
33	0.00386316497635141\\
34	0.00707736151486232\\
35	0.00985567041575126\\
36	0.0122321576740294\\
37	0.0140763022530623\\
38	0.0152988373118962\\
39	0.0158477106609429\\
40	0.0157141379481106\\
41	0.0149101142922279\\
42	0.0134928565535913\\
43	0.0115227823956818\\
44	0.00916244690492867\\
45	0.00648885867539654\\
46	0.00361680905873646\\
47	0.000682476268889974\\
48	-0.00219782608290516\\
49	-0.0048887780132168\\
50	-0.00748996668985373\\
51	-0.00983454641230558\\
52	-0.0116260099253913\\
53	-0.0130777220457245\\
54	-0.01410755595654\\
55	-0.0146643253090967\\
56	-0.014691430941652\\
57	-0.01414460425318\\
58	-0.0130287286211954\\
59	-0.0113817611085712\\
60	-0.00921400004825336\\
61	-0.0065974141409281\\
62	-0.00362925057524199\\
63	-0.000439751871576113\\
64	0.000693084827859266\\
65	0.00918766272285188\\
66	0.0170636150754878\\
67	0.0143694802708719\\
68	0.0145836361983097\\
69	0.0154571778496466\\
70	0.016092999620603\\
71	0.0161838161273543\\
72	0.0156334691191352\\
73	0.0144552831706055\\
74	0.0127112045735006\\
75	0.0104758433771591\\
76	0.00788761686780993\\
77	0.00509451013037413\\
78	0.00216539160147211\\
79	-0.000752793574008406\\
80	-0.00349173243738278\\
81	-0.00625604417314171\\
82	-0.00870267708825006\\
83	-0.0108398550974827\\
84	-0.0123679650848165\\
85	-0.0135642073454043\\
86	-0.0143200850788771\\
87	-0.0145714134299789\\
88	-0.014260602531161\\
89	-0.0133802794713839\\
90	-0.0119544327906056\\
91	-0.00998564091734785\\
92	-0.00753423740951568\\
93	-0.0046877170454179\\
94	-0.00156522284935047\\
95	0.00502657170583021\\
96	0.0119775627475487\\
97	0.0219721323193938\\
98	0.0252212631085454\\
99	0.0178207917659696\\
100	0.0143694802708719\\
};
\addlegendentry{$\text{e}_{\text{2,1}}$};

\addplot [color=mycolor2,solid]
  table[row sep=crcr]{%
1	0\\
2	0.0942561495966403\\
3	-0.170952238805503\\
4	-0.193583841832215\\
5	-0.106125910732988\\
6	0.0306776723488525\\
7	0.0412526862444037\\
8	0.138954376118198\\
9	0.443665368798181\\
10	0.633691554077828\\
11	0.617033841650252\\
12	0.629497001473648\\
13	0.605583204702762\\
14	0.546491338812166\\
15	0.444766149473266\\
16	0.292559868747122\\
17	0.110248281456588\\
18	0.0556387066688593\\
19	0.0467252690809705\\
20	0.04087338295588\\
21	0.0338642080337582\\
22	0.0255772999363582\\
23	0.016295264730663\\
24	0.00638243491791177\\
25	-0.00376204154651759\\
26	-0.0137264640306249\\
27	-0.0231068409077982\\
28	-0.0315254071262659\\
29	-0.0386455634546002\\
30	-0.044186337011555\\
31	-0.047933506388919\\
32	-0.0497474124538697\\
33	-0.0495679022011788\\
34	-0.0474131330526205\\
35	-0.0433830727754197\\
36	-0.0376468605714824\\
37	-0.0304402277757237\\
38	-0.0220540398576366\\
39	-0.0128226001064395\\
40	-0.0031106308826877\\
41	0.00670014892769413\\
42	0.0162254622537896\\
43	0.0250916376977848\\
44	0.0329506331904903\\
45	0.0394902446230111\\
46	0.0444470071040838\\
47	0.0476163701162829\\
48	0.04886117008567\\
49	0.0481182873244569\\
50	0.0454044324751687\\
51	0.0408148552415565\\
52	0.0345200792894749\\
53	0.0267683943654471\\
54	0.0178670479922624\\
55	0.00817344671964214\\
56	-0.00192068304473431\\
57	-0.0120061619426184\\
58	-0.021673730477436\\
59	-0.0305323342513254\\
60	-0.0382267679432\\
61	-0.0444515254182926\\
62	-0.0489635703240883\\
63	-0.0515915583631062\\
64	-0.0522421087577428\\
65	-0.0508105602844021\\
66	-0.0474677334822013\\
67	-0.0424976218791236\\
68	-0.0358987260285628\\
69	-0.0279755827584256\\
70	-0.0190577372765537\\
71	-0.00950302629699401\\
72	0.000310765483940814\\
73	0.009998449103747\\
74	0.019180423373929\\
75	0.0274958742393287\\
76	0.0346165541305675\\
77	0.0402580513980204\\
78	0.0441898092374192\\
79	0.0462458273741171\\
80	0.0463314586929367\\
81	0.044430661974302\\
82	0.0406048414319447\\
83	0.0349957092426535\\
84	0.0278163976171344\\
85	0.0193542799905562\\
86	0.00994691749440877\\
87	-2.65855683810282e-05\\
88	-0.010162414537734\\
89	-0.0200491105504161\\
90	-0.0292860700294709\\
91	-0.0375016396370791\\
92	-0.0443679463498232\\
93	-0.0496147872299852\\
94	-0.0530401838744449\\
95	-0.0544776892283678\\
96	-0.0538701542972689\\
97	-0.0512479028054888\\
98	-0.0469178281539428\\
99	-0.0410260918314233\\
100	-0.0424976218791236\\
};
\addlegendentry{$\text{e}_{\text{2,2}}$};

\addplot [color=mycolor3,solid]
  table[row sep=crcr]{%
1	0\\
2	0.23455063894055\\
3	-0.787831252768654\\
4	0.0169550161860064\\
5	0.186457922457437\\
6	0.137653554287325\\
7	-0.133203158839146\\
8	0.457443146535863\\
9	0.561425888317004\\
10	-0.0964884514321317\\
11	0.0310625161573333\\
12	-0.0216158748100579\\
13	-0.0570262542674939\\
14	-0.106103338127805\\
15	-0.15999116179454\\
16	-0.229344896087054\\
17	-0.231872999782621\\
18	-0.139913659429474\\
19	-0.0547328454850554\\
20	-0.027257877101979\\
21	-0.0192397311337429\\
22	-0.0172970443938536\\
23	-0.0168620744140591\\
24	-0.0164221186610755\\
25	-0.0155270891463772\\
26	-0.0140125534834047\\
27	-0.0119548251646782\\
28	-0.00944422965808113\\
29	-0.0065999074664162\\
30	-0.00355317361911811\\
31	-0.000439799487257069\\
32	0.00262363455204751\\
33	0.00545090545671102\\
34	0.00828670066899676\\
35	0.0105298071244399\\
36	0.0124286032641317\\
37	0.0139009215836584\\
38	0.0148953777826784\\
39	0.0153770275966799\\
40	0.0153257187552726\\
41	0.014719975736487\\
42	0.0135624359635485\\
43	0.0118946372601036\\
44	0.00969559077815298\\
45	0.00706608405158053\\
46	0.00410112869261826\\
47	0.000911686108136247\\
48	-0.00237188902508899\\
49	-0.00556211040108279\\
50	-0.00862284692903929\\
51	-0.0113204271749145\\
52	-0.0134310525378583\\
53	-0.0149872179912447\\
54	-0.0158797214309788\\
55	-0.0160564182770905\\
56	-0.0155244881238278\\
57	-0.0143209412353206\\
58	-0.0125073233338218\\
59	-0.0101798189082694\\
60	-0.00746433129914624\\
61	-0.00449168091498123\\
62	-0.00139570307923423\\
63	0.00169800623609664\\
64	0.0115188551865277\\
65	0.0171288624352802\\
66	0.0151096837416641\\
67	0.0129739290983575\\
68	0.0137456535693184\\
69	0.0147857626307879\\
70	0.0154672885814087\\
71	0.0156437187602347\\
72	0.0152692942207796\\
73	0.014340174891033\\
74	0.0128751804676153\\
75	0.010905502313192\\
76	0.00846431150866587\\
77	0.00561225151533611\\
78	0.0025114067730711\\
79	-0.000741696483661597\\
80	-0.00394272791142917\\
81	-0.00711796418066636\\
82	-0.00996338668291435\\
83	-0.0123884331052561\\
84	-0.0141318165149777\\
85	-0.0152737413562896\\
86	-0.0157393783668444\\
87	-0.0154981286031821\\
88	-0.0145626325417722\\
89	-0.012986585324509\\
90	-0.0108570193619288\\
91	-0.00828951125466484\\
92	-0.00541166765554877\\
93	-0.00235695842374425\\
94	0.000745025402269989\\
95	0.0119621061057049\\
96	0.020673618642626\\
97	0.0190138313753853\\
98	0.0110421304568029\\
99	0.0120654897258721\\
100	0.0129739290983575\\
};
\addlegendentry{$\text{e}_{\text{2,3}}$};

\end{axis}
\end{tikzpicture}%}
      \caption{The evolution of the error states of agent 2 over time.}
      \label{fig:d_ON_e2_2_errors_agent_2}
    \end{figure}
  \end{minipage}
\end{minipage}
}


%-------------------------------------------------------------------------------
\subsection{Distances between actors}
\label{subsection:d_ON_distances_2_2}

\begin{figure}[H]\centering
  \scalebox{0.7}{% This file was created by matlab2tikz.
%
%The latest updates can be retrieved from
%  http://www.mathworks.com/matlabcentral/fileexchange/22022-matlab2tikz-matlab2tikz
%where you can also make suggestions and rate matlab2tikz.
%
\definecolor{mycolor1}{rgb}{0.00000,1.00000,1.00000}%
\definecolor{mycolor2}{rgb}{0.00000,0.44700,0.74100}%

%
\begin{tikzpicture}

\begin{axis}[%
width=4.133in,
height=3.26in,
at={(0.693in,0.44in)},
scale only axis,
xmin=0,
xmax=100,
xmajorgrids,
ymin=0.8,
ymax=2.2,
ymajorgrids,
xlabel={time [iterations]},
ylabel={distance},
axis background/.style={fill=white},
axis x line*=bottom,
axis y line*=left,
legend style={at={(0.709,0.616)},anchor=south west,legend cell align=left,align=left,draw=white!15!black}
]
\addplot [color=mycolor1,solid]
  table[row sep=crcr]{%
0	1.01\\
100	1.01\\
};
\addlegendentry{$\text{d}_{\text{max}}$};

\addplot [color=mycolor1,solid]
  table[row sep=crcr]{%
0	2.01\\
100	2.01\\
};
\addlegendentry{$\text{d}_{\text{min}}$};

\addplot [color=mycolor2,solid]
  table[row sep=crcr]{%
1	1.5\\
2	1.50083126157479\\
3	1.90802850407836\\
4	1.98566379574154\\
5	1.51417169579336\\
6	1.05378983025117\\
7	1.03717322383287\\
8	1.04085531142189\\
9	1.03988946155391\\
10	1.03681523186882\\
11	1.11150482252664\\
12	1.18074341597948\\
13	1.26908281251512\\
14	1.36692807176105\\
15	1.43877220918611\\
16	1.4980998949487\\
17	1.46152258825147\\
18	1.49275900867495\\
19	1.49748891312852\\
20	1.49792851315055\\
21	1.49797224726479\\
22	1.49797783655406\\
23	1.49797955238484\\
24	1.49798163717184\\
25	1.49798460006658\\
26	1.49798775930857\\
27	1.49799079247671\\
28	1.49799374445112\\
29	1.49799646627223\\
30	1.49799883644745\\
31	1.49800076832522\\
32	1.49800219174698\\
33	1.49808204140662\\
34	1.49811455158693\\
35	1.49802961953908\\
36	1.49800318283209\\
37	1.49799355989479\\
38	1.4979893369758\\
39	1.49798671161758\\
40	1.49798445065978\\
41	1.49798220908182\\
42	1.49797994165761\\
43	1.49797805737971\\
44	1.4979756394931\\
45	1.4979739259131\\
46	1.49797221395158\\
47	1.49797085340955\\
48	1.49796991151323\\
49	1.49796949375205\\
50	1.49796959643636\\
51	1.49797026612426\\
52	1.49797158321572\\
53	1.49797340742434\\
54	1.49797568874685\\
55	1.4979783695316\\
56	1.4979809787792\\
57	1.4979840564244\\
58	1.49798743554357\\
59	1.49799039870926\\
60	1.49799321563086\\
61	1.49799574596327\\
62	1.4979978759431\\
63	1.49801390648784\\
64	1.49813314056037\\
65	1.49822649327842\\
66	1.49820303821838\\
67	1.498140992446\\
68	1.4981249467501\\
69	1.49811932313123\\
70	1.49811656414702\\
71	1.49811446044501\\
72	1.49811241222619\\
73	1.49811031629057\\
74	1.4981082560086\\
75	1.49810625134224\\
76	1.49810438532469\\
77	1.49810253131562\\
78	1.49810108267708\\
79	1.4981000052064\\
80	1.49809940511073\\
81	1.49809926143506\\
82	1.49809966648003\\
83	1.49810060225596\\
84	1.49810214272405\\
85	1.49810411698296\\
86	1.49810646957547\\
87	1.49810882138823\\
88	1.4981116330981\\
89	1.49811478756559\\
90	1.4981176009658\\
91	1.49812032903239\\
92	1.49812283412389\\
93	1.49812500078185\\
94	1.49815643322859\\
95	1.49824610464434\\
96	1.49851431000025\\
97	1.49877170651175\\
98	1.49868413833321\\
99	1.49862686650785\\
100	1.50747121997847\\
};
\addlegendentry{$\text{d}_{\text{12,a}}$};

\end{axis}
\end{tikzpicture}%
}
  \caption{The distance between the two agents over time. The maximum allowed
    distance has a value of $2.01$ and the minimum allowed distance a value
    of $1.01$.}
  \label{fig:d_ON_2_2_distance_agents}
\end{figure}

\noindent\makebox[\linewidth][c]{%
\begin{minipage}{\linewidth}
  \begin{minipage}{0.45\linewidth}
    \begin{figure}[H]
      \scalebox{0.6}{% This file was created by matlab2tikz.
%
%The latest updates can be retrieved from
%  http://www.mathworks.com/matlabcentral/fileexchange/22022-matlab2tikz-matlab2tikz
%where you can also make suggestions and rate matlab2tikz.
%
\definecolor{mycolor1}{rgb}{0.00000,0.44700,0.74100}%
\definecolor{mycolor2}{rgb}{0.85000,0.32500,0.09800}%
\definecolor{mycolor3}{rgb}{0.00000,1.00000,1.00000}%
%
\begin{tikzpicture}

\begin{axis}[%
width=4.133in,
height=3.26in,
at={(0.693in,0.44in)},
scale only axis,
xmin=0,
xmax=100,
xmajorgrids,
ymin=1.2,
ymax=7,
ymajorgrids,
xlabel={time [iterations]},
ylabel={distance},
axis background/.style={fill=white},
legend style={at={(0.705,0.572)},anchor=south west,legend cell align=left,align=left,draw=white!15!black}
]
\addplot [color=mycolor1,solid]
  table[row sep=crcr]{%
1	6.4621977685614\\
2	5.77342365648139\\
3	5.13953125073559\\
4	4.45577518834934\\
5	3.67653432780378\\
6	2.92019357211574\\
7	2.35706002276644\\
8	1.93663782306074\\
9	1.82215494547734\\
10	2.17588134962582\\
11	2.7758627527564\\
12	3.44636525171151\\
13	4.15176564977888\\
14	4.870918478485\\
15	5.55685670784799\\
16	6.184512139916\\
17	6.38506713895202\\
18	6.44608518232096\\
19	6.4622858640847\\
20	6.46437540229293\\
21	6.46196456118633\\
22	6.45817383469753\\
23	6.45406853535106\\
24	6.45010074050291\\
25	6.44653462047598\\
26	6.44356702793008\\
27	6.44133994438276\\
28	6.43998547676853\\
29	6.43957445678675\\
30	6.44013003284419\\
31	6.44162963075208\\
32	6.44398500024544\\
33	6.45301658069248\\
34	6.45811082569937\\
35	6.45738582698246\\
36	6.46001327021509\\
37	6.4638109949674\\
38	6.46784656679855\\
39	6.47169837865593\\
40	6.47514312284882\\
41	6.47802562096869\\
42	6.48024856383819\\
43	6.48174578184368\\
44	6.48246465547821\\
45	6.48245479503116\\
46	6.48164199289328\\
47	6.48011042662648\\
48	6.47790195921122\\
49	6.47511714823907\\
50	6.47167313755442\\
51	6.46776070097008\\
52	6.46371968039077\\
53	6.45945329390456\\
54	6.45515685151227\\
55	6.45101387252769\\
56	6.44724499361798\\
57	6.44400905600602\\
58	6.44145892862497\\
59	6.43974158260337\\
60	6.43894536169521\\
61	6.43911275892206\\
62	6.44023896050537\\
63	6.44014387030188\\
64	6.44770577199336\\
65	6.45838130956439\\
66	6.46561163096152\\
67	6.46117431717493\\
68	6.46259623298884\\
69	6.46593794904962\\
70	6.46968950737199\\
71	6.47325909485781\\
72	6.47636463789235\\
73	6.47886005437408\\
74	6.48065747172313\\
75	6.4816868492063\\
76	6.481955263064\\
77	6.48146717326543\\
78	6.48023082711676\\
79	6.47829354658849\\
80	6.47577772177869\\
81	6.47249066178186\\
82	6.46877240306506\\
83	6.46467163233192\\
84	6.46054756538046\\
85	6.45625916516831\\
86	6.45203720782322\\
87	6.44810008301778\\
88	6.44462299068274\\
89	6.44177054969541\\
90	6.43970393105542\\
91	6.4385281815893\\
92	6.43830500740896\\
93	6.4390486503228\\
94	6.44064197815333\\
95	6.44678006646274\\
96	6.45985674861404\\
97	6.47531006239829\\
98	6.47659839146644\\
99	6.46600757613679\\
100	6.46550038484404\\
};
\addlegendentry{$\text{d}_{\text{1,o}_\text{1}}$};

\addplot [color=mycolor2,solid]
  table[row sep=crcr]{%
1	6.06712452484701\\
2	5.30039011238323\\
3	4.32139025573568\\
4	4.24628707766915\\
5	3.47758594970692\\
6	2.75071295342263\\
7	2.0304232122775\\
8	1.62598115043089\\
9	1.52118298776769\\
10	1.53476993563286\\
11	1.74681187187881\\
12	2.26790050461024\\
13	2.90235452290828\\
14	3.588956183504\\
15	4.29748744649038\\
16	5.01273709105748\\
17	5.73771774539431\\
18	6.00528743889623\\
19	6.05524498245576\\
20	6.06269772612364\\
21	6.06093652007169\\
22	6.05799569105404\\
23	6.05542349611715\\
24	6.05343909752514\\
25	6.05207955773985\\
26	6.05135261418431\\
27	6.05127958654341\\
28	6.05190719369877\\
29	6.05322041795152\\
30	6.05517114123554\\
31	6.05768176669241\\
32	6.06062315334293\\
33	6.06379482310594\\
34	6.06727968639287\\
35	6.07059780740217\\
36	6.07376229818487\\
37	6.07661484436792\\
38	6.0790302854235\\
39	6.08091337027148\\
40	6.08220577832947\\
41	6.08286518370089\\
42	6.0828912067785\\
43	6.08228581220915\\
44	6.0811536896598\\
45	6.07951892250805\\
46	6.07744970684583\\
47	6.07504460608421\\
48	6.07239396264282\\
49	6.06961999228542\\
50	6.06662723762277\\
51	6.06359750916885\\
52	6.06085396817951\\
53	6.05822895069672\\
54	6.05585570084415\\
55	6.05384358470338\\
56	6.05231072619261\\
57	6.05136330715808\\
58	6.0510562747382\\
59	6.05140664911602\\
60	6.05245156946402\\
61	6.05415871517598\\
62	6.05646143958602\\
63	6.05925078328695\\
64	6.06028142419677\\
65	6.0688927384116\\
66	6.07715935395142\\
67	6.07519140222731\\
68	6.07633831583656\\
69	6.07833461096116\\
70	6.08024952204565\\
71	6.08173105041705\\
72	6.08263146255242\\
73	6.08290798308096\\
74	6.08256431781182\\
75	6.08161725699269\\
76	6.08014981066819\\
77	6.07826005356934\\
78	6.07597595321165\\
79	6.07341322566092\\
80	6.07072109659625\\
81	6.06769444544534\\
82	6.06468309800724\\
83	6.06170406879047\\
84	6.05909074475946\\
85	6.0566183462154\\
86	6.05445021752989\\
87	6.05271075730646\\
88	6.05151940118446\\
89	6.05094449206611\\
90	6.0510191259672\\
91	6.05179109385355\\
92	6.05324310820677\\
93	6.05532198815587\\
94	6.05793369936167\\
95	6.06426022734065\\
96	6.07122804140718\\
97	6.08149064661975\\
98	6.08531350559172\\
99	6.07881594200097\\
100	6.07519140222731\\
};
\addlegendentry{$\text{d}_{\text{2,o}_\text{1}}$};

\addplot [color=mycolor3,solid]
  table[row sep=crcr]{%
0	1.51\\
100	1.51\\
};
\addlegendentry{$\text{d}_{\text{min}}$};

\end{axis}
\end{tikzpicture}%
}
      \caption{The distance between each agent and obstacle 1 over time. The
        minimum allowed distance has a value of $1.51$.}
      \label{fig:d_ON_2_2_distance_obstacle_1_agents}
    \end{figure}
  \end{minipage}
  \hfill
  \begin{minipage}{0.45\linewidth}
    \begin{figure}[H]
      \scalebox{0.6}{% This file was created by matlab2tikz.
%
%The latest updates can be retrieved from
%  http://www.mathworks.com/matlabcentral/fileexchange/22022-matlab2tikz-matlab2tikz
%where you can also make suggestions and rate matlab2tikz.
%
\definecolor{mycolor1}{rgb}{0.00000,0.44700,0.74100}%
\definecolor{mycolor2}{rgb}{0.85000,0.32500,0.09800}%
\definecolor{mycolor3}{rgb}{0.00000,1.00000,1.00000}%
%
\begin{tikzpicture}

\begin{axis}[%
width=4.133in,
height=3.26in,
at={(0.693in,0.44in)},
scale only axis,
xmin=0,
xmax=100,
xmajorgrids,
ymin=1.2,
ymax=7,
ymajorgrids,
xlabel={time [iterations]},
ylabel={distance},
axis background/.style={fill=white},
legend style={at={(0.703,0.551)},anchor=south west,legend cell align=left,align=left,draw=white!15!black}
]
\addplot [color=mycolor1,solid]
  table[row sep=crcr]{%
1	6.06712452484701\\
2	5.26828661345623\\
3	4.46565947445785\\
4	3.67905114880299\\
5	3.10793075806622\\
6	2.60634334634072\\
7	1.98007616019039\\
8	1.57128616623272\\
9	1.50644893729648\\
10	1.71298434211247\\
11	2.16775708998318\\
12	2.80588086419354\\
13	3.52711537261097\\
14	4.29061494570799\\
15	5.04597367657315\\
16	5.73942275126988\\
17	5.95695227708652\\
18	6.02372023681609\\
19	6.0431225017917\\
20	6.04831143897077\\
21	6.04953480351191\\
22	6.05000383625344\\
23	6.05068483280017\\
24	6.05185894479611\\
25	6.05358899701461\\
26	6.0558592724439\\
27	6.05859936333842\\
28	6.06174248632669\\
29	6.06517957603333\\
30	6.06878179005008\\
31	6.07240878490192\\
32	6.0758918052847\\
33	6.08533699707277\\
34	6.08955729604324\\
35	6.08663973856659\\
36	6.08633079315731\\
37	6.08646048342225\\
38	6.08620296349747\\
39	6.08529349738286\\
40	6.08369225148132\\
41	6.0814400051734\\
42	6.07864013848699\\
43	6.07542332389518\\
44	6.0719218324317\\
45	6.06835700901969\\
46	6.06479316957723\\
47	6.06143180949661\\
48	6.05839324888524\\
49	6.05582057358071\\
50	6.05361748788807\\
51	6.05193743825116\\
52	6.05105174309697\\
53	6.05072235533096\\
54	6.05099066000755\\
55	6.05185740450534\\
56	6.05334371351968\\
57	6.0553941792128\\
58	6.05794707142602\\
59	6.06094452034783\\
60	6.06428552079395\\
61	6.06784816355494\\
62	6.07149499951304\\
63	6.07281443062888\\
64	6.08112543628693\\
65	6.09162273649923\\
66	6.09748818140035\\
67	6.090117969883\\
68	6.0880586792734\\
69	6.08731511112954\\
70	6.08646812693264\\
71	6.08508405815834\\
72	6.08306716129057\\
73	6.0804695449133\\
74	6.07740208589191\\
75	6.0739847308369\\
76	6.07040245454994\\
77	6.06681440129834\\
78	6.06335482969631\\
79	6.0601654165451\\
80	6.0574296103949\\
81	6.05495152314629\\
82	6.05306258993346\\
83	6.05173890037584\\
84	6.05124884063154\\
85	6.05128556883584\\
86	6.05191141878674\\
87	6.05315223468708\\
88	6.05497322030988\\
89	6.05732402845939\\
90	6.06015665322765\\
91	6.06337874850182\\
92	6.06687639302428\\
93	6.07051729080271\\
94	6.07405151960656\\
95	6.08129118814656\\
96	6.09468762285261\\
97	6.10952916777585\\
98	6.10858905491351\\
99	6.0941865298178\\
100	6.08965435363753\\
};
\addlegendentry{$\text{d}_{\text{1,o}_\text{2}}$};

\addplot [color=mycolor2,solid]
  table[row sep=crcr]{%
1	6.46219776856141\\
2	5.69403589346889\\
3	4.97520838945301\\
4	4.92530276247792\\
5	4.21236691759357\\
6	3.50912369611341\\
7	2.96653850333047\\
8	2.58393417469882\\
9	2.08225988968856\\
10	1.7672448892015\\
11	1.98215245650911\\
12	2.43694326751472\\
13	3.06215816469494\\
14	3.77144052729873\\
15	4.52580841024218\\
16	5.3053405178492\\
17	6.09456858753754\\
18	6.37544208347549\\
19	6.42709926962555\\
20	6.4371219804223\\
21	6.43905643144365\\
22	6.44053581724761\\
23	6.44287241609713\\
24	6.44608414752614\\
25	6.44999972460617\\
26	6.45441423552884\\
27	6.45914001895828\\
28	6.46402725614429\\
29	6.46889002434773\\
30	6.47354055937923\\
31	6.47779820067625\\
32	6.48147251240271\\
33	6.48434698418076\\
34	6.48650980659575\\
35	6.4875639510955\\
36	6.48760801333001\\
37	6.48661340532272\\
38	6.48460992459504\\
39	6.48167698805247\\
40	6.47794390943378\\
41	6.47356376813867\\
42	6.46873073969138\\
43	6.4636364294873\\
44	6.4585568061453\\
45	6.453674512603\\
46	6.44918946010687\\
47	6.44530053008744\\
48	6.44216459856278\\
49	6.4399306016765\\
50	6.43850112882885\\
51	6.43799943372197\\
52	6.43864258211983\\
53	6.44014492199112\\
54	6.44247357407834\\
55	6.44554709855607\\
56	6.4492745045068\\
57	6.45354464957363\\
58	6.45819856161141\\
59	6.46305158876396\\
60	6.46795691614691\\
61	6.47272877666769\\
62	6.47718185118585\\
63	6.48112832305919\\
64	6.48242306998436\\
65	6.48974643327627\\
66	6.49577961866794\\
67	6.49141239470263\\
68	6.48913082934009\\
69	6.48697082535554\\
70	6.48422819743118\\
71	6.48075401038853\\
72	6.47660053245826\\
73	6.4719224166043\\
74	6.46691563931753\\
75	6.46177960708749\\
76	6.45676021414043\\
77	6.45209594934776\\
78	6.44793230757266\\
79	6.44446473719609\\
80	6.44188380873835\\
81	6.44000570762618\\
82	6.43912953168391\\
83	6.43919906017739\\
84	6.4404186532365\\
85	6.44242869916116\\
86	6.44520890127579\\
87	6.44868071595769\\
88	6.45274817413953\\
89	6.45726361361359\\
90	6.46205234619895\\
91	6.46696886224526\\
92	6.47182977008524\\
93	6.47644825316793\\
94	6.48063468492347\\
95	6.4872802354926\\
96	6.49348542380214\\
97	6.50175089060168\\
98	6.50312987161974\\
99	6.49405693405999\\
100	6.49141239470263\\
};
\addlegendentry{$\text{d}_{\text{2,o}_\text{2}}$};

\addplot [color=mycolor3,solid]
  table[row sep=crcr]{%
0	1.51\\
100	1.51\\
};
\addlegendentry{$\text{d}_{\text{min}}$};

\end{axis}
\end{tikzpicture}%
}
      \caption{The distance between each agent and obstacle 2 over time. The
        minimum allowed distance has a value of $1.51$.}
      \label{fig:d_ON_2_2_distance_obstacle_2_agents}
    \end{figure}
  \end{minipage}
\end{minipage}
}


%-------------------------------------------------------------------------------
\subsection{Input signals}
\label{subsection:d_ON_inputs_2_2}

\noindent\makebox[\linewidth][c]{%
\begin{minipage}{\linewidth}
  \begin{minipage}{0.45\linewidth}
    \begin{figure}[H]
      \scalebox{0.6}{% This file was created by matlab2tikz.
%
%The latest updates can be retrieved from
%  http://www.mathworks.com/matlabcentral/fileexchange/22022-matlab2tikz-matlab2tikz
%where you can also make suggestions and rate matlab2tikz.
%
\definecolor{mycolor1}{rgb}{0.00000,1.00000,1.00000}%
\definecolor{mycolor2}{rgb}{0.00000,0.44700,0.74100}%
\definecolor{mycolor3}{rgb}{0.85000,0.32500,0.09800}%
%
\begin{tikzpicture}

\begin{axis}[%
width=4.133in,
height=3.26in,
at={(0.693in,0.44in)},
scale only axis,
xmin=0,
xmax=100,
xmajorgrids,
ymin=-9,
ymax=9,
ymajorgrids,
xlabel={time [iterations]},
ylabel={component magnitude},
axis background/.style={fill=white},
axis x line*=bottom,
axis y line*=left,
legend style={at={(0.709,0.605)},anchor=south west,legend cell align=left,align=left,draw=white!15!black}
]
\addplot [color=mycolor1,solid]
  table[row sep=crcr]{%
0	-8\\
100 -8\\
};
\addlegendentry{$\text{u}_{\text{max}}$};

\addplot [color=mycolor1,solid]
  table[row sep=crcr]{%
0	8\\
100 8\\
};
\addlegendentry{$\text{u}_{\text{min}}$};

\addplot [color=mycolor2,solid]
  table[row sep=crcr]{%
1	8\\
2	8\\
3	8\\
4	8\\
5	8\\
6	8\\
7	8\\
8	8\\
9	8\\
10	8\\
11	8\\
12	8\\
13	8\\
14	7.66788618594458\\
15	6.97259436997886\\
16	2.18952981040143\\
17	0.686643615994262\\
18	0.225618154735611\\
19	0.0975575613268285\\
20	0.0709709668674278\\
21	0.0743739220980101\\
22	0.0846979553496463\\
23	0.0947245907478176\\
24	0.10207192372652\\
25	0.105900429888404\\
26	0.105741899134656\\
27	0.101841410721385\\
28	0.0940947440794647\\
29	0.082725275935647\\
30	0.0681243547272631\\
31	0.0505782660739636\\
32	0.0943430712515486\\
33	0.0246994290617078\\
34	-0.0637137489415634\\
35	-0.0519251505469398\\
36	-0.0602660122467095\\
37	-0.0745803736069695\\
38	-0.0887493073482792\\
39	-0.100172106510432\\
40	-0.107835175985164\\
41	-0.111042154789332\\
42	-0.109594650171562\\
43	-0.103693228608381\\
44	-0.0927849550295056\\
45	-0.0789256149801698\\
46	-0.0612927748158287\\
47	-0.0413463354189086\\
48	-0.0195415713574043\\
49	0.00100463782285709\\
50	0.0221276699534937\\
51	0.0443750753137383\\
52	0.0620399907966086\\
53	0.0774667886598208\\
54	0.0898847404823612\\
55	0.0992641170342889\\
56	0.104729674074946\\
57	0.106246926201318\\
58	0.104012264350103\\
59	0.097866686314123\\
60	0.0879819901274997\\
61	0.0746754601307608\\
62	0.0354895898914054\\
63	0.0899270360340999\\
64	0.0954045025646298\\
65	0.0319474770924931\\
66	-0.117663676854774\\
67	-0.0773958883492131\\
68	-0.0754861192994645\\
69	-0.0853106361747212\\
70	-0.0965044527527023\\
71	-0.10539017854743\\
72	-0.11034962915426\\
73	-0.110806718649485\\
74	-0.106829191380048\\
75	-0.0980535500124932\\
76	-0.0851607689952353\\
77	-0.0689295378206389\\
78	-0.0498963064740069\\
79	-0.0283087659736756\\
80	-0.00869431547272148\\
81	0.0135755436601388\\
82	0.0342506908788711\\
83	0.0556789361389288\\
84	0.0715423995906386\\
85	0.0851546840004763\\
86	0.0958591714248238\\
87	0.102885859762525\\
88	0.106052357047541\\
89	0.105487199420508\\
90	0.100970960676051\\
91	0.0926258876570317\\
92	0.0807044086030396\\
93	0.0645833540739154\\
94	0.0860493712924999\\
95	0.132024243953427\\
96	0.130072606383223\\
97	-0.0462655450822756\\
98	-0.197056774470244\\
99	-0.10941131759403\\
100	-0.09166952573058\\
};
\addlegendentry{$\text{u}_{\text{1,1}}$};

\addplot [color=mycolor3,solid]
  table[row sep=crcr]{%
1	2.35661211394876\\
2	-1.42461937604012\\
3	-2.69362902330883\\
4	-6.79898510958896\\
5	8\\
6	-0.0252354807116152\\
7	-1.14274303755604\\
8	1.4030986766222\\
9	2.12459555554724\\
10	-0.553556520024329\\
11	-0.410053180645047\\
12	-0.512086674684963\\
13	-0.624063091613137\\
14	-1.01587942084752\\
15	0.118888140059841\\
16	0.0358417089519331\\
17	0.0465397431357431\\
18	0.0422627772332266\\
19	0.0446225063738079\\
20	0.0560158800265511\\
21	0.0712102419468153\\
22	0.0862922862277634\\
23	0.0989536574959375\\
24	0.108025900962322\\
25	0.112825263956051\\
26	0.112851310691334\\
27	0.108128765931069\\
28	0.0988279006716051\\
29	0.0853860526117819\\
30	0.0683926486907367\\
31	0.0487365300850772\\
32	0.106627123856253\\
33	-0.0272002725833642\\
34	-0.0557761822774831\\
35	-0.0464713702247711\\
36	-0.0592767173908956\\
37	-0.0750076980058788\\
38	-0.088771855720806\\
39	-0.0991437906659301\\
40	-0.105784678353068\\
41	-0.108401767505215\\
42	-0.106880035133055\\
43	-0.101563018038628\\
44	-0.0925457520440085\\
45	-0.0794992418419354\\
46	-0.0636274036996905\\
47	-0.0451585289833731\\
48	-0.0243020052327533\\
49	-0.00338446819912551\\
50	0.0187730591080539\\
51	0.0413916722936273\\
52	0.0611610781650925\\
53	0.0789660920879788\\
54	0.0937861168698994\\
55	0.104620426584634\\
56	0.111450918876069\\
57	0.113435941930636\\
58	0.110665169901012\\
59	0.103202336942825\\
60	0.0913926563534233\\
61	0.0757879277277473\\
62	0.133633735539507\\
63	0.0921145894679673\\
64	0.0139448320176313\\
65	-0.0939578386314368\\
66	-0.0727777216881382\\
67	-0.0562273434089761\\
68	-0.0680768737416832\\
69	-0.0828492168217202\\
70	-0.0949857139917517\\
71	-0.103416725777806\\
72	-0.107775500353983\\
73	-0.10803036858455\\
74	-0.104072740067847\\
75	-0.0965045742592464\\
76	-0.085423938981311\\
77	-0.0705768832642808\\
78	-0.0530783934890183\\
79	-0.0327215586425914\\
80	-0.0126285758064075\\
81	0.00977104558910234\\
82	0.0315345252957736\\
83	0.0537183405584654\\
84	0.0722250352642798\\
85	0.0881568461271081\\
86	0.100558827566344\\
87	0.109308793440871\\
88	0.113214028358938\\
89	0.112409402448846\\
90	0.106843890537877\\
91	0.0967631044402834\\
92	0.0826402684449111\\
93	0.156914676924059\\
94	0.138816432203538\\
95	0.0836022235156449\\
96	-0.083741263041827\\
97	-0.214977704588803\\
98	-0.0378814681753265\\
99	-0.0474521326340353\\
100	-0.0698474582919354\\
};
\addlegendentry{$\text{u}_{\text{1,2}}$};

\end{axis}
\end{tikzpicture}%
}
      \caption{The inputs signals directing agent 1 over time. Their value is
        constrained between $-8$ and $8$.}
      \label{fig:d_ON_2_2_inputs_agent_1}
    \end{figure}
  \end{minipage}
  \hfill
  \begin{minipage}{0.45\linewidth}
    \begin{figure}[H]
      \scalebox{0.6}{% This file was created by matlab2tikz.
%
%The latest updates can be retrieved from
%  http://www.mathworks.com/matlabcentral/fileexchange/22022-matlab2tikz-matlab2tikz
%where you can also make suggestions and rate matlab2tikz.
%
\definecolor{mycolor1}{rgb}{0.00000,1.00000,1.00000}%
\definecolor{mycolor2}{rgb}{0.00000,0.44700,0.74100}%
\definecolor{mycolor3}{rgb}{0.85000,0.32500,0.09800}%
%
\begin{tikzpicture}

\begin{axis}[%
width=4.133in,
height=3.26in,
at={(0.693in,0.44in)},
scale only axis,
xmin=0,
xmax=100,
xmajorgrids,
ymin=-9,
ymax=9,
ymajorgrids,
xlabel={time [iterations]},
axis background/.style={fill=white},
axis x line*=bottom,
axis y line*=left,
legend style={at={(0.705,0.636)},anchor=south west,legend cell align=left,align=left,draw=white!15!black}
]
\addplot [color=mycolor1,solid]
  table[row sep=crcr]{%
0	-8\\
100 -8\\
};
\addlegendentry{$\text{u}_{\text{max}}$};

\addplot [color=mycolor1,solid]
  table[row sep=crcr]{%
0	8\\
100 8\\
};
\addlegendentry{$\text{u}_{\text{min}}$};

\addplot [color=mycolor2,solid]
  table[row sep=crcr]{%
1	8\\
2	8\\
3	0.749052503697055\\
4	8\\
5	8\\
6	7.82909198833962\\
7	5.53867546850249\\
8	6.04678319180804\\
9	7.96071575121366\\
10	8\\
11	8\\
12	8\\
13	8\\
14	8\\
15	8\\
16	8\\
17	2.8729878120162\\
18	0.557966684492267\\
19	0.137691041427859\\
20	0.0618228525335849\\
21	0.0648086166229744\\
22	0.0797305394434244\\
23	0.0926234365819296\\
24	0.101366359545663\\
25	0.105580309581302\\
26	0.105511225707598\\
27	0.101704843072868\\
28	0.0939883128257405\\
29	0.0826223368995541\\
30	0.0680218927685399\\
31	0.0504680003133632\\
32	0.0301007340477179\\
33	0.0106660454327669\\
34	-0.0126350057199177\\
35	-0.0339849370477624\\
36	-0.0543386566215979\\
37	-0.0726840884105472\\
38	-0.0881650323555307\\
39	-0.0999996819489373\\
40	-0.107780139763549\\
41	-0.111011160337786\\
42	-0.10977664411957\\
43	-0.103325205066614\\
44	-0.0929257537494659\\
45	-0.0787404406162692\\
46	-0.0612002509150149\\
47	-0.0412277590360822\\
48	-0.0194070746589375\\
49	0.00111893240406995\\
50	0.022232653843033\\
51	0.0444916476792812\\
52	0.0621297605191854\\
53	0.0775338361635221\\
54	0.0899478464022247\\
55	0.0991190790235867\\
56	0.104769548418817\\
57	0.106411417096247\\
58	0.103875858510656\\
59	0.0977529627859463\\
60	0.0878782710982476\\
61	0.074571864782885\\
62	0.0581743394648081\\
63	0.0179499532714781\\
64	0.071652045248253\\
65	0.0460725084693395\\
66	-0.0777302821803384\\
67	-0.0647095478031645\\
68	-0.0715163086934932\\
69	-0.0840958726613499\\
70	-0.0961422382352343\\
71	-0.105280506871894\\
72	-0.110306812479751\\
73	-0.11078490951004\\
74	-0.106795087483008\\
75	-0.0980547129314212\\
76	-0.0849578181253173\\
77	-0.068899791890815\\
78	-0.0497942553058029\\
79	-0.0281844592983615\\
80	-0.00858918041638553\\
81	0.013677651514329\\
82	0.0343425853131338\\
83	0.0557836880065294\\
84	0.0716199506634562\\
85	0.0852090111777936\\
86	0.0957409216947225\\
87	0.102930641921765\\
88	0.106213448835596\\
89	0.105372284514097\\
90	0.100872915536343\\
91	0.0925341973493364\\
92	0.0806097395037381\\
93	0.0654160088810491\\
94	0.0807949984203845\\
95	0.0644846562722662\\
96	0.0752181379784675\\
97	-0.0109739471420994\\
98	-0.134466603922318\\
99	-0.0894972238659467\\
100	-0.0894972238659468\\
};
\addlegendentry{$\text{u}_{\text{2,1}}$};

\addplot [color=mycolor3,solid]
  table[row sep=crcr]{%
1	2.33553967832613\\
2	-8\\
3	8\\
4	1.63071460993305\\
5	-0.566245883440631\\
6	-2.79753940696045\\
7	5.81026774796186\\
8	0.94024408521065\\
9	-6.67814468368725\\
10	1.18103730496762\\
11	-0.612961050028013\\
12	-0.428550093717312\\
13	-0.550518357516959\\
14	-0.581545030317212\\
15	-0.717422420891024\\
16	-0.0294321765528332\\
17	0.935341695139122\\
18	0.886828027566842\\
19	0.327645036040629\\
20	0.148843505207763\\
21	0.101118267160349\\
22	0.095813675479085\\
23	0.101989729051518\\
24	0.108776050334233\\
25	0.113226957641614\\
26	0.113004526105841\\
27	0.108194057387099\\
28	0.0988797232004565\\
29	0.0854441576885152\\
30	0.0684591261731335\\
31	0.0488202455794712\\
32	0.0265941199141224\\
33	0.00688178872302542\\
34	-0.0179864862499206\\
35	-0.0387596566063429\\
36	-0.0580522867102584\\
37	-0.0749574314887324\\
38	-0.088827244356052\\
39	-0.099165299423876\\
40	-0.105785157373488\\
41	-0.108402811316995\\
42	-0.106744883600874\\
43	-0.10170616142117\\
44	-0.0924815506365459\\
45	-0.0796681775011335\\
46	-0.0637511056045636\\
47	-0.0452604616477739\\
48	-0.0243996206393985\\
49	-0.00347657343487866\\
50	0.0187015690553981\\
51	0.0412966848951878\\
52	0.0610790427661214\\
53	0.0788979989303582\\
54	0.09373717369713\\
55	0.104697103754319\\
56	0.111325051800423\\
57	0.113379180279558\\
58	0.110674425965566\\
59	0.103226300702665\\
60	0.0914372437124779\\
61	0.0758496210285437\\
62	0.057216422255423\\
63	0.104829631115382\\
64	0.0427990625857412\\
65	-0.0528846790597785\\
66	-0.0721389583379372\\
67	-0.0591281923338803\\
68	-0.0698434549354641\\
69	-0.0836292977458421\\
70	-0.0952745280823458\\
71	-0.103508715960509\\
72	-0.107804009942303\\
73	-0.107983711833433\\
74	-0.104130577315461\\
75	-0.0965796229973135\\
76	-0.085545135009429\\
77	-0.070616414360702\\
78	-0.0531433867156468\\
79	-0.0328053061224148\\
80	-0.0126982984174225\\
81	0.00968926995809281\\
82	0.0314618006866616\\
83	0.0536261285506236\\
84	0.0721554755043916\\
85	0.0881012545743342\\
86	0.100655069296501\\
87	0.109165857944257\\
88	0.113160549668016\\
89	0.112401883560389\\
90	0.106855333800474\\
91	0.0967963170150848\\
92	0.0826909433644709\\
93	0.065210848537286\\
94	0.127045888858432\\
95	0.0820812561689438\\
96	-0.0413400084173026\\
97	-0.123181020601522\\
98	-0.0502195214192875\\
99	-0.0562612116721654\\
100	-0.0562612116721654\\
};
\addlegendentry{$\text{u}_{\text{2,2}}$};

\end{axis}
\end{tikzpicture}%
}
      \caption{The inputs signals directing agent 2 over time. Their value is
        constrained between $-8$ and $8$.}
      \label{fig:d_ON_2_2_inputs_agent_2}
    \end{figure}
  \end{minipage}
\end{minipage}
}

%-------------------------------------------------------------------------------
\subsection{Energy of the system}
\label{subsection:d_ON_V_2_2}

\noindent\makebox[\linewidth][c]{%
\begin{minipage}{\linewidth}
  \begin{minipage}{0.45\linewidth}
    \begin{figure}[H]
      \scalebox{0.6}{% This file was created by matlab2tikz.
%
%The latest updates can be retrieved from
%  http://www.mathworks.com/matlabcentral/fileexchange/22022-matlab2tikz-matlab2tikz
%where you can also make suggestions and rate matlab2tikz.
%
\definecolor{mycolor1}{rgb}{0.00000,0.44700,0.74100}%
\definecolor{mycolor2}{rgb}{0.85000,0.32500,0.09800}%
\definecolor{mycolor3}{rgb}{1.00000,0.00000,1.00000}%
%
\begin{tikzpicture}

\begin{axis}[%
width=4.133in,
height=3.26in,
at={(0.693in,0.44in)},
scale only axis,
xmin=0,
xmax=100,
xmajorgrids,
ymin=0,
ymax=60,
ymajorgrids,
axis background/.style={fill=white},
legend style={legend cell align=left,align=left,draw=white!15!black}
]
\addplot [color=mycolor1,solid]
  table[row sep=crcr]{%
1	52.9128\\
2	45.7933148327525\\
3	39.5643264205968\\
4	33.9771358503653\\
5	30.4535593320497\\
6	25.2180676222914\\
7	20.5542151716812\\
8	16.5073376637802\\
9	12.7070425324801\\
10	9.29232909068961\\
11	6.51530861453138\\
12	4.23439680535812\\
13	2.45330168489064\\
14	1.16376458041373\\
15	0.383705181714741\\
16	0.0386690491832373\\
17	0.00463858364213219\\
18	0.00124044811908014\\
19	0.000789541917142893\\
20	0.000603838155494596\\
21	0.000449422831184455\\
22	0.000320288994727641\\
23	0.000230833172205893\\
24	0.000193703198275785\\
25	0.000214733412702286\\
26	0.000291074643294761\\
27	0.000411668523289761\\
28	0.000557988324156255\\
29	0.000707736245419558\\
30	0.000838108947357173\\
31	0.000929408998431883\\
32	0.000968129416084063\\
33	0.000979701917962921\\
34	0.000930433192856305\\
35	0.000776371238574415\\
36	0.000622619036257866\\
37	0.000471782311096627\\
38	0.000341320132360345\\
39	0.000247868357630047\\
40	0.000203366879899845\\
41	0.000212184634520403\\
42	0.000270828759405245\\
43	0.000368233040493023\\
44	0.000486916191259772\\
45	0.000607227505747259\\
46	0.000708417043528672\\
47	0.000773485621234676\\
48	0.000791014784198107\\
49	0.000757486652796928\\
50	0.000678065858712613\\
51	0.000566457192026254\\
52	0.00043936692652748\\
53	0.000320693526647387\\
54	0.000231698944878754\\
55	0.000189444859676935\\
56	0.000203611544614106\\
57	0.000274914013997471\\
58	0.00039534889070063\\
59	0.000548358123029875\\
60	0.000712357869229395\\
61	0.000863817736207294\\
62	0.000980937655495778\\
63	0.00104778595375908\\
64	0.00109289704724276\\
65	0.0011269524885195\\
66	0.00103076600853819\\
67	0.000781293554532506\\
68	0.000598255304359541\\
69	0.000439319123866099\\
70	0.000313136860037863\\
71	0.000231928108493619\\
72	0.000203005048126752\\
73	0.000226616442730245\\
74	0.000295170763165554\\
75	0.000394304818636667\\
76	0.00050526711208666\\
77	0.000607696060521911\\
78	0.000683371946101067\\
79	0.000718335134810706\\
80	0.000705939930766081\\
81	0.000647336766382039\\
82	0.000553173442343357\\
83	0.000440133710553358\\
84	0.00032584063862452\\
85	0.000235057460973606\\
86	0.000186354448430026\\
87	0.00019207450163124\\
88	0.000255827588952434\\
89	0.000372841818766233\\
90	0.00052895900211385\\
91	0.000703879020823926\\
92	0.000873945278589423\\
93	0.00101579757069975\\
94	0.00109411279878882\\
95	0.00121669191743388\\
96	0.00142006404111444\\
97	0.00149590832423762\\
98	0.00116260772473556\\
99	0.000762021320560704\\
100	0.000551836517740404\\
};
\addlegendentry{$\text{V}_\text{1}$};

\addplot [color=mycolor2,solid]
  table[row sep=crcr]{%
1	52.9128\\
2	45.7951305315887\\
3	39.9515540718125\\
4	38.3021027644722\\
5	32.2845574114086\\
6	26.9614445257535\\
7	22.4400372087634\\
8	18.9635785500736\\
9	16.1056107507204\\
10	12.8495333517029\\
11	9.53059613194501\\
12	6.78890778419011\\
13	4.51527683427096\\
14	2.71729421359142\\
15	1.38464012717291\\
16	0.512725695527091\\
17	0.0724489275354855\\
18	0.010118952295599\\
19	0.00181078087770487\\
20	0.000808066190591654\\
21	0.00052826218484184\\
22	0.00036313203295658\\
23	0.000254431270922396\\
24	0.00020160200136842\\
25	0.000207903085657705\\
26	0.000269816955398773\\
27	0.000377091739221984\\
28	0.000511498016756036\\
29	0.000651183560546286\\
30	0.000773735867546576\\
31	0.000859747834968435\\
32	0.000895899767647476\\
33	0.000876929462864635\\
34	0.000806139711493936\\
35	0.000694662030495311\\
36	0.000560313768610894\\
37	0.000424176022660283\\
38	0.000307315567884002\\
39	0.000227631484526472\\
40	0.000197257251541988\\
41	0.000220273559207136\\
42	0.00029271025418008\\
43	0.000402893789866389\\
44	0.000533223616933995\\
45	0.000662943507379919\\
46	0.000771332474825223\\
47	0.000840980066543026\\
48	0.000860318761390459\\
49	0.000825700388963141\\
50	0.000742308414076433\\
51	0.000623987412298298\\
52	0.000487712795934384\\
53	0.000357762593281023\\
54	0.000255857974811801\\
55	0.000199564003586135\\
56	0.000199250298385351\\
57	0.000256172882296872\\
58	0.000362623973373167\\
59	0.000503144543005069\\
60	0.000656292387170458\\
61	0.000798967311610867\\
62	0.000909717551589474\\
63	0.000972129464698728\\
64	0.000987888582379451\\
65	0.000974115679417983\\
66	0.000901597457962636\\
67	0.000711041130293335\\
68	0.00054592831251887\\
69	0.000399738826407367\\
70	0.000286013047784797\\
71	0.000217739072115284\\
72	0.000202053856920003\\
73	0.000238759319214197\\
74	0.000319823622831595\\
75	0.000430200264227361\\
76	0.000550755083764205\\
77	0.000660933362807145\\
78	0.000741960714898013\\
79	0.000779744043800003\\
80	0.000767467322111827\\
81	0.00070625883004365\\
82	0.00060685898407157\\
83	0.000486154017924623\\
84	0.000362076555678725\\
85	0.000259788308575083\\
86	0.000198336777914065\\
87	0.000190597196609031\\
88	0.000240829624332644\\
89	0.000344435908734463\\
90	0.000488350947520227\\
91	0.000652415433423462\\
92	0.000813406570402242\\
93	0.000948326626431853\\
94	0.00103799991525062\\
95	0.00106799768557678\\
96	0.00113177749826181\\
97	0.00112102653227822\\
98	0.000959813354651905\\
99	0.000696542657038867\\
100	0.000711041130293337\\
};
\addlegendentry{$\text{V}_\text{2}$};

\addplot [color=mycolor3,solid]
  table[row sep=crcr]{%
0	0.7117\\
100	0.7117\\
};
\addlegendentry{$\varepsilon_{\Psi}$};

\end{axis}
\end{tikzpicture}%
}
      \caption{The $\mat{P}-$norms of the errors of the three agents through time.}
      \label{}
    \end{figure}
  \end{minipage}
  \hfill
  \begin{minipage}{0.45\linewidth}
    \begin{figure}[H]
      \scalebox{0.6}{% This file was created by matlab2tikz.
%
%The latest updates can be retrieved from
%  http://www.mathworks.com/matlabcentral/fileexchange/22022-matlab2tikz-matlab2tikz
%where you can also make suggestions and rate matlab2tikz.
%
\definecolor{mycolor1}{rgb}{0.00000,0.44700,0.74100}%
\definecolor{mycolor2}{rgb}{0.85000,0.32500,0.09800}%
\definecolor{mycolor3}{rgb}{0.00000,1.00000,1.00000}%
%
\begin{tikzpicture}

\begin{axis}[%
width=4.133in,
height=3.26in,
at={(0.693in,0.44in)},
scale only axis,
xmin=1,
xmax=100,
xmajorgrids,
ymin=0,
ymax=0.01,
restrict y to domain=0:1,
ymajorgrids,
axis background/.style={fill=white},
legend style={legend cell align=left,align=left,draw=white!15!black}
]
\addplot [color=mycolor1,solid]
  table[row sep=crcr]{%
1	52.9128\\
2	45.7933148327525\\
3	39.5643264205968\\
4	33.9771358503653\\
5	30.4535593320497\\
6	25.2180676222914\\
7	20.5542151716812\\
8	16.5073376637802\\
9	12.7070425324801\\
10	9.29232909068961\\
11	6.51530861453138\\
12	4.23439680535812\\
13	2.45330168489064\\
14	1.16376458041373\\
15	0.383705181714741\\
16	0.0386690491832373\\
17	0.00463858364213219\\
18	0.00124044811908014\\
19	0.000789541917142893\\
20	0.000603838155494596\\
21	0.000449422831184455\\
22	0.000320288994727641\\
23	0.000230833172205893\\
24	0.000193703198275785\\
25	0.000214733412702286\\
26	0.000291074643294761\\
27	0.000411668523289761\\
28	0.000557988324156255\\
29	0.000707736245419558\\
30	0.000838108947357173\\
31	0.000929408998431883\\
32	0.000968129416084063\\
33	0.000979701917962921\\
34	0.000930433192856305\\
35	0.000776371238574415\\
36	0.000622619036257866\\
37	0.000471782311096627\\
38	0.000341320132360345\\
39	0.000247868357630047\\
40	0.000203366879899845\\
41	0.000212184634520403\\
42	0.000270828759405245\\
43	0.000368233040493023\\
44	0.000486916191259772\\
45	0.000607227505747259\\
46	0.000708417043528672\\
47	0.000773485621234676\\
48	0.000791014784198107\\
49	0.000757486652796928\\
50	0.000678065858712613\\
51	0.000566457192026254\\
52	0.00043936692652748\\
53	0.000320693526647387\\
54	0.000231698944878754\\
55	0.000189444859676935\\
56	0.000203611544614106\\
57	0.000274914013997471\\
58	0.00039534889070063\\
59	0.000548358123029875\\
60	0.000712357869229395\\
61	0.000863817736207294\\
62	0.000980937655495778\\
63	0.00104778595375908\\
64	0.00109289704724276\\
65	0.0011269524885195\\
66	0.00103076600853819\\
67	0.000781293554532506\\
68	0.000598255304359541\\
69	0.000439319123866099\\
70	0.000313136860037863\\
71	0.000231928108493619\\
72	0.000203005048126752\\
73	0.000226616442730245\\
74	0.000295170763165554\\
75	0.000394304818636667\\
76	0.00050526711208666\\
77	0.000607696060521911\\
78	0.000683371946101067\\
79	0.000718335134810706\\
80	0.000705939930766081\\
81	0.000647336766382039\\
82	0.000553173442343357\\
83	0.000440133710553358\\
84	0.00032584063862452\\
85	0.000235057460973606\\
86	0.000186354448430026\\
87	0.00019207450163124\\
88	0.000255827588952434\\
89	0.000372841818766233\\
90	0.00052895900211385\\
91	0.000703879020823926\\
92	0.000873945278589423\\
93	0.00101579757069975\\
94	0.00109411279878882\\
95	0.00121669191743388\\
96	0.00142006404111444\\
97	0.00149590832423762\\
98	0.00116260772473556\\
99	0.000762021320560704\\
100	0.000551836517740404\\
};
\addlegendentry{$\text{V}_\text{1}$};

\addplot [color=mycolor2,solid]
  table[row sep=crcr]{%
1	52.9128\\
2	45.7951305315887\\
3	39.9515540718125\\
4	38.3021027644722\\
5	32.2845574114086\\
6	26.9614445257535\\
7	22.4400372087634\\
8	18.9635785500736\\
9	16.1056107507204\\
10	12.8495333517029\\
11	9.53059613194501\\
12	6.78890778419011\\
13	4.51527683427096\\
14	2.71729421359142\\
15	1.38464012717291\\
16	0.512725695527091\\
17	0.0724489275354855\\
18	0.010118952295599\\
19	0.00181078087770487\\
20	0.000808066190591654\\
21	0.00052826218484184\\
22	0.00036313203295658\\
23	0.000254431270922396\\
24	0.00020160200136842\\
25	0.000207903085657705\\
26	0.000269816955398773\\
27	0.000377091739221984\\
28	0.000511498016756036\\
29	0.000651183560546286\\
30	0.000773735867546576\\
31	0.000859747834968435\\
32	0.000895899767647476\\
33	0.000876929462864635\\
34	0.000806139711493936\\
35	0.000694662030495311\\
36	0.000560313768610894\\
37	0.000424176022660283\\
38	0.000307315567884002\\
39	0.000227631484526472\\
40	0.000197257251541988\\
41	0.000220273559207136\\
42	0.00029271025418008\\
43	0.000402893789866389\\
44	0.000533223616933995\\
45	0.000662943507379919\\
46	0.000771332474825223\\
47	0.000840980066543026\\
48	0.000860318761390459\\
49	0.000825700388963141\\
50	0.000742308414076433\\
51	0.000623987412298298\\
52	0.000487712795934384\\
53	0.000357762593281023\\
54	0.000255857974811801\\
55	0.000199564003586135\\
56	0.000199250298385351\\
57	0.000256172882296872\\
58	0.000362623973373167\\
59	0.000503144543005069\\
60	0.000656292387170458\\
61	0.000798967311610867\\
62	0.000909717551589474\\
63	0.000972129464698728\\
64	0.000987888582379451\\
65	0.000974115679417983\\
66	0.000901597457962636\\
67	0.000711041130293335\\
68	0.00054592831251887\\
69	0.000399738826407367\\
70	0.000286013047784797\\
71	0.000217739072115284\\
72	0.000202053856920003\\
73	0.000238759319214197\\
74	0.000319823622831595\\
75	0.000430200264227361\\
76	0.000550755083764205\\
77	0.000660933362807145\\
78	0.000741960714898013\\
79	0.000779744043800003\\
80	0.000767467322111827\\
81	0.00070625883004365\\
82	0.00060685898407157\\
83	0.000486154017924623\\
84	0.000362076555678725\\
85	0.000259788308575083\\
86	0.000198336777914065\\
87	0.000190597196609031\\
88	0.000240829624332644\\
89	0.000344435908734463\\
90	0.000488350947520227\\
91	0.000652415433423462\\
92	0.000813406570402242\\
93	0.000948326626431853\\
94	0.00103799991525062\\
95	0.00106799768557678\\
96	0.00113177749826181\\
97	0.00112102653227822\\
98	0.000959813354651905\\
99	0.000696542657038867\\
100	0.000711041130293337\\
};
\addlegendentry{$\text{V}_\text{2}$};

\addplot [color=mycolor3,solid]
  table[row sep=crcr]{%
0	0.0035\\
100	0.0035\\
};
\addlegendentry{$\varepsilon_{\Omega}$};

\end{axis}
\end{tikzpicture}%
}
      \caption{The $\mat{P}-$norms of the errors of the three agents through time,
        focused. The colour cyan is used to illustrate the threshold
        $\varepsilon_{\Omega}$.}
      \label{}
    \end{figure}
  \end{minipage}
\end{minipage}
}
