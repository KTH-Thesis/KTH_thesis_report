The section with the agents' trajectories is omitted here, as
there is nothing interestingly different compared to those
in the case where disturbances were absent.

\section{Test case one: two agents $-$ one obstacle}

In this case the initial configurations of the two agents are
$\vect{z}_1$ $=$ $[-6, 3.5, 0]^{\top}$ and
$\vect{z}_2$ $=$ $[-6, 2.3, 0]^{\top}$.
Their desired configurations in steady-state are
$\vect{z}_{1,des} = [6, 3.5, 0]^{\top}$ and
$\vect{z}_{2,des} = [6, 2.3, 0]^{\top}$.
The related constants concerned with the execution
of this simulation are as follows: $L_{g_i} = 9.073$, $L_{V_i} = 0.0942$,
$\varepsilon_{\Psi_i} = 0.0649$ and $\varepsilon_{\Omega_i} = 0.0071$ for
all $i \in \mathcal{V}$.
The obstacle is placed between the two, at $[0, 2.9]^{\top}$. The penalty
matrices $\mat{Q}$, $\mat{R}$, $\mat{P}$ were set to
$\mat{Q} = 0.5 (I_3 + 0.5 \dagger_3)$, $\mat{R} = 0.005 I_2$ and
$\mat{P} = I_3 + 0.5 \dagger_3$, where $\dagger_N$ is a $N \times N$ matrix whose
elements are chosen at random between the values $0.0$ and $1.0$.

Subsections \ref{subsection:d_ON_errors_2_1} and \ref{subsection:d_ON_inputs_2_1}
illustrate the evolution of the error states and the input signals of the two agents
respectively. Subsection \ref{subsection:d_ON_distances_2_1} features the
figures relating to the evolution of the distance between the two agents
along with that between them and the obstacle. Subsection
\ref{subsection:d_ON_V_2_1} features the evolution of the quadratic function
$\vect{e}^{\top} \mat{P} \vect{e}$ through time for all three agents.

%-------------------------------------------------------------------------------
\subsection{State errors}
\label{subsection:d_ON_errors_2_1}

\noindent\makebox[\linewidth][c]{%
\begin{minipage}{\linewidth}
  \begin{minipage}{0.45\linewidth}
    \begin{figure}[H]
      \scalebox{0.6}{% This file was created by matlab2tikz.
%
%The latest updates can be retrieved from
%  http://www.mathworks.com/matlabcentral/fileexchange/22022-matlab2tikz-matlab2tikz
%where you can also make suggestions and rate matlab2tikz.
%
\definecolor{mycolor1}{rgb}{0.00000,0.44700,0.74100}%
\definecolor{mycolor2}{rgb}{0.85000,0.32500,0.09800}%
\definecolor{mycolor3}{rgb}{0.92900,0.69400,0.12500}%
%
\begin{tikzpicture}

\begin{axis}[%
width=4.133in,
height=3.26in,
at={(0.693in,0.44in)},
scale only axis,
xmin=1,
xmax=100,
xmajorgrids,
ymin=-1,
ymax=1,
ymajorgrids,
xlabel={time [iterations]},
ylabel={component magnitude},
axis background/.style={fill=white},
axis x line*=bottom,
axis y line*=left,
legend style={legend cell align=left,align=left,draw=white!15!black}
]
\addplot [color=mycolor1,solid]
  table[row sep=crcr]{%
1	-12\\
2	-11.0020879344844\\
3	-10.0059266675617\\
4	-9.00734027603941\\
5	-8.02172467048207\\
6	-7.0479597079904\\
7	-6.05221501143781\\
8	-5.06166144279791\\
9	-4.20105195281281\\
10	-3.25246237597068\\
11	-2.30101486094709\\
12	-1.3593239268567\\
13	-0.567240275020211\\
14	-0.208745278320353\\
15	-0.038853547261119\\
16	0.0508433506331672\\
17	0.0818026239561635\\
18	0.0623064611147452\\
19	0.0146795331498069\\
20	-0.00341214687245002\\
21	-0.0110030686090831\\
22	-0.014553468501571\\
23	-0.0163598672443341\\
24	-0.0172147713979644\\
25	-0.0174070923279572\\
26	-0.0168476350037753\\
27	-0.0156374659032625\\
28	-0.0138124304957658\\
29	-0.0113851006618936\\
30	-0.00842720487232232\\
31	-0.00510398934346732\\
32	-0.00144548469969384\\
33	0.00230262146637074\\
34	0.00611123760064174\\
35	0.0096397349292239\\
36	0.0126688198982627\\
37	0.0151460690705084\\
38	0.0169479667712581\\
39	0.0179881587677024\\
40	0.0182279707168161\\
41	0.017673504274255\\
42	0.0163730241052583\\
43	0.0143851729275796\\
44	0.0118569718195872\\
45	0.00890934467946317\\
46	0.0056836722362356\\
47	0.00231515150054097\\
48	-0.000927536927351005\\
49	-0.00418792354900735\\
50	-0.00756794948127916\\
51	-0.0103313185007057\\
52	-0.0127649140067192\\
53	-0.014547216540282\\
54	-0.0159506838977964\\
55	-0.0168581421096525\\
56	-0.0172428384192016\\
57	-0.0170049878570853\\
58	-0.0160431393922319\\
59	-0.0144622572836828\\
60	-0.0122637270895096\\
61	-0.00949995806352856\\
62	-0.00632522173660174\\
63	-0.00275073629998616\\
64	-0.00756128112615093\\
65	0.00996508440737142\\
66	0.0207281285284708\\
67	0.0173886313471786\\
68	0.0170331307057409\\
69	0.0177250408363538\\
70	0.0184634776028786\\
71	0.0187650435968083\\
72	0.0184201558801358\\
73	0.0173708732081776\\
74	0.0156474098493219\\
75	0.0133160399485258\\
76	0.0105158517996916\\
77	0.00737781693939987\\
78	0.00403259047477317\\
79	0.000628713426351475\\
80	-0.00260783728519184\\
81	-0.00584398991606017\\
82	-0.00884595464700738\\
83	-0.0115188851444183\\
84	-0.0137845229831169\\
85	-0.0152924724342553\\
86	-0.0163924468910469\\
87	-0.0170242553658102\\
88	-0.0169863534395016\\
89	-0.0163189267596454\\
90	-0.0150151983595692\\
91	-0.0130689459372679\\
92	-0.0105235888578034\\
93	-0.00745791877451285\\
94	-0.00403346073391947\\
95	-0.00945514028246371\\
96	0.0174277842444661\\
97	0.0508266838473874\\
98	0.0303055128914847\\
99	0.0223963252532742\\
100	0.0198757535813219\\
};
\addlegendentry{$\text{e}_{\text{1,1}}$};

\addplot [color=mycolor2,solid]
  table[row sep=crcr]{%
1	0\\
2	0.0689029654831547\\
3	0.187594901307665\\
4	0.303214351465324\\
5	0.508116479769482\\
6	0.774591851291735\\
7	0.93510758943837\\
8	0.854990150330949\\
9	0.733286494818328\\
10	0.653371442098809\\
11	0.50990503050784\\
12	0.315991734517668\\
13	0.149516898851886\\
14	0.0864301787285211\\
15	0.0633436854671989\\
16	0.0545190641173825\\
17	0.0526737453275298\\
18	0.0513324422098547\\
19	0.0470477050545468\\
20	0.0416301544949264\\
21	0.0347672622482801\\
22	0.0265419337912444\\
23	0.017277243871411\\
24	0.00735850597814213\\
25	-0.00280365334562136\\
26	-0.0127950976724238\\
27	-0.0222065957260312\\
28	-0.0306563084380567\\
29	-0.0378050972374492\\
30	-0.0433695403342108\\
31	-0.0471342375734041\\
32	-0.0489590424035134\\
33	-0.0487826430339885\\
34	-0.0466253228744424\\
35	-0.0425882066982226\\
36	-0.0368456767720126\\
37	-0.0296350821547191\\
38	-0.0212498068466172\\
39	-0.0120260755626055\\
40	-0.00232957941287529\\
41	0.0074583943545567\\
42	0.0169549558296628\\
43	0.025788831852172\\
44	0.0336149079947524\\
45	0.0401244624370454\\
46	0.0450568918638899\\
47	0.0482093678250503\\
48	0.0494454551453718\\
49	0.0487010015017122\\
50	0.0459919989842196\\
51	0.0414067495639011\\
52	0.0351176626224793\\
53	0.0273634312829562\\
54	0.0184548613327123\\
55	0.00874742927763013\\
56	-0.00136638089106844\\
57	-0.0114770684340423\\
58	-0.021175077441292\\
59	-0.0300651583415191\\
60	-0.037789189268284\\
61	-0.0440393422088994\\
62	-0.0485713233493625\\
63	-0.051213133160791\\
64	-0.0519140500789717\\
65	-0.0501410681548469\\
66	-0.0466573085580659\\
67	-0.0417409355556475\\
68	-0.0351715927397587\\
69	-0.0272670607938463\\
70	-0.018364472527225\\
71	-0.00882704934408446\\
72	0.00096469537860539\\
73	0.0106249803970489\\
74	0.019775519833874\\
75	0.0280580740222262\\
76	0.0351474652738193\\
77	0.0407618319836315\\
78	0.0446732744980742\\
79	0.0467169479894334\\
80	0.0467980686862925\\
81	0.0448982792292087\\
82	0.0410775467420521\\
83	0.0354738734606042\\
84	0.0282998665109107\\
85	0.0198315819880299\\
86	0.0104127187064723\\
87	0.000422318754036958\\
88	-0.00973750848731313\\
89	-0.0196525517969602\\
90	-0.0289201040343127\\
91	-0.0371657772717388\\
92	-0.044059082025193\\
93	-0.0493275902599559\\
94	-0.0527684774988828\\
95	-0.054305523878086\\
96	-0.0528569012213672\\
97	-0.0492342636841928\\
98	-0.045469227027958\\
99	-0.0396523850848615\\
100	-0.0323052012743396\\
};
\addlegendentry{$\text{e}_{\text{1,2}}$};

\addplot [color=mycolor3,solid]
  table[row sep=crcr]{%
1	0\\
2	0.136354137427072\\
3	0.0959831084463661\\
4	0.126440468946998\\
5	0.273775523269373\\
6	0.249703832565598\\
7	0.0553417255941795\\
8	-0.237706968107665\\
9	-0.0693882474298512\\
10	-0.12143680390748\\
11	-0.200520810994423\\
12	-0.227132107029498\\
13	-0.209347203883673\\
14	-0.177834963477421\\
15	-0.149813672347275\\
16	-0.105973779610936\\
17	-0.0421004323212959\\
18	0.0157118231391863\\
19	0.0194513431233624\\
20	0.000264616769817704\\
21	-0.00964725043631704\\
22	-0.0148736188204264\\
23	-0.0174893108285808\\
24	-0.0184527590248068\\
25	-0.0182334906414094\\
26	-0.0170577760789333\\
27	-0.0151058089006684\\
28	-0.0125160866425463\\
29	-0.00944929349758468\\
30	-0.00606345704475622\\
31	-0.00274324014817472\\
32	0.000801469954864693\\
33	0.00430234923045529\\
34	0.00770703327635361\\
35	0.0106732785655589\\
36	0.013075551890689\\
37	0.0150456850353415\\
38	0.0165023164007838\\
39	0.0174013174320794\\
40	0.0177044057910857\\
41	0.0173853908133828\\
42	0.0164328623517154\\
43	0.0148571627926578\\
44	0.0126542495806362\\
45	0.00990348096320805\\
46	0.0066861222410866\\
47	0.00312668890740956\\
48	-0.000610509294324318\\
49	-0.00434733694954511\\
50	-0.00819784350181009\\
51	-0.0115086190442537\\
52	-0.0143852915800636\\
53	-0.0164837027844319\\
54	-0.0178953047331866\\
55	-0.0185319891013038\\
56	-0.0183486281828056\\
57	-0.0173623712280691\\
58	-0.0156658952681713\\
59	-0.0132662274890235\\
60	-0.0103579823929482\\
61	-0.00708104580173683\\
62	-0.00384228618569613\\
63	-0.000311124578757628\\
64	0.0193500462052713\\
65	0.0356556308003023\\
66	0.0218831529091205\\
67	0.0168343466411654\\
68	0.016137387246197\\
69	0.0167674904474506\\
70	0.0175303297331236\\
71	0.0179524616983718\\
72	0.017846489173597\\
73	0.0171393829380513\\
74	0.0158116691965228\\
75	0.0138607048560219\\
76	0.0113407459329375\\
77	0.00831330255473623\\
78	0.00489136637731365\\
79	0.00120970746990966\\
80	-0.00252067530896893\\
81	-0.00624798910386866\\
82	-0.00977362577303395\\
83	-0.0128920174836624\\
84	-0.0154415395124986\\
85	-0.0170939447695919\\
86	-0.0180316130346509\\
87	-0.0181713289953755\\
88	-0.0174975539657646\\
89	-0.0160628411143736\\
90	-0.0139359470886639\\
91	-0.0112354146534871\\
92	-0.00810193947009598\\
93	-0.00468703895614119\\
94	-0.00136700186118963\\
95	0.0268785063847024\\
96	0.0450877388865616\\
97	0.0294730710237827\\
98	0.0175750340057538\\
99	0.015431965749045\\
100	0.0159172095200686\\
};
\addlegendentry{$\text{e}_{\text{1,3}}$};

\end{axis}
\end{tikzpicture}%
}
      \caption{The evolution of the error states of agent 1 over time.}
      \label{fig:d_ON_2_1_errors_agent_1}
    \end{figure}
  \end{minipage}
  \hfill
  \begin{minipage}{0.45\linewidth}
    \begin{figure}[H]
      \scalebox{0.6}{% This file was created by matlab2tikz.
%
%The latest updates can be retrieved from
%  http://www.mathworks.com/matlabcentral/fileexchange/22022-matlab2tikz-matlab2tikz
%where you can also make suggestions and rate matlab2tikz.
%
\definecolor{mycolor1}{rgb}{0.00000,0.44700,0.74100}%
\definecolor{mycolor2}{rgb}{0.85000,0.32500,0.09800}%
\definecolor{mycolor3}{rgb}{0.92900,0.69400,0.12500}%
%
\begin{tikzpicture}

\begin{axis}[%
width=4.133in,
height=3.26in,
at={(0.693in,0.44in)},
scale only axis,
xmin=1,
xmax=100,
xmajorgrids,
ymin=-2,
ymax=2.5,
ymajorgrids,
axis background/.style={fill=white},
axis x line*=bottom,
axis y line*=left,
legend style={legend cell align=left,align=left,draw=white!15!black}
]
\addplot [color=mycolor1,solid]
  table[row sep=crcr]{%
1	-12\\
2	-11.0025155780855\\
3	-10.0067830149439\\
4	-9.04331062409336\\
5	-8.31314286028116\\
6	-7.57380598224384\\
7	-7.01676427367618\\
8	-6.17128261562075\\
9	-5.22677936426417\\
10	-4.28198674052196\\
11	-3.33338463858648\\
12	-2.38641858405666\\
13	-1.45982580478962\\
14	-0.592901728822402\\
15	-0.175218996521479\\
16	0.0155458180305408\\
17	0.0814492252572294\\
18	0.0730768074857377\\
19	0.027659485553113\\
20	0.00517218303685139\\
21	-0.00625392854934139\\
22	-0.012010832979139\\
23	-0.0150665013042607\\
24	-0.016569558553366\\
25	-0.0170668505298616\\
26	-0.0166988747092271\\
27	-0.0155636108605857\\
28	-0.013762362231661\\
29	-0.0113440287932911\\
30	-0.00838855891675612\\
31	-0.00506405496608146\\
32	-0.00140349019811436\\
33	0.00234673727780162\\
34	0.0061575370735798\\
35	0.00968860209349263\\
36	0.0127204577410364\\
37	0.0152000586066926\\
38	0.0170039243169628\\
39	0.0180457081069549\\
40	0.0182866941589393\\
41	0.0177329116709174\\
42	0.016432582857734\\
43	0.0144441234048131\\
44	0.011910456500658\\
45	0.00896790147674494\\
46	0.00573970057447094\\
47	0.00236862511036366\\
48	-0.000877448085024831\\
49	-0.00414102161354713\\
50	-0.00752372815335668\\
51	-0.0102900761164661\\
52	-0.0127265431409371\\
53	-0.0145118560307464\\
54	-0.0159174152844596\\
55	-0.016826943452654\\
56	-0.0172006882896825\\
57	-0.016920468986705\\
58	-0.0159830661260608\\
59	-0.014418501921126\\
60	-0.0122255823345684\\
61	-0.00946287953743827\\
62	-0.00628634154953279\\
63	-0.00270967901486246\\
64	-0.00270618643174041\\
65	0.0190279513064084\\
66	0.034284757108574\\
67	0.0231655190963145\\
68	0.0195585701191076\\
69	0.0188371190381029\\
70	0.0189705006463705\\
71	0.019011092332268\\
72	0.0185526811829833\\
73	0.0174531297041144\\
74	0.0157068395467299\\
75	0.0133635592782063\\
76	0.010559665678478\\
77	0.00741799955660272\\
78	0.00407006799452955\\
79	0.000663824032288402\\
80	-0.00257520704466269\\
81	-0.00581350739861375\\
82	-0.00881762267275386\\
83	-0.0114924967744732\\
84	-0.0137600017465637\\
85	-0.0152698175088686\\
86	-0.0163713485502493\\
87	-0.0169960841730451\\
88	-0.0169630762486363\\
89	-0.0162946729674243\\
90	-0.0149927017414689\\
91	-0.0130467823234011\\
92	-0.010500710480397\\
93	-0.00743372392797708\\
94	-0.00400715122715268\\
95	-0.00405266004954686\\
96	0.0238158193519805\\
97	0.0546505946655873\\
98	0.0362596738032701\\
99	0.024782285086816\\
100	0.0208621808926534\\
};
\addlegendentry{$\text{e}_\text{1}$};

\addplot [color=mycolor2,solid]
  table[row sep=crcr]{%
1	0\\
2	0.0734493405915497\\
3	0.195383281647995\\
4	0.46573015758082\\
5	0.705112833886585\\
6	1.0673623498366\\
7	1.7230130897688\\
8	2.20403885464816\\
9	2.0557815100345\\
10	1.71078923470268\\
11	1.37696513179543\\
12	1.04016344255604\\
13	0.65403116762763\\
14	0.267401732547045\\
15	0.109245235997813\\
16	0.0596921376689867\\
17	0.0494812011685947\\
18	0.0483960983040716\\
19	0.0460519488598485\\
20	0.0410881506742685\\
21	0.0343002552335189\\
22	0.0260885188845493\\
23	0.0168295191652619\\
24	0.00691403862005501\\
25	-0.00324650009057758\\
26	-0.0132361064310308\\
27	-0.0226467216831958\\
28	-0.0310958676303905\\
29	-0.0382441109283438\\
30	-0.0438080485118995\\
31	-0.0475723194433145\\
32	-0.0493967886613313\\
33	-0.04922016703472\\
34	-0.0470627364050644\\
35	-0.0430256259081372\\
36	-0.0372832114445414\\
37	-0.030072829215219\\
38	-0.0216878465859082\\
39	-0.0124644707199715\\
40	-0.0027683764022702\\
41	0.00701916398813573\\
42	0.0165152756459844\\
43	0.0253486793397487\\
44	0.033174188444298\\
45	0.0396833344773872\\
46	0.0446153562404934\\
47	0.0477675121028986\\
48	0.0490033780035441\\
49	0.0482588054259644\\
50	0.0455497843782094\\
51	0.0409646616013608\\
52	0.0346758271726263\\
53	0.026921989560198\\
54	0.0180139071291142\\
55	0.00830705448788835\\
56	-0.00180634326925798\\
57	-0.0119171497646936\\
58	-0.0216139437749607\\
59	-0.0305028921607426\\
60	-0.0382261187861672\\
61	-0.0444756362916693\\
62	-0.0490071142357973\\
63	-0.0516485289774798\\
64	-0.0523030194782968\\
65	-0.0503765121465703\\
66	-0.0467874073713813\\
67	-0.0419723950597396\\
68	-0.0354375716039633\\
69	-0.0275481242340815\\
70	-0.0186516222440958\\
71	-0.00911662340741814\\
72	0.000674097638470058\\
73	0.0103338480089214\\
74	0.0194840147147401\\
75	0.0277662279663165\\
76	0.0348553419387259\\
77	0.0404694381909897\\
78	0.0443806529143333\\
79	0.0464241523578531\\
80	0.0465051675954086\\
81	0.0446053422919864\\
82	0.0407846552638454\\
83	0.0351811105285506\\
84	0.0280073154891398\\
85	0.0195393341166263\\
86	0.0101208336539157\\
87	0.000130699422856801\\
88	-0.0100286293645188\\
89	-0.0199432769564974\\
90	-0.0292103626702026\\
91	-0.0374555908867751\\
92	-0.0443484891211955\\
93	-0.0496166415633784\\
94	-0.0530572403985589\\
95	-0.0545259430485439\\
96	-0.053064874298164\\
97	-0.0496329339102015\\
98	-0.0456639576535213\\
99	-0.0397827411468923\\
100	-0.032418406827508\\
};
\addlegendentry{$\text{e}_\text{2}$};

\addplot [color=mycolor3,solid]
  table[row sep=crcr]{%
1	0\\
2	0.145492094313047\\
3	0.0933855810949526\\
4	0.447365369518162\\
5	0.175691844587656\\
6	0.726878981486458\\
7	1.00874555658001\\
8	0.018365087158827\\
9	-0.353755180859761\\
10	-0.371896082187484\\
11	-0.329062813452435\\
12	-0.376545771026098\\
13	-0.432827723317859\\
14	-0.423125403671976\\
15	-0.325883694870049\\
16	-0.212535212119303\\
17	-0.109511318447447\\
18	-0.0346572688615711\\
19	-0.0208871052155876\\
20	-0.0172721000561035\\
21	-0.0173130614031795\\
22	-0.0181916114365638\\
23	-0.0189125755430272\\
24	-0.019056220496669\\
25	-0.0184675665590513\\
26	-0.0171276735828986\\
27	-0.0150988413384514\\
28	-0.0124763711093395\\
29	-0.00939482655476086\\
30	-0.00600285938715657\\
31	-0.00268135836751927\\
32	0.00086294627583336\\
33	0.00436213098929583\\
34	0.00776474841744381\\
35	0.0107286814121144\\
36	0.0131280400860346\\
37	0.0150953824223149\\
38	0.0165493753197204\\
39	0.017446105674404\\
40	0.0177473716272179\\
41	0.0174270197693528\\
42	0.0164735879029829\\
43	0.014900593114318\\
44	0.0127043354393308\\
45	0.00995091228057111\\
46	0.00673256581837178\\
47	0.00317393605875815\\
48	-0.000561243329616692\\
49	-0.00429609478936252\\
50	-0.00814480676289307\\
51	-0.0114531592718063\\
52	-0.0143273544080035\\
53	-0.0164234176816058\\
54	-0.0178331534679063\\
55	-0.0184687757896019\\
56	-0.0182834435665139\\
57	-0.0173025359105412\\
58	-0.0155709143441468\\
59	-0.0131875094808596\\
60	-0.0102862286983344\\
61	-0.00701266846563071\\
62	-0.00377674218664663\\
63	-0.000247751646181056\\
64	0.0234864210446452\\
65	0.0352866126409965\\
66	0.0183734745654194\\
67	0.0144006728055928\\
68	0.0149405351678634\\
69	0.0161283531892577\\
70	0.017216291557199\\
71	0.0178121419984086\\
72	0.0177937009995102\\
73	0.0171289719087627\\
74	0.0158212255863991\\
75	0.0138813940874819\\
76	0.0113659861268496\\
77	0.00834103592488018\\
78	0.00492096386409153\\
79	0.00124099365871119\\
80	-0.00248767270536675\\
81	-0.00621360313807954\\
82	-0.00973751304939458\\
83	-0.0128544002845235\\
84	-0.0154024224198051\\
85	-0.017053359913468\\
86	-0.0179901176565013\\
87	-0.018127292379558\\
88	-0.0174595032838833\\
89	-0.016021720020405\\
90	-0.0138925655376046\\
91	-0.0111912109422883\\
92	-0.0080576784988853\\
93	-0.00464329945627855\\
94	-0.00132451676197667\\
95	0.027734906652175\\
96	0.0425620714284822\\
97	0.0252742255008026\\
98	0.00822122237916307\\
99	0.0107726451511536\\
100	0.0137609742308591\\
};
\addlegendentry{$\text{e}_\text{3}$};

\end{axis}
\end{tikzpicture}%}
      \caption{The evolution of the error states of agent 2 over time.}
      \label{fig:d_ON_e2_1_errors_agent_1}
    \end{figure}
  \end{minipage}
\end{minipage}
}


%-------------------------------------------------------------------------------
\subsection{Distances between actors}
\label{subsection:d_ON_distances_2_1}

\noindent\makebox[\linewidth][c]{%
\begin{minipage}{\linewidth}
  \begin{minipage}{0.45\linewidth}
    \begin{figure}[H]
      \scalebox{0.6}{% This file was created by matlab2tikz.
%
%The latest updates can be retrieved from
%  http://www.mathworks.com/matlabcentral/fileexchange/22022-matlab2tikz-matlab2tikz
%where you can also make suggestions and rate matlab2tikz.
%
\definecolor{mycolor1}{rgb}{0.00000,1.00000,1.00000}%
%
\begin{tikzpicture}

\begin{axis}[%
width=4.133in,
height=3.26in,
at={(0.693in,0.44in)},
scale only axis,
xmin=0,
xmax=100,
xmajorgrids,
ymin=0.8,
ymax=2.2,
ymajorgrids,
axis background/.style={fill=white},
axis x line*=bottom,
axis y line*=left,
legend style={at={(0.705,0.603)},anchor=south west,legend cell align=left,align=left,draw=white!15!black}
]
\addplot [color=mycolor1,solid]
  table[row sep=crcr]{%
0	1.01\\
100	1.01\\
};
\addlegendentry{$\text{d}_{\text{max}}$};

\addplot [color=mycolor1,solid]
  table[row sep=crcr]{%
0	2.01\\
100 2.01\\
};
\addlegendentry{$\text{d}_{\text{min}}$};

\addplot [color=blue,solid]
  table[row sep=crcr]{%
1	1.2\\
2	1.195453701381\\
3	1.19221192721025\\
4	1.03810756595803\\
5	1.04448115109855\\
6	1.04860844573022\\
7	1.04889330055182\\
8	1.11958682710872\\
9	1.03301585242218\\
10	1.03935080847165\\
11	1.08472869139608\\
12	1.13196112931488\\
13	1.13155173557831\\
14	1.08903404565209\\
15	1.16212682905938\\
16	1.19534819193989\\
17	1.20319259605862\\
18	1.2029845584422\\
19	1.20106589559527\\
20	1.20057269403302\\
21	1.20047640096017\\
22	1.20045610763432\\
23	1.2004484214441\\
24	1.20044464075202\\
25	1.20044289496236\\
26	1.2004410179759\\
27	1.20044012822907\\
28	1.20043956023646\\
29	1.20043901439351\\
30	1.20043850879976\\
31	1.20043808253415\\
32	1.20043774699236\\
33	1.20043752481135\\
34	1.20043741442348\\
35	1.20043742020455\\
36	1.20043753578315\\
37	1.20043774827459\\
38	1.2004380410435\\
39	1.20043839653683\\
40	1.20043879842572\\
41	1.2004392318364\\
42	1.20043968166116\\
43	1.20044015395988\\
44	1.20044072074194\\
45	1.20044112938784\\
46	1.20044153693091\\
47	1.20044185691314\\
48	1.20044207818681\\
49	1.20044219699199\\
50	1.20044221542051\\
51	1.200442088671\\
52	1.2004418360631\\
53	1.20044144224355\\
54	1.2004409546646\\
55	1.20044037519516\\
56	1.20043996311818\\
57	1.20044008430599\\
58	1.20043886783678\\
59	1.20043773461665\\
60	1.20043693012392\\
61	1.2004362946554\\
62	1.20043579151607\\
63	1.20043539651881\\
64	1.20039878782014\\
65	1.20026965993915\\
66	1.20020666397755\\
67	1.20024536192205\\
68	1.20026863570747\\
69	1.2002815786186\\
70	1.20028725680473\\
71	1.20028959928224\\
72	1.20029060505626\\
73	1.20029113520667\\
74	1.2002915065904\\
75	1.20029184699655\\
76	1.20029212413476\\
77	1.20029239446525\\
78	1.20029262216883\\
79	1.2002927961451\\
80	1.20029290153441\\
81	1.20029293732429\\
82	1.20029289181258\\
83	1.20029276322213\\
84	1.20029255127225\\
85	1.2002922480852\\
86	1.20029188523799\\
87	1.20029161966177\\
88	1.20029112110291\\
89	1.20029072540458\\
90	1.20029025884671\\
91	1.20028981381966\\
92	1.20028940731404\\
93	1.20028905154728\\
94	1.20028876318802\\
95	1.2002325780391\\
96	1.20022497292368\\
97	1.20040476080958\\
98	1.20020949981807\\
99	1.20013272780385\\
100	1.20011361094771\\
};
\addlegendentry{$\text{d}_{\text{12,a}}$};

\end{axis}
\end{tikzpicture}%
}
      \caption{The distance between the two agents over time. The maximum allowed
        distance has a value of $2.01$ and the minimum allowed distance a value
        of $1.01$.}
      \label{fig:d_ON_2_1_distance_agents}
    \end{figure}
  \end{minipage}
  \hfill
  \begin{minipage}{0.45\linewidth}
    \begin{figure}[H]
      \scalebox{0.6}{% This file was created by matlab2tikz.
%
%The latest updates can be retrieved from
%  http://www.mathworks.com/matlabcentral/fileexchange/22022-matlab2tikz-matlab2tikz
%where you can also make suggestions and rate matlab2tikz.
%
\definecolor{mycolor1}{rgb}{0.00000,0.44700,0.74100}%
\definecolor{mycolor2}{rgb}{0.85000,0.32500,0.09800}%
\definecolor{mycolor3}{rgb}{0.00000,1.00000,1.00000}%
%
\begin{tikzpicture}

\begin{axis}[%
width=4.133in,
height=3.26in,
at={(0.693in,0.44in)},
scale only axis,
xmin=0,
xmax=100,
xmajorgrids,
ymin=1.2,
ymax=7,
ymajorgrids,
xlabel={time [iterations]},
ylabel={distance},
axis background/.style={fill=white},
legend style={at={(0.705,0.525)},anchor=south west,legend cell align=left,align=left,draw=white!15!black}
]
\addplot [color=mycolor1,solid]
  table[row sep=crcr]{%
1	6.02992537267253\\
2	5.04661419979244\\
3	4.08261609687314\\
4	3.14004644879367\\
5	2.30549187289233\\
6	1.7285029092278\\
7	1.53599535109021\\
8	1.73132191850394\\
9	2.23916657570257\\
10	3.01991770870094\\
11	3.86191406373587\\
12	4.73021302623353\\
13	5.48421861444518\\
14	5.8317936899076\\
15	5.99794063625596\\
16	6.0861400295421\\
17	6.11672348358171\\
18	6.09719555031234\\
19	6.04938349082843\\
20	6.03081712004551\\
21	6.02254213114628\\
22	6.01814963059456\\
23	6.01539528494394\\
24	6.01353502079789\\
25	6.01232578752891\\
26	6.01189835410392\\
27	6.01219096149841\\
28	6.01320163111203\\
29	6.01494567899549\\
30	6.01737334955559\\
31	6.02033543333336\\
32	6.02381128605292\\
33	6.02755981590251\\
34	6.0315500272912\\
35	6.03543516665966\\
36	6.03898411403335\\
37	6.04212697415099\\
38	6.04471791078027\\
39	6.04664326836689\\
40	6.04783249099733\\
41	6.04825595563847\\
42	6.04792342731819\\
43	6.04685378278672\\
44	6.04515443154672\\
45	6.04290913716121\\
46	6.04022638356428\\
47	6.03721480174999\\
48	6.03412378199234\\
49	6.03080222238011\\
50	6.02715086444961\\
51	6.02391347319329\\
52	6.02082705451037\\
53	6.01824142901702\\
54	6.01592325691155\\
55	6.01403025634987\\
56	6.01263215774704\\
57	6.01187063702355\\
58	6.01188639285456\\
59	6.01261073022508\\
60	6.01407238646424\\
61	6.01624326348697\\
62	6.01898753388171\\
63	6.02230568438414\\
64	6.01745128837445\\
65	6.03506629298516\\
66	6.04610249101238\\
67	6.04322919668786\\
68	6.04348565197756\\
69	6.0449183358223\\
70	6.04650332986415\\
71	6.04772834274243\\
72	6.0483501665988\\
73	6.04827372829794\\
74	6.0474900127749\\
75	6.04602569798101\\
76	6.04398156080734\\
77	6.04145379530313\\
78	6.03854377961569\\
79	6.03537801361386\\
80	6.0321688052515\\
81	6.02874780230341\\
82	6.02535536014983\\
83	6.02210371106584\\
84	6.0190976449576\\
85	6.01671965281181\\
86	6.01466233771948\\
87	6.01302799941419\\
88	6.01205972224294\\
89	6.01175867337426\\
90	6.01216893670189\\
91	6.01332901215083\\
92	6.01522218907228\\
93	6.01779035012912\\
94	6.02088673529218\\
95	6.01534788499869\\
96	6.04225146027028\\
97	6.07584121371346\\
98	6.05574842253065\\
99	6.04840862938056\\
100	6.04658428148128\\
};
\addlegendentry{$\text{d}_{\text{1,o}}$};

\addplot [color=mycolor2,solid]
  table[row sep=crcr]{%
1	6.02992537267253\\
2	5.03015086313637\\
3	4.02716088797219\\
4	3.04627115426431\\
5	2.31552987454651\\
6	1.64173470323048\\
7	1.51491517518847\\
8	1.61315789110517\\
9	1.64838404399158\\
10	2.04583046308186\\
11	2.77750108942417\\
12	3.64029044250783\\
13	4.54049568768977\\
14	5.41731790849351\\
15	5.84541820376225\\
16	6.03976192203976\\
17	6.10631604383924\\
18	6.09807582553606\\
19	6.053060161368\\
20	6.03112555027967\\
21	6.02038272622755\\
22	6.01542921598412\\
23	6.0132783731944\\
24	6.01275234856852\\
25	6.01326828030317\\
26	6.01464470090599\\
27	6.01674063224258\\
28	6.01941218475479\\
29	6.0225705381178\\
30	6.02610126568151\\
31	6.02980985554013\\
32	6.03364535554842\\
33	6.0373548164568\\
34	6.04091206234537\\
35	6.04399197961968\\
36	6.04639869629242\\
37	6.0481090859191\\
38	6.04903562593562\\
39	6.04913108411108\\
40	6.04839684947021\\
41	6.04687817539635\\
42	6.04466007709504\\
43	6.04183434512196\\
44	6.03857258279038\\
45	6.03503521171502\\
46	6.0313648085305\\
47	6.02771845994803\\
48	6.02437288606427\\
49	6.02119118069568\\
50	6.01807169367797\\
51	6.01574147398701\\
52	6.01390346344136\\
53	6.01286011209692\\
54	6.01231670764198\\
55	6.01235896852167\\
56	6.01299080982173\\
57	6.01428992255997\\
58	6.01621663173672\\
59	6.01869751417756\\
60	6.02169204256988\\
61	6.02510446692947\\
62	6.02874892940686\\
63	6.03258957658322\\
64	6.03266378273186\\
65	6.05406368368881\\
66	6.06884885956201\\
67	6.05728086078642\\
68	6.05300473207318\\
69	6.05146403060785\\
70	6.05068059951577\\
71	6.04975351485746\\
72	6.04831942885099\\
73	6.04627557583921\\
74	6.04365184210184\\
75	6.0405291809502\\
76	6.0370701486082\\
77	6.03341905315676\\
78	6.02972389419646\\
79	6.02614408624143\\
80	6.02291131236829\\
81	6.01986170388453\\
82	6.01722428368637\\
83	6.01508457913039\\
84	6.01350521308689\\
85	6.01281381233929\\
86	6.01263413732613\\
87	6.01300083440669\\
88	6.01405568311572\\
89	6.01573429494701\\
90	6.01799119646244\\
91	6.02079383976905\\
92	6.02405898996587\\
93	6.02767387589208\\
94	6.03145206411956\\
95	6.03156602497084\\
96	6.05911303373004\\
97	6.0894020044915\\
98	6.07069294197922\\
99	6.05865690880127\\
100	6.05398500515137\\
};
\addlegendentry{$\text{d}_{\text{2,o}}$};

\addplot [color=mycolor3,solid]
  table[row sep=crcr]{%
0	1.51\\
100 1.51\\
};
\addlegendentry{$\text{d}_{\text{min}}$};

\end{axis}
\end{tikzpicture}%
}
      \caption{The distance between each agent and the obstacle over time. The
        minimum allowed distance has a value of $1.51$.}
      \label{fig:d_ON_2_1_distance_obstacle_agents}
    \end{figure}
  \end{minipage}
\end{minipage}
}


%-------------------------------------------------------------------------------
\subsection{Input signals}
\label{subsection:d_ON_inputs_2_1}

\noindent\makebox[\linewidth][c]{%
\begin{minipage}{\linewidth}
  \begin{minipage}{0.45\linewidth}
    \begin{figure}[H]
      \scalebox{0.6}{% This file was created by matlab2tikz.
%
%The latest updates can be retrieved from
%  http://www.mathworks.com/matlabcentral/fileexchange/22022-matlab2tikz-matlab2tikz
%where you can also make suggestions and rate matlab2tikz.
%
\definecolor{mycolor1}{rgb}{0.00000,1.00000,1.00000}%
\definecolor{mycolor2}{rgb}{0.00000,0.44700,0.74100}%
\definecolor{mycolor3}{rgb}{0.85000,0.32500,0.09800}%
%
\begin{tikzpicture}

\begin{axis}[%
width=4.133in,
height=3.26in,
at={(0.693in,0.44in)},
scale only axis,
xmin=0,
xmax=100,
xmajorgrids,
ymin=-11,
ymax=11,
ymajorgrids,
xlabel={time [iterations]},
axis background/.style={fill=white},
axis x line*=bottom,
axis y line*=left,
legend style={at={(0.697,0.634)},anchor=south west,legend cell align=left,align=left,draw=white!15!black}
]
\addplot [color=mycolor1,solid]
  table[row sep=crcr]{%
0	-10\\
100 -10\\
};
\addlegendentry{$\text{u}_{\text{max}}$};

\addplot [color=mycolor1,solid]
  table[row sep=crcr]{%
0	10\\
100 10\\
};
\addlegendentry{$\text{u}_{\text{min}}$};

\addplot [color=mycolor2,solid]
  table[row sep=crcr]{%
1	10\\
2	10\\
3	10\\
4	10\\
5	10\\
6	10\\
7	9.88563464881492\\
8	8.617970781367\\
9	9.43082904695664\\
10	9.54579411653452\\
11	9.54828897347126\\
12	8.0369360565408\\
13	3.59236261948567\\
14	1.67874054758745\\
15	0.880325536676882\\
16	0.306328944755582\\
17	-0.179253512608004\\
18	-0.441319071247606\\
19	-0.128029813103025\\
20	-0.00724727723974293\\
21	0.0461908711891062\\
22	0.0734095407652387\\
23	0.0890554575775679\\
24	0.0979190153221654\\
25	0.103692379881643\\
26	0.104542574783857\\
27	0.101348279853206\\
28	0.0947156723739461\\
29	0.0845584531071563\\
30	0.0705583055350402\\
31	0.0547710156800228\\
32	0.0358025927754855\\
33	0.0166102894699661\\
34	-0.00513278904702978\\
35	-0.0274586667939475\\
36	-0.0480076549272814\\
37	-0.0668912520582338\\
38	-0.0832537058678003\\
39	-0.0962688787733947\\
40	-0.105288610927618\\
41	-0.109847998214286\\
42	-0.109959000386839\\
43	-0.105007808826879\\
44	-0.0956690053233345\\
45	-0.0822783136338863\\
46	-0.0655427622956162\\
47	-0.0448516662360305\\
48	-0.0251013549133526\\
49	-0.00666959620988991\\
50	0.0180445394212661\\
51	0.0380701236391535\\
52	0.0588245929074031\\
53	0.0737991685561259\\
54	0.0864438232025384\\
55	0.095547055959881\\
56	0.101684333140799\\
57	0.104875885355708\\
58	0.103219147980359\\
59	0.0980637276995874\\
60	0.0893519695847831\\
61	0.0766384456922198\\
62	0.0620243710572437\\
63	-0.0414867896360675\\
64	0.162024981749064\\
65	0.0749688258949881\\
66	-0.0841920158708048\\
67	-0.0704098698822007\\
68	-0.0733352514711834\\
69	-0.0830723063277927\\
70	-0.0940379046377687\\
71	-0.103229872877285\\
72	-0.109022377267241\\
73	-0.110583525346555\\
74	-0.107759513696461\\
75	-0.100177721778053\\
76	-0.0884093059061953\\
77	-0.0730619343328041\\
78	-0.0546514377924147\\
79	-0.0331605222433453\\
80	-0.01330758832124\\
81	0.0081241026591478\\
82	0.0289847879361911\\
83	0.0484083752578333\\
84	0.0685042003605891\\
85	0.0817704056866515\\
86	0.0919394999616318\\
87	0.100205860452808\\
88	0.104089071134004\\
89	0.104155350409535\\
90	0.100650902150989\\
91	0.0934759701181015\\
92	0.0828023610761555\\
93	0.0684359690155335\\
94	-0.0393461403539833\\
95	0.26396831867213\\
96	0.309463171063702\\
97	-0.248745135657556\\
98	-0.13956371912648\\
99	-0.100250002879208\\
100	-0.0914249701536794\\
};
\addlegendentry{$\text{u}_{\text{1,1}}$};

\addplot [color=mycolor3,solid]
  table[row sep=crcr]{%
1	1.35357466319134\\
2	-0.433213081726238\\
3	0.256710915459715\\
4	1.40903609044249\\
5	-0.318919108777255\\
6	-2.03259334540993\\
7	-3.02668224280666\\
8	1.58360387417738\\
9	-0.619486850972193\\
10	-0.885312441796456\\
11	-0.352290100704846\\
12	0.103402732315297\\
13	0.255374885148664\\
14	0.237546117651611\\
15	0.414513849397488\\
16	0.634582333299252\\
17	0.593870846212467\\
18	0.0724150879644183\\
19	-0.138971911325582\\
20	-0.0304566265359456\\
21	0.0294277159203628\\
22	0.0653070555995963\\
23	0.0879556895594226\\
24	0.102018439021225\\
25	0.109838746636651\\
26	0.111946914701224\\
27	0.108985324902349\\
28	0.101104432733423\\
29	0.0888351837438188\\
30	0.0705275538203381\\
31	0.0536330062168194\\
32	0.033330203623393\\
33	0.0125706770591512\\
34	-0.0107550979122996\\
35	-0.03372488475196\\
36	-0.053074138459\\
37	-0.0703356798245091\\
38	-0.084653732183112\\
39	-0.0956213274197397\\
40	-0.102917876962661\\
41	-0.106352698204285\\
42	-0.105823892157001\\
43	-0.10174482872188\\
44	-0.0936941695451027\\
45	-0.0821922111327252\\
46	-0.0674510130965137\\
47	-0.0497966923328603\\
48	-0.0298656834316683\\
49	-0.0113742736779633\\
50	0.0125696160897144\\
51	0.0336362131665264\\
52	0.0556565852563023\\
53	0.0737070138401524\\
54	0.0891372984770748\\
55	0.101211411406675\\
56	0.109152152462715\\
57	0.112207760863548\\
58	0.11139605950152\\
59	0.105153875572186\\
60	0.0944801057829419\\
61	0.0772774388314807\\
62	0.0615909451714991\\
63	0.203232849451359\\
64	0.149754836048525\\
65	-0.170417671035434\\
66	-0.101269474584421\\
67	-0.0738150309931737\\
68	-0.0739435135376231\\
69	-0.0828161643953188\\
70	-0.0928175102181253\\
71	-0.100824195813705\\
72	-0.105583879000294\\
73	-0.106610905014541\\
74	-0.103943439176237\\
75	-0.0973673041828961\\
76	-0.0872989688581445\\
77	-0.0738273287122778\\
78	-0.0574289432223596\\
79	-0.038098819633525\\
80	-0.0182190736740477\\
81	0.00288712828891981\\
82	0.0245283478037947\\
83	0.0455647423594786\\
84	0.0670506713465768\\
85	0.0833809420292917\\
86	0.0968454120526324\\
87	0.106548647626267\\
88	0.111747206009294\\
89	0.112375164191684\\
90	0.108185577079603\\
91	0.099352632857835\\
92	0.0862928561859735\\
93	0.0673913812266591\\
94	0.297330164283001\\
95	0.177058455818324\\
96	-0.180888814372683\\
97	-0.162444381595988\\
98	-0.0818837966770668\\
99	-0.0701797042052921\\
100	-0.0759990736352368\\
};
\addlegendentry{$\text{u}_{\text{1,2}}$};

\end{axis}
\end{tikzpicture}%
}
      \caption{The inputs signals directing agent 1 over time. Their value is
        constrained between $-10$ and $10$.}
      \label{fig:d_ON_2_1_inputs_agent_1}
    \end{figure}
  \end{minipage}
  \hfill
  \begin{minipage}{0.45\linewidth}
    \begin{figure}[H]
      \scalebox{0.6}{% This file was created by matlab2tikz.
%
%The latest updates can be retrieved from
%  http://www.mathworks.com/matlabcentral/fileexchange/22022-matlab2tikz-matlab2tikz
%where you can also make suggestions and rate matlab2tikz.
%
\definecolor{mycolor1}{rgb}{0.00000,0.44700,0.74100}%
\definecolor{mycolor2}{rgb}{0.85000,0.32500,0.09800}%
%
\begin{tikzpicture}

\begin{axis}[%
width=4.133in,
height=3.26in,
at={(0.693in,0.44in)},
scale only axis,
xmin=1,
xmax=30,
xmajorgrids,
ymin=-11,
ymax=11,
ymajorgrids,
axis background/.style={fill=white},
axis x line*=bottom,
axis y line*=left,
legend style={legend cell align=left,align=left,draw=white!15!black}
]
\addplot [color=mycolor1,solid]
  table[row sep=crcr]{%
1	10\\
2	10\\
3	10\\
4	10\\
5	6.39502727333906\\
6	9.26117667051881\\
7	10\\
8	10\\
9	10\\
10	10\\
11	10\\
12	10\\
13	10\\
14	5.03664982982595\\
15	1.76150438951348\\
16	0.456755785072588\\
17	0.0364939937185682\\
18	-0.053013623162716\\
19	-0.0471726037375023\\
20	-0.0265529264844818\\
21	-0.0118668471519676\\
22	-0.00425305409491201\\
23	-0.00198774644576558\\
24	-0.00111671976246029\\
25	-0.000848054125706662\\
26	-0.000428643583212237\\
27	-0.000462561746358677\\
28	-0.000241444708741716\\
29	-0.000252791355176646\\
30	-0.000105560470847361\\
};
\addlegendentry{$\text{u}_\text{1}$};

\addplot [color=mycolor2,solid]
  table[row sep=crcr]{%
1	1.18896306588512\\
2	-0.554929637943449\\
3	0.531715506550048\\
4	3.25138271839666\\
5	6.01072465942911\\
6	-0.00947211052508098\\
7	-10\\
8	-4.37119939505131\\
9	0.519509515987139\\
10	-0.225162067041581\\
11	-0.212762059326957\\
12	-0.29875721286155\\
13	0.30083656311633\\
14	1.22842124126557\\
15	1.2403291698237\\
16	0.785218948224705\\
17	0.387124542229106\\
18	0.159422961055464\\
19	0.0546330812512465\\
20	0.0145394010805992\\
21	0.00242247163636238\\
22	-0.000403453774772748\\
23	-0.000289072252966553\\
24	-0.000189559885506888\\
25	-0.00021596819873607\\
26	-6.01178646140153e-05\\
27	-0.000120711378976395\\
28	-3.02985185867415e-05\\
29	-6.43983025829581e-05\\
30	-8.76032382184133e-06\\
};
\addlegendentry{$\text{u}_\text{2}$};

\end{axis}
\end{tikzpicture}%}
      \caption{The inputs signals directing agent 2 over time. Their value is
        constrained between $-10$ and $10$.}
      \label{fig:d_ON_2_1_inputs_agent_2}
    \end{figure}
  \end{minipage}
\end{minipage}
}


%-------------------------------------------------------------------------------
\subsection{Energy of the system}
\label{subsection:d_ON_V_2_1}

\noindent\makebox[\linewidth][c]{%
\begin{minipage}{\linewidth}
  \begin{minipage}{0.45\linewidth}
    \begin{figure}[H]
      \scalebox{0.6}{% This file was created by matlab2tikz.
%
%The latest updates can be retrieved from
%  http://www.mathworks.com/matlabcentral/fileexchange/22022-matlab2tikz-matlab2tikz
%where you can also make suggestions and rate matlab2tikz.
%
\definecolor{mycolor1}{rgb}{0.00000,0.44700,0.74100}%
\definecolor{mycolor2}{rgb}{0.85000,0.32500,0.09800}%
%
\begin{tikzpicture}

\begin{axis}[%
width=4.133in,
height=3.26in,
at={(0.693in,0.44in)},
scale only axis,
xmin=0,
xmax=100,
xmajorgrids,
ymin=0,
ymax=120,
ymajorgrids,
axis background/.style={fill=white},
legend style={legend cell align=left,align=left,draw=white!15!black}
]
\addplot [color=mycolor1,solid]
  table[row sep=crcr]{%
1	105.8256\\
2	88.5197703883773\\
3	73.0757661228202\\
4	58.9843550950124\\
5	46.3713470904517\\
6	35.6895969856726\\
7	26.5029577140376\\
8	18.8596636668326\\
9	12.8769774330224\\
10	7.79198963677452\\
11	3.98261675183155\\
12	1.44496040345556\\
13	0.289213059582676\\
14	0.062393388548671\\
15	0.0194258150986443\\
16	0.0104879775111681\\
17	0.00782708964388554\\
18	0.00591086880945457\\
19	0.00243361611974285\\
20	0.00125921499620399\\
21	0.00093229679727065\\
22	0.000733437503499326\\
23	0.000590541321784187\\
24	0.000525865351944483\\
25	0.000560003386923408\\
26	0.000687167547900528\\
27	0.000890475898954524\\
28	0.00113639668115087\\
29	0.00138341293371106\\
30	0.00159081460882521\\
31	0.00172930174503505\\
32	0.00176777477239631\\
33	0.0017057235734831\\
34	0.0015549832572817\\
35	0.0013415246510069\\
36	0.00109585813418824\\
37	0.000858798604741566\\
38	0.000667767160581342\\
39	0.000552797454256946\\
40	0.000531866243874965\\
41	0.000608331130990726\\
42	0.000770519151400696\\
43	0.000992842053285007\\
44	0.00124103908100298\\
45	0.00147695817951453\\
46	0.00166355238070082\\
47	0.00177124168755246\\
48	0.00178347468227451\\
49	0.00169437739622965\\
50	0.00152069720853092\\
51	0.00128825823835884\\
52	0.00103504030195128\\
53	0.000794364591855559\\
54	0.000614344764704369\\
55	0.000525345112868331\\
56	0.000544507192173329\\
57	0.000669477619722971\\
58	0.00088085241137267\\
59	0.00114691738251924\\
60	0.00142802813609124\\
61	0.00168061543434384\\
62	0.00187166542163392\\
63	0.00196143858750091\\
64	0.00213016698782692\\
65	0.00247716462731309\\
66	0.00198517970520583\\
67	0.00150179558547883\\
68	0.0011507856114707\\
69	0.000868811785873721\\
70	0.000667963253956648\\
71	0.000563544097308099\\
72	0.000561340707351126\\
73	0.000654374942896707\\
74	0.000822804477445303\\
75	0.00103523137438299\\
76	0.00125606961871475\\
77	0.00144745782900755\\
78	0.0015769316898952\\
79	0.0016221785585538\\
80	0.0015748654441193\\
81	0.001439346189043\\
82	0.0012400004939556\\
83	0.00101059983239069\\
84	0.000790838761108283\\
85	0.000608840354512352\\
86	0.00051100505959537\\
87	0.000519363933016718\\
88	0.000634611817633899\\
89	0.000846662895274051\\
90	0.00112713613441192\\
91	0.00143614104772456\\
92	0.00172943970520515\\
93	0.00196496331483187\\
94	0.0021131972896462\\
95	0.00250138714821464\\
96	0.00328881693201465\\
97	0.00390109802931801\\
98	0.00212882038295638\\
99	0.00149130150728147\\
100	0.00109279590684254\\
};
\addlegendentry{$\text{V}_\text{1}$};

\addplot [color=mycolor2,solid]
  table[row sep=crcr]{%
1	105.8256\\
2	88.4988808217006\\
3	73.0820114293456\\
4	58.8379140293222\\
5	49.8523764936694\\
6	40.9303879119087\\
7	35.8131971502958\\
8	29.1405831004297\\
9	21.625324864333\\
10	14.6452690847422\\
11	8.97098930021547\\
12	4.75899254726218\\
13	1.92730100206102\\
14	0.445176293699289\\
15	0.110974995741141\\
16	0.032851299186318\\
17	0.0131761039106419\\
18	0.00626614515274698\\
19	0.00234551887868292\\
20	0.00135273362079745\\
21	0.000977502668755061\\
22	0.000743874489871317\\
23	0.000589109604074539\\
24	0.000524372640087532\\
25	0.000562165400912033\\
26	0.000696091399794357\\
27	0.000905120751557018\\
28	0.00115601884142373\\
29	0.00140732031333358\\
30	0.00161808853023366\\
31	0.00175885921104426\\
32	0.00179844409922636\\
33	0.00173628876437514\\
34	0.00158422400369226\\
35	0.00136827987879705\\
36	0.00111906661655291\\
37	0.000877537182104774\\
38	0.000681285588445642\\
39	0.000560552875074352\\
40	0.00053354230693129\\
41	0.000603849124086834\\
42	0.00076004050860511\\
43	0.000976855169631668\\
44	0.00122013750640919\\
45	0.00145197048749996\\
46	0.00163533934883739\\
47	0.00174098838060901\\
48	0.00175242071474737\\
49	0.00166379250034198\\
50	0.00149183942932662\\
51	0.00126232199930783\\
52	0.00101310230971752\\
53	0.000777350598414899\\
54	0.000602967055403644\\
55	0.000520116036605207\\
56	0.000545258355114128\\
57	0.000675440867986194\\
58	0.000892531348309225\\
59	0.00116515335712637\\
60	0.00145144189429687\\
61	0.00170802660751181\\
62	0.00190190396180502\\
63	0.0019933161357493\\
64	0.00218117576814105\\
65	0.00265733528159595\\
66	0.00239674541228951\\
67	0.00161768767604294\\
68	0.00120008321384223\\
69	0.000892330529635885\\
70	0.000679895256260982\\
71	0.000568723276559871\\
72	0.000561508605341029\\
73	0.000650211421779837\\
74	0.000814713569809419\\
75	0.00102364918194269\\
76	0.00124153438542621\\
77	0.00143053360268434\\
78	0.00155834392130059\\
79	0.00160272019561704\\
80	0.00155536793179367\\
81	0.00142064877605188\\
82	0.00122291523254841\\
83	0.000995876929719734\\
84	0.000779134927597963\\
85	0.000600692836145254\\
86	0.000506811917685846\\
87	0.000519058552966314\\
88	0.000638852594929042\\
89	0.000854884385718898\\
90	0.00113916770603491\\
91	0.00145155275268497\\
92	0.00174766079012039\\
93	0.00198532443165766\\
94	0.00213494972590831\\
95	0.00246776994187737\\
96	0.00334986162085096\\
97	0.00403560956389923\\
98	0.00224308706167746\\
99	0.00149525699912459\\
100	0.00108228547300793\\
};
\addlegendentry{$\text{V}_\text{2}$};

\end{axis}
\end{tikzpicture}%}
      \caption{The $\mat{P}-$norms of the errors of the three agents through time.}
      \label{}
    \end{figure}
  \end{minipage}
  \hfill
  \begin{minipage}{0.45\linewidth}
    \begin{figure}[H]
      \scalebox{0.6}{% This file was created by matlab2tikz.
%
%The latest updates can be retrieved from
%  http://www.mathworks.com/matlabcentral/fileexchange/22022-matlab2tikz-matlab2tikz
%where you can also make suggestions and rate matlab2tikz.
%
\definecolor{mycolor1}{rgb}{0.00000,0.44700,0.74100}%
\definecolor{mycolor2}{rgb}{0.85000,0.32500,0.09800}%
\definecolor{mycolor3}{rgb}{1.00000,0.00000,1.00000}%
\definecolor{mycolor4}{rgb}{0.00000,1.00000,1.00000}%
%
\begin{tikzpicture}

\begin{axis}[%
width=4.133in,
height=3.26in,
at={(0.693in,0.44in)},
scale only axis,
xmin=1,
xmax=100,
xmajorgrids,
ymin=0,
ymax=0.08,
restrict y to domain=0:1,
ymajorgrids,
axis background/.style={fill=white},
legend style={at={(0.699,0.545)},anchor=south west,legend cell align=left,align=left,draw=white!15!black}
]
\addplot [color=mycolor1,solid]
  table[row sep=crcr]{%
1	105.8256\\
2	88.5197703883773\\
3	73.0757661228202\\
4	58.9843550950124\\
5	46.3713470904517\\
6	35.6895969856726\\
7	26.5029577140376\\
8	18.8596636668326\\
9	12.8769774330224\\
10	7.79198963677452\\
11	3.98261675183155\\
12	1.44496040345556\\
13	0.289213059582676\\
14	0.062393388548671\\
15	0.0194258150986443\\
16	0.0104879775111681\\
17	0.00782708964388554\\
18	0.00591086880945457\\
19	0.00243361611974285\\
20	0.00125921499620399\\
21	0.00093229679727065\\
22	0.000733437503499326\\
23	0.000590541321784187\\
24	0.000525865351944483\\
25	0.000560003386923408\\
26	0.000687167547900528\\
27	0.000890475898954524\\
28	0.00113639668115087\\
29	0.00138341293371106\\
30	0.00159081460882521\\
31	0.00172930174503505\\
32	0.00176777477239631\\
33	0.0017057235734831\\
34	0.0015549832572817\\
35	0.0013415246510069\\
36	0.00109585813418824\\
37	0.000858798604741566\\
38	0.000667767160581342\\
39	0.000552797454256946\\
40	0.000531866243874965\\
41	0.000608331130990726\\
42	0.000770519151400696\\
43	0.000992842053285007\\
44	0.00124103908100298\\
45	0.00147695817951453\\
46	0.00166355238070082\\
47	0.00177124168755246\\
48	0.00178347468227451\\
49	0.00169437739622965\\
50	0.00152069720853092\\
51	0.00128825823835884\\
52	0.00103504030195128\\
53	0.000794364591855559\\
54	0.000614344764704369\\
55	0.000525345112868331\\
56	0.000544507192173329\\
57	0.000669477619722971\\
58	0.00088085241137267\\
59	0.00114691738251924\\
60	0.00142802813609124\\
61	0.00168061543434384\\
62	0.00187166542163392\\
63	0.00196143858750091\\
64	0.00213016698782692\\
65	0.00247716462731309\\
66	0.00198517970520583\\
67	0.00150179558547883\\
68	0.0011507856114707\\
69	0.000868811785873721\\
70	0.000667963253956648\\
71	0.000563544097308099\\
72	0.000561340707351126\\
73	0.000654374942896707\\
74	0.000822804477445303\\
75	0.00103523137438299\\
76	0.00125606961871475\\
77	0.00144745782900755\\
78	0.0015769316898952\\
79	0.0016221785585538\\
80	0.0015748654441193\\
81	0.001439346189043\\
82	0.0012400004939556\\
83	0.00101059983239069\\
84	0.000790838761108283\\
85	0.000608840354512352\\
86	0.00051100505959537\\
87	0.000519363933016718\\
88	0.000634611817633899\\
89	0.000846662895274051\\
90	0.00112713613441192\\
91	0.00143614104772456\\
92	0.00172943970520515\\
93	0.00196496331483187\\
94	0.0021131972896462\\
95	0.00250138714821464\\
96	0.00328881693201465\\
97	0.00390109802931801\\
98	0.00212882038295638\\
99	0.00149130150728147\\
100	0.00109279590684254\\
};
\addlegendentry{$\text{V}_\text{1}$};

\addplot [color=mycolor2,solid]
  table[row sep=crcr]{%
1	105.8256\\
2	88.4988808217006\\
3	73.0820114293456\\
4	58.8379140293222\\
5	49.8523764936694\\
6	40.9303879119087\\
7	35.8131971502958\\
8	29.1405831004297\\
9	21.625324864333\\
10	14.6452690847422\\
11	8.97098930021547\\
12	4.75899254726218\\
13	1.92730100206102\\
14	0.445176293699289\\
15	0.110974995741141\\
16	0.032851299186318\\
17	0.0131761039106419\\
18	0.00626614515274698\\
19	0.00234551887868292\\
20	0.00135273362079745\\
21	0.000977502668755061\\
22	0.000743874489871317\\
23	0.000589109604074539\\
24	0.000524372640087532\\
25	0.000562165400912033\\
26	0.000696091399794357\\
27	0.000905120751557018\\
28	0.00115601884142373\\
29	0.00140732031333358\\
30	0.00161808853023366\\
31	0.00175885921104426\\
32	0.00179844409922636\\
33	0.00173628876437514\\
34	0.00158422400369226\\
35	0.00136827987879705\\
36	0.00111906661655291\\
37	0.000877537182104774\\
38	0.000681285588445642\\
39	0.000560552875074352\\
40	0.00053354230693129\\
41	0.000603849124086834\\
42	0.00076004050860511\\
43	0.000976855169631668\\
44	0.00122013750640919\\
45	0.00145197048749996\\
46	0.00163533934883739\\
47	0.00174098838060901\\
48	0.00175242071474737\\
49	0.00166379250034198\\
50	0.00149183942932662\\
51	0.00126232199930783\\
52	0.00101310230971752\\
53	0.000777350598414899\\
54	0.000602967055403644\\
55	0.000520116036605207\\
56	0.000545258355114128\\
57	0.000675440867986194\\
58	0.000892531348309225\\
59	0.00116515335712637\\
60	0.00145144189429687\\
61	0.00170802660751181\\
62	0.00190190396180502\\
63	0.0019933161357493\\
64	0.00218117576814105\\
65	0.00265733528159595\\
66	0.00239674541228951\\
67	0.00161768767604294\\
68	0.00120008321384223\\
69	0.000892330529635885\\
70	0.000679895256260982\\
71	0.000568723276559871\\
72	0.000561508605341029\\
73	0.000650211421779837\\
74	0.000814713569809419\\
75	0.00102364918194269\\
76	0.00124153438542621\\
77	0.00143053360268434\\
78	0.00155834392130059\\
79	0.00160272019561704\\
80	0.00155536793179367\\
81	0.00142064877605188\\
82	0.00122291523254841\\
83	0.000995876929719734\\
84	0.000779134927597963\\
85	0.000600692836145254\\
86	0.000506811917685846\\
87	0.000519058552966314\\
88	0.000638852594929042\\
89	0.000854884385718898\\
90	0.00113916770603491\\
91	0.00145155275268497\\
92	0.00174766079012039\\
93	0.00198532443165766\\
94	0.00213494972590831\\
95	0.00246776994187737\\
96	0.00334986162085096\\
97	0.00403560956389923\\
98	0.00224308706167746\\
99	0.00149525699912459\\
100	0.00108228547300793\\
};
\addlegendentry{$\text{V}_\text{2}$};

\addplot [color=mycolor3,solid]
  table[row sep=crcr]{%
0	0.0649\\
100	0.0649\\
};
\addlegendentry{$\varepsilon_{\Psi}$};

\addplot [color=mycolor4,solid]
  table[row sep=crcr]{%
0	0.0071\\
100	0.0071\\
};
\addlegendentry{$\varepsilon_{\Omega}$};

\end{axis}
\end{tikzpicture}%
}
      \caption{The $\mat{P}-$norms of the errors of the three agents through time,
        focused. The colour magenta is used to illustrate the threshold
        $\varepsilon_{\Psi}$, while cyan is used for $\varepsilon_{\Omega}$.}
      \label{}
    \end{figure}
  \end{minipage}
\end{minipage}
}
