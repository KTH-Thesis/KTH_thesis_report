\section{Proof of lemma \getrefnumber{lemma:V_Lipschitz_e_0}}

For every $\vect{e}_1, \vect{e}_2 \in \Omega_i$, it holds that
\begin{align}
  \big|V_i(\vect{e}_1) - V_i(\vect{e}_2)\big| &= \big|\vect{e}_1^{\top} \mat{P}_i \vect{e}_1 - \vect{e}_2^{\top} \mat{P}_i \vect{e}_2\big| \\[2.5ex]
    &= \big|\vect{e}_1^{\top} \mat{P}_i \vect{e}_1 - \vect{e}_2^{\top} \mat{P}_i \vect{e}_2 \pm \vect{e}_1^{\top} \mat{P}_i \vect{e}_2\big| \\[2.5ex]
    &= \big|\vect{e}_1^{\top} \mat{P}_i (\vect{e}_1 -\vect{e}_2) - \vect{e}_2^{\top} \mat{P}_i (\vect{e}_1-\vect{e}_2)\big| \\[2.5ex]
    &\leq \big|\vect{e}_1^{\top} \mat{P}_i (\vect{e}_1 -\vect{e}_2)\big| + \big|\vect{e}_2^{\top} \mat{P}_i (\vect{e}_1-\vect{e}_2)\big|
\end{align}

But for any $\vect{e}_1, \vect{e}_2 \in \mathbb{R}^n$
$$\big|\vect{e}_1^{\top} \mat{P}_i \vect{e}_2\big| \leq \sigma_{max}(\mat{P}_i) \|\vect{e}_1\| \|\vect{e}_2\|$$
where $\sigma_{max}(\mat{P}_i)$ denotes the largest singular value of matrix
$\mat{P}_i$. Hence:
\begin{align}
\big|V_i(\vect{e}_1) - V_i(\vect{e}_2)\big| &\leq
  \sigma_{max}(\mat{P}_i) \|\vect{e}_1\| \|\vect{e}_1 - \vect{e}_2\| +
  \sigma_{max}(\mat{P}_i) \|\vect{e}_2\| \|\vect{e}_1 - \vect{e}_2\| \\[2.5ex]
  &= \sigma_{max}(\mat{P}_i) (\|\vect{e}_1\| + \|\vect{e}_2\|)\|\vect{e}_1 - \vect{e}_2\| \\[2.5ex]
  & \leq \sigma_{max}(\mat{P}_i) (\overline{\varepsilon}_{i,\Omega_i} + \overline{\varepsilon}_{i,\Omega_i})\|\vect{e}_1 - \vect{e}_2\| \\[2.5ex]
  &= 2 \sigma_{max}(\mat{P}_i) \overline{\varepsilon}_{i,\Omega_i} \|\vect{e}_1 - \vect{e}_2\|
\end{align}
