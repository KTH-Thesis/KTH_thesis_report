%-------------------------------------------------------------------------------
\section{Objective}
\label{sec:objective}

Given the aforementioned structure of the system, the objective to be
pursued is the \textit{stabilization of all agents} $i \in \mathcal{V}$ starting
from an initial configuration abiding by assumption
\eqref{ass:initial_conditions} to a desired feasible configuration
$\vect{x}_{i,des}, \vect{v}_{i,des}$, while
satisfying all communication constraints, i.e. sustaining connectivity between
neighbouring agents, and avoiding collisions between agents, obstacles, and the
workspace boundary. The concept of a \textit{desired feasible configuration}
or \textit{feasible steady-state configuration} is given in definition
\eqref{definition:feasible_steady_state_conf}.

\begin{bw_box}
\begin{definition} (\textit{Feasible Steady-state Configuration})
\label{definition:feasible_steady_state_conf}

The desired steady-state configuration $\vect{x}_{i,des}$ of agents
$\forall i \in \mathcal{V}, j \in \mathcal{N}_i$ is \textit{feasible} if and
only if
\begin{enumerate}
  \item it is a collision-free configuration according to definition
    \eqref{definition:collision_free_conf}
  \item it does not result in violation of the communication constraints
    between neighbouring agents $i,j$, i.e. the following inequalities hold
    true simultaneously:
    \begin{align}
      \|\vect{p}_{i,des} - \vect{p}_{j,des}\| < d_i \\[2.5ex]
      \|\vect{p}_{i,des} - \vect{p}_{j,des}\| < d_j \\[2.5ex]
    \end{align}
\end{enumerate}

\end{definition}
\end{bw_box}

At this point we must address an issue
that refers to the feasibility of its solution and relates to the avoidance of
collisions from an intra-environmental perspective. Namely, we demand that a
solution be feasible if and only if the agent with the largest radius is able to
pass through the spaces demarcated by (a) the two least distant obstacles,
and (b) the obstacle closest to the boundary of the workspace and the
boundary of the workspace itself. To this end, we formalize the relevant notions
in Definition \eqref{def:intra_environmental_arrangement}. \\[2.5ex]

\begin{bw_box}
\begin{definition} (\textit{Intra-environmental Arrangement})
\label{def:intra_environmental_arrangement}

  Let us define $\underline{d}_{\ell'\ell}$
  \begin{align}
    \underline{d}_{\ell'\ell} &\triangleq \min\{\| \vect{p}_{\ell} - \vect{p}_{\ell'}\| + r_{\ell} + r_{\ell'} :
      \ell,\ell' \in \mathcal{L}, \ell \neq \ell' \},
  \end{align}
  as the distance between the two least distant obstacles in the workspace,
  $\underline{d}_{\ell,W}$
  \begin{align}
    \underline{d}_{\ell,W} & \triangleq \min\{r_W - (\|\vect{p}_W - \vect{p}_{\ell}\| + r_{\ell}) : \ell \in \mathcal{L}\},
  \end{align}
  as the distance between the least distant obstacle from the boundary of the
  workspace and the boundary itself, and $D$
  \begin{equation}
    D \triangleq \min\{\underline{d}_{\ell'\ell}, \underline{d}_{\ell,W}\}
  \end{equation}
  as the least of these two distances.
\end{definition}
\end{bw_box}
Given these notions, we can state an assumption on the feasibility of a
solution to the problem that this work addresses:

\begin{gg_box}
  \begin{assumption}(\textit{Intra-environmental Arrangement of Obstacles})
  \label{ass:intra_environmental_arrangement}

    All obstacles $\ell \in \mathcal{L}$ are situated inside the workspace $W$
    in such a way that

  \begin{equation}
    D > 2 r_i,\ i \in \mathcal{V} : r_i = max\{r_j\}\ \forall j \in \mathcal{V}
  \label{eq:geometric_constraint}
  \end{equation}
  where $D$ is defined in Definition \eqref{def:intra_environmental_arrangement}.
\end{assumption}
\end{gg_box}

The designed desirable control scheme should provide feasible control
inputs $\vect{u}_i$
per agent $i \in \mathcal{V}$, that is, inputs that abide by the input
constraints $\mathcal{U}_i$:

\begin{bw_box}
  \begin{definition} (\textit{Feasible Control Input})
    \label{definition:feasible_control_input}
    The control input $\vect{u}_i (\cdot)$ is called feasible when it abides
    by its respective constraint:
  \begin{align}
    \vect{u}_i(t) \in \mathcal{U}_i = \{ \vect{u}_i(t) \in \mathbb{R}^6 : \|\vect{u}_i (t)\| \leq \overline{u}_i\}
  \end{align}

\end{definition}
\end{bw_box}

Overall then, the objective of each agent $i$ is for
$\lim\limits_{t \to \infty} \|\vect{x}_i(t) - \vect{x}_{i,des}\| = 0$.
If we design feasible control inputs $\vect{u}_i \in \mathcal{U}_i$,
$\forall i \in \mathcal{V}$ such that the signal
$\lim\limits_{t \to \infty} \vect{x}_i(t) = \vect{x}_{i,des}$ with
dynamics given in \eqref{eq:system}, constrained under
assumptions \eqref{ass:measurements_access}, \eqref{ass:initial_conditions},
and \eqref{ass:intra_environmental_arrangement} satisfies
$\lim\limits_{t \to \infty} \|\vect{x}_i(t) - \vect{x}_{i,des}\| = 0$,
while all system related signals remain bounded in their respective regions,
$-$ if all of the above are achieved, then problem \eqref{problem} has been
solved.
