\section{Model Predictive control for non-linear continuous-time systems}

The equations describing a model of a system are doubly powerful: they
are \textit{expressive} inasmuch as they are a collection of properties of the
system, and they are \textit{predictive}: they provide the ability to predict
future values of relevant variables. A natural use for a model for control
design purposes is the calculation of expected future values of the controlled
variables as a function of possible control actions. This way, it is possible
to choose a control action which is the optimal according to a given criterion,
possibly while satisfying constraints on the controlled variables and control
inputs. It is reasonable for a closed-form solution of the state-feedback law
to be sought. Although this is possible when the system is linear and constraints
are absent, it is impossible or infeasible in the general case. Fortunately,
the problem can be bybassed altogether by solving an open-loop optimization
problem periodically (or even aperiodically), applying a fragment of
the resulting control input to the system, and repeating the process ad
infinitum. Control approaches using this strategy (with variations) are referred
to as \textit{Model Predictive Control} (MPC), or \textit{Receding Horizon Control}.
In this context, the term horizon refers to the period of time or timesteps
considered during the solution of the optimization problem.

In greater detail, in its pure form, the MPC strategy assumes the following
form: at an initial sampling time $t$, the controller, having measurements of
the states to be controlled, predicts the dynamic behaviour of the system over a
prediction horizon $T_p$, that is until $t+T_p$, and determines the appropriate
control input that minimizes a predetermined (open-loop) objective, typically
under constraints involving the states and the inputs. The calculated input
is then applied to the system until the next sampling time, at which time
the above process is repeated. Hence the term ``receding horizon" control.

When the system's dynamics are linear and there are is no system uncertainty
(disturbances affecting the states of the system or model-plant mismatch)
and no constraints are enforced, the input can be determined off-line by
solving the optimization problem over an infinite horizon. In this case the
input is applied to the open loop once, and the process is not repeated from
then on. However, in general, systems may be non-linear, subject to disturbances,
partially descriptive, under constraints, while an infinite horizon approach
may be practically infeasible. Hence, there are occasions when the optimization
problem must be solved on-line under a finite time-horizon.

In its general form, nonlinear MPC (NMPC) possesses a unique usefulness
in that it can directly incorporate nonlinear models for prediction,
nonlinear constraints, and explicit constraints on the states of the system
and its calculated/implemented inputs. Detailed descriptions and analyses of the
NMPC strategy can be found in the notable \cite{262032} and \cite{FINDEISEN2003190}.
