\subsection{Graph Theory}
\note{?? ISTORISOU // EXPAND}
An \textit{undirected graph} $\mathcal{G}$ is a pair
$(\mathcal{V}, \mathcal{E})$, where $\mathcal{V}$ is a finite set of nodes,
representing a team of agents, and
$\mathcal{E} \subseteq \big\{ \{i,j\} : i,j \in \mathcal{V}, i \neq j \big\}$,
with $M = |\mathcal{E}|$, is the set of edges that model the communication
capability between neighboring agents. For each agent, its neighbors' set
$\mathcal{N}_i$ is defined as
$\mathcal{N}_i = \{i_1, \ldots, i_{N_i}\} = \big\{ j \in \mathcal{V} : \{i,j\} \in \mathcal{E}\big\}$,
where $i_1, \ldots, i_{N_i}$ is an enumeration of the neighbors of agent $i$
and $N_i = |\mathcal N_i|$.
%Moreover, the notation
%$\bar{\vect{x}}_i = (\vect{x}_{i_1}, \dots, \vect{x}_{i_{N_i}})$
%is used to denote the vector of the neighbors of agent i, where
%$i_1, \dots, i_{N_i} \in \mathcal{N}_i$.

If there is an edge $\{i, j\} \in \mathcal{E}$, then $i, j$ are called
\textit{adjacent}. A \textit{path} of length $r$ from vertex $i$ to vertex
$j$ is a sequence of $r+1$ distinct vertices, starting with $i$ and ending
with $j$, such that consecutive vertices are adjacent. For $i = j$, the path
is called a \textit{cycle}. If there is a path between any two vertices of the
graph $\mathcal{G}$, then $\mathcal{G}$ is called \textit{connected}.
A connected graph is called a \textit{tree} if it contains no cycles.
