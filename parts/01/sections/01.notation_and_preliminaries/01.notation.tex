\subsection{Notation}

The set of positive integers is denoted by $\mathbb{N}$. The real $n$-coordinate
space, with $n\in\mathbb{N}$, is denoted by $\mathbb{R}^n$;
$\mathbb{R}^n_{\geq 0}$ and $\mathbb{R}^n_{> 0}$ are the sets of real
$n$-vectors with all elements nonnegative and positive, respectively. Given a
set $S$, we denote as $\lvert S\lvert$ its cardinality. The notation
$\|\vect{x}\|$ is used for the Euclidean norm of a vector
$\vect{x} \in \mathbb{R}^n$. Given a symmetric matrix
$\mat{A} = \mat{A}^T, \lambda_{\text{min}}(\mat{A})$
denotes the minimum eigenvalue of $\mat{A}$, respectively, where
$\sigma(\mat{A})$ is the set of all the eigenvalues of
$\mat{A}$ and $\text{rank}(\mat{A})$ is its rank. Given two sets $S_1$ and $S_2$,
the operation $S_1 \oplus S_2$ denotes the Minkowski addition, defined by
$S_1 \oplus S_2 = \{s_1 + s_2 : s_1 \in S_1, s_2 \in S_2$. Define by
$\vect{1}_n \in \mathbb{R}^n, \mat{I_n} \in \mathbb{R}^{n \times n},
\mat{0}_{m \times n} \in \mathbb{R}^{m \times n}$
the column vector with all entries $1$, the unit matrix and the $m \times n$
matrix with all entries zeros, respectively.
A matrix $\mat{A} \in \mathbb{R}^{n \times n}$ is called skew-symmetric if and only
if $\mat{A}^\top = -\mat{A}$.
$\mathcal{B}(\vect{c},r) \overset{\Delta}{=} \{\vect{x} \in \mathbb{R}^3: \|\vect{x}-\vect{c}\| \leq r\}$
is the $3$D sphere of radius $r \in \mathbb{R}_{\ge 0}$ and center
$\vect{c}\in\mathbb{R}^{3}$.

The vector expressing the coordinates of the origin of frame $\{j\}$ in
frame $\{i\}$ is denoted by $\vect{p}_{j \triangleright i}$. When this vector is
expressed in 3D space in a third frame, frame $\{k\}$, it is denoted by
$\vect{p}_{j \triangleright i}^{k}$.
%Given $\vect{a}\in\mathbb{R}^3$, $\mat{S}(\vect{a})$ is the skew-symmetric
%matrix defined according to $\mat{S}(\vect{a})\vect{b} = \vect{a}\times \vect{b}$.
The angular velocity of frame $\{j\}$ with respect to frame $\{i\}$, expressed
in frame $\{k\}$ coordinates, is denoted by
$\vect{\omega}^k_{j \triangleright i}\in \mathbb{R}^{3}$.
We also use the notation $\mathbb{M} = \mathbb{R}^3\times \mathbb{T}^3$.
We further denote as $\vect{q}_{j \triangleright i} \in \mathbb{T}^3$
the Euler angles representing the orientation of frame $\{j\}$ with respect to
frame $\{i\}$, where $\mathbb{T}^3$ is the $3$D torus.
For notational brevity, when a coordinate frame corresponds to the inertial frame
of reference $\{\mathcal{O}\}$, we will omit its explicit notation
(e.g., $\vect{p}_i = \vect{p}_{i \triangleright \mathcal{O}} = \vect{p}^\mathcal{O}_{i \triangleright \mathcal{O}}$,
$\vect{\omega}_i = \vect{\omega}_{i \triangleright \mathcal{O}} = \vect{\omega}^\mathcal{O}_{i \triangleright \mathcal{O}}$).
All vector and matrix differentiations are derived with respect to the inertial
frame $\{\mathcal{O}\}$ unless stated otherwise.

\begin{bw_box}
\begin{definition}\cite{khalil_nonlinear_systems} (\textit{Class $\mathcal{K}$ function})

  A continuous function $f : [0, \alpha] \to \mathbb{R}_{\geq 0}$,
  $\alpha \in \mathbb{R}_{>0}$ is said to belong to class $\mathcal{K}$ if
  \begin{enumerate}
    \item it is strictly increasing
    \item $f(0) = 0$
  \end{enumerate}
  \label{def:k_class}
\end{definition}
\end{bw_box}
