\subsection{Notation}

The set of positive integers is denoted by $\mathbb{N}$. The real $n$-coordinate
space, with $n\in\mathbb{N}$, is denoted by $\mathbb{R}^n$;
$\mathbb{R}^n_{\geq 0}$ and $\mathbb{R}^n_{> 0}$ are the sets of real
$n$-vectors with all elements nonnegative and positive, respectively. Given a
set $S$, we denote as $\lvert S\lvert$ its cardinality. The notation
$\|\vect{x}\|$ is used for the Euclidean norm of a vector
$\vect{x} \in \mathbb{R}^n$. Given a symmetric matrix
$\mat{A} = \mat{A}^T, \lambda_{\text{min}}(\mat{A})$
denotes the minimum eigenvalue of $\mat{A}$, respectively, where
$\sigma(\mat{A})$ is the set of all the eigenvalues of
$\mat{A}$ and $\text{rank}(\mat{A})$ is its rank. Given two sets $A$ and $B$,
the operation $A \oplus B$ denotes the Minkowski addition, defined by
$A \oplus B = \{\vect{a} + \vect{b}, \vect{a} \in A, \vect{b} \in B$. Similarly,
the Minkowski $-$ or Pontryagin difference is defined by
$A \ominus B = \{\vect{a} - \vect{b}, \vect{a} \in A, \vect{b} \in B$.
$\mat{I_n} \in \mathbb{R}^{n \times n}$ and
$\mat{0}_{m \times n} \in \mathbb{R}^{m \times n}$
are the unit matrix and the $m \times n$ matrix with all entries zeros,
respectively.  A matrix $\mat{A} \in \mathbb{R}^{n \times n}$ is called
skew-symmetric if and only if $\mat{A}^\top = -\mat{A}$.
$\mathcal{B}(\vect{c},r) \overset{\Delta}{=} \{\vect{x} \in \mathbb{R}^3: \|\vect{x}-\vect{c}\| \leq r\}$
is the $3$D sphere of radius $r \in \mathbb{R}_{\ge 0}$ and center
$\vect{c}\in\mathbb{R}^{3}$.

The vector expressing the coordinates of the origin of frame $\{j\}$ in
frame $\{i\}$ is denoted by $\vect{p}_{j \triangleright i}$. When this vector is
expressed in 3D space in a third frame, frame $\{k\}$, it is denoted by
$\vect{p}_{j \triangleright i}^{k}$.
%Given $\vect{a}\in\mathbb{R}^3$, $\mat{S}(\vect{a})$ is the skew-symmetric
%matrix defined according to $\mat{S}(\vect{a})\vect{b} = \vect{a}\times \vect{b}$.
The angular velocity of frame $\{j\}$ with respect to frame $\{i\}$, expressed
in frame $\{k\}$ coordinates, is denoted by
$\vect{\omega}^k_{j \triangleright i}\in \mathbb{R}^{3}$.
We also use the notation $\mathbb{M} = \mathbb{R}^3\times \mathbb{T}^3$.
We further denote as $\vect{q}_{j \triangleright i} \in \mathbb{T}^3$
the Euler angles representing the orientation of frame $\{j\}$ with respect to
frame $\{i\}$, where $\mathbb{T}^3$ is the $3$D torus.
For notational brevity, when a coordinate frame corresponds to the inertial frame
of reference $\{\mathcal{O}\}$, we will omit its explicit notation
(e.g., $\vect{p}_i = \vect{p}_{i \triangleright \mathcal{O}} = \vect{p}^\mathcal{O}_{i \triangleright \mathcal{O}}$,
$\vect{\omega}_i = \vect{\omega}_{i \triangleright \mathcal{O}} = \vect{\omega}^\mathcal{O}_{i \triangleright \mathcal{O}}$).
All vector and matrix differentiations are derived with respect to the inertial
frame $\{\mathcal{O}\}$ unless stated otherwise.

\begin{bw_box}
\begin{definition}\cite{khalil_nonlinear_systems} (\textit{Class $\mathcal{K}$ function})

  A continuous function $\alpha : [0, a) \to [0, \infty)$
  is said to belong to class $\mathcal{K}$ if
  \begin{enumerate}
    \item it is strictly increasing
    \item $f(0) = 0$
  \end{enumerate}
  If $a = \infty$ and $\lim\limits_{r \to \infty} a(r) = \infty$, then function
  $\alpha$ is said to belong to class $\mathcal{K}_{\infty}$
  \label{def:k_class}
\end{definition}
\end{bw_box}

\begin{bw_box}
\begin{definition}\cite{khalil_nonlinear_systems} (\textit{Class $\mathcal{KL}$ function})

  A continuous function $\beta : [0, a) \times [0, \infty) \to [0, \infty)$
  is said to belong to class $\mathcal{KL}$ if
  \begin{enumerate}
    \item for a fixed $s$, the mapping $\beta(r,s)$ belongs to class $\mathcal{K}$ with respect to $r$
    \item for a fixed $r$, the mapping $\beta(r,s)$ is decreasing with respect
      to $s$
    \item $\lim\limits_{s \to \infty} \beta(r,s) = 0$
  \end{enumerate}
  \label{def:kl_class}
\end{definition}
\end{bw_box}


\begin{bw_box}
\begin{definition}\cite{Sontag2008} (\textit{Input-to-State Stability})

  A nonlinear system $\dot{x} = f(x,u)$ with initial condition $x(t_0)$ is said
  to be \textit{Input-to-State Stable (ISS)} if there exist functions
  $\sigma \in \mathcal{K}_{\infty}$ and $\beta \in \mathcal{KL}$ such that
  \begin{align}
    \|x(t)\| \leq \beta\big(\|x(t_0)\|,t\big) + \sigma\big(\|u\|_{infty}\big)
  \end{align}
  \label{def:ISS}
\end{definition}
\end{bw_box}

\begin{bw_box}
\begin{definition}\cite{Sontag2008} (\textit{ISS Lyapunov function})

  A Lyapunov function $V(x,u)$ for the nonlinear system $\dot{x} = f(x,u)$
  with initial condition $x(t_0)$ is said to be a \textit{ISS-Lyapunov function}
  if there exist functions $\sigma, \alpha \in \mathcal{K}_{\infty}$ such that
  \begin{align}
    \dot{V}(x,u) \leq -\alpha\big(\|x\|\big) + \sigma\big(\|u\|\big),\ \forall x,u
  \end{align}
  \label{def:ISS_Lyapunov}
\end{definition}
\end{bw_box}

\begin{bw_box}
\begin{theorem}\cite{Sontag2008}

  A nonlinear system $\dot{x} = f(x,u)$ with initial condition $x(t_0)$ is said
  to be \textit{Input-to-State Stable} if and only if it admits an
  ISS-Lyapunov function.
  \label{def:ISS_Lyapunov_admit_theorem}
\end{theorem}
\end{bw_box}
