\subsection{Notation}

The set of positive integers is denoted by $\mathbb{N}$. The real $n$-coordinate
space, $n\in\mathbb{N}$, is denoted by $\mathbb{R}^n$;
$\mathbb{R}^n_{\geq 0}$ and $\mathbb{R}^n_{> 0}$ are the sets of real
$n$-vectors with all elements nonnegative and positive, respectively. Given a
set $S$, we denote by $\lvert S\lvert$ its cardinality. The notation
$\|\vect{x}\|$ is used for the Euclidean norm of a vector
$\vect{x} \in \mathbb{R}^n$. Given matrix $\mat{A}$, $\lambda_{\text{min}}(\mat{A})$
and $\lambda_{\text{max}}(\mat{A})$
denote the minimum and maximum eigenvalues of $\mat{A}$, respectively.
Its minimum and maximum singular values are denoted by
$\sigma_{\text{min}}(\mat{A})$ and $\sigma_{\text{max}}(\mat{A})$ respectively.
Given two sets $A$ and $B$, the operation $A \oplus B$ denotes the Minkowski
addition, defined by
$A \oplus B = \{\vect{a} + \vect{b},\ \vect{a} \in A, \vect{b} \in B\}$. Similarly,
the Minkowski $-$ or Pontryagin difference is defined by
$A \ominus B = \{\vect{a} - \vect{b},\ \vect{a} \in A, \vect{b} \in B\}$.
$\mat{I_n} \in \mathbb{R}^{n \times n}$ and
$\mat{0}_{m \times n} \in \mathbb{R}^{m \times n}$
are the unit matrix and the $m \times n$ matrix with all entries zeros
respectively.  A matrix $\mat{A} \in \mathbb{R}^{n \times n}$ is called
skew-symmetric if and only if $\mat{A}^\top = -\mat{A}$. The notation
$\mathcal{B}(\vect{c},r) \overset{\Delta}{=} \{\vect{x} \in \mathbb{R}^3: \|\vect{x}-\vect{c}\| \leq r\}$
is reserved for the 3D sphere of radius $r \in \mathbb{R}_{\ge 0}$ and center
located at $\vect{c}\in\mathbb{R}^{3}$.

The vector expressing the coordinates of the origin of frame $\{j\}$ in
frame $\{i\}$ is denoted by $\vect{p}_{j \triangleright i}$. When this vector is
expressed in 3D space in a third frame, frame $\{k\}$, it is denoted by
$\vect{p}_{j \triangleright i}^{k}$.
%Given $\vect{a}\in\mathbb{R}^3$, $\mat{S}(\vect{a})$ is the skew-symmetric
%matrix defined according to $\mat{S}(\vect{a})\vect{b} = \vect{a}\times \vect{b}$.
The angular velocity of frame $\{j\}$ with respect to frame $\{i\}$, expressed
in frame $\{k\}$ coordinates, is denoted by
$\vect{\omega}^k_{j \triangleright i}\in \mathbb{R}^{3}$.
We further denote by $\vect{q}_{j \triangleright i} \in \mathbb{T}^3$
the Euler angles representing the orientation of frame $\{j\}$ with respect to
frame $\{i\}$, where $\mathbb{T}^3$ is the $3$D torus.
We also use the notation $\mathbb{M} = \mathbb{R}^3\times \mathbb{T}^3$.
For notational brevity, when a coordinate frame corresponds to the inertial frame
of reference $\{\mathcal{O}\}$, we will omit its explicit notation
(e.g., $\vect{p}_i = \vect{p}_{i \triangleright \mathcal{O}} = \vect{p}^\mathcal{O}_{i \triangleright \mathcal{O}}$, and
$\vect{\omega}_i = \vect{\omega}_{i \triangleright \mathcal{O}} = \vect{\omega}^\mathcal{O}_{i \triangleright \mathcal{O}}$).
%All vector and matrix differentiations are derived with respect to the inertial
%frame $\{\mathcal{O}\}$ unless stated otherwise.

\begin{bw_box}
\begin{definition}\cite{khalil_nonlinear_systems} (\textit{Class $\mathcal{K}$ function})
\label{def:k_class}

  A continuous function $\alpha : [0, a) \to [0, \infty)$
  is said to belong to class $\mathcal{K}$ if
  \begin{enumerate}
    \item it is strictly increasing
    \item $\alpha (0) = 0$
  \end{enumerate}
  If $a = \infty$ and $\lim\limits_{r \to \infty} a(r) = \infty$, then function
  $\alpha$ is said to belong to class $\mathcal{K}_{\infty}$
\end{definition}
\end{bw_box}

\begin{bw_box}
\begin{definition}\cite{khalil_nonlinear_systems} (\textit{Class $\mathcal{KL}$ function})
\label{def:kl_class}

  A continuous function $\beta : [0, a) \times [0, \infty) \to [0, \infty)$
  is said to belong to class $\mathcal{KL}$ if
  \begin{enumerate}
    \item for a fixed $s$, the mapping $\beta(r,s)$ belongs to class $\mathcal{K}$ with respect to $r$
    \item for a fixed $r$, the mapping $\beta(r,s)$ decreases with respect to $s$
    \item $\lim\limits_{s \to \infty} \beta(r,s) = 0$
  \end{enumerate}
\end{definition}
\end{bw_box}

\begin{bw_box}
  \begin{lemma} \cite{Fontes2007} (\textit{A modification of Barbalat's lemma})
  \label{lemma:barbalat}

    Let $f$ be a continuous, positive-definite function, and $\vect{x}$ be an
    absolutely continuous function in $\mathbb{R}$. If the following hold:
  \begin{itemize}
    \item $\|\vect{x}(\cdot)\| < \infty$
    \item $\|\dot{\vect{x}}(\cdot)\| < \infty$
    \item $\lim\limits_{t \to \infty} \int\limits_0^t f\big(\vect{x}(s)\big)ds < \infty$
  \end{itemize}
  then $\lim\limits_{t \to \infty} \|\vect{x}(t)\| = 0$
  \end{lemma}
\end{bw_box}


\begin{bw_box}
\begin{definition}\cite{Sontag2008} (\textit{Input-to-State Stability})
\label{def:ISS}

A nonlinear system $\dot{\vect{x}} = f(\vect{x},\vect{u})$ with initial condition $\vect{x}(t_0)$ is said
  to be \textit{Input-to-State Stable (ISS)} if there exist functions
  $\sigma \in \mathcal{K}_{\infty}$ and $\beta \in \mathcal{KL}$ such that
  \begin{align}
      \|\vect{x}(t)\| \leq \beta\big(\|\vect{x}(t_0)\|,t\big) + \sigma\big(\|\vect{u}\|_{\infty}\big)
  \end{align}
\end{definition}
\end{bw_box}

\begin{bw_box}
\begin{definition}\cite{Sontag2008} (\textit{ISS Lyapunov function})
\label{def:ISS_Lyapunov}

A continuous function $V(\vect{x}, \vect{u}) : X \times U \to \mathbb{R}$ for the
nonlinear system $\dot{\vect{x}} = f(\vect{x}, \vect{u})$ is said to be a
\textit{ISS Lyapunov function} on $X$ if there exist functions
$\lambda_1, \lambda_2, \alpha, \sigma \in \mathcal{K}$ such that
  \begin{align}
    \lambda_1(\|\vect{x}\|) \leq &V(\vect{x},\vect{u}) \leq \lambda_2(\|\vect{x}\|), \forall \vect{x} \in X \\
     &\dot{V}(\vect{x},\vect{u}) \leq -\alpha\big(\|\vect{x}\|\big) + \sigma\big(\|\vect{u}\|\big),\ \forall \vect{x} \in X, \vect{u} \in U
  \end{align}
\end{definition}
\end{bw_box}

\begin{bw_box}
\begin{theorem}\cite{Sontag2008}
\label{def:ISS_Lyapunov_admit_theorem}

  A nonlinear system $\dot{\vect{x}} = f(\vect{x},\vect{u})$ is said
  to be \textit{Input-to-State Stable} if and only if it admits an
  ISS Lyapunov function.
\end{theorem}
\end{bw_box}
