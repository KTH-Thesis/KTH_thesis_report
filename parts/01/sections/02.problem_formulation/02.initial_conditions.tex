%-------------------------------------------------------------------------------
\subsection{Initial Conditions}

We assume that at time $t=0$ \textit{all} agents are in a
\textit{collision-free configuration}, i.e.
\begin{subequations}
\begin{align}
    d_{ij,a}(0) &> \underline{d}_{ij,a} \\[2.5ex]
    d_{il,o}(0) &> \underline{d}_{il,o} \\[2.5ex]
    d_{i,W}(0)  &< \overline{d}_{i,W}
\label{eq:initially_coll_free}
\end{align}
\end{subequations}
$\forall i \in \mathcal{V}, \ell \in \mathcal{L}$. Before declaring further
assumptions that relate to the initial conditions of the system's configuration,
we give the definition of the \textit{neighbour set} $\mathcal{N}_i$ of a
generic agent $i \in \mathcal{V}$:
\begin{bw_box}
\begin{definition} (\textit{Neighbours Set})

Agents $j \in \mathcal{N}_i$ are defined as the \textit{neighbours} of
agent $i \in \mathcal{V}$. The set $\mathcal{N}_i$ is composed of the
indices of agents $j \in \mathcal{V}$ which
\begin{enumerate}
  \item are within the sensing range of agent $i$ at time $t=0$, i.e.
    $j \in \mathcal{R}_i(0)$, \textit{and}
  \item are \textit{intended} to be kept within the sensing range of agent $i$ at all
    times $t \in \mathbb{R}_{> 0}$
\end{enumerate}
\end{definition}
\end{bw_box}

Therefore, while the composition of the set $\mathcal{R}_i(t)$ evolves and
varies through time in general, the set $\mathcal{N}_i$ \textit{should} remain
invariant over time\footnote{This reason, and the fact that the proposed
control scheme guarantees that $\mathcal{N}_i$ \textit{will} remain invariant
over time, is why we do not refer to this set as $\mathcal{N}_i(t)$.}.

It is not necessary that all agents are assigned a set of agents with whom they
should maintain connectivity; however, if \textit{all} agents were
neighbour-less, then the concept of \textit{cooperation} between them would be
void, since in that case the problem would break down into $|\mathcal{V}|$
individual problems of smaller significance or interest, while weakening the
``multi-agent" perspective as well. Therefore, we assume that
$\sum\limits_i |\mathcal{N}_i| > 0$.

It is further assumed that $\mathcal{N}_i$ is given at $t=0$, and that
neighbouring relations are reciprocal, i.e. agent $i$ is a neighbour of
agent $j$ if and only if $j$ is a neighbour of $i$:
\begin{subequations}
\begin{align}
  j \in \mathcal{N}_i \Leftrightarrow i \in \mathcal{N}_j,\ \forall i,j \in \mathcal{V}, i\not=j
\end{align}
\end{subequations}
%Topologically, this means that the graph $\mathcal{G}$ constructed by the edges
%connecting neighbouring agents $i \in \mathcal{V}$ is undirected.

Furthermore, it is assumed that at time $t=0$ the Jacobians $\mat{J}_i$ are
well-defined $\forall i \in \mathcal{V}$, and that the system \eqref{eq:system}
enjoys a continuous solution for all initial conditions. The assumptions which
concern the initial conditions of the problem are formally summarized in
assumption \ref{ass:initial_conditions}:

\begin{gg_box}
\begin{assumption}(\textit{Initial Conditions Assumption})\\[2.5ex]
  \label{ass:initial_conditions}

  At time $t = 0$

  \begin{enumerate}

    \item the sets $\mathcal{N}_i$ are known for all $i \in \mathcal{V}$
      and $\sum\limits_i |\mathcal{N}_i| > 0$

    \item all agents are in a collision-free configuration with each other,
      the obstacles $\ell \in \mathcal{L}$ and the workspace $W$ boundary

    \item all agents are in a singularity-free configuration:

      $$ -\frac{\pi}{2} < \theta_i(0) < \frac{\pi}{2},\ \forall i \in \mathcal{V}$$

  \end{enumerate}
\end{assumption}
\end{gg_box}
