We begin by rewriting the system equations \eqref{eq:system_1},
\eqref{eq:system_2} for a generic agent $i$ in state-space form:
\begin{align}
  \dot{\vect{x}}_i(t) &= \mat{J}_i^{-1}(\vect{x}_i) \vect{v}_i(t) \\
  \dot{\vect{v}}_i(t) &= -\mat{M}_i^{-1}(\vect{x}_i)\mat{C}_i(\vect{x}_i,\dot{\vect{x}}_i) \vect{v}_i(t)
    - \mat{M}_i^{-1}(\vect{x}_i)\vect{g}_i(\vect{x}_i)
    + \mat{M}_i^{-1}(\vect{x}_i)\vect{u}_i
\label{eq:system}
\end{align}
Denoting $\vect{z}_i(t)$ by
\begin{align}
  \vect{z}_i(t) =
    \begin{bmatrix}
      \vect{x}_i(t) \\
      \vect{v}_i(t) \\
    \end{bmatrix}
\end{align}
and
$\dot{\vect{x}}_i(t)$ and $\dot{\vect{v}}_i(t)$ by
\begin{align}
  \dot{\vect{x}}_i(t) &= f_{i,x}(\vect{z}_i, \vect{u}_i) \\
  \dot{\vect{v}}_i(t) &= f_{i,v}(\vect{z}_i, \vect{u}_i)
\end{align}
we get the compact representation of the system's model
\begin{align}
  \dot{\vect{z}}_i(t) =
    \begin{bmatrix}
      f_{i,x}(\vect{z}_i, \vect{u}_i) \\
      f_{i,v}(\vect{z}_i, \vect{u}_i) \\
    \end{bmatrix} =
 f_i \big(\vect{z}_i (t), \vect{u}_i (t)\big)
\end{align}

The state evolution of agent $i$ is modeled by a system of non-linear
continuous-time differential equations of the form
\begin{align}
  \dot{\vect{z}}_i(t) &= f_i \big(\vect{z}_i (t), \vect{u}_i (t)\big) \label{eq:non_perturbed_system}\\
  \vect{z}_i(0) &= \vect{z}_{i,0} \\
  \vect{z}_i (t) &\in \mathcal{Z}_i \subset \mathbb{R}^{9} \times \mathbb{T}^3 \\
  \vect{u}_i (t) &\in \mathcal{U}_i \subset \mathbb{R}^6
\end{align}
where state $\vect{z}_i$ is directly measurable, and sets $\mathcal{Z}_i$,
$\mathcal{U}_i$ are compact and contain the origin. Equation
\ref{eq:non_perturbed_system} does not consider model-plant mismatches or
external disturbances. The applied input $\vect{u}_i$ is a portion of the
optimal solution to an optimization problem where information on the states
of the neighbouring agents of agent $i$ are taken into account, either in the
cost function (in the displacement-based formation approach), or only in the
constraints considered in the optimization problem (in the position-based
approach). These constraints pertain to the set of its neighbours
$\mathcal{N}_i$ and, in total, to the set of all agents within its sensing
range $\mathcal{R}_i$.

Specifically, at time $t$, agent $i$ has access to\footnote{Although
    $\mathcal{N}_i \subseteq \mathcal{R}_i$, we make the distinction between
    the two because all agents $j \in \mathcal{R}_i$ need to avoid collision
    with agent $i$, but only agents $j' \in \mathcal{N}_i$ need to remain
    within the sensing range of agent $i$.}

\begin{enumerate}
  \item measurements of the states of
    \begin{itemize}
      \item all agents within its sensing range at time $t$
      \item its neighbouring agents at time $t$
      \end{itemize}
    \item the last applied inputs to
      \begin{itemize}
        \item all agents within its sensing range
        \item its neighbouring agents
      \end{itemize}
\end{enumerate}

We assume that these pieces of information are (a) always available and
accurate, and (b) exchanged without delay. We encapsulate these pieces of
information in four stacked vectors:

\begin{align}
  \vect{z}_{\mathcal{R}_i}(t) &\triangleq col[\vect{z}_j(t)], \forall j \in \mathcal{R}_i(t) \\
  \vect{z}_{\mathcal{N}_i}(t) &\triangleq col[\vect{z}_j(t)], \forall j \in \mathcal{N}_i \\
  \vect{u}_{\mathcal{R}_i}(t) &\triangleq col[\vect{u}_j(t)], \forall j \in \mathcal{R}_i(t) \\
  \vect{u}_{\mathcal{N}_i}(t) &\triangleq col[\vect{u}_j(t)], \forall j \in \mathcal{N}_i
\end{align}

