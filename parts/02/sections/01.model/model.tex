We begin by rewriting the system equations \eqref{eq:system_1},
\eqref{eq:system_2} for a generic agent $i \in \mathcal{V}$ in state-space form:
\begin{subequations}
\begin{align}
  \dot{\vect{x}}_i(t) &= \mat{J}_i^{-1}(\vect{x}_i) \vect{v}_i(t) \\
  \dot{\vect{v}}_i(t) &= -\mat{M}_i^{-1}(\vect{x}_i)\mat{C}_i(\vect{x}_i,\dot{\vect{x}}_i) \vect{v}_i(t)
    - \mat{M}_i^{-1}(\vect{x}_i)\vect{g}_i(\vect{x}_i)
    + \mat{M}_i^{-1}(\vect{x}_i)\vect{u}_i
\end{align}
\label{eq:state_space_system}
\end{subequations}
where the inversion of $\mat{M}_i$ is possible due to it being
positive-definite $\forall i \in \mathcal{V}$. Denoting $\vect{z}_i(t)$ by
\begin{align}
  \vect{z}_i(t) =
    \begin{bmatrix}
      \vect{x}_i(t) \\
      \vect{v}_i(t) \\
    \end{bmatrix}
\end{align}
and
$\dot{\vect{x}}_i(t)$ and $\dot{\vect{v}}_i(t)$ by
\begin{subequations}
\begin{align}
  \dot{\vect{x}}_i(t) &= f_{i,x}(\vect{z}_i, \vect{u}_i) \\
  \dot{\vect{v}}_i(t) &= f_{i,v}(\vect{z}_i, \vect{u}_i)
\end{align}
\end{subequations}
we get the compact representation of the system's model
\begin{align}
  \dot{\vect{z}}_i(t) =
    \begin{bmatrix}
      f_{i,x}(\vect{z}_i, \vect{u}_i) \\
      f_{i,v}(\vect{z}_i, \vect{u}_i) \\
    \end{bmatrix} =
 f_i \big(\vect{z}_i (t), \vect{u}_i (t)\big)
\end{align}
The state evolution of agent $i$ is modeled by a system of non-linear
continuous-time differential equations of the form
\begin{align}
  \dot{\vect{z}}_i(t) &= f_i \big(\vect{z}_i (t), \vect{u}_i (t)\big) \label{eq:non_perturbed_system}\\
  \vect{z}_i(0) &= \vect{z}_{i,0} \\
  \vect{z}_i (t) & \in \mathcal{Z}_i \subset \mathbb{R}^{9} \times \mathbb{T}^3 \\
  \vect{u}_i (t) & \in \mathcal{U}_i \subset \mathbb{R}^6
\end{align}
where state $\vect{z}_i$ is directly measurable. It should be noted that
equation \eqref{eq:non_perturbed_system} does not consider model-plant
mismatches or external disturbances.

The set $\mathcal{Z}_i$ captures all the state constraints of the system's
dynamics posed by the problem \eqref{problem}, therefore $\mathcal{Z}_i$ is
such that:
\begin{align}
  \mathcal{Z}_i = \{\vect{z}_i \in \mathbb{R}^{12} : \\
  & \|\vect{p}_i(t) - \vect{p}_j(t)\| > \underline{d}_{ij,a}, \forall j \in \mathcal{R}_i(t), \label{constraint:p_1}\\
  & \|\vect{p}_i(t) - \vect{p}_j(t)\| < d_i, \forall j \in \mathcal{N}_i, \\
  & \|\vect{p}_i(t) - \vect{p}_{\ell}\| > \underline{d}_{i\ell,o}, \forall \ell \in \mathcal{L}, \\
  & \|\vect{p}_W - \vect{p}_i(t)\| < \overline{d}_{i,W}, \\
  & - \frac{\pi}{2} < \theta_i(t) < \frac{\pi}{2} \label{constraint:p_5}, \\
  &\forall t \in \mathbb{R}_{\geq 0}\}
\end{align}

The applied input $\vect{u}_i$ is a portion of the
optimal solution to an optimization problem where information on the states
of the neighbouring agents of agent $i$ are taken into account only in the
constraints considered in the optimization problem. These constraints pertain
to the set of its neighbours $\mathcal{N}_i$ and, in total, to the set of
all agents within its sensing range $\mathcal{R}_i$. Regarding these, we
make the following assumption:\\

\begin{gg_box}
\begin{assumption}
considering the context of Model Predictive Control, when
at time $t_k$, agent $i$ solves a finite horizon optimization problem, it has
access to\footnote{Although
    $\mathcal{N}_i \subseteq \mathcal{R}_i$, we make the distinction between
    the two because all agents $j \in \mathcal{R}_i$ need to avoid collision
    with agent $i$, but only agents $j' \in \mathcal{N}_i$ need to remain
    within the sensing range of agent $i$. The distinction will prove the
    justification of its existence when considering the state constraints
    in the subsequent declaration of the optimization problem.}

\begin{enumerate}
  \item measurements of the states
    \begin{itemize}
      \item $\vect{z}_j(t_k)$ of all agents $j \in \mathcal{R}_i(t_k)$ within its sensing range at time $t_k$
      \item $\vect{z}_{j'}(t_k)$ of all of its neighbouring agents $j' \in \mathcal{N}_i$
      \end{itemize}
    \item the \textit{predicted states}
      \begin{itemize}
        \item $\overline{\vect{z}}_j(\tau)$ of all agents $j \in \mathcal{R}_i(t_k)$ within its sensing range
        \item $\overline{\vect{z}}_{j'}(\tau)$ of all of its neighbouring agents $j' \in \mathcal{N}_i$
      \end{itemize}
      across the entire horizon $\tau \in [t_k + h, t_k + T_p]$, where $T_p$ is the
      set horizon and $h$ is the sampling time.
\end{enumerate}
\end{assumption}
\end{gg_box}
We assume that these pieces of information are (a) always available and
accurate, and (b) exchanged without delay. We encapsulate these pieces of
information in four stacked vectors:

\begin{align}
  \vect{z}_{\mathcal{R}_i}(t_k) &\triangleq col[\vect{z}_j(t_k)], \forall j \in \mathcal{R}_i(t_k) \\
  \vect{z}_{\mathcal{N}_i}(t_k) &\triangleq col[\vect{z}_j(t_k)], \forall j \in \mathcal{N}_i \\
  \overline{\vect{z}}_{\mathcal{R}_i}(\tau) &\triangleq col[\overline{\vect{z}}_j(\tau)], \forall j \in \mathcal{R}_i(t_k), \tau \in [t_k + h, t_k + T_p] \\
  \overline{\vect{z}}_{\mathcal{N}_i}(\tau) &\triangleq col[\overline{\vect{z}}_j(\tau)], \forall j \in \mathcal{N}_i, \tau \in [t_k + h, t_k + T_p]
\end{align}

Considering that $\mathcal{N}_i \subseteq \mathcal{R}_i$, that the state
vectors $\vect{z}_j$ are comprised of 12 real numbers that are encoded by
4 bytes, and that sampling occurs with a frequency $f$ for all agents, the
overall downstream bandwidth required by each agent is
$$BW_d = 12 \times 32\ \text{[bits]} \times |\mathcal{R}_i| \times \dfrac{T_p}{h} \times f\ [\text{sec}^{-1}]$$
Given conservative constants $f = 100$ Hz, $\dfrac{T_p}{h} = 100$, the
wireless protocol IEEE 802.11n-2009 (a standard for present-day devices)
can accomodate

$$|\mathcal{R}_i| = \dfrac{600\ [\text{Mbit}\cdot \text{sec}^{-1}] }{12\times32[\text{bit}]\times10^4 [\text{sec}^{-1}]} \sim
16 \cdot 10^2 \text{ agents}$$ within the range of one agent, a number that we
deem is large enough for practical applications so as to consider this
approach to be legal.
