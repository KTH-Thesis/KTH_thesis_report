Here we are interested in steering each agent $i \in \mathcal{V}$ into
resting at a \textit{position} in 3D space, while conforming to the requirements
of the problem; that is, all agents should avoid colliding with each other, all
obstacles in the workspace, and the workspace boundary itself, while remaining
in a non-singular configuration and sustaining the connectivity to their
respective neighbours.


\subsection{The error model}

A desired configuration $\vect{z}_{i,des} \in \mathbb{R}^9 \times \mathbb{T}^3$
is associated to each agent $i \in \mathcal{V}$, with the aim of agent $i$
achieving it in steady-state:
$\lim\limits_{t \to \infty} \|\vect{z}_i(t) - \vect{z}_{i,des}\| = 0$. The
interior of this expression denotes the state error of agent $i$:

$$\vect{e}_i(t) = \vect{z}_i(t) - \vect{z}_{i,des}, \vect{e}_i(t) :
\mathbb{R}_{\geq 0} \to \mathbb{R}^9 \times \mathbb{T}^3$$

The error dynamics are denoted by $g_i(\vect{e}_i, \vect{u}_i)$:
\begin{align}
  \dot{\vect{e}}_i(t) = \dot{\vect{z}}_i(t) - \dot{\vect{z}}_{i,des} =
  \dot{\vect{z}}_i(t) = f_i(\vect{z}_i(t), \vect{u}_i(t)) = g_i(\vect{e}_i(t), \vect{u}_i(t))
  \label{eq:position_based_error_model}
\end{align}
with $\vect{e}_i(0) = \vect{z}_i(0) - \vect{z}_{i,des}$


\subsection{The optimization problem}

At a generic time $t_0$, agent $i$ solves the following optimization problem:

\begin{align}
  \min\limits_{\overline{\vect{u}}_i (\cdot)}\ &
    J_i \big(\overline{\vect{u}}_i (\cdot);\ \vect{e}_i(t_0)) \triangleq
      \int_{t_0}^{t_0 + T_p} F_i \big(\overline{\vect{e}}_i(\tau), \overline{\vect{u}}_i (\tau)\big) d \tau +
      V_i \big(\overline{\vect{e}}_i (t_0 + T_p)\big) \label{position_based_cost} \\
  \text{subject to:} & \nonumber \\
  & \dot{\overline{\vect{e}}}_i(t) = g_i (\overline{\vect{e}}_i (t), \overline{\vect{u}}_i (t)) \\
  & \overline{\vect{e}}_i (t_0) = \vect{e}_i (t_0) \\
  & \overline{\vect{u}}_i(t) \in \mathcal{U}_i, t \in [t_0, t_0 + T_p)\\
  & \overline{\vect{e}}_i (t) \in \mathcal{E}_i,\ t \in [t_0, t_0 + T_p]\\
  & \overline{\vect{e}}_i (t_0 + T_p) \in \mathcal{E}_i^f \\
  \text{and } \forall t \in [t_0, t_0 + T_p]:& \nonumber \\
  & \|\overline{\vect{p}}_i(t) - \overline{\vect{p}}_j(t)\| > \underline{d}_{ij,a}, \forall j \in \mathcal{R}_i(t) \label{constraint:p_1}\\
  & \|\overline{\vect{p}}_i(t) - \vect{p}_k(t)\| > \underline{d}_{ik,o}, \forall k \in \mathcal{K} \\
  & \|\overline{\vect{p}}_i(t)\| + r_i < r_W \\
  & \|\overline{\vect{p}}_i(t) - \overline{\vect{p}}_j(t)\| < d_i, \forall j \in \mathcal{N}_i \\
  & \overline{\theta}_i(t) \ne \pm \frac{\pi}{2} \label{constraint:p_5}
\end{align}\\
Constraints \ref{constraint:p_1}-\ref{constraint:p_5} explicitly address the
requirements posed by the problem \eqref{problem}, while the rest exist to
ensure that formation is achieved under constrained states and input signals.

The functions
$F_i : \mathcal{E}_i \times \mathcal{U}_i \to \mathbb{R}_{\geq 0}$ and
$V_i: \mathcal{E}_i^f \to \mathbb{R}_{\geq 0}$ are defined as
\begin{align}
  F_i (\vect{e}_i, \vect{u}_i)
    &\triangleq \| \vect{e}_i\|_{\mat{Q}_i}^2 + \| \vect{u}_i\|_{\mat{R}_i}^2 \\
  V_i (\vect{e}_i)
    & \triangleq \| \vect{e}_i \|_{\mat{P}_i}^2
\end{align}\\
Matrices $\mat{R}_i \in \mathbb{R}^{6 \times 6}$ are symmetric and positive
definite, while matrices $\mat{Q}_i, \mat{P}_i \in \mathbb{R}^{12 \times 12}$
are symmetric and positive semi-definite.

The set $\mathcal{E}_i$ is such that
$$\mathcal{E}_i = \{\vect{e}_i \in \mathbb{R}^{12} : \vect{e}_i \in \mathcal{Z}_i \oplus (-z_{i,des} )\}$$

The set $\mathcal{E}_i^f \subseteq \mathcal{E}_i$ is an admissible positively
invariant set \note{?? define it} for system \eqref{eq:position_based_error_model}
such that
\begin{align}
  \mathcal{E}_i^f = \{\vect{e}_i \in \mathcal{E}_i : \|\vect{e}_i\| \leq \epsilon_0 \}
\end{align}
where $\epsilon_0$ is an arbitrarily small but fixed positive real scalar.\\

The solution to the optimal control problem \eqref{position_based_cost} -
\eqref{constraint:p_5} at time $t_0$ is an optimal control input
$\vect{u}_i^{\ast}(t;\ \vect{e}_i(t_0))$, with $t \in [t_0, t_0 + T_p]$, which
is applied to the open-loop system until the next sampling instant $t_0 + h$:
\begin{align}
  \vect{u}_i(t;\ \vect{e}_i(t_0)) = \vect{u}_i^{\ast}(t_0;\ \vect{e}_i(t_0)) \\
  t \in [t_0, t_0 + h) \\
  0 < h < T_p
\end{align}

The control input $\vect{u}_i(\cdot)$ is a feedback, since it is
recalculated at each sampling instant based on the then-current state. The
solution to equation \eqref{eq:position_based_error_model}, starting at time
$t_0$, from an initial condition $\vect{e}_i(t_0)$, by application of the
control input $\vect{u}_i : [t_0, t_1] \to \mathcal{U}_i$ is denoted by
$$\vect{e}_i(\tau;\ \vect{u}_i(\cdot), \vect{e}_i(t_0))$$
with $\tau \in [t_0, t_1]$.

The \textit{predicted} state of the system \eqref{eq:position_based_error_model}
at time $t_0 + \tau$, based on the measurement of the state at time
$t_0$, $\vect{e}_i(t_0)$, by application of the control input
$\vect{u}_i(t;\ \vect{e}_i(t_0))$, for the time period $t \in [t_0, t_0 + \tau]$
is denoted by
$$\overline{\vect{e}}_i(t_0 + \tau;\ \vect{u}_i(\cdot), \vect{e}_i(t_0))$$
As is natural,
$\vect{e}_i(t_0) = \overline{\vect{e}}_i(t_0;\ \vect{u}_i(\cdot), \vect{e}_i(t_0))$.\\

We can now give the definition of an \textit{admissible input}:

\begin{bw_box}
\begin{definition} (Admissible input)\\

  A control input $\vect{u}_i : [0, T_p] \to \mathbb{R}^6$ for a state
  $\vect{e}_i(t_0)$ is called \textit{admissible} if all the following hold:

  \begin{enumerate}
    \item $\vect{u}_i(\cdot)$ is piecewise continuous
    \item $\vect{u}_i(\tau) \in \mathcal{U},\ \forall \tau \in [t_0, t_0 + T_p]$
    \item $\vect{e}_i(\tau;\ \vect{u}_i(\cdot), \vect{e}_i(t_0)) \in \mathcal{E}_i,\ \forall \tau \in [t_0, t_0 + T_p]$
    \item $\vect{e}_i(T_p;\ \vect{u}_i(\cdot), \vect{e}_i(t_0)) \in \mathcal{E}_i^f$
  \end{enumerate}

\end{definition}
\end{bw_box}


\begin{bw_box}
\begin{theorem} Suppose that

  \begin{enumerate}
    \item The optimal control problem \eqref{position_based_cost} -
      \eqref{constraint:p_5} is feasible at time $t=0$, that is, assumptions
      \eqref{ass:measurements_access}, \eqref{ass:initial_conditions}, and
      \eqref{ass:after_formation_geometry} hold at time $t=0$
    \item
  \end{enumerate}

\end{theorem}
\end{bw_box}



\begin{gg_box}
\begin{assumption} (Functions $g_i$ are Lipschitz continuous)
  \label{ass:g_i_Lipschitz}
\end{assumption}
\end{gg_box}

\begin{gg_box}
\begin{assumption} (Functions $F_i$ are Lipschitz continuous)
  \label{ass:F_i_Lipschitz}
\end{assumption}
\end{gg_box}


\begin{gg_box}
\begin{assumption} (Functions $V_i$ are Lipschitz continuous)
  \label{ass:V_i_Lipschitz}
\end{assumption}
\end{gg_box}
