%-------------------------------------------------------------------------------
\subsection{The error model}

A feasible desired configuration
$\vect{z}_{i,des} \in \mathbb{R}^9 \times \mathbb{T}^3$
is associated to each agent $i \in \mathcal{V}$, with the aim of agent $i$
achieving it in steady-state:
$\lim\limits_{t \to \infty} \|\vect{z}_i(t) - \vect{z}_{i,des}\| = 0$. The
interior of the norm of this expression denotes the state error of agent $i$:
$$\vect{e}_i(t) = \vect{z}_i(t) - \vect{z}_{i,des},\ \vect{e}_i(t) :
\mathbb{R}_{\geq 0} \to \mathbb{R}^9 \times \mathbb{T}^3$$
The error dynamics are denoted by $g_i^R(\vect{e}_i, \vect{u}_i)$:
\begin{align}
  \dot{\vect{e}}_i(t) &= \dot{\vect{z}}_i(t) - \dot{\vect{z}}_{i,des} =
  \dot{\vect{z}}_i(t) = f_i^R\big(\vect{z}_i(t), \vect{u}_i(t)\big) =  f_i \big(\vect{z}_i (t), \vect{u}_i (t)\big) + \vect{\delta}_i(t) \\
                      &=g_i \big(\vect{e}_i (t), \vect{u}_i (t)\big) + \vect{\delta}_i(t) \\
                      &= g_i^R(\vect{e}_i(t), \vect{u}_i(t)\big)
    \label{eq:position_based_error_model_with_disturbance}
\end{align}
with $\vect{e}_i(0) = \vect{z}_i(0) - \vect{z}_{i,des}$.
In order to translate
the constraints that are dictated for the state $\vect{z}_i(t)$ into constraints
regarding the error state $\vect{e}_i(t)$, we define the set
$\mathcal{E}_i \subset \mathbb{R}^9 \times \mathbb{T}^3$ as:
$$\mathcal{E}_i = \{\vect{e}_i(t) \in \mathbb{R}^9 \times \mathbb{T}^3 :
\vect{e}_i(t) \in \mathcal{Z}_i \oplus (-\vect{z}_{i,des} )\}$$
as the set that captures all constraints for the error dynamics
\eqref{eq:position_based_error_model_with_disturbance} dictated by the problem \eqref{problem}.

If we design control laws $\vect{u}_i \in \mathcal{U}_i$,
$\forall i \in \mathcal{V}$ such that the error signal $\vect{e}_i(t)$ with
dynamics given in \eqref{eq:position_based_error_model_with_disturbance}, constrained under
$\vect{e}_i(t) \in \mathcal{E}_i$, satisfies
$\lim\limits_{t \to \infty} \|\vect{e}_i(t)\| = 0$, while all system related
signals remain bounded in their respective regions,$-$ if all of the above are
achieved, then problem \eqref{problem} has been solved.

In order to achieve this task, we employ a Nonlinear Receding Horizon scheme.
