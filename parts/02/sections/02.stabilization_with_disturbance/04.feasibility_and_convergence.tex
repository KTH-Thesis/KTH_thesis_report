%-------------------------------------------------------------------------------
\subsection{Feasibility and Convergence}

Under these considerations, we can now state the theorem that relates to
the guaranteeing of the stability of the compound system of agents
$i \in \mathcal{V}$, when each of them is assigned a desired
position which results in feasible displacements:\\

\begin{bw_box}
\begin{theorem}

  Suppose that

  \begin{enumerate}
    \item the terminal region $\mathcal{E}_{i,f} \subseteq \mathcal{E}_i$ is
      closed with $\vect{0} \in \mathcal{E}_{i,f}$
    \item a solution to the optimal control problem \eqref{position_based_cost}
      is feasible at time $t=0$, that is, assumptions
      \eqref{ass:measurements_access}, \eqref{ass:initial_conditions}, and
      \eqref{ass:intra_environmental_arrangement} hold at time $t=0$
    \item there exists an admissible control input
      $\vect{u}_{i,f} : [0, h] \to \mathcal{U}_i$ such that for all
      $\vect{e}_i \in \mathcal{E}_{i,f}$ and $\forall \tau \in [0,h]$:

      \begin{enumerate}
        \item $\vect{e}_i(\tau) \in \mathcal{E}_{i,f}$
        \item $\dfrac{\partial V_i}{\partial \vect{e}_i} g_i\big(\vect{e}_i(\tau), \vect{u}_{i,f}(\tau)\big)
          + F_i\big(\vect{e}_i(\tau), \vect{u}_{i,f}(\tau)\big) \leq 0$
      \end{enumerate}

  \end{enumerate}

  then the closed loop system \eqref{eq:without_disturbances_closed_loop} under
  the control input \eqref{eq:position_based_optimal_u} converges to the set
  $\mathcal{E}_{i,f}$ when $t \to \infty$.
  \label{theorem:with_disturbances}
\end{theorem}
\end{bw_box}

\textbf{Proof}. The proof of the above theorem consists of two parts:
in the first, recursive feasibility is established, that is, initial
feasibility is shown to imply subsequent feasibility; in the second, and based
on the first part, it is shown that the error state $\vect{e}_i(t)$ converges
to the terminal set $\mathcal{E}_{i,f}$.\\

\textbf{Feasibility analysis}
Consider a sampling instant $t_k$ for which a
solution $\overline{\vect{u}}_i^{\star}\big(\cdot;\ \vect{e}_i(t_k)\big)$ to
\eqref{position_based_cost} exists.
%In between $t_k$ and $t_k + h$,
%where $h$ is the sampling time, the optimal control signal
%$\vect{u}_i^{\star}\big(\cdot;\ \vect{e}_i(t_k)\big)$ is applied to the open-loop
%system.
Suppose now a time instant $t_{k+1}$ such that\footnote{It is not strictly necessary
that $t_{k+1} = t_k + h$ here, however it is necessary for the following that
$t_{k+1} - t_k \leq h$} $t_k < t_{k+1} < t_k + T_p$, and consider that the
optimal control signal calculated at $t_k$ is comprised by the following two
portions:

\begin{equation}
  \overline{\vect{u}}_i^{\star}\big(\cdot;\ \vect{e}_i(t_k)\big) = \left\{
      \begin{array}{ll}
        \overline{\vect{u}}_i^{\star}\big(\tau_1;\ \vect{e}_i(t_k)\big), & \tau_1 \in [t_k, t_{k+1}] \\
        \overline{\vect{u}}_i^{\star}\big(\tau_2;\ \vect{e}_i(t_k)\big), & \tau_2 \in [t_{k+1}, t_k + T_p]
      \end{array}
      \right.
  \label{eq:optimal_input_portions}
\end{equation}

Both portions are admissible since the calculated optimal control input is
admissible, and hence they both conform to the input constraints.
As for the resulting predicted states, they satisfy the state constraints, and,
crucially: $\overline{\vect{e}}_i\big(t_k + T_p;\ \overline{\vect{u}}_i^{\star}(\cdot), \vect{e}_i(t_k)\big) \in \mathcal{E}_{i,f}$.
Furthermore, according to assumption (3) of the theorem, there exists an
admissible (and certainly not guaranteed optimal) input $\vect{u}_{i,f}$ that
renders $\mathcal{E}_{i,f}$ invariant over $[t_k + T_p, t_k + T_p + h]$.

Given the above facts, we can construct an admissible input
$\widetilde{\vect{u}}_i(\cdot)$  for time $t_{k+1}$ by sewing together the second
portion of \eqref{eq:optimal_input_portions} and the input
$\vect{u}_{i,f}(\cdot)$:

\begin{equation}
  \widetilde{\vect{u}}_i(\tau) = \left\{
      \begin{array}{ll}
        \overline{\vect{u}}_i^{\star}\big(\tau;\ \vect{e}_i(t_k)\big), & \tau \in [t_{k+1}, t_k + T_p] \\
        %\vect{u}_{i,f}\Big(\overline{\vect{e}}_i\big(t_k + T_p;\ \vect{u}_i^{\star}, \vect{e}_i(t_k)\big)\Big), & \tau_2 \in (t_k + T_p, t_{k+1} + T_p]
        \vect{u}_{i,f}(\tau - t_k - T_p), & \tau \in (t_k + T_p, t_{k+1} + T_p]
      \end{array}
      \right.
\label{eq:optimal_input_t_plus_one}
\end{equation}

Applied at time $t_{k+1}$, $\widetilde{\vect{u}}_i(\cdot)$
is an admissible control input as a composition of admissible control inputs.

This means that feasibility of a solution to the optimization problem at time
$t_k$ implies feasibility at time $t_{k+1} > t_k$, and, thus, since at time $t=0$
a solution is assumed to be feasible, a solution to the optimal control problem
is feasible for all $t \geq 0$.\\

\textbf{Convergence analysis}
The second part of the proof involves demonstrating the convergence of the
state $\vect{e}_i$ to the terminal set $\mathcal{E}_{i,f}$. In order for this
to be proved, it must be shown that a proper value function decreases along
the solution trajectories starting at some initial time $t_k$. We consider the
\textit{optimal} cost $J_i^{\star}\big(\vect{e}_i(t)\big)$ as a candidate
Lyapunov function:
$$J_i^{\star}\big(\vect{e}_i(t)\big) \triangleq J_i \Big(\vect{e}_i(t), \overline{\vect{u}}_i^{\star}\big(\cdot;\ \vect{e}_i(t)\big)\Big)$$
and, in particular, our goal is to show that that this cost decreases over
consecutive sampling instants $t_{k+1} = t_k + h$, i.e.
$J_i^{\star}\big(\vect{e}_i(t_{k+1})\big) - J_i^{\star}\big(\vect{e}_i(t_k)\big) \leq 0$.\\

In order not to wreak notational havoc, let us define the following terms:
\begin{gg_box}
\begin{itemize}
  \item $\vect{u}_{0,i}(\tau) \triangleq \overline{\vect{u}}_i^{\star}\big(\tau;\ \vect{e}_i(t_k)\big)$
    as the \textit{optimal} input that results from the solution to problem
    \eqref{problem:opt_without_disturbances} based on the measurement of state
    $\vect{e}_i(t_k)$, applied at time $\tau \geq t_k$
  \item $\vect{e}_{0,i}(\tau) \triangleq \overline{\vect{e}}_i\big(\tau;\ \overline{\vect{u}}_i^{\star}\big(\cdot;\ \vect{e}_i(t_k)\big), \vect{e}_i(t_k)\big)$
    as the \textit{predicted} state at time $\tau \geq t_k$, that is,
    the state that results from the application of the above input
    $\overline{\vect{u}}_i^{\star}\big(\cdot;\ \vect{e}_i(t_k)\big)$ to the
    state $\vect{e}_i(t_k)$, at time $\tau$
  \item $\vect{u}_{1,i}(\tau) \triangleq \widetilde{\vect{u}}_i(\tau)$
    as the \textit{admissible} input at $\tau \geq t_{k+1}$ (see eq. \eqref{eq:optimal_input_t_plus_one})
  \item $\vect{e}_{1,i}(\tau) \triangleq \overline{\vect{e}}_i\big(\tau;\ \widetilde{\vect{u}}_i(\cdot), \vect{e}_i(t_{k+1})\big)$
    as the \textit{predicted} state at time $\tau \geq t_{k+1}$, that is,
    the state that results from the application of the above input
    $\widetilde{\vect{u}}_i(\cdot)$ to the state
    $\vect{e}_i\big(t_{k+1};\ \overline{\vect{u}}_i^{\star}\big(\cdot;\ \vect{e}_i(t_k)\big), \vect{e}_i(t_k)\big)$, at time $\tau$
\end{itemize}
\end{gg_box}

\begin{bw_box}
  \begin{remark}
    Given that disturbances \textit{are} present, for the predicted and actual
    states at time $\tau_1 \geq \tau_0 \in \mathbb{R}_{\geq 0}$ it holds that:
    \begin{align}
      \vect{e}_i\big(\tau_1;\ \vect{u}_i(\cdot), \vect{e}_i(\tau_0)\big) &=
        \vect{e}_i(\tau_0) + \int_{\tau_0}^{\tau_1} g_i^R\big(\vect{e}_i(s;\ \vect{e}_i(\tau_0)), \vect{u}_i(s)\big) ds \\
      \overline{\vect{e}}_i\big(\tau_1;\ \vect{u}_i(\cdot), \vect{e}_i(\tau_0)\big) &=
        \vect{e}_i(\tau_0) + \int_{\tau_0}^{\tau_1} g_i\big(\overline{\vect{e}}_i(s;\ \vect{e}_i(\tau_0)), \vect{u}_i(s)\big) ds
    \end{align}
    \label{remark:predicted_actual_equations_with_disturbance}
  \end{remark}
\end{bw_box}

The following proof of convergence to the terminal set relies heavily on the
Gr\"{o}nwall-Bellman inequality. We state it here for reference purposes.
\begin{bw_box}
  \begin{lemma} \cite{khalil_nonlinear_systems} \textit{Gr\"{o}nwall-Bellman Inequality}

    Let $\lambda : [a,b] \to \mathbb{R}$ be continuous and
    $\mu : [a,b] \to \mathbb{R}$ be continuous and non-negative. If a
    continuous function $y : [a,b] \to \mathbb{R}$ satisfies
    \begin{align}
      y(t) \leq \lambda(t) + \int_a^t \mu(s) y(s) ds
    \end{align}
    for $a \leq t \leq b$, then on the same interval
    \begin{align}
      y(t) \leq \lambda(t) + \int_a^t \lambda(s) \mu(s) e^{\int_s^t \mu(\tau)d\tau} ds
    \end{align}
    In particular, if $\lambda(t) \equiv \lambda$ is a constant, then
    \begin{align}
      y(t) \leq \lambda e^{\int_a^t \mu(\tau)d\tau} ds
    \end{align}
    If $\lambda(t) \equiv \lambda$ and $\mu(t) \equiv \mu$ are both constants,
    then
    \begin{align}
      y(t) \leq \lambda e^{\mu (t - a)} ds
    \end{align}
    \label{lemma:bellman_inequality}
  \end{lemma}
\end{bw_box}

Before beginning to prove convergence, it is worth noting that while the cost
$$J_i \Big(\vect{e}_i(t), \overline{\vect{u}}_i^{\star}\big(\cdot;\ \vect{e}_i(t)\big)\Big)$$
is optimal (in the sense that it is based on the optimal input, which provides
its minimum realization), a cost that is based on a plainly admissible
(and thus, without loss of generality, sub-optimal) input
$\vect{u}_i \not= \overline{\vect{u}}_i^{\star}$ will result in a configuration where
\begin{equation}
J_i \Big(\vect{e}_i(t), \vect{u}_i\big(\cdot;\ \vect{e}_i(t)\big)\Big)
\geq J_i \Big(\vect{e}_i(t), \overline{\vect{u}}_i^{\star}\big(\cdot;\ \vect{e}_i(t)\big)\Big)
\end{equation}

Let us now begin our investigation on the sign of the difference between the cost
that results from the application of the feasible input $\vect{u}_{1,i}$,
which we shall denote by $\overline{J}_i\big(\vect{e}_i(t_{k+1})\big)$,
and the optimal cost $J_i^{\star}\big(\vect{e}_i(t_k)\big)$, while reminding
ourselves that
$J_i \big(\vect{e}_i(t), \overline{\vect{u}}_i (\cdot)\big)$ $=$
$\int_{t}^{t + T_p} F_i \big(\overline{\vect{e}}_i(s), \overline{\vect{u}}_i (s)\big) ds$ $+$
$V_i \big(\overline{\vect{e}}_i (t + T_p)\big)$:
\begin{align}
  \overline{J}_i\big(\vect{e}_i(t_{k+1})\big) - J_i^{\star}\big(\vect{e}_i(t_k)\big) =\
   & V_i \big(\vect{e}_{1,i} (t_{k+1} + T_p)\big) + \int_{t_{k+1}}^{t_{k+1} + T_p} F_i \big(\vect{e}_{1,i}(s), \vect{u}_{1,i} (s)\big) ds \\
  -&V_i \big(\vect{e}_{0,i} (t_k + T_p)\big) - \int_{t_k}^{t_k + T_p} F_i \big(\vect{e}_{0,i}(s), \vect{u}_{0,i} (s)\big) ds
\end{align}
Considering that $t_k < t_{k+1} < t_k + T_p < t_{k+1} + T_p$, we break down the
two integrals above in between these intervals:
\begin{align}
  \overline{J}_i\big(\vect{e}_i(t_{k+1})\big) - J_i^{\star}\big(\vect{e}_i(t_k)\big) &= \\
    V_i \big(\vect{e}_{1,i} (t_{k+1} + T_p)\big)
    &+ \int_{t_{k+1}}^{t_k + T_p} F_i \big(\vect{e}_{1,i}(s), \vect{u}_{1,i} (s)\big) d s
    + \int_{t_k + T_p}^{t_{k+1} + T_p} F_i \big(\vect{e}_{1,i}(s), \vect{u}_{1,i} (s)\big) d s \\
    -V_i \big(\vect{e}_{0,i} (t_k + T_p)\big)
    &- \int_{t_k}^{t_{k+1}} F_i \big(\vect{e}_{0,i}(s), \vect{u}_{0,i} (s)\big) d s
    - \int_{t_{k+1}}^{t_k + T_p} F_i \big(\vect{e}_{0,i}(s), \vect{u}_{0,i} (s)\big) d s
\label{eq:convergence_4_integrals_2}
\end{align}

\begin{gg_box}

Since there are disturbances present, consulting remark
\eqref{remark:predicted_actual_equations_with_disturbance} and substituting
for $\tau_0 = t_k$ and $\tau_1 = t_{k+1}$ yields:
\begin{align}
  \vect{e}_i\big(t_{k+1};\ \overline{\vect{u}}_i^{\star}\big(\cdot;\ \vect{e}_i(t_k)\big), \vect{e}_i(t_k)\big) &=
    \vect{e}_i(t_k)
    + \int_{t_k}^{t_{k+1}} g_i\big(\vect{e}_i(s;\ \vect{e}_i(t_k)), \overline{\vect{u}}_i^{\star}(s)\big) ds
    + \int_{t_k}^{t_{k+1}}\delta_i(s)ds\\
  \overline{\vect{e}}_i\big(t_{k+1};\ \overline{\vect{u}}_i^{\star}\big(\cdot;\ \vect{e}_i(t_k)\big), \vect{e}_i(t_k)\big) &=
    \vect{e}_i(t_k) + \int_{t_k}^{t_{k+1}} g_i\big(\overline{\vect{e}}_i(s;\ \vect{e}_i(t_k)), \overline{\vect{u}}_i^{\star}(s)\big) ds
\end{align}
Subtracting the latter from the former and taking norms on either side yields:
\begin{align}
  &\bigg\| \vect{e}_i\big(t_{k+1};\ \overline{\vect{u}}_i^{\star}\big(\cdot;\ \vect{e}_i(t_k)\big), \vect{e}_i(t_k)\big) -
  \overline{\vect{e}}_i\big(t_{k+1};\ \overline{\vect{u}}_i^{\star}\big(\cdot;\ \vect{e}_i(t_k)\big), \vect{e}_i(t_k)\big) \bigg\| \\
  &=\bigg\| \int_{t_k}^{t_{k+1}} g_i\big(\vect{e}_i(s;\ \vect{e}_i(t_k)), \overline{\vect{u}}_i^{\star}(s)\big) ds
     - \int_{t_k}^{t_{k+1}} g_i\big(\overline{\vect{e}}_i(s;\ \vect{e}_i(t_k)), \overline{\vect{u}}_i^{\star}(s)\big) ds
    + \int_{t_k}^{t_{k+1}}\delta_i(s)ds \bigg\| \\
  & \leq \bigg\| \int_{t_k}^{t_{k+1}} g_i\big(\vect{e}_i(s;\ \vect{e}_i(t_k)), \overline{\vect{u}}_i^{\star}(s)\big) ds
     - \int_{t_k}^{t_{k+1}} g_i\big(\overline{\vect{e}}_i(s;\ \vect{e}_i(t_k)), \overline{\vect{u}}_i^{\star}(s)\big) ds \bigg\|
     + (t_{k+1} - t_k)\overline{\delta}_i \\
  &=
     \int_{t_k}^{t_{k+1}} \bigg\| g_i\big(\vect{e}_i(s;\ \vect{e}_i(t_k)), \overline{\vect{u}}_i^{\star}(s)\big)s
     - g_i\big(\overline{\vect{e}}_i(s;\ \vect{e}_i(t_k)), \overline{\vect{u}}_i^{\star}(s)\big) \bigg\| ds + h \overline{\delta}_i\\
  &\leq L_{g_i} \int_{t_k}^{t_{k+1}} \bigg\| \vect{e}_i\big(s;\ \overline{\vect{u}}_i^{\star}\big(\cdot;\ \vect{e}_i(t_k)\big), \vect{e}_i(t_k)\big) -
  \overline{\vect{e}}_i\big(s;\ \overline{\vect{u}}_i^{\star}\big(\cdot;\ \vect{e}_i(t_k)\big), \vect{e}_i(t_k)\big) \bigg\| ds + h \overline{\delta}_i  \\
\end{align}
since $g_i$ is Lipschitz continuous in $\mathcal{E}_i$ with Lipschitz constant
$L_{g_i}$. Reformulation yields
\begin{align}
  &\bigg\| \vect{e}_i\big(t_k+h;\ \overline{\vect{u}}_i^{\star}\big(\cdot;\ \vect{e}_i(t_k)\big), \vect{e}_i(t_k)\big) -
  \overline{\vect{e}}_i\big(t_k+h;\ \overline{\vect{u}}_i^{\star}\big(\cdot;\ \vect{e}_i(t_k)\big), \vect{e}_i(t_k)\big) \bigg\| \\
  &\leq h \overline{\delta}_i
     + L_{g_i} \int_{0}^{h} \bigg\| \vect{e}_i\big(t_k + s;\ \overline{\vect{u}}_i^{\star}\big(\cdot;\ \vect{e}_i(t_k)\big), \vect{e}_i(t_k)\big) -
  \overline{\vect{e}}_i\big(t_k + s;\ \overline{\vect{u}}_i^{\star}\big(\cdot;\ \vect{e}_i(t_k)\big), \vect{e}_i(t_k)\big) \bigg\| ds \\
\end{align}
By applying the Gr\"{o}nwall-Bellman inequality we get:
\begin{align}
  &\bigg\| \vect{e}_i\big(t_{k+1};\ \overline{\vect{u}}_i^{\star}\big(\cdot;\ \vect{e}_i(t_k)\big), \vect{e}_i(t_k)\big) -
    \overline{\vect{e}}_i\big(t_{k+1};\ \overline{\vect{u}}_i^{\star}\big(\cdot;\ \vect{e}_i(t_k)\big), \vect{e}_i(t_k)\big) \bigg\| \\
  &\leq h \overline{\delta}_i +  L_{g_i} \int_{0}^{h} s \overline{\delta}_i e^{L_{g_i}(h - s)} ds \\
  &= h \overline{\delta}_i - \overline{\delta}_i \int_{0}^{h} s  \big(e^{L_{g_i}(h - s)}\big)' ds \\
  &= h \overline{\delta}_i -
    \overline{\delta}_i \bigg( \big[s e^{L_{g_i}(h - s)}\big]_0^h
      - \int_{0}^{h} e^{L_{g_i}(h - s)}ds\bigg) \\
  &= h \overline{\delta}_i - \overline{\delta}_i \bigg( h + \dfrac{1}{L_{g_i}} (1- e^{L_{g_i}h})\bigg) \\
  &= \dfrac{\overline{\delta}_i}{\L_{g_i}} (e^{L_{g_i}h} - 1)
\end{align}
\end{gg_box}

\begin{bw_box}
  \begin{lemma}
    Suppose that the real system, which is under the existence of bounded
    additive disturbances, and the model are both at time $t_k$ at state
    $\vect{e}_i(t_k)$. Applying at time $t_k$ a control law $\vect{u}(\cdot)$
    to the system model deemed ``real" and its model will cause at time $t_k + \tau$,
    $\tau \geq 0$ a divergence between the states of the real system and its
    model. The norm of the difference between the state of the real system
    and the state of the model system is bounded by
    \begin{align}
      \bigg\| \vect{e}_i\big(t_k + \tau;\ \vect{u}(\cdot), \vect{e}_i(t_k)\big) -
        \overline{\vect{e}}_i\big(t_k + \tau;\ \vect{u}(\cdot), \vect{e}_i(t_k)\big) \bigg\|
        \leq \dfrac{\overline{\delta}_i}{\L_{g_i}} (e^{L_{g_i} \tau} - 1)
    \end{align}
    where $\overline{\delta}_i$ is the upper bound of the disturbance,
    and $L_{g_i}$ the Lipschitz constant of both models.
    \label{lemma:diff_state_from_same_conditions}
  \end{lemma}
\end{bw_box}
Let us now begin working on \eqref{eq:convergence_4_integrals_2}, focusing
first on the difference between the two intervals over $[t_{k+1}, t_{k+1} + T_p]$:
\begin{align}
  &\int_{t_{k+1}}^{t_k + T_p} F_i \big(\vect{e}_{1,i}(s), \vect{u}_{1,i} (s)\big) d s
    - \int_{t_{k+1}}^{t_k + T_p} F_i \big(\vect{e}_{0,i}(s), \vect{u}_{0,i} (s)\big) d s \\
  &\int_{t_k+h}^{t_k + T_p} F_i \big(\vect{e}_{1,i}(s), \vect{u}_{1,i} (s)\big) d s
    - \int_{t_k+h}^{t_k + T_p} F_i \big(\vect{e}_{0,i}(s), \vect{u}_{0,i} (s)\big) d s \\
  & \leq \bigg\| \int_{t_k+h}^{t_k + T_p} F_i \big(\vect{e}_{1,i}(s), \vect{u}_{1,i} (s)\big) d s
    - \int_{t_k+h}^{t_k + T_p} F_i \big(\vect{e}_{0,i}(s), \vect{u}_{0,i} (s)\big) d s \bigg\|
    \text{ (for } x \leq |x|, x \in \mathbb{R} \text{)}\\
  & = \bigg\| \int_{t_k+h}^{t_k + T_p} \bigg( F_i \big(\vect{e}_{1,i}(s), \vect{u}_{1,i} (s)\big)
    -  F_i \big(\vect{e}_{0,i}(s), \vect{u}_{0,i} (s)\big) \bigg) d s \bigg\| \\
  & = \int_{t_k+h}^{t_k + T_p} \bigg\| F_i \big(\vect{e}_{1,i}(s), \vect{u}_{1,i} (s)\big)
    -  F_i \big(\vect{e}_{0,i}(s), \vect{u}_{0,i} (s)\big) \bigg\| d s \\
  & \leq L_{F_i}\int_{t_k+h}^{t_k + T_p} \bigg\| \overline{\vect{e}}_i\big(s;\ \vect{u}_{1,i} (\cdot), \vect{e}_i(t_k + h) \big)
    -  \overline{\vect{e}}_i\big(s;\ \vect{u}_{0,i} (\cdot), \overline{\vect{e}}_i(t_k + h)\big) \bigg\| d s
    \text{ (for } F_i \text{ is Lipschitz continuous in } \mathcal{E}_i \text{)} \label{eq:integrals_over_same_u_LV}
\end{align}


\begin{gg_box}
Consulting with remark \eqref{remark:predicted_actual_equations_with_disturbance}
and substituting for $\tau_1 = t_k + T_p$ and $\tau_0 = t_k +h$ in the second
equation for the two different initial conditions we get
\begin{align}
  \overline{\vect{e}}_i\big(t_k + T_p;\ \vect{u}_i^{\star}(\cdot), \vect{e}_i(t_k +h)\big) &=
    \vect{e}_i(t_k +h) + \int_{t_k +h}^{t_k + T_p} g_i\big(\overline{\vect{e}}_i(s;\ \vect{e}_i(t_k + h)), \vect{u}_i^{\star}(s)\big) ds \\
  \overline{\vect{e}}_i\big(t_k + T_p;\ \vect{u}_i^{\star}(\cdot), \overline{\vect{e}}_i(t_k +h)\big) &=
    \overline{\vect{e}}_i(t_k +h) + \int_{t_k +h}^{t_k + T_p} g_i\big(\overline{\vect{e}}_i(s;\ \overline{\vect{e}}_i(t_k + h)), \vect{u}_i^{\star}(s)\big) ds
\end{align}
Subtracting the latter from the former and taking norms on either side yields
\begin{align}
  &\bigg\| \overline{\vect{e}}_i\big(t_k + T_p;\ \overline{\vect{u}}_i^{\star}(\cdot), \vect{e}_i(t_k +h)\big) -
    \overline{\vect{e}}_i\big(t_k + T_p;\ \overline{\vect{u}_i^{\star}}(\cdot), \overline{\vect{e}}_i(t_k +h)\big) \bigg\| \\
  &\leq \bigg \| \vect{e}_i(t_k +h) - \overline{\vect{e}}_i(t_k +h) \\
  &+ \int_{t_k +h}^{t_k + T_p} g_i\big(\overline{\vect{e}}_i(s;\ \vect{e}_i(t_k + h)), \overline{\vect{u}}_i^{\star}(s)\big) ds
    - \int_{t_k +h}^{t_k + T_p} g_i\big(\overline{\vect{e}}_i(s;\ \overline{\vect{e}}_i(t_k + h)), \overline{\vect{u}}_i^{\star}(s)\big) ds\bigg\| \\
  &\leq \big \| \vect{e}_i(t_k +h) - \overline{\vect{e}}_i(t_k +h) \big \| \\
  &+ \bigg\| \int_{t_k +h}^{t_k + T_p} g_i\big(\overline{\vect{e}}_i(s;\ \vect{e}_i(t_k + h)), \overline{\vect{u}}_i^{\star}(s)\big) ds
    - \int_{t_k +h}^{t_k + T_p} g_i\big(\overline{\vect{e}}_i(s;\ \overline{\vect{e}}_i(t_k + h)), \overline{\vect{u}}_i^{\star}(s)\big) ds\bigg\| \\
  &\leq \big \| \vect{e}_i(t_k +h) - \overline{\vect{e}}_i(t_k +h) \big \| \\
  &+ \bigg\| \int_{t_k +h}^{t_k + T_p} \bigg( g_i\big(\overline{\vect{e}}_i(s;\ \vect{e}_i(t_k + h)), \overline{\vect{u}}_i^{\star}(s)\big)
    - g_i\big(\overline{\vect{e}}_i(s;\ \overline{\vect{e}}_i(t_k + h)), \overline{\vect{u}}_i^{\star}(s)\big) \bigg)ds\bigg\| \\
  &= \big \| \vect{e}_i(t_k +h) - \overline{\vect{e}}_i(t_k +h) \big \| \\
  &+ \int_{t_k +h}^{t_k + T_p} \bigg\| g_i\big(\overline{\vect{e}}_i(s;\ \vect{e}_i(t_k + h)), \overline{\vect{u}}_i^{\star}(s)\big)
  - g_i\big(\overline{\vect{e}}_i(s;\ \overline{\vect{e}}_i(t_k + h)), \overline{\vect{u}}_i^{\star}(s)\big) \bigg\| ds \\
  &\leq \big \| \vect{e}_i(t_k +h) - \overline{\vect{e}}_i(t_k +h) \big \| \\
  &+ L_{g_i}\int_{t_k +h}^{t_k + T_p} \bigg\| \overline{\vect{e}}_i\big(s;\ \overline{\vect{}u}_i^{\star}(\cdot), \vect{e}_i(t_k + h) \big)
    - \overline{\vect{e}}_i\big(s;\ \overline{\vect{u}}_i^{\star}(\cdot), \overline{\vect{e}}_i(t_k + h)\big)  \bigg\| ds \\
  &= \big \| \vect{e}_i(t_k +h) - \overline{\vect{e}}_i(t_k +h) \big \| \\
  &+ L_{g_i}\int_{h}^{T_p} \bigg\| \overline{\vect{e}}_i\big(t_k + s;\ \overline{\vect{u}}_i^{\star}(\cdot), \vect{e}_i(t_k + h) \big)
    - \overline{\vect{e}}_i\big(t_k + s;\ \overline{\vect{u}}_i^{\star}(\cdot), \overline{\vect{e}}_i(t_k + h)\big)  \bigg\| ds \\
\end{align}
By applying the Gr\"{o}nwall-Bellman inequality we get:
\begin{align}
  \bigg\| \overline{\vect{e}}_i\big(t_k + T_p;\ \overline{\vect{u}}_i^{\star}(\cdot), \vect{e}_i(t_k +h)\big)
    & - \overline{\vect{e}}_i\big(t_k + T_p;\ \overline{\vect{u}}_i^{\star}(\cdot), \overline{\vect{e}}_i(t_k +h)\big) \bigg\| \\
  & \leq \big \| \vect{e}_i(t_k +h) - \overline{\vect{e}}_i(t_k +h) \big \| e^{L_{g_i}(T_p - h)}
\end{align}
\end{gg_box}


\begin{bw_box}
  \begin{lemma}
    Suppose that the model of the real system (\textit{not} the real model itself),
    at time $t_k$ is at state $\vect{e}_i(t_k)$, and that another identical
    model is at time $t_k$ at state $\overline{\vect{e}}_i(t_k)$. Applying at
    time $t_k$ a control law $\vect{u}(\cdot)$ to both will cause at time
    $t_k + \tau$, $\tau \geq 0$ a divergence between their states.
    The norm of the difference between these states is bounded by
    \begin{align}
      \bigg\| \overline{\vect{e}}_i\big(t_k + \tau;\ \vect{u}(\cdot), \vect{e}_i(t_k)\big) -
        \overline{\vect{e}}_i\big(t_k + \tau;\ \vect{u}(\cdot), \overline{\vect{e}}_i(t_k)\big) \bigg\|
        \leq \big\| \vect{e}_i(t_k) - \overline{\vect{e}}_i(t_k) \big\| e^{L_{g_i}\tau}
    \end{align}
    where $\overline{\delta}_i$ is the upper bound of the disturbance,
    and $L_{g_i}$ the Lipschitz constant of the models.
    \label{lemma:diff_state_from_diff_conditions}
  \end{lemma}
\end{bw_box}

Given lemma \eqref{lemma:diff_state_from_diff_conditions},
\eqref{eq:integrals_over_same_u_LV} becomes
\begin{align}
  &\int_{t_{k+1}}^{t_k + T_p} F_i \big(\vect{e}_{1,i}(s), \vect{u}_{1,i} (s)\big) d s
    - \int_{t_{k+1}}^{t_k + T_p} F_i \big(\vect{e}_{0,i}(s), \vect{u}_{0,i} (s)\big) d s \\
  &\leq L_{F_i}\int_{t_k+h}^{t_k + T_p} \bigg\| \overline{\vect{e}}_i\big(s;\ \vect{u}_{1,i} (\cdot), \vect{e}_i(t_k + h) \big)
    -  \overline{\vect{e}}_i\big(s;\ \vect{u}_{0,i} (\cdot), \overline{\vect{e}}_i(t_k + h)\big) \bigg\| d s \\
  & = L_{F_i}\int_{h}^{T_p} \bigg\| \overline{\vect{e}}_i\big(t_k + s;\ \vect{u}_{1,i} (\cdot), \vect{e}_i(t_k + h) \big)
    -  \overline{\vect{e}}_i\big(t_k + s;\ \vect{u}_{0,i} (\cdot), \overline{\vect{e}}_i(t_k + h)\big) \bigg\| d s \\
  & \leq L_{F_i}\int_{h}^{T_p} \bigg\| \vect{e}_i(t_k +h) - \overline{\vect{e}}_i(t_k + h) \bigg\| e^{L_{g_i}(s-h)} ds \\
  & \leq L_{F_i} \dfrac{\overline{\delta}_i}{\L_{g_i}} (e^{L_{g_i} h} - 1) \int_{h}^{T_p}  e^{L_{g_i}(s-h)} ds \text{, from lemma \eqref{lemma:diff_state_from_same_conditions} for } \tau = h \\
  & = L_{F_i} \dfrac{\overline{\delta}_i}{\L_{g_i}} (e^{L_{g_i} h} - 1) \dfrac{1}{L_{g_i}}(e^{L_{g_i}(T_p-h)} - 1) \\
  & = L_{F_i} \dfrac{\overline{\delta}_i}{\L_{g_i}^2} (e^{L_{g_i} h} - 1) (e^{L_{g_i}(T_p-h)} - 1)
\end{align}
Hence we discovered that
\begin{bw_box}
\begin{align}
  \int_{t_{k+1}}^{t_k + T_p} F_i \big(\vect{e}_{1,i}(s), \vect{u}_{1,i} (s)\big) d s
  &- \int_{t_{k+1}}^{t_k + T_p} F_i \big(\vect{e}_{0,i}(s), \vect{u}_{0,i} (s)\big) d s \\
  & \leq L_{F_i} \dfrac{\overline{\delta}_i}{\L_{g_i}^2} (e^{L_{g_i} h} - 1) (e^{L_{g_i}(T_p-h)} - 1)
\label{eq:end_result_two_integrals}
\end{align}
\end{bw_box}

With this partial result established, we turn back to the remaining terms
found in \eqref{eq:convergence_4_integrals_2} and, in particular, we focus on
the integral
\begin{align}
  \int_{t_k + T_p}^{t_{k+1} + T_p} F_i \big(\vect{e}_{1,i}(s), \vect{u}_{1,i} (s)\big) d s
\end{align}
\begin{gg_box}
  We discern that the range of the above integral has a length\footnote{$(t_{k+1} + T_p) - (t_k + T_p) = t_{k+1} - t_k = h$}
  equal to the length of the interval where assumption (3b) (??) of theorem
  \eqref{theorem:with_disturbances} holds.
  Integrating the expression found in the assumption over the
  interval $[t_k + T_p, t_{k+1} + T_p]$, for the controls and states applicable
  in it we get
  \begin{align}
    \int_{t_k + T_p}^{t_{k+1} + T_p} \Bigg(\dfrac{\partial V_i}{\partial \vect{e}_{1,i}} g_i\big(\vect{e}_{1,i}(s), \vect{u}_{1,i}(s)\big)
    + F_i\big(\vect{e}_{1,i}(s), \vect{u}_{1,i}(s)\big)\Bigg) ds \leq 0 \\\\
    \int_{t_k + T_p}^{t_{k+1} + T_p} \dfrac{d}{ds} V_i\big(\vect{e}_{1,i}(s)\big) d s
    + \int_{t_k + T_p}^{t_{k+1} + T_p} F_i\big(\vect{e}_{1,i}(s), \vect{u}_{1,i}(s)\big) ds \leq 0 \\\\
    V_i\big(\vect{e}_{1,i}(t_{k+1} + T_p)\big) - V_i\big(\vect{e}_{1,i}(t_k + T_p)\big)
    + \int_{t_k + T_p}^{t_{k+1} + T_p} F_i\big(\vect{e}_{1,i}(s), \vect{u}_{1,i}(s)\big) ds \leq 0 \\\\
    V_i\big(\vect{e}_{1,i}(t_{k+1} + T_p)\big)
    + \int_{t_k + T_p}^{t_{k+1} + T_p} F_i\big(\vect{e}_{1,i}(s), \vect{u}_{1,i}(s)\big) ds \leq V_i\big(\vect{e}_{1,i}(t_k + T_p)\big)
  \end{align}

  The left-hand side expression is the same as the first two terms in the
  right-hand side of equality \eqref{eq:convergence_4_integrals_2}. We can
  introduce the third one by subtracting it from both sides:
  \begin{align}
    &V_i\big(\vect{e}_{1,i}(t_{k+1} + T_p)\big)
    + \int_{t_k + T_p}^{t_{k+1} + T_p} F_i\big(\vect{e}_{1,i}(s), \vect{u}_{1,i}(s)\big) ds
    - V_i\big(\vect{e}_{0,i}(t_k + T_p)\big) \\
    &\leq V_i\big(\vect{e}_{1,i}(t_k + T_p)\big)
    - V_i\big(\vect{e}_{0,i}(t_k + T_p)\big) \\
    &\leq \Big\|V_i\big(\vect{e}_{1,i}(t_k + T_p)\big)
    - V_i\big(\vect{e}_{0,i}(t_k + T_p)\big)\Big\| \text{, for } x \leq |x|, \forall x \in \mathbb{R} \\
    &\leq L_{V_i}\bigg \|\overline{\vect{e}}_i\big(t_k + T_p;\ \overline{\vect{u}}_i^{\star}(\cdot), \vect{e}_i(t_{k+1})\big)
    - \overline{\vect{e}}_i\big(t_k + T_p;\ \overline{\vect{u}}_i^{\star}(\cdot), \vect{e}_i(t_k) \big)\Big\|
    \text{, from lemma } \eqref{lemma:V_Lipschitz_e_0}
    \label{eq:V_diff_intermediate}
  \end{align}
  Consulting with remark \eqref{remark:predicted_actual_equations_with_disturbance}
  we get that the two terms interior to the norm are respectively equal to
  \begin{alignat}{2}
    \overline{\vect{e}}_i\big(t_k + T_p;\ \overline{\vect{u}}_i^{\star}(\cdot), \vect{e}_i(t_{k+1})\big)
      &= \vect{e}_i(t_{k+1}) &&+ \int_{t_{k+1}}^{t_k + T_p} g_i\big(\overline{\vect{e}}_i(s;\ \vect{e}_i(t_{k+1})), \overline{\vect{u}}_i^{\star}(s) \big)ds \\
      &\text{and}\\
    \overline{\vect{e}}_i\big(t_k + T_p;\ \overline{\vect{u}}_i^{\star}(\cdot), \vect{e}_i(t_k)\big)
      &= \vect{e}_i(t_k)     &&+ \int_{t_k}^{t_k + T_p} g_i\big(\overline{\vect{e}}_i(s;\ \vect{e}_i(t_k)), \overline{\vect{u}}_i^{\star}(s) \big)ds \\
      &= \vect{e}_i(t_k)     &&+ \int_{t_k}^{t_{k+1}} g_i\big(\overline{\vect{e}}_i(s;\ \vect{e}_i(t_k)), \overline{\vect{u}}_i^{\star}(s) \big)ds \\
      &                      &&+ \int_{t_{k+1}}^{t_k + T_p} g_i\big(\overline{\vect{e}}_i(s;\ \vect{e}_i(t_k)), \overline{\vect{u}}_i^{\star}(s) \big)ds \\
      &= \overline{\vect{e}}_i(t_{k+1}) &&+ \int_{t_{k+1}}^{t_k + T_p} g_i\big(\overline{\vect{e}}_i(s;\ \vect{e}_i(t_k)), \overline{\vect{u}}_i^{\star}(s) \big)ds \\
  \end{alignat}
  Subtracting the latter from the former and taking norms on either side we get
  \begin{alignat}{2}
    \bigg \|\overline{\vect{e}}_i\big(t_k + T_p;\ \overline{\vect{u}}_i^{\star}(\cdot), \vect{e}_i(t_{k+1})\big)
      - \overline{\vect{e}}_i\big(t_k + T_p;\ \overline{\vect{u}}_i^{\star}(\cdot), \vect{e}_i(t_k)\big) \bigg\| \\
    = \bigg \| \vect{e}_i(t_{k+1}) - \overline{\vect{e}}_i(t_{k+1}) \\
    + \int_{t_{k+1}}^{t_k + T_p} g_i\big(\overline{\vect{e}}_i(s;\ \vect{e}_i(t_{k+1})), \overline{\vect{u}}_i^{\star}(s) \big)ds
      - \int_{t_{k+1}}^{t_k + T_p} g_i\big(\overline{\vect{e}}_i(s;\ \vect{e}_i(t_k)), \overline{\vect{u}}_i^{\star}(s) \big)ds \bigg \| \\
    \leq \bigg \| \vect{e}_i(t_{k+1}) - \overline{\vect{e}}_i(t_{k+1}) \bigg\| \\
    + \bigg \| \int_{t_{k+1}}^{t_k + T_p} g_i\big(\overline{\vect{e}}_i(s;\ \vect{e}_i(t_{k+1})), \overline{\vect{u}}_i^{\star}(s) \big)ds
      - \int_{t_{k+1}}^{t_k + T_p} g_i\big(\overline{\vect{e}}_i(s;\ \vect{e}_i(t_k)), \overline{\vect{u}}_i^{\star}(s) \big)ds \bigg \| \\
    = \bigg \| \vect{e}_i(t_{k+1}) - \overline{\vect{e}}_i(t_{k+1}) \bigg\| \\
    + \bigg \| \int_{t_{k+1}}^{t_k + T_p} \bigg( g_i\big(\overline{\vect{e}}_i(s;\ \vect{e}_i(t_{k+1})), \overline{\vect{u}}_i^{\star}(s) \big)
    -  g_i\big(\overline{\vect{e}}_i(s;\ \vect{e}_i(t_k)), \overline{\vect{u}}_i^{\star}(s) \big) \bigg) ds \bigg \| \\
    = \bigg \| \vect{e}_i(t_{k+1}) - \overline{\vect{e}}_i(t_{k+1}) \bigg\| \\
    +  \int_{t_{k+1}}^{t_k + T_p} \bigg\| g_i\big(\overline{\vect{e}}_i(s;\ \vect{e}_i(t_{k+1})), \overline{\vect{u}}_i^{\star}(s) \big)
    -  g_i\big(\overline{\vect{e}}_i(s;\ \vect{e}_i(t_k)), \overline{\vect{u}}_i^{\star}(s) \big) \bigg\| ds \\
    \leq \bigg \| \vect{e}_i(t_{k+1}) - \overline{\vect{e}}_i(t_{k+1}) \bigg\| \\
    +  L_{g_i} \int_{t_{k+1}}^{t_k + T_p} \bigg\| \overline{\vect{e}}_i\big(s;\ \overline{\vect{u}}_i^{\star}(\cdot), \vect{e}_i(t_{k+1})\big)
    - \overline{\vect{e}}_i\big(s;\ \overline{\vect{u}}_i^{\star}(\cdot), \vect{e}_i(t_k)\big) \bigg\| ds \text{, for } g_i \text{ is Lipschitz continuous in } \mathcal{E}\\
    = \bigg \| \vect{e}_i(t_{k+1}) - \overline{\vect{e}}_i(t_{k+1}) \bigg\| \\
    +  L_{g_i} \int_{h}^{T_p} \bigg\| \overline{\vect{e}}_i\big(t_k + s;\ \overline{\vect{u}}_i^{\star}(\cdot), \vect{e}_i(t_{k+1})\big)
    - \overline{\vect{e}}_i\big(t_k + s;\ \overline{\vect{u}}_i^{\star}(\cdot), \vect{e}_i(t_k)\big) \bigg\| ds
  \end{alignat}
  By applying the  Gr\"{o}nwall-Bellman inequality we get
  \begin{alignat}{2}
    \bigg \|\overline{\vect{e}}_i\big(t_k + T_p;\ \overline{\vect{u}}_i^{\star}(\cdot), \vect{e}_i(t_{k+1})\big)
      - \overline{\vect{e}}_i\big(t_k + T_p;\ \overline{\vect{u}}_i^{\star}(\cdot), \vect{e}_i(t_k)\big) \bigg\|
      \leq \bigg \| \vect{e}_i(t_{k+1}) - \overline{\vect{e}}_i(t_{k+1}) \bigg\| e^{L_{g_i} (T_p - h)}
  \end{alignat}
  By applying lemma \eqref{lemma:diff_state_from_same_conditions} for $\tau = h$
  we get
  \begin{alignat}{2}
    \bigg \|\overline{\vect{e}}_i\big(t_k + T_p;\ \overline{\vect{u}}_i^{\star}(\cdot), \vect{e}_i(t_{k+1})\big)
      - \overline{\vect{e}}_i\big(t_k + T_p;\ \overline{\vect{u}}_i^{\star}(\cdot), \vect{e}_i(t_k)\big) \bigg\|
      \leq \dfrac{\overline{\delta}_i}{L_{g_i}} (e^{L_{g_i}h} - 1) e^{L_{g_i} (T_p - h)}
  \end{alignat}
\end{gg_box}
With the above result established, we substitute for the norm in
\eqref{eq:V_diff_intermediate}, getting
\begin{bw_box}
\begin{align}
  V_i\big(\vect{e}_{1,i}(t_{k+1} + T_p)\big)
  + \int_{t_k + T_p}^{t_{k+1} + T_p} F_i\big(\vect{e}_{1,i}(s), \vect{u}_{1,i}(s)\big) ds
  &- V_i\big(\vect{e}_{0,i}(t_k + T_p)\big) \\
  &\leq L_{V_i}\dfrac{\overline{\delta}_i}{L_{g_i}} (e^{L_{g_i}h} - 1) e^{L_{g_i} (T_p - h)}
  \label{eq:end_result_diff_V_plus_int}
\end{align}
\end{bw_box}

Adding the milestone inequalities \eqref{eq:end_result_two_integrals} and
\eqref{eq:end_result_diff_V_plus_int} yields
\begin{alignat}{2}
  &\int_{t_{k+1}}^{t_k + T_p} F_i \big(\vect{e}_{1,i}(s), \vect{u}_{1,i} (s)\big) d s
  - \int_{t_{k+1}}^{t_k + T_p} F_i \big(\vect{e}_{0,i}(s), \vect{u}_{0,i} (s)\big) d s \\
  &+ V_i\big(\vect{e}_{1,i}(t_{k+1} + T_p)\big)
  + \int_{t_k + T_p}^{t_{k+1} + T_p} F_i\big(\vect{e}_{1,i}(s), \vect{u}_{1,i}(s)\big) ds
  - V_i\big(\vect{e}_{0,i}(t_k + T_p)\big) \\
  &\leq L_{F_i} \dfrac{\overline{\delta}_i}{\L_{g_i}^2} (e^{L_{g_i} h} - 1) (e^{L_{g_i}(T_p-h)} - 1)
  + L_{V_i}\dfrac{\overline{\delta}_i}{L_{g_i}} (e^{L_{g_i}h} - 1) e^{L_{g_i} (T_p - h)}
\end{alignat}
and therefore \eqref{eq:convergence_4_integrals_2}, by bringing the integral
ranging from $t_k$ to $t_{k+1}$ to the left-hand side, becomes
\begin{align}
  \overline{J}_i\big(\vect{e}_i(t_{k+1})\big)
    - J_i^{\star}\big(\vect{e}_i(t_k)\big)
    + \int_{t_k}^{t_{k+1}} F_i \big(\vect{e}_{0,i}(s), \vect{u}_{0,i} (s)\big) d s \\
    \leq L_{F_i} \dfrac{\overline{\delta}_i}{\L_{g_i}^2} (e^{L_{g_i} h} - 1) (e^{L_{g_i}(T_p-h)} - 1)
  + L_{V_i}\dfrac{\overline{\delta}_i}{L_{g_i}} (e^{L_{g_i}h} - 1) e^{L_{g_i} (T_p - h)}
\end{align}

By rearranging terms, the cost difference becomes bounded by
\begin{align}
  \overline{J}_i\big(\vect{e}_i(t_{k+1})\big) &- J_i^{\star}\big(\vect{e}_i(t_k)\big) \\
  &\leq -\int_{t_k}^{t_{k+1}} F_i \big(\vect{e}_{0,i}(s), \vect{u}_{0,i} (s)\big) d s
    + \dfrac{\overline{\delta}_i}{L_{g_i}} \bigg(e^{L_{g_i}h} - 1\bigg)
    \bigg(\big(L_{V_i} + \dfrac{L_{F_i}}{\L_{g_i}}\big) e^{L_{g_i}(T_p-h)}  - \dfrac{L_{F_i}}{\L_{g_i}}\bigg) \\
  &= -\int_{t_k}^{t_{k+1}} F_i \big(\vect{e}_{0,i}(s), \vect{u}_{0,i} (s)\big) d s + \xi_i
\end{align}
where
\begin{align}
  \xi_i = \dfrac{\overline{\delta}_i}{L_{g_i}} \bigg(e^{L_{g_i}h} - 1\bigg)
    \bigg(\big(L_{V_i} + \dfrac{L_{F_i}}{\L_{g_i}}\big) e^{L_{g_i}(T_p-h)}  - \dfrac{L_{F_i}}{\L_{g_i}}\bigg)
\end{align}

\begin{gg_box}
  $F_i$ is a positive-definite function as a sum of a positive-definite
  $\|\vect{u}_i\|^2_{\mat{R}_i}$ and a positive semi-definite function
  $\|\vect{e}_i\|^2_{\mat{Q}_i}$. If we denote by
  $m_i = \lambda_{min}(\mat{Q}_i, \mat{R}_i) \geq 0$ the minimum eigenvalue
  between those of matrices $\mat{R}_i, \mat{Q}_i$, this means that
  \begin{align}
    F_i \big(\vect{e}_{0,i}(s), \vect{u}_{0,i} (s)\big) \geq m_i \|\vect{e}_{0,i}(s)\|^2
  \end{align}
  By integrating the above between our interval of interest $[t_k, t_{k+1}]$ we get
  \begin{align}
    \int_{t_k}^{t_{k+1}} F_i \big(\vect{e}_{0,i}(s), \vect{u}_{0,i} (s)\big) &\geq \int_{t_k}^{t_{k+1}} m_i \|\vect{e}_{0,i}(s)\|^2 ds \\
    \text{or}\\
    -\int_{t_k}^{t_{k+1}} F_i \big(\vect{e}_{0,i}(s), \vect{u}_{0,i} (s)\big) &\leq -m_i \int_{t_k}^{t_{k+1}} \|\vect{e}_{0,i}(s)\|^2 ds
  \end{align}
\end{gg_box}
This means that the cost difference is upper-bounded by
\begin{align}
  \overline{J}_i\big(\vect{e}_i(t_{k+1})\big) - J_i^{\star}\big(\vect{e}_i(t_k)\big)
  &\leq \xi_i -m_i \int_{t_k}^{t_{k+1}} \|\vect{e}_{0,i}(s)\|^2 ds \leq 0
\end{align}
and since the cost $\overline{J}_i\big(\vect{e}_i(t_{k+1})\big)$ is, in general,
sub-optimal: $J_i^{\star}\big(\vect{e}_i(t_{k+1})\big) - \overline{J}_i\big(\vect{e}_i(t_{k+1})\big) \leq 0$:
\begin{align}
 J_i^{\star}\big(\vect{e}_i(t_{k+1})\big) - J_i^{\star}\big(\vect{e}_i(t_k)\big) \leq \xi_i -m_i \int_{t_k}^{t_{k+1}} \|\vect{e}_{0,i}(s)\|^2 ds
 \label{eq:J_opt_between_consecutive_k_2}
\end{align}

\textcolor{red}{=======================================================}

With this milestone result established, we need to trace the time $t_k$ back
to $t_0 = 0$.

\begin{gg_box}
  The integral of $\|\vect{e}_{0,i}(\tau)\|^2$ over the interval $[t_0, t_{k+1}]$,
  $t_0 < t_k < t_{k+1}$ can be decomposed into the addition of two integrals
  with limits ranging from (a) $t_0$ to $t_k$ and (b) $t_k$ to $t_{k+1}$:
  \begin{align}
    \int_{t_0}^{t_{k+1}} \|\vect{e}_{0,i}(s)\|^2 ds = \int_{t_0}^{t_{k}} \|\vect{e}_{0,i}(s)\|^2 ds + \int_{t_k}^{t_{k+1}} \|\vect{e}_{0,i}(s)\|^2 ds
  \end{align}
  By rearranging terms, this means that
  \begin{align}
    \int_{t_k}^{t_{k+1}} \|\vect{e}_{0,i}(s)\|^2 ds = \int_{t_0}^{t_{k+1}} \|\vect{e}_{0,i}(s)\|^2 ds - \int_{t_0}^{t_{k}} \|\vect{e}_{0,i}(s)\|^2 ds
  \end{align}
  making the optimal cost difference between the consecutive sampling times
  $t_k$ and $t_{k+1}$ in \eqref{eq:J_opt_between_consecutive_k}
  \begin{align}
    J_i^{\star}\big(\vect{e}_i(t_{k+1})\big) - J_i^{\star}\big(\vect{e}_i(t_k)\big) \leq
      -m \int_{t_0}^{t_{k+1}} \|\vect{e}_{0,i}(s)\|^2 ds +m \int_{t_0}^{t_{k}} \|\vect{e}_{0,i}(s)\|^2 ds
  \end{align}
  Similarly, the optimal cost difference between the sampling times $t_{k-1}$
  and $t_{k}$ is
  \begin{align}
    J_i^{\star}\big(\vect{e}_i(t_{k})\big) - J_i^{\star}\big(\vect{e}_i(t_{k-1})\big) \leq
      -m \int_{t_0}^{t_{k}} \|\vect{e}_{0,i}(s)\|^2 ds +m \int_{t_0}^{t_{k-1}} \|\vect{e}_{0,i}(s)\|^2 ds
  \end{align}
  and we can apply this rationale all the way back to the cost difference
  between $t_0$ and $t_1$. Summing all the inequalities between the pairs of
  consecutive sampling times $(t_0, t_1)$, $(t_1, t_2)$, $\dots$,
  $(t_{k-1}, t_k)$, we get
  \begin{align}
    J_i^{\star}\big(\vect{e}_i(t_{k})\big) - J_i^{\star}\big(\vect{e}_i(t_0)\big) \leq
      -m \int_{t_0}^{t_{k}} \|\vect{e}_{0,i}(s)\|^2 ds
  \end{align}
\end{gg_box}

Hence, for $t_0 = 0$
\begin{align}
  J_i^{\star}\big(\vect{e}_i(t_{k})\big) - J_i^{\star}\big(\vect{e}_i(0)\big) \leq
    -m \int_{0}^{t_{k}} \|\vect{e}_{0,i}(s)\|^2 ds \leq 0
\label{eq:J_opt_between_k_and_0}
\end{align}
which implies that the value function $J_i^{\star}\big(\vect{e}_i(t_{k})\big)$
in non-increasing for all sampling times:
\begin{align}
  J_i^{\star}\big(\vect{e}_i(t_{k})\big) \leq J_i^{\star}\big(\vect{e}_i(0)\big),\ \forall t_k \in \mathbb{R}_{\geq0}
\end{align}
Let us now define the function $V_i(\vect{e}_i(t))$:
\begin{align}
  V_i\big(\vect{e}(t)\big) \triangleq J_i^{\star}\big(\vect{e}_i(\tau)\big) \leq J_i^{\star}\big(\vect{e}_i(0)\big),\ t \in \mathbb{R}_{\geq0}
\end{align}
where $\tau = max\{t_k : t_k \leq t\}$. Since $J_i^{\star}\big(\vect{e}_i(0)\big)$
is bounded, this implies that $V_i\big(\vect{e}(t)\big)$ is also bounded. The
signals $\vect{e}_i(t) \in \mathcal{E}_i$ and $\vect{u}_i(t) \in \mathcal{U}_i$
are also bounded. According to \eqref{eq:position_based_error_model}, this
means that $\dot{\vect{e}}_i(t)$ is bounded as well. From inequality
\eqref{eq:J_opt_between_k_and_0} we then have
\begin{align}
  V_i\big(\vect{e}_i(t)\big) = J_i^{\star}\big(\vect{e}_i(\tau)\big) \leq J_i^{\star}\big(\vect{e}_i(0)\big)
    -m \int_{0}^{\tau} \|\vect{e}_{0,i}(s)\|^2 ds \leq 0
\end{align}
which, due to the fact that $\tau \leq t$, is equivalent to
\begin{align}
  V_i\big(\vect{e}_i(t)\big) \leq J_i^{\star}\big(\vect{e}_i(0)\big) -m \int_{0}^{t} \|\vect{e}_{0,i}(s)\|^2 ds \leq 0,\ t \in \mathbb{R}_{t\geq 0}
\end{align}
Solving for the integral we get
\begin{align}
  \int_{0}^{t} \|\vect{e}_{0,i}(s)\|^2 ds \leq
    \dfrac{1}{m}\Big(J_i^{\star}\big(\vect{e}_i(0)\big) - V_i\big(\vect{e}_i(t)\big)\Big),\ t \in \mathbb{R}_{t\geq 0}
\end{align}
Both $J_i^{\star}\big(\vect{e}_i(0)\big)$ and $V_i\big(\vect{e}_i(t)\big)$
are bounded, and therefore so is their difference, which means that the
integral $\int\limits_{0}^{t} \|\vect{e}_{0,i}(s)\|^2 ds$ is bounded as well.
We make use of the following lemma to show that the error internal to the
norm of the integral goes to zero in steady-state:

\begin{bw_box}
  \begin{lemma} (\textit{A modification of Barbalat's lemma}\cite{Fontes2007})

    Let $f$ be a continuous, positive-definite function, and $\vect{x}$ be an
    absolutely continuous function in $\mathbb{R}$. If the following hold:
  \begin{itemize}
    \item $\|\vect{x}(\cdot)\| < \infty, \|\dot{\vect{x}}(\cdot)\| < \infty$
    \item $\lim\limits_{t \to \infty} \int\limits_0^t f\big(\vect{x}(s)\big) < \infty$
  \end{itemize}
  then $\lim\limits_{t \to \infty} \|\vect{x}(t)\| = 0$
  \label{lemma:barbalat}
  \end{lemma}
\end{bw_box}
Lemma \eqref{lemma:barbalat} assures us that under these conditions for the
error and its dynamics, which are fulfilled in our case, the error
\begin{align}
  \lim\limits_{t \to \infty}&\|\vect{e}_{0,i}(t)\| = 0 \Leftrightarrow \\
  \lim\limits_{t \to \infty}
  &\Big\| \overline{\vect{e}}_i\Big(t;\ \overline{\vect{u}}_i^{\star}\big(\cdot;\ \vect{e}_i(t_k)\big), \vect{e}_i(t_k) \Big) \Big\| = 0,\
\forall t_k \in \mathbb{R}_{\geq 0}
\end{align}
which, given \eqref{eq:error_now_to_predicted_error} and substituting
for $\tau_1 = t$ while dropping the initial condition at $\tau_0 = t_k$, means that
$$\lim\limits_{t \to \infty}\|\vect{e}_i(t)\| = 0$$
which implies that
$$\lim\limits_{t \to \infty}\vect{e}_i(t) \in \mathcal{E}_{i,f}$$
Therefore, the closed-loop trajectory of the error state $\vect{e}_i$ converges
to the terminal set $\mathcal{E}_{i,f}$ as $t \to \infty$.\\ \qedsymbol
