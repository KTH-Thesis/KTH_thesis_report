%-------------------------------------------------------------------------------
\subsection{The optimization problem}
Consider a sequence of sampling times $\{t_k\}_{k \geq 0}$, with a constant
sampling time $h$, $0 < h < T_p$, where $T_p$ is the finite time-horizon, such
that $t_{k+1} = t_k + h$. In sampling data NMPC, a finite-horizon open-loop
optimal control problem (OCP) is solved at discrete sampling time instants $t_k$
based on the then-current state error measurement $\vect{e}_i(t_k)$. The
solution is an optimal control signal $\overline{\vect{u}}_i^{\star}(t)$, computed over
$t \in [t_k, t_k+T_p]$. This signal is applied to the open-loop system in
between sampling times $t_k$ and $t_k + h$.

At a generic time $t_k$, agent $i$ solves the following optimization problem:
\begin{problem}
\begin{align}
  \text{Find }& \\
              &J_i^{\star} \big(\vect{e}_i(t_k)\big) \triangleq \text{min }\limits_{\overline{\vect{u}}_i (\cdot)}\
    J_i \big(\vect{e}_i(t_k), \overline{\vect{u}}_i (\cdot) \big) \\
    \text{where}& \\
    &J_i \big(\vect{e}_i(t_k), \overline{\vect{u}}_i (\cdot) \big) \triangleq
      \int_{t_k}^{t_k + T_p} F_i \big(\overline{\vect{e}}_i(s), \overline{\vect{u}}_i (s)\big) ds +
      V_i \big(\overline{\vect{e}}_i (t_k + T_p)\big) \label{position_based_cost_2} \\
  \text{subject to:} & \nonumber \\
                     & \dot{\overline{\vect{e}}}_i(s) = g_i \big(\overline{\vect{e}}_i (s), \overline{\vect{u}}_i (s)\big) \label{eq:internal_error_model_2},\ \overline{\vect{e}}_i (t_k) = \vect{e}_i (t_k) \\
                     & \overline{\vect{u}}_i(s) \in \mathcal{U}_i,\ \overline{\vect{e}}_i (s) \in \mathcal{E}_{i, s - t_k},\ s \in [t_k, t_k + T_p]\\
                     & \overline{\vect{e}}_i (t_k + T_p) \in \mathcal{E}_{i,f} \subseteq \mathcal{E}_{i,F} \subseteq \mathcal{E}_i
\end{align}
\label{problem:opt_with_disturbances}
\end{problem}

The notation $\overline{\cdot}$ is used to distinguish predicted states which
are internal to the controller, as opposed to their actual values, because,
even in the nominal case, the predicted values will not be equal to the
actual closed-loop values. This means
that $\overline{\vect{e}}_i(\cdot)$ is the solution to
\eqref{eq:internal_error_model_2} driven by the control input
$\overline{\vect{u}}_i(\cdot) : [t_k, t_k + T_p] \to \mathcal{U}_i$ with
initial condition $\vect{e}_i(t_k)$.

The applied input signal is a portion of the optimal solution to an
optimization problem where information on the states of the neighbouring agents
of agent $i$ are taken into account only in the constraints considered in the
optimization problem. These constraints pertain to the set of its neighbours
$\mathcal{N}_i$ and, in total, to the set of all agents within its sensing
range $\mathcal{R}_i$. Regarding these, we make the following assumption:

\begin{gg_box}
  \begin{assumption} (\textit{Access to Predicted Information from an
    Inter-agent Perspective})

Considering the context of Receding Horizon Control, when
at time $t_k$ agent $i$ solves a finite horizon optimization problem, he has
access to\footnote{Although
  $\mathcal{N}_i \subseteq \mathcal{R}_i$, we make the distinction between
  the two because all agents $j \in \mathcal{R}_i$ need to avoid collision
  with agent $i$, but only agents $j' \in \mathcal{N}_i$ need to remain
  within the sensing range of agent $i$.
  %The distinction will prove the
  %justification of its existence when considering the state constraints
  %in the subsequent declaration of the optimization problem.
}

\begin{enumerate}
  \item measurements of the states\footnote{as per assumption
    \eqref{ass:measurements_access_2}}
    \begin{itemize}
      \item $\vect{z}_j(t_k)$ of all agents $j \in \mathcal{R}_i(t_k)$ within its sensing range at time $t_k$
      \item $\vect{z}_{j'}(t_k)$ of all of its neighbouring agents $j' \in \mathcal{N}_i$ at time $t_k$
      \end{itemize}
    \item the \textit{predicted states}
      \begin{itemize}
        \item $\overline{\vect{z}}_j(\tau)$ of all agents $j \in \mathcal{R}_i(t_k)$ within its sensing range
        \item $\overline{\vect{z}}_{j'}(\tau)$ of all of its neighbouring agents $j' \in \mathcal{N}_i$
      \end{itemize}
      across the entire horizon $\tau \in (t_k, t_k + T_p]$
\end{enumerate}
\end{assumption}
\end{gg_box}
In other words, each time an agent solves its own individual
optimization problem, he knows the error predictions that have been generated
by the solution of the optimization problem of all agents within
its range at that time, for the next $T_p$ timesteps. This assumption is
crucial to satisfying the constraints regarding collision aversion and
connectivity maintenance between neighbouring agents.
We assume that the above pieces of information are (a) always available and
accurate, and (b) exchanged without delay. We encapsulate these pieces of
information in four stacked vectors:
\begin{subequations}
\begin{align}
  \vect{z}_{\mathcal{R}_i}(t_k) &\triangleq col[\vect{z}_j(t_k)], \forall j \in \mathcal{R}_i(t_k) \\
  \vect{z}_{\mathcal{N}_i}(t_k) &\triangleq col[\vect{z}_j(t_k)], \forall j \in \mathcal{N}_i \\
  \overline{\vect{z}}_{\mathcal{R}_i}(\tau) &\triangleq col[\overline{\vect{z}}_j(\tau)], \forall j \in \mathcal{R}_i(t_k), \tau \in [t_k, t_k + T_p] \\
  \overline{\vect{z}}_{\mathcal{N}_i}(\tau) &\triangleq col[\overline{\vect{z}}_j(\tau)], \forall j \in \mathcal{N}_i, \tau \in [t_k, t_k + T_p]
\end{align}
\end{subequations}

\begin{bw_box}
\begin{remark}
The justification for this assumption is the following: considering that
$\mathcal{N}_i \subseteq \mathcal{R}_i$, that the state
vectors $\vect{z}_j$ are comprised of 12 real numbers that are encoded by
4 bytes, and that sampling occurs with a frequency $f$ for all agents, the
overall downstream bandwidth required by each agent is
$$BW_d = 12 \times 32\ \text{[bits]} \times |\mathcal{R}_i| \times \dfrac{T_p}{h} \times f\ [\text{sec}^{-1}]$$
Given conservative constants $f = 100$ Hz, $\dfrac{T_p}{h} = 100$, the
wireless protocol IEEE 802.11n-2009 (a standard for present-day devices)
can accommodate up to

$$|\mathcal{R}_i| = \dfrac{600\ [\text{Mbit}\cdot \text{sec}^{-1}] }{12\times32[\text{bit}]\times10^4 [\text{sec}^{-1}]} \approx
16 \cdot 10^2 \text{ agents}$$ within the range of one agent.
We deem this number to be large enough for practical applications
for the approach of assuming access to the predicted states of agents
within the range of one agent to be legal.
\end{remark}
\end{bw_box}

\note{?? more on the actual $\mathcal{E}_i$}

While in the disturbance-free case the constraint set is $\mathcal{E}_i$,
due to the existence of disturbances here, the constraint set is replaced in
problem \eqref{problem:opt_with_disturbances} by
\begin{align}
  \mathcal{E}_{i, s-t_k} \equiv \mathcal{E}_i \oplus \mathcal{B}_{i,s-t_k}
\label{eq:restricted_constraint_set}
\end{align}
where
\begin{align}
  \mathcal{B}_{i,s-t_k} \equiv \big\{ \vect{e}_i \in \mathbb{R}^9 \times \mathbb{T}^3 :
    \|\vect{e}_i(s)\| \leq \dfrac{\overline{\delta}_i}{L_{g_i}}\big( e^{L_{g_i}(s - t_k)} - 1\big),\ \forall s \in [t_k, t_k + T_p] \big\}
\label{eq:b_restricted_constraint_set}
\end{align}
The reason for this substitution lies in the following. Consider that there
are no disturbances affecting the states of the plant; the state evolution of
the plant and its model considered in the solution to the optimization problem
abide both by the state constraints since the two models are identical. Consider
now that there are disturbances affecting the states of the plant, disturbances
that are unknown to the model considered in the solution to the optimization
problem. If the state constraint set was left unchanged during the solution of
the optimization problem, then the uncertainty affecting the plant could,
without loss of generality\footnote{Receding Horizon Control is inherently robust
under certain considerations, see \cite{Fontes2007} for more.}, force the states
of the plant to escape their intended bounds.

If the state constraint set considered in the solution of the optimization
problem \eqref{problem:opt_with_disturbances} is equal to
\eqref{eq:restricted_constraint_set}, then the state of the real system,
the plant, is guaranteed to abide by the original state constraint set
$\mathcal{E}_i$. We formalize this statement in property
\eqref{property:restricted_constraint_set}.

\begin{bw_box}
\begin{property}

  For every $s \in [t_k, t_k + T_p]$
  \begin{align}
    \overline{\vect{e}}_i\big( s;\ \vect{u}_i(\cdot,\ \vect{e}_i(t_k)), \vect{e}_i(t_k) \big) \in \mathcal{E}_i \oplus \mathcal{B}_{i,s-t_k}
    \Rightarrow
    \vect{e}_i(s) \in \mathcal{E}_i
  \end{align}
  where $\mathcal{B}_{i,s-t_k}$ is given by \eqref{eq:b_restricted_constraint_set}.
\label{property:restricted_constraint_set}
\end{property}
\end{bw_box}

\begin{gg_box}
\textbf{Proof of property \eqref{property:restricted_constraint_set}}

Let us define for convenience $\vect{\zeta}_i : \mathbb{R}_{\geq 0} \to \mathbb{R}^9 \times \mathbb{T}^3$:
$\vect{\zeta}_i(s) \triangleq \vect{e}_i(s) - \overline{\vect{e}}_i(s;\ \vect{u}_i(s;\ \vect{e}_i(t_k)), \vect{e}_i(t_k))$,
for $s \in [t_k, t_k + T_p]$.

According to lemma
\eqref{lemma:diff_state_from_same_conditions}
\begin{align}
 \|\vect{e}_i(s) - \overline{\vect{e}}_i(s;\ \vect{u}_i(s;\ \vect{e}_i(t_k)), \vect{e}_i(t_k))\|
   &\leq \dfrac{\overline{\delta}_i}{\L_{g_i}} (e^{L_{g_i} (s-t_k)} - 1) \\
 \|\vect{\zeta}_i(s)\| &\leq \dfrac{\overline{\delta}_i}{\L_{g_i}} (e^{L_{g_i} (s-t_k)} - 1)
\end{align}
which means that $\vect{\zeta}_i(s) \in \mathcal{B}_{i,s-t_k}$.
Now let us assume that
$\overline{\vect{e}}_i\big( s;\ \vect{u}_i(\cdot,\ \vect{e}_i(t_k)), \vect{e}_i(t_k) \big) \in \mathcal{E}_i \oplus \mathcal{B}_{i,s-t_k}$.
Then,
\begin{align}
  \overline{\vect{e}}_i\big( s;\ \vect{u}_i(\cdot,\ \vect{e}_i(t_k)), \vect{e}_i(t_k) \big) &\in \mathcal{E}_i \oplus \mathcal{B}_{i,s-t_k} \\
  \vect{e}_i(s) - \vect{\zeta}_i(s) &\in \mathcal{E}_i \oplus \mathcal{B}_{i,s-t_k}
\end{align}
If $\vect{\zeta}_i(s)$ resides in $\mathcal{B}_{i,s-t_k}$, then
$-\vect{\zeta}_i(s)$ also resides in $\mathcal{B}_{i,s-t_k}$, which means that
\begin{align}
  \vect{e}_i(s) + \big(-\vect{\zeta}_i(s)\big) &\in \mathcal{E}_i \oplus \mathcal{B}_{i,s-t_k} \\
  \vect{e}_i(s) &\in \mathcal{E}_i
\end{align}
\qedsymbol
\end{gg_box}




Before defining the terminal set
$\mathcal{E}_{i,f}$ it is necessary to state the definition of a positively
invariant set:
\begin{bw_box}
  \begin{definition} (\textit{Positively Invariant Set})

    Consider a dynamical system $\dot{\vect{x}} = f(\vect{x})$,
    $\vect{x} \in \mathbb{R}^n$, and a trajectory $\vect{x}(t;\ \vect{x}_0)$,
    where $\vect{x}_0$ is the initial condition. The set
    $S = \{\vect{x} \in \mathbb{R}^n : \gamma(\vect{x}) = 0\}$, where
    $\gamma$ is a valued function, is said to be \textit{positively invariant}
    if the following holds:
    \begin{align}
      \vect{x}_0 \in S \Rightarrow \vect{x}(t;\ \vect{x}_0) \in S,\ \forall t \geq t_0
    \end{align}

    Intuitively, this means that the set $S$ is positively invariant if a
    trajectory of the system does not exit it once it enters it.
    \label{def:positively_invariant_2}
  \end{definition}
\end{bw_box}

The terminal set
$\mathcal{E}_{i,f} \subseteq \mathcal{E}_{i,F}$ is a
subset of an admissible positively invariant set
$\mathcal{E}_{i,F} \subseteq \mathcal{E}_i$ for system
\eqref{eq:position_based_error_model_2}, where $\mathcal{E}_{i,F}$ is defined as
\begin{align}
  \mathcal{E}_{i,F} \triangleq \big\{\vect{e}_i \in \mathcal{E}_i : V_i(\vect{e}_i) \leq \varepsilon_F,\ \varepsilon_F > 0 \big\}
\end{align}

In particular, $\mathcal{E}_{i,F}$ is a subset of the restricted constraint
set that corresponds to the constraints on the predicted state
$\overline{\vect{e}}_i(t_k + T_p)$, which is the set
$\mathcal{E}_{i,t_k + T_p - t_k} \equiv \mathcal{E}_{i,T_p}$. Using our
adopted notation, $\mathcal{E}_{i,T_p} = \mathcal{E}_i \oplus \mathcal{B}_{i, T_p}$

and is such that for all
$\vect{e}_i \in \mathcal{E}_{i,F} \Rightarrow g_i(\vect{e}_i, \vect{h}_i(\vect{e}_i)) \in \mathcal{E}_{i,f}$.


The terminal set $\mathcal{E}_{i,f}$ is defined as
\begin{align}
  \mathcal{E}_{i,f} \triangleq \big\{\vect{e}_i \in \mathcal{E}_i : V_i(\vect{e}_i) \leq \varepsilon_f,\ \varepsilon_f \in (0, \varepsilon_F) \big\}
\end{align}

\begin{figure}[ht!]
  \centering
  \begin{tikzpicture}[scale = 1]
  \draw (2,2) ellipse (6cm and 3cm);
    \node at ($(2.5,2.5)+(75:6 and 3)$) {$\mathcal{E}_i$};
  \draw[dashed] (2,2) ellipse (5cm and 2.5cm);
    \node at ($(1.7,1.7)+(75:5 and 2.5)$) {$\mathcal{E}_i \oplus \mathcal{B}_{i,t_{k+1} - t_k}$};
  \draw[dashed] (2,2) ellipse (2cm and 1cm);
    \node at ($(1.6,1.6)+(75:2 and 1)$) {$\mathcal{E}_i \oplus \mathcal{B}_{i,t_k + T_p - t_k}$};

  \node at (5.5,2) {$\dots$};
  \node at (-1.5,2) {$\dots$};
\end{tikzpicture}

  \caption{The nominal constraint set $\mathcal{E}_i$ in bold and the
    consecutive restricted constraint sets $\mathcal{E}_i \oplus \mathcal{B}_{i, s-t_k}$,
    $s \in [t_k, t_k + T_p]$, dashed.}
\end{figure}

\begin{figure}[ht!]
  \centering
  \begin{tikzpicture}[scale = 1, rotate=-30]
  \draw[dashed](2,2) ellipse (5cm and 2.5cm);
    \node at ($(2.7,2.7)+(75:5 and 2.5)$) {$\mathcal{E}_i \ominus \mathcal{B}_{i,T_p-h}$};
  \draw[dashdotted](2,2) ellipse (4cm and 2cm);
    \node at ($(2.2,2.2)+(75:4 and 2)$) {$\Phi_i$};
  \draw[dashdotted] (2,2) ellipse (3cm and 1.5cm);
    \node at ($(2.2,2.2)+(75:3 and 1.5)$) {$\Psi_i$};
  \draw (2,2) ellipse (2cm and 1cm);
    \node at ($(2.2,2.2)+(75:2 and 1)$) {$\Omega_i$};
\pgflowlevel{\pgftransformrotate{30}}
\end{tikzpicture}

    %\caption{}
\end{figure}


The functions
$F_i : \mathcal{E}_i \times \mathcal{U}_i \to \mathbb{R}_{\geq 0}$ and
$V_i: \mathcal{E}_{i,f} \to \mathbb{R}_{\geq 0}$ are defined as
\begin{align}
  F_i \big(\overline{\vect{e}}_i(t), \overline{\vect{u}}_i(t)\big)
  &\triangleq \overline{\vect{e}}_i(t)^{\top} \mat{Q}_i \overline{\vect{e}}_i(t) + \overline{\vect{u}}_i(t)^{\top} \mat{R}_i \overline{\vect{u}}_i(t)\\
  V_i \big(\overline{\vect{e}}_i(t)\big) & \triangleq \overline{\vect{e}}_i(t)^{\top} \mat{P}_i \overline{\vect{e}}_i(t)
\end{align}
Matrices $\mat{R}_i \in \mathbb{R}^{6 \times 6}$ are symmetric and positive
definite, while matrices $\mat{Q}_i, \mat{P}_i \in \mathbb{R}^{12 \times 12}$
are symmetric and positive semi-definite. The running costs $F_i$ are
upper- and lower-bounded:
\begin{align}
  \lambda_{min}(\mat{Q}_i, \mat{R}_i)\|\vect{e}_i(t)\|^2 &\leq
  \lambda_{min}(\mat{Q}_i, \mat{R}_i) \Bigg\| \begin{bmatrix}
      \vect{e}_i(t) \\
      \vect{u}_i(t)
    \end{bmatrix}\Bigg\|^2  \\
    &\leq F_i \big(\vect{e}_i(t), \vect{u}_i(t)\big) \\
    &\leq \lambda_{max}(\mat{Q}_i, \mat{R}_i) \Bigg\| \begin{bmatrix}
      \vect{e}_i(t) \\
      \vect{u}_i(t)
    \end{bmatrix}\Bigg\|^2 \leq
  \lambda_{max}(\mat{Q}_i, \mat{R}_i) \|\vect{e}_i(t)\|^2
\end{align}
where $\lambda_{min}(\mat{Q}_i, \mat{R}_i)$ is the smallest eigenvalue between
those of matrices $\mat{Q}_i$ and $\mat{R}_i$, and
$\lambda_{max}(\mat{Q}_i, \mat{R}_i)$ the largest. Since the terms
$\lambda_{min}(\mat{Q}_i, \mat{R}_i) \|\vect{e}_i(t)\|$ and
$\lambda_{max}(\mat{Q}_i, \mat{R}_i) \|\vect{e}_i(t)\|$ are themselves
class $\mathcal{K}$ functions according to definition \eqref{def:k_class_2},
$F_i$ is lower- and upper-bounded by class $\mathcal{K}$ functions. As is
obvious, $F_i(\vect{0}, \vect{0}) = 0$.


With regard to the terminal penalty function $V_i$, the following lemma will
prove to be useful in guaranteeing the convergence of the solution to the
optimal control problem to the terminal region $\mathcal{E}_{i,f}$:

\begin{bw_box}
\begin{lemma} ($V_i$ is Lipschitz continuous in $\mathcal{E}_{i,F}$)

  The terminal penalty function $V_i$ is Lipschitz continuous in
  $\mathcal{E}_{i,F}$
  $$|V(\vect{e}_{1,i}) - V(\vect{e}_{2,i})| \leq L_{V_i} \|\vect{e}_{1,i} - \vect{e}_{2,i}\|$$
  where $\vect{e}_{1,i}, \vect{e}_{2,i} \in \mathcal{E}_{i,F}$,
  with Lipschitz constant $L_{V_i} = 2 \varepsilon_F \lambda_{max}(P_i)$\\

\label{lemma:V_Lipschitz_e_0_2}
\end{lemma}
\end{bw_box}


\begin{gg_box}
\textbf{Proof} For every $\vect{e}_i \in \mathcal{E}_{i,f}$, it holds that
\begin{align}
  |V(\vect{e}_{1,i}) - V(\vect{e}_{2,i})| &= |\vect{e}_{1,i}^{\top} \mat{P}_i \vect{e}_{1,i} - \vect{e}_{2,i}^{\top} \mat{P}_i \vect{e}_{2,i}| \\
    &= |\vect{e}_{1,i}^{\top} \mat{P}_i \vect{e}_{1,i} - \vect{e}_{2,i}^{\top} \mat{P}_i \vect{e}_{2,i} \pm \vect{e}_{1,i}^{\top} \mat{P}_i \vect{e}_{2,i}| \\
    &= |\vect{e}_{1,i}^{\top} \mat{P}_i (\vect{e}_{1,i}-\vect{e}_{2,i}) - \vect{e}_{2,i}^{\top} \mat{P}_i (\vect{e}_{1,i}-\vect{e}_{2,i})| \\
    &\leq |\vect{e}_{1,i}^{\top} \mat{P}_i (\vect{e}_{1,i}-\vect{e}_{2,i})| + |\vect{e}_{2,i}^{\top} \mat{P}_i (\vect{e}_{1,i}-\vect{e}_{2,i})|
\end{align}

But for any $\vect{x}, \vect{y} \in \mathbb{R}^n$
$$|\vect{x}^{\top} \mat{A} \vect{y}| \leq \lambda_{max}(A) \|\vect{x}\| \|\vect{y}\|$$
where $\lambda_{max}(A)$ denotes the largest eigenvalue of matrix $\mat{A}$.
Hence:
\begin{align}
  |V(\vect{e}_{1,i}) - V(\vect{e}_{2,i})| &\leq
  \lambda_{max}(\mat{P}_i) \|\vect{e}_{1,i}\| \|\vect{e}_{1,i} - \vect{e}_{2,i}\| +
  \lambda_{max}(\mat{P}_i) \|\vect{e}_{2,i}\| \|\vect{e}_{1,i} - \vect{e}_{2,i}\| \\
  &= \lambda_{max}(\mat{P}_i) (\|\vect{e}_{1,i}\| + \|\vect{e}_{2,i}\|)\|\vect{e}_{1,i} - \vect{e}_{2,i}\| \\
  & \leq \lambda_{max}(\mat{P}_i) (\varepsilon_F + \varepsilon_F)\|\vect{e}_{1,i} - \vect{e}_{2,i}\| \\
  &= 2 \varepsilon_F \lambda_{max}(\mat{P}_i) \|\vect{e}_{1,i} - \vect{e}_{2,i}\|
\end{align}
\qedsymbol
\end{gg_box}


The solution to the optimal control problem \eqref{position_based_cost_2}
at time $t_k$ is an optimal control input, denoted by
$\overline{\vect{u}}_i^{\star}(\cdot;\ \vect{e}_i(t_k))$, which
is applied to the open-loop system until the next sampling instant $t_k + h$,
with $h \in (0,T_p)$, at which time a new optimal control problem is solved in
the same manner:
\begin{align}
  %\vect{u}_i\big(t;\ \vect{e}_i(t_k)\big) = \overline{\vect{u}}_i^{\star}\big(t;\ \vect{e}_i(t_k)\big),\  t \in [t_k, t_k + h) \nonumber \\
  \vect{u}_i(t) = \overline{\vect{u}}_i^{\star}\big(t;\ \vect{e}_i(t_k)\big),\  t \in [t_k, t_k + h]
 \label{eq:position_based_optimal_u_2}
\end{align}
The control input $\vect{u}_i(\cdot)$ is of feedback form,
since it is recalculated at each sampling instant based on the then-current
state. The solution to equation \eqref{eq:position_based_error_model_with_disturbance}, starting at time
$t_1$, from an initial condition $\vect{e}_i(t_1)$, by application of the
control input $\vect{u}_i : [t_1, t_2] \to \mathcal{U}_i$ is denoted by
\begin{align}
  \vect{e}_i\big(t;\ \vect{u}_i(\cdot), \vect{e}_i(t_1)\big),\ t \in [t_1, t_2]
\end{align}

The \textit{predicted} state of the system \eqref{eq:position_based_error_model_with_disturbance}
at time $t_k + \tau$, based on the measurement of the state at time
$t_k$, $\vect{e}_i(t_k)$, by application of the control input
$\vect{u}_i\big(t;\ \vect{e}_i(t_k)\big)$, for the time period $t \in [t_k, t_k + \tau]$
is denoted by
\begin{align}
  \overline{\vect{e}}_i\big(t_k + \tau;\ \vect{u}_i(\cdot), \vect{e}_i(t_k)\big) \label{eq:position_based_predicted_error_0_2}
\end{align}
In contrast to the disturbance-free case:
\begin{align}
  \overline{\vect{e}}_i\big(\tau_1;\ \vect{u}_i(\cdot), \vect{e}_i(\tau_0)\big) \not=
  \vect{e}_i\big(\tau_1;\ \vect{u}_i(\cdot), \vect{e}_i(\tau_0)\big)
  \label{eq:error_now_to_predicted_error_2}
\end{align}
holds true here because \textit{there are} disturbances acting on the system.

The closed-loop system for which stability is to be guaranteed is
\begin{align}
  \vect{e}_i(\tau) = g_i^R\big(\vect{e}_i(\tau), \overline{\vect{u}}_i^{\star}(\tau)\big),\ \tau \geq t_0 = 0
  \label{eq:without_disturbances_closed_loop_2}
\end{align}
where $\overline{\vect{u}}_i^{\star}(\tau) = \overline{\vect{u}}_i^{\star}(\tau;\ \vect{e}_i(t_k))$,
$\tau \in [t_k, t_k + h)$.

We can now give the definition of an \textit{admissible input}:

\begin{bw_box}
\begin{definition} (Admissible input)\\

  A control input $\vect{u}_i : [t_k, t_k + T_p] \to \mathbb{R}^6$ for a state
  $\vect{e}_i(t_k)$ is called \textit{admissible} if all the following hold:

  \begin{enumerate}
    \item $\vect{u}_i(\cdot)$ is piecewise continuous
    \item $\vect{u}_i(\tau) \in \mathcal{U}_i,\ \forall \tau \in [t_k, t_k + T_p]$
    \item $\vect{e}_i\big(\tau;\ \vect{u}_i(\cdot), \vect{e}_i(t_k)\big) \in \mathcal{E}_i,\ \forall \tau \in [t_k, t_k + T_p]$
    \item $\vect{e}_i\big(t_k + T_p;\ \vect{u}_i(\cdot), \vect{e}_i(t_k)\big) \in \mathcal{E}_{i,f}$
  \end{enumerate}

\end{definition}
\end{bw_box}
