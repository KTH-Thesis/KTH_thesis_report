%-------------------------------------------------------------------------------
\subsection{The optimization problem}
Consider a sequence of sampling times $\{t_k\}_{k \geq 0}$, with a constant
sampling time $h$, $0 < h < T_p$, where $T_p$ is the finite time-horizon, such
that $t_{k+1} = t_k + h$. In sampling data NMPC, a finite-horizon open-loop
optimal control problem (FHOCP) is solved at discrete sampling time instants
$t_k$ based on the then-current state error measurement $\vect{e}_i(t_k)$. The
solution is an optimal control signal $\overline{\vect{u}}_i^{\star}(t)$,
computed over $t \in [t_k, t_k+T_p]$. This signal is applied to the open-loop
system in between sampling times $t_k$ and $t_k + h$.

At a generic time $t_k$ then, agent $i$ solves the following optimization
problem:
\begin{problem}
\label{problem:opt_with_disturbances}
\begin{align}
  \text{Find }& \\[2.5ex]
              &J_i^{\star} \big(\vect{e}_i(t_k)\big) \triangleq \text{min }\limits_{\overline{\vect{u}}_i (\cdot)}\
    J_i \big(\vect{e}_i(t_k), \overline{\vect{u}}_i (\cdot) \big) \label{position_based_cost_2} \\[2.5ex]
    \text{where}& \\[2.5ex]
    &J_i \big(\vect{e}_i(t_k), \overline{\vect{u}}_i (\cdot) \big) \triangleq
      \int_{t_k}^{t_k + T_p} F_i \big(\overline{\vect{e}}_i(s), \overline{\vect{u}}_i (s)\big) ds +
      V_i \big(\overline{\vect{e}}_i (t_k + T_p)\big)  \\[2.5ex]
  \text{subject to:} & \nonumber \\[2.5ex]
                     & \dot{\overline{\vect{e}}}_i(s) = g_i \big(\overline{\vect{e}}_i (s), \overline{\vect{u}}_i (s)\big) \label{eq:internal_error_model_2},\ \overline{\vect{e}}_i (t_k) = \vect{e}_i (t_k) \\[2.5ex]
                     & \overline{\vect{u}}_i(s) \in \mathcal{U}_i,\ \overline{\vect{e}}_i (s) \in \mathcal{E}_{i, s - t_k},\ s \in [t_k, t_k + T_p]\\[2.5ex]
                     & \overline{\vect{e}}_i (t_k + T_p) \in \Omega_i
\end{align}
\end{problem}
The notation $\overline{\cdot}$ is used to distinguish predicted states which
are internal to the controller, as opposed to their actual values, because,
even in the nominal case, the predicted values will not be equal to the
actual closed-loop values. This means
that $\overline{\vect{e}}_i(\cdot)$ is the solution to
\eqref{eq:internal_error_model_2} driven by the control input
$\overline{\vect{u}}_i(\cdot) : [t_k, t_k + T_p] \to \mathcal{U}_i$ with
initial condition $\vect{e}_i(t_k)$.

The applied input signal is a portion of the optimal solution to an
optimization problem where information on the states of the neighbouring agents
of agent $i$ are taken into account only in the constraints considered in the
optimization problem. These constraints pertain to the set of its neighbours
$\mathcal{N}_i$ and, in total, to the set of all agents within its sensing
range $\mathcal{R}_i$. Regarding these, we assume assumption
\eqref{ass:access_to_predicted_info_n}, i.e. at time $t_k$ when agent $i$
solves the optimization problem, he has access to the measurements of the states
and the values of the predicted states for all agents within its sensing range.


\note{?? more on the actual $\mathcal{E}_i$}


\begin{figure}[ht!]
  \centering
  \begin{tikzpicture}[scale = 1, rotate=-30]
  \draw (2,2) ellipse (6cm and 3cm);
    \node at ($(2.5,2.5)+(75:6 and 3)$) {$\mathcal{E}_i$};
  \draw[dashed] (2,2) ellipse (5cm and 2.5cm);
    \node at ($(-2.3,1.2)+(75:5 and 2.5)$) {$\mathcal{E}_i \ominus \mathcal{B}_{i,t_{k+1} - t_k}$};
  \draw[dashed] (2,2) ellipse (2cm and 1cm);
    \node at ($(1.2,1.2)+(75:2 and 1)$) {$\mathcal{E}_i \ominus \mathcal{B}_{i,t_k + T_p - t_k}$};

  \node at (5.5,2) {$\dots$};
  \node at (-1.5,2) {$\dots$};

\pgflowlevel{\pgftransformrotate{30}}
\end{tikzpicture}

  \caption{The nominal constraint set $\mathcal{E}_i$ in bold and the
    consecutive restricted constraint sets $\mathcal{E}_i \ominus \mathcal{B}_{i, s-t_k}$,
    $s \in [t_k, t_k + T_p]$, dashed.}
\end{figure}

While in the disturbance-free case the constraint set is $\mathcal{E}_i$,
due to the existence of disturbances here, the constraint set is replaced in
problem \eqref{problem:opt_with_disturbances} by
\begin{align}
  \mathcal{E}_{i, s-t_k} \equiv \mathcal{E}_i \ominus \mathcal{B}_{i,s-t_k}
\label{eq:restricted_constraint_set}
\end{align}
where
\begin{align}
  \mathcal{B}_{i,s-t} \equiv \big\{ \vect{e}_i \in \mathbb{R}^9 \times \mathbb{T}^3 :
    \|\vect{e}_i(s)\| \leq \dfrac{\overline{\delta}_i}{L_{g_i}}\big( e^{L_{g_i}(s - t)} - 1\big),\ \forall s \in [t, t + T_p] \big\}
\label{eq:b_restricted_constraint_set}
\end{align}
The reason for this substitution lies in the following. Consider that there
are no disturbances affecting the states of the plant; the state evolution of
the plant and its model considered in the solution to the optimization problem
abide both by the state constraints since the two models are identical. Consider
now that there are disturbances affecting the states of the plant, disturbances
that are unknown to the model considered in the solution to the optimization
problem. If the state constraint set was left unchanged during the solution of
the optimization problem, the applied input to the plant, coupled with the
uncertainty affecting the states of the plant could, without loss of
generality\footnote{Receding Horizon Control is inherently robust
under certain considerations, see \cite{Fontes2007} for more.}, force the states
of the plant to escape their intended bounds.

If the state constraint set considered in the solution of the optimization
problem \eqref{problem:opt_with_disturbances} is equal to
\eqref{eq:restricted_constraint_set}, then the state of the real system,
the plant, is guaranteed to abide by the original state constraint set
$\mathcal{E}_i$. We formalize this statement in property
\eqref{property:restricted_constraint_set}.

\begin{bw_box}
  \begin{property}
  \label{property:restricted_constraint_set}

  For every $s \in [t, t + T_p]$
  \begin{align}
    \overline{\vect{e}}_i\big( s;\ \vect{u}_i(\cdot,\ \vect{e}_i(t)), \vect{e}_i(t) \big) \in \mathcal{E}_i \ominus \mathcal{B}_{i,s-t}
    \Rightarrow
    \vect{e}_i(s) \in \mathcal{E}_i
  \end{align}
  where $\mathcal{B}_{i,s-t}$ is given by \eqref{eq:b_restricted_constraint_set}.
\end{property}
\end{bw_box}

\begin{gg_box}
\textbf{Proof of property \eqref{property:restricted_constraint_set}}

Let us define for convenience $\vect{\zeta}_i : \mathbb{R}_{\geq 0} \to \mathbb{R}^9 \times \mathbb{T}^3$:
$\vect{\zeta}_i(s) \triangleq \vect{e}_i(s) - \overline{\vect{e}}_i(s;\ \vect{u}_i(s;\ \vect{e}_i(t)), \vect{e}_i(t))$,
for $s \in [t, t + T_p]$.

According to lemma
\eqref{lemma:diff_state_from_same_conditions}
\begin{align}
 \|\vect{e}_i(s) - \overline{\vect{e}}_i\big(s;\ \vect{u}_i(s;\ \vect{e}_i(t)), \vect{e}_i(t)\big)\|
   &\leq \dfrac{\overline{\delta}_i}{\L_{g_i}} (e^{L_{g_i} (s-t)} - 1) \\[2.5ex]
 \|\vect{\zeta}_i(s)\| &\leq \dfrac{\overline{\delta}_i}{\L_{g_i}} (e^{L_{g_i} (s-t)} - 1)
\end{align}
which means that $\vect{\zeta}_i(s) \in \mathcal{B}_{i,s-t}$.
Now let us assume that
$\overline{\vect{e}}_i\big( s;\ \vect{u}_i(\cdot,\ \vect{e}_i(t)), \vect{e}_i(t) \big) \in \mathcal{E}_i \ominus \mathcal{B}_{i,s-t}$.
Then, we add the two include statements:
\begin{align}
  \overline{\vect{e}}_i\big( s;\ \vect{u}_i(\cdot,\ \vect{e}_i(t)), \vect{e}_i(t) \big) &\in \mathcal{E}_i \ominus \mathcal{B}_{i,s-t} \\[2.5ex]
  \vect{\zeta}_i(s) &\in \mathcal{B}_{i,s-t}
\end{align}
which yields
\begin{align}
  \vect{\zeta}_i(s) + \overline{\vect{e}}_i\big(s;\ \vect{u}_i(s;\ \vect{e}_i(t)), \vect{e}_i(t)\big)
    &\in \big(\mathcal{E}_i \ominus \mathcal{B}_{i,s-t}\big) \oplus \mathcal{B}_{i,s-t}
\end{align}
Utilizing Theorem 2.1 (ii) from \cite{kolmanovsky} yields
\begin{align}
  \vect{\zeta}_i(s) + \overline{\vect{e}}_i\big(s;\ \vect{u}_i(s;\ \vect{e}_i(t)), \vect{e}_i(t)\big) &\in \mathcal{E}_i \\[2.5ex]
  \vect{e}_i(s) &\in \mathcal{E}_i
\end{align}
\qedsymbol
\end{gg_box}

\begin{bw_box}
  \begin{assumption}
  \label{ass:psi}
  The terminal set $\Omega_i \subseteq \Psi_i$ is a subset of an admissible and
  positively invariant set $\Psi_i$ as per definition
  \eqref{def:positively_invariant}, where $\Psi_i$ is defined as
  \begin{align}
    \Psi_i \triangleq \big\{\vect{e}_i \in \mathcal{E}_i : V_i(\vect{e}_i)
      \leq \varepsilon_{\Psi_i} \big\},\ \varepsilon_{\Psi_i} > 0
  \end{align}
  \end{assumption}
\end{bw_box}

\begin{bw_box}
  \begin{assumption}
  \label{ass:phi_psi}
  The set $\Psi_i$ belongs to the set $\Phi_i$, $\Psi_i \subseteq \Phi_i$,
  which is the set of states within $\mathcal{E}_{i,T_p-h}$ for which there is an
  admissible control input whose form is of linear feedback with regard to the
  state:
  \begin{align}
    \Phi_i \triangleq \big\{\vect{e}_i \in \mathcal{E}_{i,T_p-h} : \vect{h}_i(\vect{e}_i) \in \mathcal{U}_i \big\}
  \end{align}
  \end{assumption}
\end{bw_box}


\begin{bw_box}
  \begin{assumption}
  \label{ass:psi_omega}
  The admissible and positively invariant set $\Psi_i$ is such that
  \begin{align}
    \forall \vect{e}_i \in \Psi_i \Rightarrow g_i(\vect{e}_i, \vect{h}_i(\vect{e}_i)) \in \Omega_i \subseteq \Psi_i
  \end{align}
  \end{assumption}
\end{bw_box}

\begin{bw_box}
  \begin{assumption}
  \label{ass:omega}
  The terminal set $\Omega_i$ is defined as
  \begin{align}
    \Omega_i \triangleq \big\{\vect{e}_i \in \mathcal{E}_i : V_i(\vect{e}_i)
      \leq \varepsilon_{\Omega_i}\big\}\ \text{, where } \varepsilon_{\Omega_i} \in (0, \varepsilon_{\Psi_i})
  \end{align}
  \end{assumption}
\end{bw_box}

\begin{figure}[ht!]
  \centering
  \begin{tikzpicture}[scale = 1, rotate=-30]
  \draw[dashed](2,2) ellipse (5cm and 2.5cm);
    \node at ($(2.7,2.7)+(75:5 and 2.5)$) {$\mathcal{E}_i \ominus \mathcal{B}_{i,T_p}$};
  \draw[dashdotted](2,2) ellipse (4cm and 2cm);
    \node at ($(2.2,2.2)+(75:4 and 2)$) {$\Phi_i$};
  \draw[dashdotted] (2,2) ellipse (3cm and 1.5cm);
    \node at ($(2.2,2.2)+(75:3 and 1.5)$) {$\Psi_i$};
  \draw (2,2) ellipse (2cm and 1cm);
    \node at ($(2.2,2.2)+(75:2 and 1)$) {$\Omega_i$};
\pgflowlevel{\pgftransformrotate{30}}
\end{tikzpicture}

  \caption{The hierarchy of sets
  $\Omega_i \subseteq \Psi_i \subseteq \Phi_i \subseteq \mathcal{E}_{i,T_p-h}$,
  in bold, dash-dotted, dash-dotted, and dashed, respectively.
  For every state in $\Phi_i$ there is a linear state feedback control
  $\vect{h}_i(\vect{e}_i)$ which, when applied to a state
  $\vect{e}_i \in \Psi_i$, causes the trajectory of the state of the system to
  fall into the terminal set $\Omega_i$.}
\end{figure}


Functions
$F_i : \mathcal{E}_i \times \mathcal{U}_i \to \mathbb{R}_{\geq 0}$ and
$V_i: \Psi_i \to \mathbb{R}_{\geq 0}$ are defined by \eqref{eq:F_i_def}
and \eqref{eq:V_i_def} respectively, as in the disturbance-free case.
Matrices $\mat{R}_i \in \mathbb{R}^{6 \times 6}$,
$\mat{Q}_i, \mat{P}_i \in \mathbb{R}^{12 \times 12}$ are positive definite.
Consequently, lemmas \eqref{lemma:F_i_bounded_K_class},
\eqref{lemma:F_Lipschitz}, \eqref{lemma:V_Lipschitz_e_0} and
\eqref{lemma:V_i_lower_upper_bounded} hold true here, as in the
disturbance-free case: the running costs $F_i$ are Lipschitz continuous in
$\mathcal{E}_i \times \mathcal{U}_i$ with Lipschitz constant $L_{F_i}$ and
they are lower- and upper-bounded by class $\mathcal{K}_{\infty}$ functions;
the terminal penalty functions $V_i$ are Lipschitz continuous in $\Psi_i$
with Lipschitz constant $L_{V_i}$, and they are lower- and upper-bounded
by class $\mathcal{K}_{\infty}$ functions.


The solution to the optimal control problem \eqref{position_based_cost_2}
at time $t_k$ is an optimal control input, denoted by
$\overline{\vect{u}}_i^{\star}(\cdot;\ \vect{e}_i(t_k))$, which
is applied to the open-loop system until the next sampling instant $t_k + h$,
with $h \in (0,T_p)$.
\begin{align}
  %\vect{u}_i\big(t;\ \vect{e}_i(t_k)\big) = \overline{\vect{u}}_i^{\star}\big(t;\ \vect{e}_i(t_k)\big),\  t \in [t_k, t_k + h) \nonumber \\[2.5ex]
  \vect{u}_i(t) = \overline{\vect{u}}_i^{\star}\big(t;\ \vect{e}_i(t_k)\big),\  t \in [t_k, t_k + h]
 \label{eq:position_based_optimal_u_2}
\end{align}
At time $t_{k+1}$ a new finite horizon optimal control problem is solved in the
same manner, leading to a receding horizon approach.

The control input $\vect{u}_i(\cdot)$ is of feedback form,
since it is recalculated at each sampling instant based on the then-current
state. The solution to equation \eqref{eq:position_based_error_model_with_disturbance}, starting at time
$t_1$, from an initial condition $\vect{e}_i(t_1) = \overline{\vect{e}}_i(t_1)$,
by application of the control input $\vect{u}_i : [t_1, t_2] \to \mathcal{U}_i$
is denoted by
\begin{align}
  \vect{e}_i\big(t;\ \vect{u}_i(\cdot), \vect{e}_i(t_1)\big),\ t \in [t_1, t_2]
\end{align}

As before, the \textit{predicted} state of the system
\eqref{eq:position_based_error_model_with_disturbance}
at time $t_k + \tau$, based on the measurement of the state at time
$t_k$, $\vect{e}_i(t_k)$, by application of the control input
$\vect{u}_i\big(t;\ \vect{e}_i(t_k)\big)$, for the time period $t \in [t_k, t_k + \tau]$
is denoted by
\begin{align}
  \overline{\vect{e}}_i\big(t_k + \tau;\ \vect{u}_i(\cdot), \vect{e}_i(t_k)\big) \label{eq:position_based_predicted_error_0_2}
\end{align}

On the existence of solutions to
\eqref{eq:position_based_error_model_with_disturbance} we assume the following:
\begin{bw_box}
\begin{assumption}
  \label{ass:existence_of_solutions_with_disturbance}

  The system \eqref{eq:position_based_error_model_with_disturbance} has a
  \textit{continuous solution} for any $\vect{e}_i(0) \in \mathcal{E}_i$,
  any \textit{piecewise continuous} input
  $\vect{u}_i(\cdot) :[0,T_p] \to \mathcal{U}_i$, and any
  \textit{exogenous disturbance} $\delta_i(\cdot) : [0,T_p] \to \Delta_i$.
\end{assumption}
\end{bw_box}


In contrast to the disturbance-free case where the predicted state coincided
with the state of the actual system, due to the existence of disturbances
this equality is void.

\begin{bw_box}
\begin{remark}
  The following holds true here because \textit{there are} disturbances
  acting on the system.
  \begin{align}
    \overline{\vect{e}}_i\big(\tau_1;\ \vect{u}_i(\cdot), \vect{e}_i(\tau_0)\big) \not=
    \vect{e}_i\big(\tau_1;\ \vect{u}_i(\cdot), \vect{e}_i(\tau_0)\big)
    \label{eq:error_now_to_predicted_error_2}
  \end{align}
\end{remark}
\end{bw_box}

The closed-loop system for which stability is to be guaranteed is
\begin{align}
  \vect{e}_i(\tau) = g_i^R\big(\vect{e}_i(\tau), \overline{\vect{u}}_i^{\star}(\tau)\big),\ \tau \geq t_0 = 0
  \label{eq:with_disturbances_closed_loop}
\end{align}
where $\overline{\vect{u}}_i^{\star}(\tau) = \overline{\vect{u}}_i^{\star}(\tau;\ \vect{e}_i(t_k))$,
$\tau \in [t_k, t_k + h)$.

We can now give the definition of an \textit{admissible input} for the FHOCP
\eqref{problem:opt_with_disturbances}:

\begin{bw_box}
  \begin{definition} (\textit{Admissible input for the FHOCP
    \eqref{problem:opt_with_disturbances}})
  \label{definition:admissible_input_with_disturbance}

  A control input $\vect{u}_i : [t_k, t_k + T_p] \to \mathbb{R}^6$ for a state
  $\vect{e}_i(t_k)$ is called \textit{admissible} for the problem
  \eqref{problem:opt_with_disturbances} if all the following hold:

  \begin{enumerate}
    \item $\vect{u}_i(\cdot)$ is piecewise continuous
    \item $\vect{u}_i(\tau) \in \mathcal{U}_i,\ \forall \tau \in [t_k, t_k + T_p]$
    \item $\overline{\vect{e}}_i\big(t_k + \tau;\ \vect{u}_i(\cdot), \vect{e}_i(t_k)\big) \in \mathcal{E}_i \ominus \mathcal{B}_{i,\tau},\ \forall \tau \in [0, T_p]$
    \item $\overline{\vect{e}}_i\big(t_k + T_p;\ \vect{u}_i(\cdot), \vect{e}_i(t_k)\big) \in \Omega_i$
  \end{enumerate}

  In other words, $\vect{u}_i$ is admissible if it conforms to the constraints
  on the input and its application yields states that conform to the
  prescribed state constraints of problem \eqref{problem:opt_with_disturbances}
  along the entire horizon $[t_k, t_k + T_p]$, and the terminal predicted
  state conforms to the terminal constraint.

\end{definition}
\end{bw_box}
