%-------------------------------------------------------------------------------
\section{Stabilization: Feasibility and Convergence}

Under these considerations, we can now state the theorem that relates to
the guaranteeing of the stability of the compound system of agents
$i \in \mathcal{V}$, when each of them is assigned a desired
position which results in feasible displacements:\\[2.5ex]

\begin{bw_box}
\begin{theorem}

  Suppose that

  \begin{enumerate}
    \item the terminal region $\Omega_i \subseteq \mathcal{E}_{i,s}$ is
      closed with $\vect{0} \in \Omega_i$
    \item a solution to the optimal control problem \eqref{position_based_cost}
      is feasible at time $t=0$, that is, assumptions
      \eqref{ass:measurements_access}, \eqref{ass:initial_conditions}, and
      \eqref{ass:intra_environmental_arrangement} hold at time $t=0$
    \item assumptions \eqref{ass:g_i_Lipschitz}, \eqref{ass:access_to_predicted_info_n},
      and \eqref{ass:existence_of_solutions_without_disturbance} hold
    \item there exists an admissible control input
      $h_i(\vect{e}_i) : [t_k + T_p, t_{k+1} + T_p] \to \mathcal{U}_i$
      such that for all $\vect{e}_i \in \Omega_i$ and
      $\forall \tau \in [t_k + T_p, t_{k+1} + T_p]$:

      \begin{enumerate}
        \item $\vect{e}_i(\tau) \in \Omega_i$
        \item $\dfrac{\partial V_i}{\partial \vect{e}_i} g_i\big(\vect{e}_i(\tau),
        h_i\big(\vect{e}_i(\tau)\big)
          + F_i\big(\vect{e}_i(\tau), h_i\big(\vect{e}_i(\tau)\big)\big) \leq 0$
      \end{enumerate}

  \end{enumerate}

  then the closed loop system \eqref{eq:without_disturbances_closed_loop} under
  the control input \eqref{eq:position_based_optimal_u} converges to the set
  $\Omega_i$ when $t \to \infty$.

\end{theorem}
\end{bw_box}

\textbf{Proof}. The proof of the above theorem consists of two parts:
in the first, recursive feasibility is established, that is, initial
feasibility is shown to imply subsequent feasibility; in the second, and based
on the first part, it is shown that the error state $\vect{e}_i(t)$ converges
to the terminal set $\Omega_i$.\\[2.5ex]

\textbf{Feasibility analysis}
Consider a sampling instant $t_k$ for which a
solution $\overline{\vect{u}}_i^{\star}\big(\cdot;\ \vect{e}_i(t_k)\big)$ to
\eqref{position_based_cost} exists.
%In between $t_k$ and $t_k + h$,
%where $h$ is the sampling time, the optimal control signal
%$\vect{u}_i^{\star}\big(\cdot;\ \vect{e}_i(t_k)\big)$ is applied to the open-loop
%system.
Suppose now a time instant $t_{k+1}$ such that\footnote{It is not strictly necessary
that $t_{k+1} = t_k + h$ here, however it is necessary for the following that
$t_{k+1} - t_k \leq h$} $t_k < t_{k+1} < t_k + T_p$, and consider that the
optimal control signal calculated at $t_k$ is comprised by the following two
portions:

\begin{equation}
  \overline{\vect{u}}_i^{\star}\big(\cdot;\ \vect{e}_i(t_k)\big) = \left\{
      \begin{array}{ll}
        \overline{\vect{u}}_i^{\star}\big(\tau_1;\ \vect{e}_i(t_k)\big), & \tau_1 \in [t_k, t_{k+1}] \\[2.5ex]
        \overline{\vect{u}}_i^{\star}\big(\tau_2;\ \vect{e}_i(t_k)\big), & \tau_2 \in [t_{k+1}, t_k + T_p]
      \end{array}
      \right.
  \label{eq:optimal_input_portions_without_disturbances}
\end{equation}

Both portions are admissible since the calculated optimal control input is
admissible, and hence they both conform to the input constraints.
As for the resulting predicted states, they satisfy the state constraints, and,
crucially: $\overline{\vect{e}}_i\big(t_k + T_p;\ \overline{\vect{u}}_i^{\star}(\cdot), \vect{e}_i(t_k)\big) \in \Omega_i$.
Furthermore, according to assumption (3) of the theorem, there exists an
admissible (and certainly not guaranteed optimal) input $h_i(\vect{e}_i)$ that
renders $\Omega_i$ invariant over $[t_k + T_p, t_{k+1} + T_p]$.

Given the above facts, we can construct an admissible input
$\widetilde{\vect{u}}_i(\cdot)$  starting at time $t_{k+1}$ by sewing together
the second portion of \eqref{eq:optimal_input_portions_without_disturbances}
and the input $h_i(\vect{e}_i)$:

\begin{equation}
  \widetilde{\vect{u}}_i(\tau) = \left\{
      \begin{array}{ll}
        \overline{\vect{u}}_i^{\star}\big(\tau;\ \vect{e}_i(t_k)\big), & \tau \in [t_{k+1}, t_k + T_p] \\[2.5ex]
        h_i\big(\vect{e}_i(\tau)\big), & \tau \in (t_k + T_p, t_{k+1} + T_p]
      \end{array}
      \right.
\label{eq:optimal_input_t_plus_one_without_disturbances}
\end{equation}

The control input $\widetilde{\vect{u}}_i(\cdot)$ is admissible as a
composition of admissible control inputs. This means that feasibility of a
solution to the optimization problem at time $t_k$ implies feasibility at
time $t_{k+1} > t_k$, and, thus, since at time $t=0$ a solution is assumed to
be feasible, a solution to the optimal control problem is feasible for all
$t \geq 0$.\\[2.5ex]

\textbf{Convergence analysis}
The second part of the proof involves demonstrating the convergence of the
state $\vect{e}_i$ to the terminal set $\Omega_i$. In order for this
to be proved, it must be shown that a proper value function decreases along
closed-loop trajectories starting at some initial time $t_k$. We consider the
\textit{optimal} cost $J_i^{\star}\big(\vect{e}_i(t)\big)$ as a candidate
Lyapunov function:
$$J_i^{\star}\big(\vect{e}_i(t)\big) \triangleq J_i \Big(\vect{e}_i(t), \overline{\vect{u}}_i^{\star}\big(\cdot;\ \vect{e}_i(t)\big)\Big)$$
and, in particular, our goal is to show that that this cost decreases over
consecutive sampling instants $t_{k+1} = t_k + h$, i.e.
$J_i^{\star}\big(\vect{e}_i(t_{k+1})\big) - J_i^{\star}\big(\vect{e}_i(t_k)\big) \leq 0$.\\[2.5ex]

In order not to wreak notational havoc, let us define the following terms:
\begin{gg_box}
\begin{itemize}
  \item $\vect{u}_{0,i}(\tau) \triangleq \overline{\vect{u}}_i^{\star}\big(\tau;\ \vect{e}_i(t_k)\big)$
    as the \textit{optimal} input that results from the solution to problem
    \eqref{problem:opt_without_disturbances} based on the measurement of state
    $\vect{e}_i(t_k)$, applied at time $\tau \geq t_k$
  \item $\vect{e}_{0,i}(\tau) \triangleq \overline{\vect{e}}_i\big(\tau;\ \overline{\vect{u}}_i^{\star}\big(\cdot;\ \vect{e}_i(t_k)\big), \vect{e}_i(t_k)\big)$
    as the \textit{predicted} state at time $\tau \geq t_k$, that is,
    the state that results from the application of the above input
    $\overline{\vect{u}}_i^{\star}\big(\cdot;\ \vect{e}_i(t_k)\big)$ to the
    state $\vect{e}_i(t_k)$, at time $\tau$
  \item $\vect{u}_{1,i}(\tau) \triangleq \widetilde{\vect{u}}_i(\tau)$
    as the \textit{admissible} input at $\tau \geq t_{k+1}$
    (see eq. \eqref{eq:optimal_input_t_plus_one_without_disturbances})
  \item $\vect{e}_{1,i}(\tau) \triangleq \overline{\vect{e}}_i\big(\tau;\ \widetilde{\vect{u}}_i(\cdot), \vect{e}_i(t_{k+1})\big)$
    as the \textit{predicted} state at time $\tau \geq t_{k+1}$, that is,
    the state that results from the application of the above input
    $\widetilde{\vect{u}}_i(\cdot)$ to the state
    $\vect{e}_i\big(t_{k+1};\ \overline{\vect{u}}_i^{\star}\big(\cdot;\ \vect{e}_i(t_k)\big), \vect{e}_i(t_k)\big)$, at time $\tau$
\end{itemize}
\end{gg_box}

\begin{bw_box}
  \begin{remark}
  \label{remark:predicted_actual_equations_without_disturbance}
    Given that no model mismatch or disturbances exist, for
    the predicted and actual states at time
    $\tau_1 \geq \tau_0 \in \mathbb{R}_{\geq 0}$ it holds that:
    \begin{align}
      \vect{e}_i\big(\tau_1;\ \vect{u}_i(\cdot), \vect{e}_i(\tau_0)\big) &=
        \vect{e}_i(\tau_0) + \int_{\tau_0}^{\tau_1} g_i\big(\vect{e}_i(s;\ \vect{e}_i(\tau_0)), \vect{u}_i(s)\big) ds \\[2.5ex]
      \overline{\vect{e}}_i\big(\tau_1;\ \vect{u}_i(\cdot), \vect{e}_i(\tau_0)\big) &=
        \vect{e}_i(\tau_0) + \int_{\tau_0}^{\tau_1} g_i\big(\overline{\vect{e}}_i(s;\ \vect{e}_i(\tau_0)), \vect{u}_i(s)\big) ds
    \end{align}
  \end{remark}
\end{bw_box}

Before beginning to prove convergence, it is worth noting that while the cost
$$J_i \Big(\vect{e}_i(t), \overline{\vect{u}}_i^{\star}\big(\cdot;\ \vect{e}_i(t)\big)\Big)$$
is optimal (in the sense that it is based on the optimal input, which provides
its minimum realization), a cost that is based on a plainly admissible
(and thus, without loss of generality, sub-optimal) input
$\vect{u}_i \not= \overline{\vect{u}}_i^{\star}$ will result in a configuration where
\begin{equation}
J_i \Big(\vect{e}_i(t), \vect{u}_i\big(\cdot;\ \vect{e}_i(t)\big)\Big)
\geq J_i \Big(\vect{e}_i(t), \overline{\vect{u}}_i^{\star}\big(\cdot;\ \vect{e}_i(t)\big)\Big)
\end{equation}

Let us now begin our investigation on the difference between the cost
that results from the application of the feasible input $\vect{u}_{1,i}$,
which we shall denote by $\overline{J}_i\big(\vect{e}_i(t_{k+1})\big)$,
and the optimal cost $J_i^{\star}\big(\vect{e}_i(t_k)\big)$. We remind
ourselves that
$J_i \big(\vect{e}_i(t), \overline{\vect{u}}_i (\cdot)\big)$ $=$
$\int_{t}^{t + T_p} F_i \big(\overline{\vect{e}}_i(s), \overline{\vect{u}}_i (s)\big) ds$ $+$
$V_i \big(\overline{\vect{e}}_i (t + T_p)\big)$:
\begin{align}
  \overline{J}_i\big(\vect{e}_i(t_{k+1})\big) - J_i^{\star}\big(\vect{e}_i(t_k)\big) =\
   & V_i \big(\vect{e}_{1,i} (t_{k+1} + T_p)\big) + \int_{t_{k+1}}^{t_{k+1} + T_p} F_i \big(\vect{e}_{1,i}(s), \vect{u}_{1,i} (s)\big) ds \\[2.5ex]
  -&V_i \big(\vect{e}_{0,i} (t_k + T_p)\big) - \int_{t_k}^{t_k + T_p} F_i \big(\vect{e}_{0,i}(s), \vect{u}_{0,i} (s)\big) ds
\end{align}
Considering that $t_k < t_{k+1} < t_k + T_p < t_{k+1} + T_p$, we break down the
two integrals above in between these intervals:
\begin{align}
  \overline{J}_i\big(\vect{e}_i(t_{k+1})\big) - J_i^{\star}\big(\vect{e}_i(t_k)\big) &= \\[2.5ex]
    V_i \big(\vect{e}_{1,i} (t_{k+1} + T_p)\big)
    &+ \int_{t_{k+1}}^{t_k + T_p} F_i \big(\vect{e}_{1,i}(s), \vect{u}_{1,i} (s)\big) d s
    + \int_{t_k + T_p}^{t_{k+1} + T_p} F_i \big(\vect{e}_{1,i}(s), \vect{u}_{1,i} (s)\big) d s \\[2.5ex]
    -V_i \big(\vect{e}_{0,i} (t_k + T_p)\big)
    &- \int_{t_k}^{t_{k+1}} F_i \big(\vect{e}_{0,i}(s), \vect{u}_{0,i} (s)\big) d s
    - \int_{t_{k+1}}^{t_k + T_p} F_i \big(\vect{e}_{0,i}(s), \vect{u}_{0,i} (s)\big) d s
\label{eq:convergence_4_integrals}
\end{align}

\begin{gg_box}

Since no model mismatch or disturbances are present, consulting with remark
\eqref{remark:predicted_actual_equations_without_disturbance} and substituting
for $\tau_0 = t_k$ and $\tau_1 = t_{k+1}$ yields:
\begin{align}
  \vect{e}_i\big(t_{k+1};\ \overline{\vect{u}}_i^{\star}\big(\cdot;\ \vect{e}_i(t_k)\big), \vect{e}_i(t_k)\big) &=
    \vect{e}_i(t_k) + \int_{t_k}^{t_{k+1}} g_i\big(\vect{e}_i(s;\ \vect{e}_i(t_k)), \overline{\vect{u}}_i^{\star}(s)\big) ds \\[2.5ex]
  \overline{\vect{e}}_i\big(t_{k+1};\ \overline{\vect{u}}_i^{\star}\big(\cdot;\ \vect{e}_i(t_k)\big), \vect{e}_i(t_k)\big) &=
    \vect{e}_i(t_k) + \int_{t_k}^{t_{k+1}} g_i\big(\overline{\vect{e}}_i(s;\ \vect{e}_i(t_k)), \overline{\vect{u}}_i^{\star}(s)\big) ds
\end{align}
Subtracting the second expression from the first, we get
\begin{align}
  \vect{e}_i\big(t_{k+1};\ &\overline{\vect{u}}_i^{\star}\big(\cdot;\ \vect{e}_i(t_k)\big), \vect{e}_i(t_k)\big) -
  \overline{\vect{e}}_i\big(t_{k+1};\ \overline{\vect{u}}_i^{\star}\big(\cdot;\ \vect{e}_i(t_k)\big), \vect{e}_i(t_k)\big) \\[2.5ex]
  &=\int_{t_k}^{t_{k+1}} g_i\big(\vect{e}_i(s;\ \vect{e}_i(t_k)), \overline{\vect{u}}_i^{\star}(s)\big) ds -
    \int_{t_k}^{t_{k+1}} g_i\big(\overline{\vect{e}}_i(s;\ \vect{e}_i(t_k)), \overline{\vect{u}}_i^{\star}(s)\big) ds\\[2.5ex]
  &=\int_{t_k}^{t_{k+1}} \bigg( g_i\big(\vect{e}_i(s;\ \vect{e}_i(t_k)), \overline{\vect{u}}_i^{\star}(s)\big)
  - g_i\big(\overline{\vect{e}}_i(s;\ \vect{e}_i(t_k)), \overline{\vect{u}}_i^{\star}(s)\big) \bigg) ds \\[2.5ex]
\end{align}
Taking norms on either side yields
\begin{align}
  \bigg\|\vect{e}_i\big(t_{k+1};\ &\overline{\vect{u}}_i^{\star}\big(\cdot;\ \vect{e}_i(t_k)\big), \vect{e}_i(t_k)\big)
    -\overline{\vect{e}}_i\big(t_{k+1};\ \overline{\vect{u}}_i^{\star}\big(\cdot;\ \vect{e}_i(t_k)\big), \vect{e}_i(t_k)\big) \bigg\| \\[2.5ex]
  &=\bigg\|\int_{t_k}^{t_{k+1}} \bigg( g_i\big(\vect{e}_i(s;\ \vect{e}_i(t_k)), \overline{\vect{u}}_i^{\star}(s)\big)
  - g_i\big(\overline{\vect{e}}_i(s;\ \vect{e}_i(t_k)), \overline{\vect{u}}_i^{\star}(s)\big) \bigg) ds \bigg\|\\[2.5ex]
  &=\int_{t_k}^{t_{k+1}} \bigg\| g_i\big(\vect{e}_i(s;\ \vect{e}_i(t_k)), \overline{\vect{u}}_i^{\star}(s)\big)
  - g_i\big(\overline{\vect{e}}_i(s;\ \vect{e}_i(t_k)), \overline{\vect{u}}_i^{\star}(s)\big) \bigg\| ds\\[2.5ex]
  &\leq L_{g_i}\int_{t_k}^{t_{k+1}} \bigg\|
    \vect{e}_i\big(s;\ \overline{\vect{u}}_i^{\star}\big(\cdot;\ \vect{e}_i(t_k)\big), \vect{e}_i(t_k)\big)
    -\overline{\vect{e}}_i\big(s;\ \overline{\vect{u}}_i^{\star}\big(\cdot;\ \vect{e}_i(t_k)\big), \vect{e}_i(t_k)\big) \bigg\| ds
\end{align}
since $g_i$ is Lipschitz continuous in $\mathcal{E}_{i,s}$ with Lipschitz constant
$L_{g_i}$. Reformulation yields
\begin{align}
  \bigg\|\vect{e}_i\big(t_k&+h;\ \overline{\vect{u}}_i^{\star}\big(\cdot;\ \vect{e}_i(t_k)\big), \vect{e}_i(t_k)\big)
    -\overline{\vect{e}}_i\big(t_k+h;\ \overline{\vect{u}}_i^{\star}\big(\cdot;\ \vect{e}_i(t_k)\big), \vect{e}_i(t_k)\big) \bigg\| \\[2.5ex]
  &\leq L_{g_i}\int_{0}^{h} \bigg\|
    \vect{e}_i\big(t_k + s;\ \overline{\vect{u}}_i^{\star}\big(\cdot;\ \vect{e}_i(t_k)\big), \vect{e}_i(t_k)\big)
    -\overline{\vect{e}}_i\big(t_k + s;\ \overline{\vect{u}}_i^{\star}\big(\cdot;\ \vect{e}_i(t_k)\big), \vect{e}_i(t_k)\big) \bigg\| ds
\end{align}
By applying the Gr\"{o}nwall-Bellman inequality we obtain zero as an
upper bound for the norm of the difference between the two states. Since
any norm cannot be negative, we conclude that
\begin{align}
  \bigg\|\vect{e}_i\big(t_{k+1};\ \overline{\vect{u}}_i^{\star}\big(\cdot;\ \vect{e}_i(t_k)\big), \vect{e}_i(t_k)\big)
    -\overline{\vect{e}}_i\big(t_{k+1};\ \overline{\vect{u}}_i^{\star}\big(\cdot;\ \vect{e}_i(t_k)\big), \vect{e}_i(t_k)\big) \bigg\| = 0
\end{align}
which means that
\begin{align}
  \vect{e}_i\big(t_{k+1};\ \overline{\vect{u}}_i^{\star}\big(\cdot;\ \vect{e}_i(t_k)\big), \vect{e}_i(t_k)\big)
    =\overline{\vect{e}}_i\big(t_{k+1};\ \overline{\vect{u}}_i^{\star}\big(\cdot;\ \vect{e}_i(t_k)\big), \vect{e}_i(t_k)\big)
\end{align}

In between times $t_{k+1}$ and $t_k + T_p$, the constructed admissible input
$\widetilde{\vect{u}}_i(\cdot)$ is equal to the optimal
input $\overline{\vect{u}}_i^{\star}\big(\cdot;\ \vect{e}_i(t_k)\big)$
(see eq. \ref{eq:optimal_input_t_plus_one_without_disturbances}), which means
that $\vect{u}_{1,i}(\tau) = \vect{u}_{0,i}(\tau)$ in the interval
$\tau \in [t_{k+1}, t_k + T_p]$. Since the initial conditions at $t=t_{k+1}$ are
equal and the control laws are also equal, so will the predicted states over the
same interval:
\begin{align}
  \overline{\vect{e}}_i\big(\tau;\ \widetilde{\vect{u}}_i(\cdot), \vect{e}_i(t_{k+1})\big) =
  \overline{\vect{e}}_i\big(\tau;\ \overline{\vect{u}}_i^{\star}(\cdot), \overline{\vect{e}}_i(t_{k+1})\big), \ \tau \in [t_{k+1}, t_k + T_p]
  \label{eq:equal_predicted_without_disturbance}
\end{align}
Using our notation then, in the same interval:
$\vect{e}_{1,i}(\cdot) = \vect{e}_{0,i}(\cdot)$. Coupled with the fact that
in the same interval $\vect{u}_{1,i}(\tau) = \vect{u}_{0,i}(\tau)$, the following
equality holds over $[t_{k+1}, t_k + T_p]$:
\begin{align}
  F_i \big(\vect{e}_{1,i}(s), \vect{u}_{1,i} (s)\big) =
  F_i \big(\vect{e}_{0,i}(s), \vect{u}_{0,i} (s)\big),\ s \in [t_{k+1}, t_k + T_p]
\end{align}
Integrating this equality over the interval where it is valid yields
\begin{align}
  \int_{t_{k+1}}^{t_k + T_p}F_i \big(\vect{e}_{1,i}(s), \vect{u}_{1,i} (s)\big) ds =
  \int_{t_{k+1}}^{t_k + T_p}F_i \big(\vect{e}_{0,i}(s), \vect{u}_{0,i} (s)\big) ds
\end{align}
\end{gg_box}
This means that these two integrals with ends over the interval
$[t_{k+1}, t_k + T_p]$ featured in the right-hand side of eq.
\eqref{eq:convergence_4_integrals} vanish, and thus the cost difference becomes
\begin{align}
  \overline{J}_i\big(\vect{e}_i(t_{k+1})\big) - J_i^{\star}\big(\vect{e}_i(t_k)\big) &=
    V_i \big(\vect{e}_{1,i} (t_{k+1} + T_p)\big)
    +\int_{t_k + T_p}^{t_{k+1} + T_p} F_i \big(\vect{e}_{1,i}(s), \vect{u}_{1,i} (s)\big) d s \\[2.5ex]
    &-V_i \big(\vect{e}_{0,i} (t_k + T_p)\big)
    -\int_{t_k}^{t_{k+1}} F_i \big(\vect{e}_{0,i}(s), \vect{u}_{0,i} (s)\big) d s
\label{eq:convergence_2_integrals}
\end{align}

During the course of arriving at the above result, we have concluded that,
in the absence of disturbances, the remark \eqref{eq:error_now_to_predicted_error}
holds.
\begin{bw_box}
  \begin{remark}
  \label{eq:error_now_to_predicted_error}
    In the absence of disturbances, the following equality holds over
    the entire horizon $t \in [t_k, t_k + T_p]$:
    \begin{align}
      \vect{e}_i\big(t;\ \overline{\vect{u}}_i^{\star}\big(\cdot;\ \vect{e}_i(t_k)\big), \vect{e}_i(t_k)\big)
        =\overline{\vect{e}}_i\big(t;\ \overline{\vect{u}}_i^{\star}\big(\cdot;\ \vect{e}_i(t_k)\big), \vect{e}_i(t_k)\big)
    \end{align}
  \end{remark}
\end{bw_box}

\begin{gg_box}
  We turn our attention to the first integral in the above expression, and we
  note that $[t_{k+1} + T_p, t_k + T_p]$, is exactly the
  the interval where assumption (3b) of the theorem holds. Hence,
  we decide to integrate the expression found in the assumption over the
  interval $[t_k + T_p, t_{k+1} + T_p]$, for the controls and states applicable
  in it:
  \begin{align}
    \int_{t_k + T_p}^{t_{k+1} + T_p} \Bigg(\dfrac{\partial V_i}{\partial \vect{e}_{1,i}} g_i\big(\vect{e}_{1,i}(s), \vect{u}_{1,i}(s)\big)
    + F_i\big(\vect{e}_{1,i}(s), \vect{u}_{1,i}(s)\big)\Bigg) ds &\leq 0 \\[2.5ex]
    \int_{t_k + T_p}^{t_{k+1} + T_p} \dfrac{d}{ds} V_i\big(\vect{e}_{1,i}(s)\big) d s
    + \int_{t_k + T_p}^{t_{k+1} + T_p} F_i\big(\vect{e}_{1,i}(s), \vect{u}_{1,i}(s)\big) ds &\leq 0 \\[2.5ex]
    V_i\big(\vect{e}_{1,i}(t_{k+1} + T_p)\big) - V_i\big(\vect{e}_{1,i}(t_k + T_p)\big)
    + \int_{t_k + T_p}^{t_{k+1} + T_p} F_i\big(\vect{e}_{1,i}(s), \vect{u}_{1,i}(s)\big) ds &\leq 0 \\[2.5ex]
    V_i\big(\vect{e}_{1,i}(t_{k+1} + T_p)\big)
    + \int_{t_k + T_p}^{t_{k+1} + T_p} F_i\big(\vect{e}_{1,i}(s), \vect{u}_{1,i}(s)\big) ds &\leq V_i\big(\vect{e}_{1,i}(t_k + T_p)\big)
  \end{align}

  The left-hand side expression is the same as the first two terms in the
  right-hand side of equality \eqref{eq:convergence_2_integrals}. We can
  introduce the third one by subtracting it from both sides:
  \begin{align}
    V_i\big(\vect{e}_{1,i}(t_{k+1} + T_p)\big)
    &+ \int_{t_k + T_p}^{t_{k+1} + T_p} F_i\big(\vect{e}_{1,i}(s), \vect{u}_{1,i}(s)\big) ds
    - V_i\big(\vect{e}_{0,i}(t_k + T_p)\big) \\[2.5ex]
    &\leq V_i\big(\vect{e}_{1,i}(t_k + T_p)\big)
    - V_i\big(\vect{e}_{0,i}(t_k + T_p)\big) \\[2.5ex]
    &\leq \Big|V_i\big(\vect{e}_{1,i}(t_k + T_p)\big)
    - V_i\big(\vect{e}_{0,i}(t_k + T_p)\big)\Big|
  \end{align}
  since $x \leq |x|, \forall x \in \mathbb{R}$.

  By revisiting lemma \eqref{lemma:V_Lipschitz_e_0}, the above inequality
  becomes
  \begin{align}
    V_i\big(\vect{e}_{1,i}(t_{k+1} + T_p)\big)
    + \int_{t_k + T_p}^{t_{k+1} + T_p} F_i\big(\vect{e}_{1,i}(s), \vect{u}_{1,i}(s)\big) ds
    - V_i\big(\vect{e}_{0,i}(t_k + T_p)\big) \\[2.5ex]
    \leq L_{V_i} \big\|\vect{e}_{1,i}(t_k + T_p) - \vect{e}_{0,i}(t_k + T_p)\big\|
  \end{align}

  However, as we witnessed in \eqref{eq:equal_predicted_without_disturbance},
  in the interval $[t_{k+1}, t_k + T_p]$:
  $\vect{e}_{1,i}(\cdot) = \vect{e}_{0,i}(\cdot)$, hence the right-hand
  side of the inequality equals zero:
  \begin{align}
    V_i\big(\vect{e}_{1,i}(t_{k+1} + T_p)\big)
    + \int_{t_k + T_p}^{t_{k+1} + T_p} F_i\big(\vect{e}_{1,i}(s), \vect{u}_{1,i}(s)\big) ds
    - V_i\big(\vect{e}_{0,i}(t_k + T_p)\big) \leq 0
  \end{align}

  By subtracting the fourth term needed to complete the right-hand side
  expression of \eqref{eq:convergence_2_integrals}, i.e.
  $\int_{t_k}^{t_{k+1}} F_i \big(\vect{e}_{0,i}(s), \vect{u}_{0,i} (s)\big) d s$
  from both sides we get
  \begin{align}
    V_i\big(\vect{e}_{1,i}(t_{k+1} + T_p)\big)
    + \int_{t_k + T_p}^{t_{k+1} + T_p} F_i\big(\vect{e}_{1,i}(s), \vect{u}_{1,i}(s)\big) ds& \\[2.5ex]
    - V_i\big(\vect{e}_{0,i}(t_k + T_p)\big)
    -\int_{t_k}^{t_{k+1}} F_i \big(\vect{e}_{0,i}(s), \vect{u}_{0,i} (s)\big) d s
    &\leq -\int_{t_k}^{t_{k+1}} F_i \big(\vect{e}_{0,i}(s), \vect{u}_{0,i} (s)\big) d s
  \end{align}

  The left-hand side of this inequality is now equal to the cost difference
  $\overline{J}_i\big(\vect{e}_i(t_{k+1})\big) - J_i^{\star}\big(\vect{e}_i(t_k)\big)$.
\end{gg_box}
Hence, the cost difference becomes bounded by
\begin{align}
  \overline{J}_i\big(\vect{e}_i(t_{k+1})\big) - J_i^{\star}\big(\vect{e}_i(t_k)\big) \leq
    -\int_{t_k}^{t_{k+1}} F_i \big(\vect{e}_{0,i}(s), \vect{u}_{0,i} (s)\big) ds
\end{align}
\begin{gg_box}
  $F_i$ is a positive-definite function as a sum of a positive-definite
  $\|\vect{u}_i\|^2_{\mat{R}_i}$ and a positive semi-definite function
  $\|\vect{e}_i\|^2_{\mat{Q}_i}$. If we denote by
  $m_i = \lambda_{min}(\mat{Q}_i, \mat{R}_i) \geq 0$ the minimum eigenvalue
  between those of matrices $\mat{R}_i, \mat{Q}_i$, this means that
  \begin{align}
    F_i \big(\vect{e}_{0,i}(s), \vect{u}_{0,i} (s)\big) \geq m_i \|\vect{e}_{0,i}(s)\|^2
  \end{align}

  By integrating the above between our interval of interest $[t_k, t_{k+1}]$ we get
  \begin{align}
    \int_{t_k}^{t_{k+1}} F_i \big(\vect{e}_{0,i}(s), \vect{u}_{0,i} (s)\big) &\geq \int_{t_k}^{t_{k+1}} m_i \|\vect{e}_{0,i}(s)\|^2 ds \\[2.5ex]
    \text{or}\\[2.5ex]
    -\int_{t_k}^{t_{k+1}} F_i \big(\vect{e}_{0,i}(s), \vect{u}_{0,i} (s)\big) &\leq -m_i \int_{t_k}^{t_{k+1}} \|\vect{e}_{0,i}(s)\|^2 ds
  \end{align}
\end{gg_box}

This means that the cost difference is upper-bounded by a class $\mathcal{K}$
function
\begin{align}
  \overline{J}_i\big(\vect{e}_i(t_{k+1})\big) - J_i^{\star}\big(\vect{e}_i(t_k)\big)
  &\leq -m_i \int_{t_k}^{t_{k+1}} \|\vect{e}_{0,i}(s)\|^2 ds \leq 0
\end{align}
and since the cost $\overline{J}_i\big(\vect{e}_i(t_{k+1})\big)$ is, in general,
sub-optimal: $J_i^{\star}\big(\vect{e}_i(t_{k+1})\big) - \overline{J}_i\big(\vect{e}_i(t_{k+1})\big) \leq 0$:
\begin{align}
 J_i^{\star}\big(\vect{e}_i(t_{k+1})\big) - J_i^{\star}\big(\vect{e}_i(t_k)\big) \leq -m_i \int_{t_k}^{t_{k+1}} \|\vect{e}_{0,i}(s)\|^2 ds
 \label{eq:J_opt_between_consecutive_k}
\end{align}
With this milestone result established, we need to trace the time $t_k$ back
to $t_0 = 0$ in order to prove that the closed-loop system is stable at all
times $t \in \mathbb{R}_{\geq 0}$.

\begin{gg_box}
  The integral of $\|\vect{e}_{0,i}(\tau)\|^2$ over the interval $[t_0, t_{k+1}]$,
  $t_0 < t_k < t_{k+1}$ can be decomposed into the addition of two integrals
  with limits ranging from (a) $t_0$ to $t_k$ and (b) $t_k$ to $t_{k+1}$:
  \begin{align}
    \int_{t_0}^{t_{k+1}} \|\vect{e}_{0,i}(s)\|^2 ds = \int_{t_0}^{t_{k}} \|\vect{e}_{0,i}(s)\|^2 ds + \int_{t_k}^{t_{k+1}} \|\vect{e}_{0,i}(s)\|^2 ds
  \end{align}
  By rearranging terms, this means that
  \begin{align}
    \int_{t_k}^{t_{k+1}} \|\vect{e}_{0,i}(s)\|^2 ds = \int_{t_0}^{t_{k+1}} \|\vect{e}_{0,i}(s)\|^2 ds - \int_{t_0}^{t_{k}} \|\vect{e}_{0,i}(s)\|^2 ds
  \end{align}
  making the optimal cost difference between the consecutive sampling times
  $t_k$ and $t_{k+1}$ in \eqref{eq:J_opt_between_consecutive_k}
  \begin{align}
    J_i^{\star}\big(\vect{e}_i(t_{k+1})\big) - J_i^{\star}\big(\vect{e}_i(t_k)\big) \leq
      -m_i \int_{t_0}^{t_{k+1}} \|\vect{e}_{0,i}(s)\|^2 ds +m_i \int_{t_0}^{t_{k}} \|\vect{e}_{0,i}(s)\|^2 ds
  \end{align}
  Similarly, the optimal cost difference between the sampling times $t_{k-1}$
  and $t_{k}$ is
  \begin{align}
    J_i^{\star}\big(\vect{e}_i(t_{k})\big) - J_i^{\star}\big(\vect{e}_i(t_{k-1})\big) \leq
      -m_i \int_{t_0}^{t_{k}} \|\vect{e}_{0,i}(s)\|^2 ds +m_i \int_{t_0}^{t_{k-1}} \|\vect{e}_{0,i}(s)\|^2 ds
  \end{align}
  and we can apply this rationale all the way back to the cost difference
  between $t_0$ and $t_1$. Summing all the inequalities between the pairs of
  consecutive sampling times $(t_0, t_1)$, $(t_1, t_2)$, $\dots$,
  $(t_{k-1}, t_k)$, we get
  \begin{align}
    J_i^{\star}\big(\vect{e}_i(t_{k})\big) - J_i^{\star}\big(\vect{e}_i(t_0)\big) \leq
      -m_i \int_{t_0}^{t_{k}} \|\vect{e}_{0,i}(s)\|^2 ds
  \end{align}
\end{gg_box}

Hence, for $t_0 = 0$
\begin{align}
  J_i^{\star}\big(\vect{e}_i(t_{k})\big) - J_i^{\star}\big(\vect{e}_i(0)\big) \leq
    -m_i \int_{0}^{t_{k}} \|\vect{e}_{0,i}(s)\|^2 ds \leq 0
\label{eq:J_opt_between_k_and_0}
\end{align}
which implies that the value function $J_i^{\star}\big(\vect{e}_i(t_{k})\big)$
in non-increasing for all sampling times:
\begin{align}
  J_i^{\star}\big(\vect{e}_i(t_{k})\big) \leq J_i^{\star}\big(\vect{e}_i(0)\big),\ \forall t_k \in \mathbb{R}_{\geq0}
\end{align}
Let us now define the function $\Xi_i\big(\vect{e}_i(t)\big)$:
\begin{align}
  \Xi_i\big(\vect{e}(t)\big) \triangleq J_i^{\star}\big(\vect{e}_i(\tau)\big),\ t \in \mathbb{R}_{\geq0}
\end{align}
where $\tau = max\{t_k : t_k \leq t\}$ $-$ i.e. the immediately previous to
$t$ sampling time. Then the above inequality reforms into
\begin{align}
  \Xi_i\big(\vect{e}(t)\big) \leq \Xi_i\big(\vect{e}_i(0)\big),\ t \in \mathbb{R}_{\geq0}
\end{align}

Since $\Xi_i\big(\vect{e}_i(0)\big)$
is bounded (as composition of bounded parts), this implies
that $\Xi_i\big(\vect{e}(t)\big)$ is also bounded. The signals
$\vect{e}_i(t) \in \mathcal{E}_{i,s}$ and $\vect{u}_i(t) \in \mathcal{U}_i$
are also bounded. According to \eqref{eq:position_based_error_model}, this
means that $\dot{\vect{e}}_i(t)$ is bounded as well. From inequality
\eqref{eq:J_opt_between_k_and_0} we then have
\begin{align}
  \Xi_i\big(\vect{e}_i(t)\big) \leq \Xi_i\big(\vect{e}_i(0)\big)
    -m_i \int_{0}^{\tau} \|\vect{e}_{0,i}(s)\|^2 ds \leq 0
\end{align}
which, due to the fact that $\tau \leq t$, is equivalent to
\begin{align}
  \Xi_i\big(\vect{e}_i(t)\big) \leq \Xi_i\big(\vect{e}_i(0)\big) -m_i \int_{0}^{t} \|\vect{e}_{0,i}(s)\|^2 ds \leq 0,\ t \in \mathbb{R}_{\geq 0}
\end{align}
Solving for the integral we get
\begin{align}
  \int_{0}^{t} \|\vect{e}_{0,i}(s)\|^2 ds \leq
    \dfrac{1}{m_i}\Big(\Xi_i\big(\vect{e}_i(0)\big) - \Xi_i\big(\vect{e}_i(t)\big)\Big),\ t \in \mathbb{R}_{\geq 0}
\end{align}
Both $\Xi_i\big(\vect{e}_i(0)\big)$ and $\Xi_i\big(\vect{e}_i(t)\big)$
are bounded, and therefore so is their difference, which means that the
integral $\int\limits_{0}^{t} \|\vect{e}_{0,i}(s)\|^2 ds$ and its limit when
$t \to \infty$ are bounded as well. Lemma \eqref{lemma:barbalat} assures us
that under these conditions for the error and its dynamics, which are
fulfilled in our case, the error
\begin{align}
  \lim\limits_{t \to \infty}\|\vect{e}_{0,i}(t)\| &= 0 \Leftrightarrow \\[2.5ex]
  \lim\limits_{t \to \infty}
  \Big\| \overline{\vect{e}}_i\Big(t;\ \overline{\vect{u}}_i^{\star}\big(\cdot;\ \vect{e}_i(t_k)\big), \vect{e}_i(t_k) \Big) \Big\| &= 0,\
\forall t_k \in \mathbb{R}_{\geq 0}
\end{align}
which, given remark \eqref{eq:error_now_to_predicted_error},
and dropping the initial condition, means that
$$\lim\limits_{t \to \infty}\|\vect{e}_i(t)\| = 0$$
which implies that
$$\lim\limits_{t \to \infty}\vect{e}_i(t) \in \Omega_i$$
Therefore, the closed-loop trajectory of the error state $\vect{e}_i$ converges
to the terminal set $\Omega_i$ as $t \to \infty$.

In turn, this means that the system \eqref{eq:original_z_system} converges
to $\vect{z}_{i,des}$ while simultaneously conforming to
all constraints $\mathcal{Z}_{i,t}$, as $t \to \infty$. This conclusion holds
for all $i \in \mathcal{V}$, and hence, the compound system of agents
$\mathcal{V}$ is stable in $\mathcal{Z}_{i,t}$.
\qedsymbol
