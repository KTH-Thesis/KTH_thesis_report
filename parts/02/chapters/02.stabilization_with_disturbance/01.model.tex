The purpose to be saught in this chapter is the steering of each agent
$i \in \mathcal{V}$ into a desired configuration in 12D space while conforming to
the requirements posed by the problem. Here, the real system and its model are
\textit{not} equivalent: we consider that additive disturbances act on the real
system. At first, the model of the perturbed system \eqref{eq:system} will
be formalized. Next, the error model and \textit{its} constraints will be
expressed. We will then pose the optimization problem to be solved periodically:
it will equip us with the optimum feasible input that steers the system towards
achieving its intended configuration regardless of the introduced uncertainty,
provided that it is bounded by a certain value. This will lead to the proof
of this statement, i.e. that the compound closed-loop system of agents
$i \in \mathcal{V}$ is stable under the proposed control regime, provided
that the disturbance is bounded.

%-------------------------------------------------------------------------------
\section{The perturbed model}

In the following, we assume that the real system is
subject to bounded additive disturbances $\vect{\delta}_i$ such that
$\vect{\delta}_i \in \Delta_i \subset \mathbb{R}^9 \times \mathbb{T}^3$, where
$\Delta_i$ is a compact set containing the origin.
The real system is described by:
\begin{align}
  \dot{\vect{z}}_i(t) &= f_i^R \big(\vect{z}_i (t), \vect{u}_i (t)\big) \label{eq:perturbed_system} \\[2.5ex]
                      &= f_i \big(\vect{z}_i (t), \vect{u}_i (t)\big) + \vect{\delta}_i(t) \\[2.5ex]
  \vect{z}_i(0) &= \vect{z}_{i,0} \\[2.5ex]
  \vect{z}_i (t) & \subset \mathbb{R}^{9} \times \mathbb{T}^3 \\[2.5ex]
  \vect{u}_i (t) & \subset \mathbb{R}^6 \\[2.5ex]
  \vect{\delta}_i(t) &\in \Delta_i \subset \mathbb{R}^9 \times \mathbb{T}^3,\ t \in \mathbb{R}_{\geq 0}\\[2.5ex]
  \text{sup}\limits_{t \in \mathbb{R}_{\geq 0}} \|\vect{\delta}_i(t)\| &\leq \overline{\delta}_i
\end{align}
where state $\vect{z}_i$ is directly measurable as per assumption
\eqref{ass:measurements_access}.

The constraint set $\mathcal{Z}_{i,t} \subset \mathbb{R}^{9} \times \mathbb{T}^3$
is unchanged: it is the set that captures all the state constraints of
the system's dynamics posed by the problem \eqref{problem},
at $t \in \mathbb{R}_{\geq 0}$. We include it again here for reference
purposes.
\begin{align}
  \mathcal{Z}_{i,t} = \big\{\vect{z}_i(t) \in \mathbb{R}^{9}\times \mathbb{T}^3 : \
      & \|\vect{p}_i(t) - \vect{p}_j(t)\| > \underline{d}_{ij,a}, \forall j \in \mathcal{R}_i(t), \label{constraint:p_1_2}\\[2.5ex]
      & \|\vect{p}_i(t) - \vect{p}_j(t)\| < d_i, \forall j \in \mathcal{N}_i, \\[2.5ex]
      & \|\vect{p}_i(t) - \vect{p}_{\ell}\| > \underline{d}_{i\ell,o}, \forall \ell \in \mathcal{L}, \\[2.5ex]
      & \|\vect{p}_W - \vect{p}_i(t)\| < \overline{d}_{i,W}, \\[2.5ex]
      & - \frac{\pi}{2} < \theta_i(t) < \frac{\pi}{2} \label{constraint:p_5_2}\big\}
\end{align}
