\subsubsection{Model}

We begin by rewriting the system equations \eqref{eq:system_1},
\eqref{eq:system_2} in state-space form:

\begin{align}
  \dot{\vect{z}}_i(t) =
    \begin{bmatrix}
      f_{i,1}(\vect{z}_i, \vect{u}_i) \\
      f_{i,2}(\vect{z}_i, \vect{u}_i) \\
    \end{bmatrix}, \text{ where }
  \vect{z}_i(t) =
    \begin{bmatrix}
      \vect{x}_i(t) \\
      \vect{v}_i(t) \\
    \end{bmatrix}, \text{ and }
\end{align}

\begin{align}
  f_{i,1}(\vect{z}_i, \vect{u}_i) &=
  \dot{\vect{x}}_i(t) = \mat{J}_i^{-1}(\vect{x}_i) \vect{v}_i(t) \\
  f_{i,2}(\vect{z}_i, \vect{u}_i) &=
  \dot{\vect{v}}_i(t) = -\mat{M}_i^{-1}(\vect{x}_i)\mat{C}_i(\vect{x}_i,\dot{\vect{x}}_i) \vect{v}_i(t)
    - \mat{M}_i^{-1}(\vect{x}_i)\vect{g}_i(\vect{x}_i)
    + \mat{M}_i^{-1}(\vect{x}_i)\vect{u}_i
\label{eq:system}
\end{align}

The state evolution of agent $i$ is modeled by a system of non-linear
continuous-time differential equations of the form

\begin{align}
  \dot{\vect{z}}_i(t) &= f_i \big(\vect{z}_i (t), \vect{u}_i (t)\big) \label{eq:non_perturbed_system}\\
  \vect{z}_i(0) &= \vect{z}_0 \\
  \vect{z}_i (t) &\in \mathcal{Z}_i \subset \mathbb{R}^n \\
  \vect{u}_i (t) &\in \mathcal{U}_i \subset \mathbb{R}^m
\end{align}
where state $\vect{z}_i$ is directly measurable. Equation
\ref{eq:non_perturbed_system} does not consider model-plant mismatches or
external disturbances. The applied input $\vect{u}_i$ is a portion of the
optimal solution to an optimization problem where the \textit{predicted} states
of the neighbouring agents of agent $i$ are taken into account in the cost
function, as opposed to their \textit{true} states.

\begin{gg_box}
\begin{assumption} (Functions $f_i$ are Lipschitz continuous)

  The functions $f_i(\vect{x}, \vect{u}), \forall i \in \mathcal{V}, x \in \mathcal{X}_i$
  are Lipschitz continuous with Lipschitz constants $L_{f_i}$:

  \begin{align}
    \| f_i(\vect{x}_1, \vect{u}) - f_i(\vect{x}_2, \vect{u})\| \leq L_{f_i} \|\vect{x}_1 - \vect{x}_2 \|
  \end{align}

  \label{ass:f_i_Lipschitz}
\end{assumption}
\end{gg_box}


\subsubsection{Cost function}

\begin{align}
  J_i \big(\overline{\vect{u}}_i (\cdot);\ \vect{x}_i(t_k), \vect{x}_{\mathcal{N}_i}(t_k)\big) &=
    J_i^U \big(\overline{\vect{u}}_i (\cdot) \big) +
    J_i^X \big(\vect{x}_i (t_k), \vect{x}_{\mathcal{N}_i}(t_k)\big)
\end{align}

where

\begin{align}
  J_i^U \big(\overline{\vect{u}}_i (\cdot)\big) &=
    \int_{t_k}^{t_k + T_p} h_i \big(\overline{\vect{u}}_i (\tau)\big) d \tau \\
  J_i^X \big(\vect{x}_i (t_k), \vect{x}_{\mathcal{N}_i}(t_k)\big) &=
    \sum_{j \in \mathcal{N}_i} \Bigg(\int_{t_k}^{t_k + T_p} g_{ij} (\overline{\vect{x}}_i (\tau), \overline{\vect{x}}_j (\tau)\big) d \tau +
    V_{ij} \big(\overline{\vect{x}}_i (t_k + T_p), \overline{\vect{x}}_j (t_k + T_p)\big)\Bigg)
\end{align}

and

\begin{align}
  h_i(\vect{u}_i) &=\| \vect{u}_i\|_{R_i}^2 \\
  g_{ij}(\vect{x}_i, \vect{x}_j) &= \| (\vect{x}_i - \vect{x}_j) - \vect{x}_{ji,des} \|_{G_{ij}}^2
\end{align}


\begin{gg_box}
\begin{assumption} (Functions $h_i$ are Lipschitz continuous)

  The functions $h_i(\vect{u}), \forall i \in \mathcal{V}, \vect{u} \in \mathcal{U}_i$ are
  Lipschitz continuous with Lipschitz constants $L_{h_i}$:

  \begin{align}
    \| h_i(\vect{u}_1) - h_i(\vect{u}_2)\| \leq L_{h_i} \|\vect{u}_1 - \vect{u}_2 \|
  \end{align}

  \label{ass:h_i_Lipschitz}
\end{assumption}
\end{gg_box}

\begin{gg_box}
\begin{assumption} (Functions $g_{ij}$ are Lipschitz continuous)

  The functions $g_{ij}(\vect{x}_i, \vect{x}_j), \forall i \in \mathcal{V}, j \in \mathcal{N}_i$ are
  Lipschitz continuous with Lipschitz constants $L_{g_i}$:

  \begin{align}
    \| g_{ij}(\vect{x}_{i,1}, \vect{x}_j) - g_{ij}(\vect{x}_{i,2}, \vect{x}_j)\| \leq L_{g_i} \|\vect{x}_{i,1} - \vect{x}_{i,2} \|
  \end{align}

  \label{ass:g_ij_Lipschitz}
\end{assumption}
\end{gg_box}

\subsubsection{Optimization problem}

\begin{align}
  \text{min}_{\overline{\vect{u}}_i (\cdot)}\ &
    J_i \big(\overline{\vect{u}}_i (\cdot);\ \vect{x}_i(t_k), \vect{x}_{\mathcal{N}_i}(t_k), \vect{x}_{\mathcal{Z}}\big) \\
  \text{subject to:} & \\
  & \dot{\overline{\vect{x}}}_i(t) = f_i (\overline{\vect{x}}_i (t), \overline{\vect{u}}_i (t)) \\
  & \overline{\vect{x}}_i (t_k) = \vect{x}_i (t_k) \\
  & \overline{\vect{u}}_i \in \mathcal{U}_i \\
  & \overline{\vect{x}}_i (t) \in \mathcal{X}_i^{t - t_k}, t \in [t_k, t_k + T_p] \\
  & \overline{\vect{x}}_i (t_k + T_p) \in \mathcal{X}_{f_i} \\
  & J_i \big(\overline{\vect{u}}_i (\cdot);\ \vect{x}_i(t_k)\big) \leq J_i^{\text{ sup}}(t)
\end{align}
where the constraint on $\overline{\vect{x}}_i (t)$ is narrowed to
$\overline{\vect{x}}_i (t) \in \mathcal{X}_i^{t - t_k} \subseteq \mathcal{X}_i$
so as to ensure that there is a robust positively invariant set for the
closed-loop system where a solution to the FHOCP exists \ref{}.


Specifically,
$\mathcal{X}_i^{t - t_k} = \mathcal{X}_i \sim \mathcal{B}_i^{t-t_k}$, where

\begin{align}
  \mathcal{B}_i^{t-t_k} =
  \Big\{\vect{x}_i \in \mathbb{R}^n : \|\vect{x}_i\| \leq \gamma_i (t-t_k) = \frac{\overline{\vect{w}}_i}{L_{f_i}} (e^{L_{f_i} (t-t_k)} - 1) \Big\}
\end{align}
and the operator $\sim$ denotes the Pontryagin difference.

The constraint $J_i \big(\overline{\vect{u}}_i (\cdot);\ \vect{x}_i(t_k)\big) \leq J_i^{\text{ sup}}(t_k)$
is imposed for ensuring the stability for each agent; it will be defined later
on in the text.

\begin{gg_box}
\begin{assumption}

  There exists a local stabilizing controller
  $\vect{\kappa}_i(\vect{x}) \in \mathcal{U}_i$ such that

  \begin{align}
    \dfrac{\partial V_i}{\partial \vect{x}} f_i \Big(\vect{x}(\tau), \vect{\kappa}_i\big(\vect{x}(\tau)\big)\Big) +
      h_i\Big(\vect{x}(\tau), \vect{\kappa}_i \big(\vect{x}(\tau)\big)\Big) \leq 0, \forall \vect{x} \in \Phi_i
  \end{align}

  where $\Phi_i = \{\vect{x} \in \mathbb{R}^n : V_i(\vect{x}) \leq \alpha_i \}$, and
  $\Phi_i \subseteq \mathcal{X}_i^{T_p} = \{x \in \mathbb{R}^n : \vect{\kappa}_i(\vect{x}) \in \mathcal{U}_i\}$

  \label{ass:local_controller_k}
\end{assumption}
\end{gg_box}


\begin{gg_box}
\begin{assumption}

  The terminal costs $V_i(\vect{x})$ are Lipschitz continuous in $\vect{x} \in \Phi_i$
  with Lipschitz constants $L_{V_i}$:

  \begin{align}
    \| V_i(\vect{x}_1) - V_i(\vect{x}_2)\| \leq L_{V_i} \|\vect{x}_1 - \vect{x}_2 \|
  \end{align}

  \label{ass:V_i_Lipschitz}
\end{assumption}
\end{gg_box}


\begin{gg_box}
\begin{assumption}

  The terminal set $\mathcal{X}_{f_i} = \{ \vect{x} \in \mathbb{R}^n : V_i(\vect{x}) \in \alpha_{V_i}\}$
  is such that for all $\vect{x} \in \Phi_i$, $f_i\big(\vect{x}, \vect{\kappa}_i(\vect{x})\big) \in \mathcal{X}_{f_i} \subseteq \Phi_i$.

  \label{ass:x_f_i}
\end{assumption}
\end{gg_box}
