\subsubsection{Model}

The state evolution of agent $i$ is modeled by a non-linear continuous-time
differential equation of the form

\begin{align}
  \dot{\vect{x}}_i(t) &= f_i \big(\vect{x}_i (t), \vect{u}_i (t)\big) \label{eq:non_perturbed_system}\\
  \vect{x}_i(0) &= \vect{x}_0 \\
  \vect{x}_i (t) &\in \mathcal{X}_i \subset \mathbb{R}^n \\
  \vect{u}_i (t) &\in \mathcal{U}_i \subset \mathbb{R}^m
\end{align}
where state $\vect{x}_i$ is directly measurable. Equation
\ref{eq:non_perturbed_system} does not consider model-plant mismatches or
disturbances. However, a more accurate approximation of the agent's dynamics
is given by

\begin{align}
  \dot{\vect{x}}_i(t) &= f_i \big(\vect{x}_i (t), \vect{u}_i (t) \big) + \vect{w}_i (t) \label{eq:perturbed_system}
\end{align}
where the additive disturbance $\vect{w}_i$ is assumed
to belong to a compact set $\vect{w}_i(t) \in \mathcal{W}$ and is bounded by
$\|\vect{w}_i(t)\| \leq \overline{\vect{w}}_i,\forall t \in \mathbb{R}_{\geq 0}$.
It will be made clearer later on that the existence of this disturbance arises
due to the fact that the input $\vect{u}_i$ is a portion of the optimal solution
to an optimization problem where the \textit{predicted} states of the
neighbouring agents of agent $i$ are taken into account in the cost function,
as opposed to their \textit{true} states.

\begin{gg_box}
\begin{assumption} (Functions $f_i$ are Lipschitz continuous)

  The functions $f_i(x,u), \forall i \in \mathcal{V}, x \in \mathcal{X}_i$
  are Lipschitz continuous with Lipschitz constants $L_{f_i}$:

  \begin{align}
    \| f_i(\vect{x}_1, \vect{u}) - f_i(\vect{x}_2, \vect{u})\| \leq L_{f_i} \|\vect{x}_1 - \vect{x}_2 \|
  \end{align}

  \label{ass:f_i_Lipschitz}
\end{assumption}
\end{gg_box}


\subsubsection{Cost function}

\begin{align}
  J_i \big(\overline{\vect{u}}_i (\cdot);\ \vect{x}_i(t_k), \vect{x}_{\mathcal{N}_i}(t_k), \vect{x}_{\mathcal{Z}}\big) =
    &J_i \big(\overline{\vect{u}}_i (\cdot);\ \vect{x}_i(t_k)\big) + \\
    &J_i \big(\vect{x}_i (t_k), \vect{x}_{\mathcal{N}_i}(t_k)\big) + \\
    &J_i \big(\vect{x}_i (t_k), \vect{x}_{\mathcal{Z}}\big)
\end{align}

where

\begin{align}
  J_i \big(\overline{\vect{u}}_i (\cdot);\ \vect{x}_i(t_k)\big) &=
    \int_{t_k}^{t_k + T_p} h_i \big(\overline{\vect{x}}_i (\tau), \overline{\vect{u}}_i (\tau)\big) d \tau +
    V_i \big(\overline{\vect{x}}_i (t_k + T_p)\big) \\
  J_i \big(\vect{x}_i (t_k), \vect{x}_{\mathcal{N}_i}(t_k)\big) &=
    \sum_{j \in \mathcal{N}_i} \int_{t_k}^{t_k + T_p} g_{ij} (\overline{\vect{x}}_i (\tau), \overline{\vect{x}}_j (\tau)\big) d \tau \\
  J_i \big(\vect{x}_i (t_k), \vect{x}_{\mathcal{Z}}\big) &=
    -\sum_{z \in \mathcal{Z}} \int_{t_k}^{t_k + T_p} l_{iz} (\overline{\vect{x}}_i (\tau), \vect{x}_z (\tau)\big) d \tau
\end{align}

and

\begin{align}
  h_i(\vect{x}_i, \vect{u}_i) &= \| \vect{x}_i - \vect{x}_{i,des}\|_{Q_i} + \| \vect{u}_i\|_{R_i} \\
  g_{ij}(\vect{x}_i, \vect{x}_j) &= \| \vect{x}_i - \vect{x}_j + \vect{x}_{ij,des} \|_{G_{ij}} \\
  l_{iz}(\vect{x}_i, \vect{x}_z) &= \| \vect{x}_i - \vect{x}_z \|_{L_{iz}} \\
\end{align}


\begin{gg_box}
\begin{assumption} (Functions $h_i$ are Lipschitz continuous)

  The functions $h_i(x,u), \forall i \in \mathcal{V}, x \in \mathcal{X}_i$ are
  Lipschitz continuous with Lipschitz constants $L_{h_i}$:

  \begin{align}
    \| h_i(\vect{x}_1, \vect{u}) - h_i(\vect{x}_2, \vect{u})\| \leq L_{h_i} \|\vect{x}_1 - \vect{x}_2 \|
  \end{align}

  \label{ass:h_i_Lipschitz}
\end{assumption}
\end{gg_box}


\begin{bw_box}
\begin{lemma}
  In the presence of additive disturbances, the difference between the state
  measurement at time $t_k + \tau$, $\vect{x}(t_k + \tau)$ and the predicted
  state at that time, $\overline{\vect{x}}(t_k + \tau;\ u(\cdot), \vect{x}(t_k))$,
  under a control law $\vect{u}(t_k + \tau;\ \vect{x}(t_k))$, where
  $0 \leq \tau \leq T_p$, having the same initial condition $x(t_k)$, is
  bounded \ref{}:

  \begin{align}
    \|\vect{x}(t_k + \tau) - \overline{\vect{x}}\big(t_k + \tau;\ u(\cdot), \vect{x}(t_k)\big)\|
    \leq \frac{\overline{\vect{w}}_i}{L_{f_i}} (e^{L_{f_i} \tau} - 1)
  \end{align}

  \label{lemma:gamma_tau}
\end{lemma}
\end{bw_box}

For convenience we let the right-hand side
$\dfrac{\overline{\vect{w}}_i}{L_{f_i}} (e^{L_{f_i} t} - 1) = \gamma(t)$.


\subsubsection{Optimization problem}

\begin{align}
  \text{min}_{\overline{\vect{u}}_i (\cdot)}\ &
    J_i \big(\overline{\vect{u}}_i (\cdot);\ \vect{x}_i(t_k), \vect{x}_{\mathcal{N}_i}(t_k), \vect{x}_{\mathcal{Z}}\big) \\
  \text{subject to:} & \\
  & \dot{\overline{\vect{x}}}_i(t) = f_i (\overline{\vect{x}}_i (t), \overline{\vect{u}}_i (t)) \\
  & \overline{\vect{x}}_i (t_k) = \vect{x}_i (t_k) \\
  & \overline{\vect{u}}_i \in \mathcal{U}_i \\
  & \overline{\vect{x}}_i (t) \in \mathcal{X}_i^{t - t_k}, t \in [t_k, t_k + T_p] \\
  & \overline{\vect{x}}_i (t_k + T_p) \in \mathcal{X}_{f_i} \\
  & J_i \big(\overline{\vect{u}}_i (\cdot);\ \vect{x}_i(t_k)\big) \leq J_i^{\text{ sup}}(t)
\end{align}
where the constraint on $\overline{\vect{x}}_i (t)$ is narrowed to
$\overline{\vect{x}}_i (t) \in \mathcal{X}_i^{t - t_k} \subseteq \mathcal{X}_i$
so as to ensure that there is a robust positively invariant set for the
closed-loop system where a solution to the FHOCP exists \ref{}. Specifically,
$\mathcal{X}_i^{t - t_k} = \mathcal{X}_i \sim \mathcal{B}_i^{t-t_k}$, where

\begin{align}
  \mathcal{B}_i^{t-t_k} =
  \Big\{\vect{x}_i \in \mathbb{R}^n : \|\vect{x}_i\| \leq \gamma_i (t-t_k) = \frac{\overline{\vect{w}}_i}{L_{f_i}} (e^{L_{f_i} (t-t_k)} - 1) \Big\}
\end{align}
and the operator $\sim$ denotes the Pontryagin difference.

The constraint $J_i \big(\overline{\vect{u}}_i (\cdot);\ \vect{x}_i(t_k)\big) \leq J_i^{\text{ sup}}(t_k)$
is imposed for ensuring the stability for each agent; it will be defined later
on in the text.

\begin{gg_box}
\begin{assumption}

  There exists a local stabilizing controller
  $\vect{\kappa}_i(\vect{x}) \in \mathcal{U}_i$ such that

  \begin{align}
    \dfrac{\partial V_i}{\partial \vect{x}} f_i \Big(\vect{x}(\tau), \vect{\kappa}_i\big(\vect{x}(\tau)\big)\Big) +
      h_i\Big(\vect{x}(\tau), \vect{\kappa}_i \big(\vect{x}(\tau)\big)\Big) \leq 0, \forall \vect{x} \in \Phi_i
  \end{align}

  where $\Phi_i = \{\vect{x} \in \mathbb{R}^n : V_i(\vect{x}) \leq \alpha_i \}$, and
  $\Phi_i \subseteq \mathcal{X}_i^{T_p} = \{x \in \mathbb{R}^n : \vect{\kappa}_i(\vect{x}) \in \mathcal{U}_i\}$

  \label{ass:local_controller_k}
\end{assumption}
\end{gg_box}


\begin{gg_box}
\begin{assumption}

  The terminal costs $V_i(\vect{x})$ are Lipschitz continuous in $\vect{x} \in \Phi_i$
  with Lipschitz constants $L_{V_i}$:

  \begin{align}
    \| V_i(\vect{x}_1) - V_i(\vect{x}_2)\| \leq L_{V_i} \|\vect{x}_1 - \vect{x}_2 \|
  \end{align}

  \label{ass:V_i_Lipschitz}
\end{assumption}
\end{gg_box}


\begin{gg_box}
\begin{assumption}

  The terminal set $\mathcal{X}_{f_i} = \{ \vect{x} \in \mathbb{R}^n : V_i(\vect{x}) \in \alpha_{V_i}\}$
  is such that for all $\vect{x} \in \Phi_i$, $f_i\big(\vect{x}, \vect{\kappa}_i(\vect{x})\big) \in \mathcal{X}_{f_i} \subseteq \Phi_i$.

  \label{ass:x_f_i}
\end{assumption}
\end{gg_box}
