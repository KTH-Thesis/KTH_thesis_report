\subsection{System Model}

Consider a set of $N$ rigid bodies, with $\mathcal{V} = \{ 1,2, \ldots, N\}$,
$N  \geq 2$, operating in a workspace $W\subseteq \mathbb{R}^3$.
A coordinate frame $\{i\}, i\in\mathcal{V}$ is attached to each body's
center of mass. The workspace is assumed to be modeled as a
bounded sphere $\mathcal{B}(\vect{0},r_W)$ expressed in an inertial frame
$\{\mathcal{O}\}$.

We consider that over time $t$ each agent $i$ occupies the space of a sphere
$\mathcal{B}(\vect{p}_i(t), r_i)$, where $\vect{p}_i : \mathbb{R}_{\geq 0} \to \mathbb{R}^3$
is the position of the agent's center of mass, and $r_i < r_W$ is the radius of the
agent's body. We denote by $\vect{q}_i(t):\mathbb{R}_{\geq 0} \to \mathbb{T}^3, i\in\mathcal{V}$,
the Euler angles representing the agents' orientation,
with $\vect{q}_i = [\phi_i,\theta_i,\psi_i]^{\top}$.
We define $\vect{x}_i = [\vect{p}^{\top},\vect{q}^{\top}_i]^{\top},
\vect{v}_i =[\dot{\vect{p}}^{\top}_i, \vect{\omega}^{\top}_i]^{\top}$,
$\vect{x}_i:\mathbb{R}_{\geq 0} \to \mathbb{R}^3\times \mathbb{T}^3 =
\mathbb{M}, \vect{v}_i : \mathbb{R}_{\geq 0} \to \mathbb{R}^3\times \mathbb{R}^3
= \mathbb{R}^6$, and model the motion of agent $i$ under second order dynamics:

\begin{subequations}
	\begin{align}
    \dot{\vect{x}}_i(t) &= \mat{J}_i^{-1}(\vect{x}_i) \vect{v}_i(t), \label{eq:system_1} \\
    \vect{u}_i &= \mat{M}_i(\vect{x}_i) \dot{\vect{v}}_i(t) + \mat{C}_i(\vect{x}_i,\dot{\vect{x}}_i) \vect{v}_i(t)+\vect{g}_i(\vect{x}_i), \label{eq:system_2}
	\end{align}
  \label{eq:system}
\end{subequations}

In equation (\ref{eq:system_1}), $\mat{J}_i:\mathbb{T}^3 \to \mathbb{R}^{6\times6}$ is
a Jacobian matrix that maps the Euler angle rates to the orthogonal angular
rates $\vect{v}_i$:

\begin{equation}
  \mat{J}_i(\vect{x}_i) =
  \begin{bmatrix}
    \mat{I}_3 & \mat{0}_{3 \times 3} \\
    \mat{0}_{3 \times 3} & \mat{J}_{ q }(\vect{x}_i) \\
  \end{bmatrix} \notag, \text{ with }
  \mat{J}_q(\vect{x}_i) =
  \begin{bmatrix}
    1 & 0 & -\sin\theta_i \\
    0 & \cos\phi_i & \cos\theta_i \sin\phi_i \\
    0 & - \sin\phi_i & \cos\phi_i \cos\theta_i
  \end{bmatrix} \notag
\end{equation}

In order for $\mat{J}_i$ to be always well-defined (and hence invertible, since
$\det(\mat{J}_i) = \cos\theta_i$) we need to make the following assumption:

\note{?? EVENTUALLY NEEDED? CHECK}
\begin{full_box}
\begin{assumption}
	The angle $\theta_i$ satisfies the inequality
  $$-\frac{\pi}{2} < \theta_i(t) < \frac{\pi}{2} ,\forall i\in\mathcal{V},t\in\mathbb{R}_{\geq 0}$$.
\label{as:J}
\end{assumption}
\end{full_box}

\begin{figure}[ht!]
	\centering
    \begin{tikzpicture}[scale = 0.5]
	%draw the global frame
	\draw [color=black,thick,->,>=stealth'](-9, -5) to (-7, -5);
	\draw [color=black,thick,->,>=stealth'](-9, -5) to (-9, -3);
	\draw [color=black,thick,->,>=stealth'](-9, -5) to (-10, -6.5);
	\node at (-9.8, -5.0) {$\{0\}$};

	%draw agent i
	\draw [color = blue, fill = blue!20] (-4.5,0) circle (2.5cm);
  \node at (-5.7, 0.0) {$\{i\}$};
	\draw[green,thick,dashed] (-4.5,0) circle (5.0cm);
	\draw [color=black,thick,->,>=stealth'](-9, -5) to (-4.5, -0.1);
	\node at (-7.7, -3.0) {$p_i$};
	\draw [color=green,thick,dashed,->,>=stealth'](-4.5, 0.0) to (-8.93, 2.43);
	\node at (-7.3, 2.15) {$d_i$};
	\draw [color=black,thick,dashed,->,>=stealth'](-4.5, 0.0) to (-2.0, 0.0);
	\node at (-3.3, 0.3) {$r_i$};
	\node at (-4.5, 0.0) {$\bullet$};
	\node at (-4.8, 3.0) {$\mathcal{B}(p_i, r_i)$};

	%draw agent j
	\draw [color = red, fill = red!20] (3.2, 0) circle (1.5cm);
	\node at (2.5, 0.3) {$\{j\}$};
	\draw[orange,thick,dashed,] (3.2, 0) circle (4.1cm);
	\draw [color=black,thick,->,>=stealth'](-9, -5) to (3.2, -0.1);
	\node at (-5.0, -3.9) {$p_j$};
	\draw [color=orange,thick,dashed,->,>=stealth'](3.2, 0.0) to (3.2, -4.0);
	\node at (3.8, -2.7) {$d_j$};
	\draw [color=black,thick,dashed,->,>=stealth'](3.2, 0.0) to (4.7, 0.0);
	\node at (4.1, 0.3) {$r_j$};
	\node at (3.2, 0.0) {$\bullet$};
	\node at (3.0, 2.1) {$\mathcal{B}(p_j, r_j)$};

	% draw the obstacle
	\draw [color = black, fill = black!20] (-1, -8) circle (1.2cm);
	\draw [color=black,thick,->,>=stealth'](-9, -5) to (-1.1, -7.98);
	\draw [color=black,thick,dashed,->,>=stealth'](-1, -8) to (-1, -6.8);
	\node at (-1, -8) {$\bullet$};
	\node at (-5.0, -6.0) {$p_{o_z}$};
	\node at (-0.40, -7.5) {$r_{o_z}$};
\end{tikzpicture}

    \caption{Illustration of two agents $i, j \in \mathcal{V}$ and an static
      obstacle $o_z$ in the workspace; $\{\mathcal{O}\}$ is the inertial frame,
      $\{i\}, \{j\}$ are the frames attached to the agents' center of mass,
      $\vect{p}_i, \vect{p}_j, \vect{p}_{o_z} \in \mathbb{R}^3$ are the
      positions of the center of mass of the agents $i,j$ and the
      obstacle $o_z$ respectively, expressed in frame
      $\{\mathcal{O}\}$. $r_i, r_j, r_{o_z}$ are the radii of the agents $i,j$
      and the obstacle $o_z$ respectively. $d_i, d_j$ with
      $d_i > d_j$ are the agents' sensing ranges.
      In this figure, agents $i$ and $j$ are not neighbours, since
      $\vect{p}_j \notin \mathcal{B}(\vect{p}_i(t), d_i)$.}
	\label{fig:two_agents_one_obstacle}
\end{figure}

In equation (\ref{eq:system_2}), $\mat{M}_i:\mathbb{M} \to \mathbb{R}^{6\times6}$ is
the symmetric and positive definite \textit{inertia matrix},
$\mat{C}_i:\mathbb{M}\times\mathbb{R}^6 \to \mathbb{R}^{6\times6}$ is the
\textit{Coriolis matrix} and $\vect{g}_i:\mathbb{M} \to \mathbb{R}^6$ is the
\textit{gravity vector}.
Finally, $\vect{u}_i\in\mathbb{R}^6$ is the control input vector representing
the $6$D generalized \textit{actuation force} acting on the agent.

%Let us also define the vectors
%$\vect{X} = [\vect{x}_1^\top, \dots, x_N^\top]^\top :
%\mathbb{R}_{\geq 0} \to \mathbb{M}^N, \vect{V} = [\vect{v}_1^\top, \dots
%\vect{v}_N^\top]^\top: \mathbb{R}_{\geq 0} \to \mathbb{R}^{6N}$.


\begin{gray_box}
  \begin{remark}
    According to \cite{Siciliano2009}, the matrices
    $\dot{\mat{M}}_i - 2\mat{C}_i, i \in \mathcal{V}$ are skew-symmetric.
    From \cite{horn_jonshon}, we have that a quadratic form of a skew-symmetric
    matrix is always equal to $\mat{0}$. Hence, for the matrices
    $\dot{\mat{M}}_i - 2 \mat{C}_i$ it holds that:

    \begin{equation} \label{eq:skew_symm}
      \vect{y}^\top \left[\dot{\mat{M}}_i - 2 \mat{C}_i\right]\vect{y} = 0,
        \forall \vect{y} \in \mathbb{R}^n, i \in \mathcal{V}.
    \end{equation}
  \end{remark}
\end{gray_box}

However, access to measurements or knowledge of these matrices and vectors
was not considered up until now. At this point we declare the following
assumption:

\begin{full_box}
\begin{assumption} (Measurements and Access to Information Assumption)
  \begin{enumerate}

    \item Agent $i$ has access to the measurements
      $\vect{p}_i, \vect{q}_i, \dot{\vect{p}}_i, \vect{\omega}_i, i\in\mathcal{V}$,
      that is, vectors $\vect{x}, \vect{v}$, pertaining to himself,

    \item Agent $i$ has a (upper-bounded) sensing range $d_i$ such that
      $$d_i > \max\{r_i + r_j : \forall i,j \in \mathcal{V}, i \neq j\}$$

    \item the inertia $\mat{M}$ and Coriolis $\mat{C}$ vector fields are
      bounded and unknown

    \item the gravity vector is bounded and known.

  \end{enumerate}
\end{assumption}
\end{full_box}

Therefore, by defining the neighboring set
$\mathcal{N}_i(t) = \{j\in\mathcal{V} : \vect{p}_j(t)\in\mathcal{B}(\vect{p}_i(t), d_i)\}$,
agent $i$ also knows at each time instant $t$ all
$\vect{p}_{j \triangleright i}(t), \vect{q}_{j \triangleright i}(t)$
and, since it knows its own $\vect{p}_i(t), \vect{q}_i(t)$, it can compute all
$\vect{p}_{j}(t), \vect{q}_{j}(t), \forall j\in \mathcal{N}_i(t),t\in\mathbb{R}_{\geq 0}$.

In the workspace there are $|Z|$ \textit{static obstacles}, modeled as
spheres with centers at positions
$\vect{p}_{o_z}$ with radii $r_{o_z}\in \mathbb{R}^3, z \in \mathcal{Z} = \{1,\dots,|Z| \}$.
Thus, the obstacles are modeled by the spheres
$\mathcal{B}(\vect{p}_{o_z}, r_{o_z}), z \in \{1,\dots,|Z|\}$. The geometry in
workspace $W$ of two agents $i$ and $j$ as well as an obstacle $z$ is depicted
in Fig. \ref{fig:two_agents_one_obstacle}.

Let us also define the distance between agents $i,j$ as
$d_{ij,a}: \mathbb{R}^6 \to \mathbb{R}_{\geq 0}$, and that between agent $i$
and obstacle $z$ as $d_{iz,o} : \mathbb{R}^3 \to \mathbb{R}_{\geq 0}$:

\begin{subequations}
	\begin{align}
    d_{ij,a}(t) &= \| \vect{p}_i(t) - \vect{p}_j(t) \|, \\
    d_{iz,o}(t) &= \| \vect{p}_i(t) - \vect{p}_z(t) \|
	\end{align}
\end{subequations}

$\forall i, j \in \mathcal{V}, i \neq j, z \in \mathcal{Z}$, as well as
constants $\underline{d}_{ij, a} = r_{i} + r_{j},
\underline{d}_{iz, o} = r_{i} + r_{o_z}$. The latter stand for the minimum
distance between two agents, and between an agent and an obstacle. These arise
spatially as physical limitations and will be utilized as collision-avoidance
constraints.

The topology of the multi-agent network is modeled through the graph
$\mathcal{G} = (\mathcal{V},\mathcal{E})$, with $\mathcal{V}=\{1,\dots,N\}$ and
$\mathcal{E}=\big\{\{i,j\}\in\mathcal{V}\times\mathcal{V} : i \neq j, j\in\mathcal{N}_i(0) \text{ and } i\in\mathcal{N}_j(0)\big\}$.
The latter implies that at $t=0$ the graph is undirected, i.e.,

\begin{equation} \label{eq:initially_connected}
  d_{ij,a}(0) < d_i, \forall i \in \mathcal{V}, j \in \mathcal{N}_i(0)
\end{equation}

We also consider that $\mathcal{G}$ is static in the sense that no edges are
added to the graph. We do not exclude, however, edge removal through
connectivity loss between initially neighboring agents, which we guarantee
to avoid, as presented in the sequel. It is also assumed that at $t=0$ the
neighboring agents are in a \textit{collision-free configuration}, i.e.,

\begin{equation}
  \underline{d}_{ij, a} < d_{ij,a}(0), \forall i,j \in \mathcal{V}, i \neq j
\label{eq:initially_coll_free}
\end{equation}
To capitulate, we assume that at time $t=0$
$$\underline{d}_{ij,a} < d_{ij,a}(0) < d_i$$
for agent $i$ and all of his adjacent agents $j$ at time $t=0$.

We will now introduce the assumption we make on the feasibility of the
problem. But first, let us denote by $d_o$

\begin{align*}
  d_o &= \min\{\| \vect{p}_{o_z} - \vect{p}_{o_{z'}}\| : z,z' \in \mathcal{Z}, z \neq z' \},
\end{align*}
the distance between the two least distant obstacles in the workspace,
by $d_{o,W}$

\begin{align*}
  d_{o,W} &= \min\{r_W - \left( r_{o_z} + \| \vect{p}_{o_z} \| \right) : z \in \mathcal{Z}\},
\end{align*}
the distance between the least distant obstacle from the boundary of the
workspace and the boundary itself, by $D$

\begin{equation*}
  D = \min\{d_o, d_{o,W}\}
\end{equation*}
the least of these two distances, and by $\Delta$

\begin{align*}
  &\hspace{-2mm} \Delta =  \max\Big\{d_{ij, a}+r_i+r_j: i, j \in \mathcal{V}, i \neq j, \notag \\
  &\hspace{-2mm} \| \vect{p}_k - \vect{p}_\ell \| = \vect{p}_{k \ell,\text{des}},
  \|\vect{q}_k - \vect{q}_\ell\| = \vect{q}_{k \ell,\text{des}},
  k \in \mathcal{V},
  \ell \in \mathcal{N}_k(0)\Big\}
\end{align*}
the \emph{diameter of formation} $\Delta$, which is the
distance between the two most distant agents when formation is achieved.

\begin{assumption}(Problem Feasibility Assumption)
	In order for the problem to be feasible, we make the following natural
  geometric assumptions:

	\begin{itemize}
		\item All the agents should be able to pass between any two obstacles
      and between all obstacles and the boundary of the workspace, and
      simultaneously, without any of them colliding to each other or with the
      obstacles or the boundary of the workspace.
      Thus, it is required $D >  \sum_{i \in \mathcal{V}}^{} 2r_i$.
		\item When the multi-agent system reach the desired formation, it should be
      able to pass between two of the obstacles and between an obstacle and the
      boundary of the workspace. Thus, it is required $D > \Delta$.
	\end{itemize}

	These geometrical assumptions can be summarized in the following
  inequality:

	\begin{equation} \label{eq:geometric_constraint}
    D > \max\left\{\Delta, \sum_{i \in \mathcal{V}}^{} 2r_i \right\}
	\end{equation}

\end{assumption}
