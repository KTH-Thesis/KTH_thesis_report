\subsection{System Model}

Consider a set of $N$ rigid bodies, with $\mathcal{V} = \{ 1,2, \ldots, N\}$,
$N  \geq 2$, operating in a workspace $W\subseteq \mathbb{R}^3$.
A coordinate frame $\{i\}, i\in\mathcal{V}$ is attached to each body's
center of mass. The workspace is assumed to be modeled as a
bounded sphere $\mathcal{B}(\vect{0},r_W)$ expressed in an inertial frame
$\{\mathcal{O}\}$.

We consider that over time $t$ each agent $i$ occupies the space of a sphere
$\mathcal{B}(\vect{p}_i(t), r_i)$, where $\vect{p}_i : \mathbb{R}_{\geq 0} \to \mathbb{R}^3$
is the position of the agent's center of mass, and $r_i < r_W$ is the radius of the
agent's body. We denote $\vect{q}_i(t):\mathbb{R}_{\geq 0} \to \mathbb{T}^3, i\in\mathcal{V}$,
the Euler angles representing the agents' orientation with respect to the
inertial frame $\{\mathcal{O}\}$,
with $\vect{q}_i \overset{\Delta}{=} [\phi_i,\theta_i,\psi_i]^{\top}$.
We define $\vect{x}_i \overset{\Delta}{=} [\vect{p}^{\top},\vect{q}^{\top}_i]^{\top},
\vect{v}_i \overset{\Delta}{=} [\dot{\vect{p}}^{\top}_i, \vect{\omega}^{\top}_i]^{\top}$,
$\vect{x}_i:\mathbb{R}_{\geq 0} \to \mathbb{R}^3\times \mathbb{T}^3 =
\mathbb{M}, \vect{v}_i : \mathbb{R}_{\geq 0} \to \mathbb{R}^3\times \mathbb{R}^3
= \mathbb{R}^6$, and model the motion of agent $i$ under second order dynamics:

\begin{subequations}
	\begin{align}
    \dot{\vect{x}}_i(t) &= \mat{J}_i^{-1}(\vect{x}_i) \vect{v}_i(t), \label{eq:system_1} \\
    \vect{u}_i &= \mat{M}_i(\vect{x}_i) \dot{\vect{v}}_i(t) + \mat{C}_i(\vect{x}_i,\dot{\vect{x}}_i) \vect{v}_i(t)+\vect{g}_i(\vect{x}_i), \label{eq:system_2}
	\end{align}
  \label{eq:system}
\end{subequations}

In equation \eqref{eq:system_1}, $\mat{J}_i:\mathbb{T}^3 \to \mathbb{R}^{6\times6}$ is
a Jacobian matrix that maps the non-orthogonal Euler angle rates to the
orthogonal angular velocities $\vect{v}_i$:

\begin{equation}
  \mat{J}_i(\vect{x}_i) =
  \begin{bmatrix}
    \mat{I}_3 & \mat{0}_{3 \times 3} \\
    \mat{0}_{3 \times 3} & \mat{J}_{ q }(\vect{x}_i) \\
  \end{bmatrix} \notag, \text{ with }
  \mat{J}_q(\vect{x}_i) =
  \begin{bmatrix}
    1 & 0 & -\sin\theta_i \\
    0 & \cos\phi_i & \cos\theta_i \sin\phi_i \\
    0 & - \sin\phi_i & \cos\phi_i \cos\theta_i
  \end{bmatrix} \notag
\end{equation}

The matrix $\mat{J}_i$ is singular when $det(\mat{J}_i)$ $=$ $\cos\theta_i = 0$
$\Leftrightarrow$ $\theta_i$ $=$ $\pm \frac{\pi}{2}$. The controller
proposed in this thesis guarantees that this is always avoided, and hence
equation \eqref{eq:system_1} is well defined. This gives rise to the following
remark:

\begin{bw_box}
  \begin{remark}
    $det(\mat{J}_i) = \cos\theta_i \leq 1$, $\forall i \in \mathcal{V}$
  \end{remark}
\end{bw_box}

\begin{figure}[ht!]
	\centering
    \begin{tikzpicture}[scale = 0.5]
	%draw the global frame
	\draw [color=black,thick,->,>=stealth'](-9, -5) to (-7, -5);
	\draw [color=black,thick,->,>=stealth'](-9, -5) to (-9, -3);
	\draw [color=black,thick,->,>=stealth'](-9, -5) to (-10, -6.5);
  \node at (-9.9, -5.0) {$\{\mathcal{O}\}$};

	%draw agent i
	\draw [color = blue, fill = blue!20] (-4.5,0) circle (2.5cm);
  \node at (-5.7, 0.0) {$\{i\}$};
	\draw[green,thick,dashed] (-4.5,0) circle (5.0cm);
	\draw [color=black,thick,->,>=stealth'](-9, -5) to (-4.5, -0.1);
	\node at (-7.7, -3.0) {$p_i$};
	\draw [color=green,thick,dashed,->,>=stealth'](-4.5, 0.0) to (-8.93, 2.43);
	\node at (-7.3, 2.15) {$d_i$};
	\draw [color=black,thick,dashed,->,>=stealth'](-4.5, 0.0) to (-2.0, 0.0);
	\node at (-3.3, 0.3) {$r_i$};
	\node at (-4.5, 0.0) {$\bullet$};
	\node at (-4.8, 3.0) {$\mathcal{B}(p_i, r_i)$};

	%draw agent j
	\draw [color = red, fill = red!20] (3.2, 0) circle (1.5cm);
	\node at (2.5, 0.3) {$\{j\}$};
	\draw[orange,thick,dashed,] (3.2, 0) circle (4.1cm);
	\draw [color=black,thick,->,>=stealth'](-9, -5) to (3.2, -0.1);
	\node at (-5.0, -3.9) {$p_j$};
	\draw [color=orange,thick,dashed,->,>=stealth'](3.2, 0.0) to (3.2, -4.0);
	\node at (3.8, -2.7) {$d_j$};
	\draw [color=black,thick,dashed,->,>=stealth'](3.2, 0.0) to (4.7, 0.0);
	\node at (4.1, 0.3) {$r_j$};
	\node at (3.2, 0.0) {$\bullet$};
	\node at (3.0, 2.1) {$\mathcal{B}(p_j, r_j)$};

	% draw the obstacle
	\draw [color = black, fill = black!20] (-1, -8) circle (1.2cm);
	\draw [color=black,thick,->,>=stealth'](-9, -5) to (-1.1, -7.98);
	\draw [color=black,thick,dashed,->,>=stealth'](-1, -8) to (-1, -6.8);
	\node at (-1, -8) {$\bullet$};
	\node at (-5.0, -6.0) {$p_{\ell}$};
	\node at (-0.40, -7.5) {$r_{\ell}$};
\end{tikzpicture}

    \caption{Illustration of two agents $i, j \in \mathcal{V}$ and an static
      obstacle $o_z$ in the workspace; $\{\mathcal{O}\}$ is the inertial frame,
      $\{i\}, \{j\}$ are the frames attached to the agents' center of mass,
      $\vect{p}_i, \vect{p}_j, \vect{p}_{o_z} \in \mathbb{R}^3$ are the
      positions of the center of mass of the agents $i,j$ and the
      obstacle $o_z$ respectively, expressed in frame
      $\{\mathcal{O}\}$. $r_i, r_j, r_{o_z}$ are the radii of the agents $i,j$
      and the obstacle $o_z$ respectively. $d_i, d_j$ with
      $d_i > d_j$ are the agents' sensing ranges.
      In this figure, agents $i$ and $j$ are not neighbours, since
      $\vect{p}_j \notin \mathcal{B}(\vect{p}_i(t), d_i)$.}
	\label{fig:two_agents_one_obstacle}
\end{figure}

In equation \eqref{eq:system_2}, $\mat{M}_i:\mathbb{M} \to \mathbb{R}^{6\times6}$ is
the symmetric and positive definite \textit{inertia matrix},
$\mat{C}_i:\mathbb{M}\times\mathbb{R}^6 \to \mathbb{R}^{6\times6}$ is the
\textit{Coriolis matrix} and $\vect{g}_i:\mathbb{M} \to \mathbb{R}^6$ is the
\textit{gravity vector}.
Finally, $\vect{u}_i\in\mathbb{R}^6$ is the control input vector representing
the $6$D generalized \textit{actuation force} acting on the agent.

%Let us also define the vectors
%$\vect{X} = [\vect{x}_1^\top, \dots, x_N^\top]^\top :
%\mathbb{R}_{\geq 0} \to \mathbb{M}^N, \vect{V} = [\vect{v}_1^\top, \dots
%\vect{v}_N^\top]^\top: \mathbb{R}_{\geq 0} \to \mathbb{R}^{6N}$.


\begin{bw_box}
  \begin{remark}
    According to \cite{Siciliano2009}, the matrices
    $\dot{\mat{M}}_i - 2\mat{C}_i, i \in \mathcal{V}$ are skew-symmetric.
    The quadratic form of a skew-symmetric matrix is always equal to 0
    \cite{horn_jonshon}, hence:

    \begin{equation}
      \vect{y}^\top \left[\dot{\mat{M}}_i - 2 \mat{C}_i\right]\vect{y} = 0,
        \forall \vect{y} \in \mathbb{R}^n, i \in \mathcal{V}.
    \label{eq:skew_symm}
    \end{equation}
  \end{remark}
\end{bw_box}

However, access to measurements of, or knowledge about these matrices and
vectors was not considered up until now. At this point we make the following
assumption:

\begin{gg_box}
\begin{assumption} (Measurements and Access to Information Assumption From an
  Inter-agent Perspective)
  \begin{enumerate}

    \item Agent $i$ has access to measurements
      $\vect{p}_i, \vect{q}_i, \dot{\vect{p}}_i, \vect{\omega}_i, i\in\mathcal{V}$,
      that is, vectors $\vect{x}_i, \vect{v}_i$ pertaining to himself,

    \item Agent $i$ has a (upper-bounded) sensing range $d_i$ such that
      $$d_i > \max\{r_i + r_j : \forall i,j \in \mathcal{V}, i \neq j\}$$

    \item the inertia $\mat{M}$ and Coriolis $\mat{C}$ vector fields are
      bounded and unknown

    \item the gravity vectors $\vect{g}$ are bounded and known

  \end{enumerate}
\end{assumption}
\end{gg_box}

The consequence of points 1 and 2 is that, by defining the neighboring set of
agent $i$ as
$\mathcal{N}_i(t) \overset{\Delta}{=} \{j\in\mathcal{V} : \vect{p}_j(t)\in\mathcal{B}(\vect{p}_i(t), d_i)\}$,
agent $i$ also knows at each time instant $t$ all
$\vect{p}_{j \triangleright i}(t), \vect{q}_{j \triangleright i}(t),
\dot{\vect{p}}_{j \triangleright i}(t), \vect{\omega}_{j \triangleright i}(t)$.
Therefore, agent $i$ assumes access to all
$\vect{p}_{j}(t)$, $\vect{q}_{j}(t)$, $\dot{\vect{p}}_j(t)$,
$\vect{\omega}_j(t)$, $\forall j\in \mathcal{N}_i(t),t\in\mathbb{R}_{\geq 0}$,
by virtue of being able to calculate them using knowledge of its own
$\vect{p}_i(t)$, $\vect{q}_i(t)$, $\dot{\vect{p}}_i(t)$, $\vect{\omega}_i(t)$.



In the workspace there are $|Z|$ \textit{static obstacles}, modeled as
spheres with centers at positions $\vect{p}_{o_z}$ with radii
$r_{o_z}\in \mathbb{R}^3, z \in \mathcal{Z} = \{1,\dots,|Z| \}$.
Thus, the obstacles are modeled by the spheres
$\mathcal{B}(\vect{p}_{o_z}, r_{o_z}), z \in \{1,\dots,|Z|\}$. The geometry in
workspace $W$ of two agents $i$ and $j$ as well as an obstacle $z$ is depicted
in Fig. \ref{fig:two_agents_one_obstacle}.

Let us also define the distance between agents $i,j$ as
$d_{ij,a}: \mathbb{R}^6 \to \mathbb{R}_{\geq 0}$, and that between agent $i$
and obstacle $z$ as $d_{iz,o} : \mathbb{R}^3 \to \mathbb{R}_{\geq 0}$:

\begin{subequations}
	\begin{align}
    d_{ij,a}(t) &\overset{\Delta}{=} \| \vect{p}_i(t) - \vect{p}_j(t) \|, \\
    d_{iz,o}(t) &\overset{\Delta}{=} \| \vect{p}_i(t) - \vect{p}_z(t) \|
	\end{align}
\end{subequations}

$\forall i, j \in \mathcal{V}, i \neq j, z \in \mathcal{Z}$, as well as
constants $\underline{d}_{ij, a} \overset{\Delta}{=} r_{i} + r_{j},
\underline{d}_{iz, o} \overset{\Delta}{=} r_{i} + r_{o_z}$. The latter stand for the minimum
distance between two agents, and between an agent and an obstacle. These arise
spatially as physical limitations and will be utilized as collision-avoidance
constraints.

The \textit{topology} of the multi-agent network is modeled through the graph
$\mathcal{G} = (\mathcal{V},\mathcal{E})$, with $\mathcal{V}=\{1,\dots,N\}$ and
$\mathcal{E}=\big\{\{i,j\}\in\mathcal{V}\times\mathcal{V} : i \neq j, j\in\mathcal{N}_i(0) \text{ and } i\in\mathcal{N}_j(0)\big\}$.
The latter implies that at $t=0$ the graph is undirected, i.e.,

\begin{align}
  d_{ij,a}(0) &< d_i, \forall i \in \mathcal{V}, j \in \mathcal{N}_i(0) \label{eq:initially_connected_0} \\
  d_{ji,a}(0) &< d_j, \forall j \in \mathcal{V}, i \in \mathcal{N}_j(0) \label{eq:initially_connected_1}
\end{align}

We also consider that the connected graph $\mathcal{G}$ is static, in the sense
that no edges are added to it through time. We do not exclude, however, edge
removal through connectivity loss between initially neighboring agents, which we
guarantee to avoid, as presented in the sequel. It is also assumed that at
$t=0$ the neighboring agents are in a \textit{collision-free configuration},
i.e.

\begin{equation}
  \underline{d}_{ij, a} < d_{ij,a}(0), \forall i,j \in \mathcal{V}, i \neq j
\label{eq:initially_coll_free}
\end{equation}

Furthermore, it is assumed that the Jacobian $\mat{J}_i$ is well-defined
$\forall i \in \mathcal{V}$. These four assumptions, which concern the initial
conditions of the problem, are summarized in assumption
\ref{ass:initial_conditions_assumption}:

\begin{gg_box}
\begin{assumption}(Initial Conditions Assumption)\\

  At time $t = 0$

  \begin{enumerate}

	  \item the communication graph $\mathcal{G}$ is connected,

    \item all agents are in a collision-free configuration:

      $$\underline{d}_{ij,a} < d_{ij,a}(0) < d_i$$

    \item all agents are in a singularity-free configuration:

      $$\theta_i(0) \neq \pm \frac{\pi}{2}$$

  \end{enumerate}
  $\forall i \in \mathbb{V}$ and all of $i$'s adjacent agents $j$.
  \label{ass:initial_conditions_assumption}
\end{assumption}
\end{gg_box}





At this point, let us define $d_o$
\begin{align*}
  d_o &\overset{\Delta}{=} \min\{\| \vect{p}_{o_z} - \vect{p}_{o_{z'}}\| : z,z' \in \mathcal{Z}, z \neq z' \},
\end{align*}
as the distance between the two least distant obstacles in the workspace,
$d_{o,W}$
\begin{align*}
  d_{o,W} &\overset{\Delta}{=} \min\{r_W - \left( r_{o_z} + \| \vect{p}_{o_z} \| \right) : z \in \mathcal{Z}\},
\end{align*}
as the distance between the least distant obstacle from the boundary of the
workspace and the boundary itself, $D$
\begin{equation*}
  D \overset{\Delta}{=} \min\{d_o, d_{o,W}\}
\end{equation*}
as the least of these two distances, and $\Delta$
\begin{align*}
  &\hspace{-2mm} \Delta \overset{\Delta}{=}  \max\Big\{d_{ij, a}+r_i+r_j: i, j \in \mathcal{V}, i \neq j, \notag \\
  &\hspace{-2mm} \| \vect{p}_k - \vect{p}_\ell \| = \vect{p}_{k \ell,\text{des}},
  \|\vect{q}_k - \vect{q}_\ell\| = \vect{q}_{k \ell,\text{des}},
  k \in \mathcal{V},
  \ell \in \mathcal{N}_k(0)\Big\}
\end{align*}
as the \emph{diameter of formation}. $\Delta$ is the distance between the two
most distant agents when formation is achieved. Given these notions, we
can state an assumption on the feasibility of a solution to the problem
that this thesis addresses:

\begin{gg_box}
\begin{assumption}(After-formation Geometric Assumption)

	\begin{itemize}
		\item When the multi-agent system reaches the desired formation, it should
      be able to pass between two of the obstacles and between an
      obstacle and the boundary of the workspace.
      Thus, it is required that $D > \Delta$.
		\item As a consequence, at the very least, all agents should be able to
      pass between any two obstacles and between all obstacles and the
      boundary of the workspace, without, simultaneously, any of them
      colliding with each other or with the obstacles or the boundary of the
      workspace.
      Thus, it is required that $D >  \sum_{i \in \mathcal{V}}^{} 2r_i$.
	\end{itemize}

	These geometric assumptions can be summarized in the following
  inequality:

	\begin{equation}
    D > \max\left\{\Delta, \sum_{i \in \mathcal{V}}^{} 2r_i \right\}
  \label{eq:geometric_constraint}
	\end{equation}

\end{assumption}
\end{gg_box}
