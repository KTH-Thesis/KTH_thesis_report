\subsection{Notation}

The set of positive integers is denoted as $\mathbb{N}$. The real $n$-coordinate
space, with $n\in\mathbb{N}$, is denoted as $\mathbb{R}^n$;
$\mathbb{R}^n_{\geq 0}$ and $\mathbb{R}^n_{> 0}$ are the sets of real
$n$-vectors with all elements nonnegative and positive, respectively. Given a
set $S$, we denote as $\lvert S\lvert$ its cardinality. The notation $\|x\|$
is used for the Euclidean norm of a vector $x \in \mathbb{R}^n$. Given a
symmetric matrix $A, \lambda_{\text{min}}(A) = \min \{|\lambda| : \lambda \in \sigma(A) \}$
denotes the minimum eigenvalue of $A$, respectively, where $\sigma(A)$ is the
set of all the eigenvalues of $A$ and $\text{rank}(A)$ is its rank;
$A \otimes B$ denotes the Kronecker product of matrices $A, B \in \mathbb{R}^{m \times n}$,
as was introduced in \cite{horn_jonshon}. Define by $\mathbbm{1}_n \in \mathbb{R}^n, I_n \in \mathbb{R}^{n \times n}, 0_{m \times n} \in \mathbb{R}^{m \times n}$
the column vector with all entries $1$, the unit matrix and the $m \times n$
matrix with all entries zeros, respectively.
A matrix $A \in \mathbb{R}^{n \times n}$ is called skew-symmetric if and only
if $A^\top = -A$. $\mathcal{B}(c,r) = \{x \in \mathbb{R}^3: \|x-c\| \leq r\}$
is the $3$D sphere of radius $r \in \mathbb{R}_{\ge 0}$ and center
$c\in\mathbb{R}^{3}$.
The vector connecting the origins of coordinate frames $\{A\}$ and $\{B$\}
expressed in frame $\{C\}$ coordinates in $3$D space is denoted as
$p^{\scriptscriptstyle C}_{{\scriptscriptstyle B/A}}\in{\mathbb{R}}^{3}$.
Given $a\in\mathbb{R}^3$, $S(a)$ is the skew-symmetric matrix
defined according to $S(a)b = a\times b$. We further denote as
$q_{\scriptscriptstyle B/A}\in\mathbb{T}^3$ the Euler angles representing
the orientation of frame $\{B\}$ with respect to frame $\{A\}$, where
$\mathbb{T}^3$ is the $3$D torus. The angular velocity of frame $\{B\}$ with
respect to $\{A\}$, expressed in frame $\{C\}$ coordinates, is denoted as
$\omega^{\scriptscriptstyle C}_{\scriptscriptstyle B/A}\in \mathbb{R}^{3}$.
We also use the notation $\mathbb{M} = \mathbb{R}^3\times \mathbb{T}^3$.
For notational brevity, when a coordinate frame corresponds to an inertial frame
of reference $\{0\}$, we will omit its explicit notation
(e.g., $p_{\scriptscriptstyle B} = p^{\scriptscriptstyle 0}_{\scriptscriptstyle B/0},
\omega_{\scriptscriptstyle B} = \omega^{\scriptscriptstyle 0}_{\scriptscriptstyle B/0}$ etc.).
All vector and matrix differentiations are derived with respect to an inertial
frame $\{0\}$, unless otherwise stated.

\subsection{Graph Theory}

An \textit{undirected graph} $\mathcal{G}$ is a pair
$(\mathcal{V}, \mathcal{E})$, where $\mathcal{V}$ is a finite set of nodes,
representing a team of agents, and
$\mathcal{E} \subseteq \{ \{i,j\} : i,j \in \mathcal{V}, i \neq j\}$,
with $M = |\mathcal{E}|$, is the set of edges that model the communication
capability between neighboring agents. For each agent, its neighbors' set
$\mathcal{N}_i$ is defined as
$\mathcal{N}_i = \{i_1, \ldots, i_{N_i}\} = \{ j \in \mathcal{V} : \{i,j\} \in \mathcal{E}\}$,
where $i_1, \ldots, i_{N_i}$ is an enumeration of the neighbors of agent $i$
and $N_i = |\mathcal N_i|$.

If there is an edge $\{i, j\} \in \mathcal{E}$, then $i, j$ are called
\textit{adjacent}. A \textit{path} of length $r$ from vertex $i$ to vertex
$j$ is a sequence of $r+1$ distinct vertices, starting with $i$ and ending
with $j$, such that consecutive vertices are adjacent. For $i = j$, the path
is called a \text{cycle}. If there is a path between any two vertices of the
graph $\mathcal{G}$, then $\mathcal{G}$ is called \textit{connected}.
A connected graph is called a \text{tree} if it contains no cycles.
