\section{Auxiliary Prerequisites}

This section features auxiliary and useful theorems, lemmas and definitions
needed to support the advocated solutions in part
\ref{part:advocated_solutions}.

% The Gronwall-Bellman inequality
%===============================================================================
%\begin{bw_box}
  \begin{lemma} \cite{khalil_nonlinear_systems} \textit{The Gr\"{o}nwall-Bellman Inequality}
    \label{lemma:bellman_inequality}

    Let $\lambda : [a,b] \to \mathbb{R}$ be continuous and
    $\mu : [a,b] \to \mathbb{R}$ be continuous and non-negative. If a
    continuous function $y : [a,b] \to \mathbb{R}$ satisfies
    \begin{align}
      y(t) \leq \lambda(t) + \int_a^t \mu(s) y(s) ds
    \end{align}
    for $a \leq t \leq b$, then on the same interval
    \begin{align}
      y(t) \leq \lambda(t) + \int_a^t \lambda(s) \mu(s) e^{\int_s^t \mu(\tau)d\tau} ds
    \end{align}
    In particular, if $\lambda(t) \equiv \lambda$ is a constant, then
    \begin{align}
      y(t) \leq \lambda e^{\int_a^t \mu(\tau)d\tau} ds
    \end{align}
    If $\lambda(t) \equiv \lambda$ and $\mu(t) \equiv \mu$ are both constants,
    then
    \begin{align}
      y(t) \leq \lambda e^{\mu (t - a)}
    \end{align}
    \\[2.5ex]
  \end{lemma}
%\end{bw_box}

% Class K functions
%===============================================================================
%\begin{bw_box}
\begin{definition}\cite{khalil_nonlinear_systems} (\textit{Class $\mathcal{K}$ function})
\label{def:k_class}

  A continuous function $\alpha : [0, a) \to [0, \infty)$
  is said to belong to class $\mathcal{K}$ if
  \begin{enumerate}
    \item it is strictly increasing
    \item $\alpha (0) = 0$
  \end{enumerate}
  If $a = \infty$ and $\lim\limits_{r \to \infty} \alpha(r) = \infty$, then function
  $\alpha$ is said to belong to class $\mathcal{K}_{\infty}$
\\[2.5ex]
\end{definition}
%\end{bw_box}


% Class KL functions
%===============================================================================
%\begin{bw_box}
\begin{definition}\cite{khalil_nonlinear_systems} (\textit{Class $\mathcal{KL}$ function})
\label{def:kl_class}

  A continuous function $\beta : [0, a) \times [0, \infty) \to [0, \infty)$
  is said to belong to class $\mathcal{KL}$ if
  \begin{enumerate}
    \item for a fixed $s$, the mapping $\beta(r,s)$ belongs to class $\mathcal{K}$ with respect to $r$
    \item for a fixed $r$, the mapping $\beta(r,s)$ decreases with respect to $s$
    \item $\lim\limits_{s \to \infty} \beta(r,s) = 0$
\\[2.5ex]
  \end{enumerate}
\end{definition}
%\end{bw_box}



% Barbalat's lemma
%===============================================================================
%\begin{bw_box}
  \begin{lemma} \cite{Fontes2007} (\textit{A modification of Barbalat's lemma})
  \label{lemma:barbalat}

    Let $f$ be a continuous, positive-definite function, and $\vect{x}$ be an
    absolutely continuous function in $\mathbb{R}$. If the following hold:
  \begin{itemize}
    \item $\|\vect{x}(\cdot)\| < \infty$
    \item $\|\dot{\vect{x}}(\cdot)\| < \infty$
    \item $\lim\limits_{t \to \infty} \int\limits_0^t f\big(\vect{x}(s)\big)ds < \infty$
  \end{itemize}
  then $\lim\limits_{t \to \infty} \|\vect{x}(t)\| = 0$
\\[2.5ex]
  \end{lemma}
%\end{bw_box}

\newpage
% ISS stability definition
%===============================================================================
%\begin{bw_box}
\begin{definition}\cite{marquez2003nonlinear} (\textit{Input-to-State Stability})
\label{def:ISS}

A nonlinear system $\dot{\vect{x}} = f(\vect{x},\vect{u})$, $\vect{x} \in X$,
$\vect{u} \in U$ with initial condition $\vect{x}(t_0)$ is said
to be \textit{locally Input-to-State Stable (ISS)} if there exist functions
$\sigma \in \mathcal{K}$ and $\beta \in \mathcal{KL}$
and constants $k_1, k_2 \in \mathbb{R}_{> 0}$ such that
\begin{align}
  \|\vect{x}(t)\| \leq \beta\big(\|\vect{x}(t_0)\|,t\big) + \sigma\big(\|\vect{u}\|_{\infty}\big),\ \forall t \geq 0
\end{align}
for all $\vect{x}(t_0) \in X$ and $\vect{u} \in U$ satisfying $\|\vect{x}(t_0)\| \leq k_1$
and $\text{sup}\limits_{t > 0} \|\vect{u}(t)\| = \|\vect{u}\|_{\infty} \leq k_2$.
\\[2.5ex]
\end{definition}
%\end{bw_box}



% remark on ISS stability definition
%===============================================================================
%\begin{bw_box}
\begin{remark} \cite{1185106}
\label{def:ISS_remark}
  A nonlinear system $\dot{\vect{x}} = f(\vect{x},\vect{u})$, $\vect{x} \in X$,
$\vect{u} \in U$ which is input-to-output stable, is asymptotically stable
in the absence of disturbances $\vect{u}$, or if the disturbance is decaying.
If the disturbance is merely bounded, then the evolution of the system is
\textit{ultimately bounded} in a set whose size depends on the bound of the
disturbance.
\\[2.5ex]
\end{remark}
%\end{bw_box}



% ISS Lyapunov function definition
%===============================================================================
%\begin{bw_box}
\begin{definition}\cite{marquez2003nonlinear} (\textit{ISS Lyapunov function})
\label{def:ISS_Lyapunov}

A continuous function $V(\vect{x}) : \Psi \to \mathbb{R}_{\geq 0}$ for the
nonlinear system $\dot{\vect{x}} = f(\vect{x}, \vect{\delta})$ is said to be a
\textit{ISS Lyapunov function} in $\Psi$ if there are class
$\mathcal{K}$ functions
$\alpha_1$, $\alpha_2$, $\alpha_3$ and $\sigma$, such that
\begin{align}
  \alpha_1\big(\|\vect{x}\|\big) \leq V\big(\vect{x}\big) \leq \alpha_2\big(\|\vect{x}\|\big),\ \forall \vect{x} \in \Psi
\end{align}
and
\begin{align}
  \dfrac{d}{dt}V\big(\vect{x}\big)
    \leq \sigma\big(\|\vect{\delta}\|\big) - \alpha_3\big(\|\vect{x}\|\big),\ \forall \vect{x} \in \Psi, \vect{\delta} \in \Delta
\end{align}
\\[2.5ex]
$V$ is an ISS Lyapunov function if $\Psi = \mathbb{R}^n$, $\Delta = \mathbb{R}^m$,
and $\alpha_1$, $\alpha_2$, $\alpha_3$, $\sigma$ $\in \mathcal{K}_{\infty}$.
\end{definition}
%\end{bw_box}


\newpage
% ISS Lyapunov function remark
%===============================================================================
%\begin{bw_box}
\begin{remark}
\label{remark:ISS_Lyapunov}

With regard to definition \eqref{def:ISS_Lyapunov}, the statement
\begin{align}
  \dfrac{d}{dt}V\big(\vect{x}\big)
    \leq \sigma\big(\|\vect{\delta}\|\big) - \alpha_3\big(\|\vect{x}\|\big),\ \forall \vect{x} \in \Psi, \vect{\delta} \in \Delta
\end{align}
is equivalent to
\begin{align}
  V\big(\vect{x}(t_1)\big) - V\big(\vect{x}(t_0)\big)
  \leq \int_{t_0}^{t_1}\Big(\sigma\big(\|\vect{\delta}(t)\|\big) - \alpha_3\big(\|\vect{x}(t)\|\big)\Big)dt,\
    \forall \vect{x} \in \Psi,\ \vect{\delta} \in \Delta,\ t \in [t_0, t_1]
\end{align}
\\[2.5ex]
\end{remark}
%\end{bw_box}


% ISS Lyapunov <==> ISS
%===============================================================================
%\begin{bw_box}
\begin{theorem}\cite{marquez2003nonlinear}
\label{def:ISS_Lyapunov_admit_theorem}

  A nonlinear system $\dot{\vect{x}} = f(\vect{x},\vect{u})$ is said
  to be \textit{Input-to-State Stable} in $\Psi$ if and only if it admits an
  ISS Lyapunov function in $\Psi$.
\\[2.5ex]
\end{theorem}
%\end{bw_box}


% Positively invariant set
%===============================================================================
%\begin{bw_box}
  \begin{definition} (\textit{Positively Invariant Set})
    \label{def:positively_invariant}

    Consider a dynamical system $\dot{\vect{x}} = f(\vect{x})$,
    $\vect{x} \in \mathbb{R}^n$, and a trajectory $\vect{x}(t;\ \vect{x}_0)$,
    where $\vect{x}_0$ is the initial condition. The set
    $S = \{\vect{x} \in \mathbb{R}^n : \gamma(\vect{x}) = 0\}$, where
    $\gamma$ is a valued function, is said to be \textit{positively invariant}
    if the following holds:
    \begin{align}
      \vect{x}_0 \in S \Rightarrow \vect{x}(t;\ \vect{x}_0) \in S,\ \forall t \geq t_0
    \end{align}

    Intuitively, this means that the set $S$ is positively invariant if a
    trajectory of the system does not exit it once it enters it. \\[2.5ex]
  \end{definition}
%\end{bw_box}


% Robust positively invariant set
%===============================================================================
%\begin{bw_box}
\begin{definition}\cite{ISS_SKATOLINI} (\textit{Robust positively-invariant set})
\label{def:robust_positively_invariant_set}

A set $\Psi \in \mathbb{R}^n$ is a robust positively invariant set for the
nonlinear system $\dot{\vect{x}} = f(\vect{x}, \vect{\delta})$ if
$f(\vect{x}, \vect{\delta}) \in \Psi$, for all $\vect{x} \in \Psi$ and for all
$\vect{\delta} \in \Delta$.
\\[2.5ex]
\end{definition}
%\end{bw_box}


%% Robust control invariant set
%%===============================================================================
%%\begin{bw_box}
%\begin{definition}\cite{1024831} (\textit{Robust control invariant set})
%\label{def:robust_control_invariant_set}

%A set $\Psi \in \mathbb{R}^n$ is a robust control invariant set for the
%nonlinear system $\dot{\vect{x}} = f(\vect{x}, \vect{u}) + \vect{\delta}$,
%$\vect{x} \in X, \vect{u} \in U, \vect{\delta} \in \Delta$, if
%$\Psi \subseteq X$ and for all $\vect{x} \in \Omega$ there is an admissible
%control action $\vect{u} = \vect{h}(\vect{u}) \in U$ such that
%$f(\vect{x}, \vect{u}) + \vect{\delta} \in \Omega$ for all
%$\vect{\delta} \in \Delta$.
%\\[2.5ex]
%\end{definition}
%%\end{bw_box}

The proofs of lemmas or properties whose reference is not cited are provided
in appendix \ref{chapter:proofs_of_lemmas}.
