\section{Process \& information flow}

The designed procedure flow can be either concurrent or sequential, meaning
that agents can solve their individual FHOCP's and apply the control inputs
either at the same time as everyone else, or one by one. The conceptual
design itself is procedure-flow agnostic, and hence it can accommodate both
without loss of feasibility or successful stabilization. For lack of a stable
and usable multi-threaded \texttt{MATLAB} framework, the simulations where
carried out under the latter approach: each agent solves its own FHOCP and
applies the corresponding admissible control input in a round robin way,
considering the current and planned (open-loop state predictions) configurations
of all agents within its sensing range. Figures \eqref{fig:process_flow}
and \eqref{fig:information_flow} depict the sequential procedural and
informational regimes.\\[2ex]

\begin{figure}[ht]\centering
  \scalebox{0.9}{% Define block styles
\tikzstyle{decision} = [diamond, draw, %fill=blue!20,
    text width=7.5em, text badly centered, node distance=3cm, inner sep=0pt]
\tikzstyle{block} = [rectangle, draw, %fill=blue!20,
    text width=7em, text centered, minimum height=4em]
\tikzstyle{line} = [draw, -latex']
\tikzstyle{cloud} = [draw, ellipse,f %ill=red!20,
    node distance=3cm, minimum height=2em]

\begin{tikzpicture}[node distance = 2cm, auto]
    % Place nodes
    \node [decision] (decide) {Current configuration feasible?};
    \node [block, below of=decide, node distance=5.5cm] (stop) {stop};

    \node [block, right of=decide, node distance=5cm] (solve_1) {Agent 1: solve FHOCP};
    \node [block, below of=solve_1, node distance=2.5cm] (solve_2) {Agent 2: solve FHOCP};
    \node [block, below of=solve_2, node distance=3cm] (solve_N) {Agent N: solve FHOCP};

    \node [block, right of=solve_1, node distance=5cm] (solve_1_) {Agent 1: solve FHOCP};
    \node [block, below of=solve_1_, node distance=2.5cm] (solve_2_) {Agent 2: solve FHOCP};
    \node [block, below of=solve_2_, node distance=3cm] (solve_N_) {Agent N: solve FHOCP};


    \node [above of=decide, node distance=3cm] (step_0) {STEP 0};
    \node [above of=solve_1, node distance=3cm] (step_1) {STEP 1};
    \node [above of=solve_1_, node distance=3cm] (step_1_) {STEP 2};


    % invisible nodes
    \node[inner sep=0,minimum size=0,right of=solve_N, node distance=2.5cm] (sN) {};
    \node[inner sep=0,minimum size=0,left of=solve_1_, node distance=2.5cm] (s1_) {};
    \node[inner sep=0,minimum size=0,right of=solve_N_, node distance=2.5cm] (sN_) {};
    \node[inner sep=0,minimum size=0,right of=solve_1_, node distance=2.5cm] (s1_inv) {};
    \node[right of=solve_1_, node distance=3.5cm] (solve_1_inv) {$\dots$};


    \path [line]        (decide)  -- node {no}(stop);
    \path [line]        (decide)  -- node {yes}(solve_1);
    \path [line]        (solve_1) -- node {}(solve_2);
    \path [line,dashed] (solve_2) -- node {}(solve_N);
    \path [line]        (solve_1_) -- node {}(solve_2_);
    \path [line,dashed] (solve_2_) -- node {}(solve_N_);

    \path [line] (solve_N) -- (sN);
    \path [line] (sN) -- (s1_);
    \path [line] (s1_) -- (solve_1_);

    \path [line] (solve_N_) -- (sN_);
    \path [line] (sN_) -- (s1_inv);
    \path [line] (s1_inv) -- (solve_1_inv);

\end{tikzpicture}
}
  \caption{The procedure is approached sequentially. Notice that the
    figure implies that recursive feasibility is established if the initial
    configuration is itself feasible.}
  \label{fig:process_flow}
\end{figure}

\begin{figure}[ht]\centering
  \scalebox{0.9}{% Define block styles
\tikzstyle{decision} = [diamond, draw, %fill=blue!20,
    text width=7.5em, text badly centered, node distance=3cm, inner sep=0pt]
\tikzstyle{block} = [rectangle, draw, %fill=blue!20,
    text width=10em, text centered, minimum height=4em]
\tikzstyle{block_rounded} = [rectangle, rounded corners, draw, %fill=blue!20,
    text width=12em, text centered, minimum height=4em]
\tikzstyle{line} = [draw, -latex']
\tikzstyle{cloud} = [draw, ellipse, %fill=red!20,
    node distance=3cm, minimum height=4em, minimum width=4em]

\begin{tikzpicture}[node distance = 2cm, auto]

    % Place nodes

    \node [block, node distance=1.5cm] (system_m) {Agent $m \in \mathcal{R}_i(t_k)$};
    \node [block, below of=system_m, node distance=2.5cm] (system_n) {Agent $n \in \mathcal{R}_i(t_k)$};

    % invisible node
    \node[right of=system_n, node distance=4.5cm] (right_of_n){};

    \node [block_rounded, below of=right_of_n, node distance=2.5cm] (latest_plans) {Latest predictions (current timestep $t_k$)};
    \node [block, below of=latest_plans, node distance=2.5cm] (agent_i) {Agent $i$};
    \node [block_rounded, below of=agent_i, node distance=2.5cm] (latest_plans_) {Latest predictions \\ (previous timestep $t_{k-1}$)};

    % invisible node
    \node[left of=latest_plans_, node distance=4.5cm] (left_of_i){};

    \node [block, below of=left_of_i, node distance=2.5cm] (system_p) {Agent $p \in \mathcal{R}_i(t_k)$};
    \node [block, below of=system_p, node distance=2.5cm] (system_q) {Agent $q \in \mathcal{R}_i(t_k)$};

    \node [above of=system_m, node distance=1.5cm] (dots_0) {$\vdots$};
    \node [below of=system_q, node distance=1.5cm] (dots_1) {$\vdots$};


    % invisible nodes
    \node[inner sep=0,minimum size=0,right of=system_m, node distance=5.5cm] (sm) {};
    \node[inner sep=0,minimum size=0,right of=system_n, node distance=4.5cm] (sn) {};
    \node[inner sep=0,minimum size=0,right of=system_p, node distance=4.5cm] (sp) {};
    \node[inner sep=0,minimum size=0,right of=system_q, node distance=5.5cm] (sq) {};

    % paths
    \path[line]           (system_m) -- (sm);
    \path[line]           (system_n) -- (sn);
    \path[line]           (sm) -- (latest_plans.38);
    \path[line]           (sn) -- (latest_plans);

    \path[line]           (system_p) -- (sp);
    \path[line]           (system_q) -- (sq);
    \path[line]           (sp) -- (latest_plans_);
    \path[line]           (sq) -- (latest_plans_.-38);

    \path[line,dashed]    (latest_plans) -- (agent_i);
    \path[line,dashed]    (latest_plans.-38) -- (agent_i.38);

    \path[line,dashed]    (latest_plans_) -- (agent_i);
    \path[line,dashed]    (latest_plans_.38) -- (agent_i.-38);

    %\path [line]        (decide)  -- node {NO}(stop);
    %\path [line]        (decide)  -- node {YES}(solve_1);
    %\path [line]        (solve_1) -- node {}(solve_2);
    %\path [line,dashed] (solve_2) -- node {}(solve_N);
    %\path [line]        (solve_1_) -- node {}(solve_2_);
    %\path [line,dashed] (solve_2_) -- node {}(solve_N_);

    %\path [line] (solve_N) -- (sN);
    %\path [line] (sN) -- (s1_);
    %\path [line] (s1_) -- (solve_1_);

    %\path [line] (solve_N_) -- (sN_);
    %\path [line] (sN_) -- (s1_inv);
    %\path [line] (s1_inv) -- (solve_1_inv);

\end{tikzpicture}
}
  \caption{The flow of information to agent $i$ regarding his perception of
    agents within its sensing range $\mathcal{R}_i$ at arbitrary FHOCP
    solution time $t_k$. Agents $m,n \in \mathcal{R}_i(t_k)$ have solved their
    FHOCP; agent $i$ is next; agents $p,q \in \mathcal{R}_i(t_k)$ have not
    solved their FHOCP yet.}
  \label{fig:information_flow}
\end{figure}
