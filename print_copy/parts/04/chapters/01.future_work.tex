\chapter{Future work}

Future work as I understand it should begin by visiting these two independent
(but not necessarily independent) topics:
(A) Since simulations are not actual experiments, the designed control regimes
appear to be less falsifiable than they could possibly be. Which alludes to
implementing the two regimes in real life and testing it on real agents $-$ the
motivating problem in itself is posed under the consideration of robotic
holonomic agents that operate in physical 3D space. Practice will measure the
non-falsifiability of theory.
(B) The equivalent of this problem $-$ in relative nature $-$ is instead of
handing each agent a desired set point, a neighbour set, and constraints on
their distances to their neighbours $-$ instead of these, giving them a
relative configuration they should hold (relative set points) with each other,
and drive the compound system(s) of neighbours to a configuration where at most
only one agents should receive a set point, and formation is achieved in
terms of relative agent configurations. This I think may be a relevant problem
and one that is more difficult in nature and implications to approach. Lastly,
and as a side quest, the answer to the question of what happens to the
trajectory of the compound system if an agent breaks down mid-way would be
beneficial to be given, as the current regime does not, and cannot, handle
disturbances of this sort.
