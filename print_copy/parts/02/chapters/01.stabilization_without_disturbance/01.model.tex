The purpose to be saught in this chapter is the steering of each agent
$i \in \mathcal{V}$ into a desired configuration in 12D space while conforming
to the requirements posed by the problem. Here, the real system and its model
are equivalent: no model-reality mismatches exist, and no disturbances act on
the real system. At first, the model of system \eqref{eq:system} will be
formalized. Next, the error model and \textit{its} constraints will be
expressed. Then, the optimization problem to be solved periodically will be
posed; it will equip us with the optimum feasible input that steers the system
towards achieving its intended collision-free steady-state configuration while
avoiding the pitfall of violating its constraints. This will lead to the proof
of stability of the compound closed-loop system of agents $i \in \mathcal{V}$
under the proposed control regime.

%; that is, all agents should avoid colliding with each other, all
%obstacles in the workspace, and the workspace boundary itself, while remaining
%in a non-singular configuration and sustaining the connectivity to their
%respective neighbours.

%-------------------------------------------------------------------------------
\section{Formalizing the system's model}

We begin by rewriting the decoupled non-linear time-invariant system equations
\eqref{eq:system_1}, \eqref{eq:system_2} for a generic agent
$i \in \mathcal{V}$ in state-space form:
\begin{subequations}
\begin{align}
  \dot{\vect{x}}_i(t) &= \mat{J}_i^{-1}(\vect{x}_i) \vect{v}_i(t) \\[2.5ex]
  \dot{\vect{v}}_i(t) &= -\mat{M}_i^{-1}(\vect{x}_i)\mat{C}_i(\vect{x}_i,\dot{\vect{x}}_i) \vect{v}_i(t)
    - \mat{M}_i^{-1}(\vect{x}_i)\vect{g}_i(\vect{x}_i)
    + \mat{M}_i^{-1}(\vect{x}_i)\vect{u}_i(t)
\end{align}
\label{eq:state_space_system}
\end{subequations}
where the inversion of $\mat{M}_i$ is possible due to it being
positive-definite $\forall i \in \mathcal{V}$. Denoting by $\vect{z}_i(t)$
\begin{align}
  \vect{z}_i(t) \triangleq
    \begin{bmatrix}
      \vect{x}_i(t) \\
      \vect{v}_i(t)
    \end{bmatrix},\
    \vect{z}_i(t) : \mathbb{R}_{\geq 0} \to \mathbb{R}^9 \times \mathbb{T}^3
\end{align}
and
$\dot{\vect{x}}_i(t)$ and $\dot{\vect{v}}_i(t)$ by
\begin{subequations}
\begin{align}
  \dot{\vect{x}}_i(t) &= f_{i,x}(\vect{z}_i, \vect{u}_i) \\[2.5ex]
  \dot{\vect{v}}_i(t) &= f_{i,v}(\vect{z}_i, \vect{u}_i)
\end{align}
\end{subequations}
we get the compact representation of the system's model
\begin{align}
  \dot{\vect{z}}_i(t) =
    \begin{bmatrix}
      f_{i,x}(\vect{z}_i, \vect{u}_i) \\
      f_{i,v}(\vect{z}_i, \vect{u}_i)
    \end{bmatrix} =
 f_i \big(\vect{z}_i (t), \vect{u}_i (t)\big)
\end{align}
The state evolution of agent $i$ is modeled by a system of non-linear
continuous-time differential equations of the form
\begin{align}
  \dot{\vect{z}}_i(t) &= f_i \big(\vect{z}_i (t), \vect{u}_i (t)\big) \label{eq:non_perturbed_system}\\[2.5ex]
  \vect{z}_i(0) &= \vect{z}_{i,0} \\[2.5ex]
  \vect{z}_i (t) & \subset \mathbb{R}^{9} \times \mathbb{T}^3 \\[2.5ex]
  \vect{u}_i (t) & \subset \mathbb{R}^6
  \label{eq:original_z_system}
\end{align}
where state $\vect{z}_i$ is directly measurable as per assumption
\eqref{ass:measurements_access}.

We define the set $\mathcal{Z}_{i,t} \subset \mathbb{R}^{9} \times \mathbb{T}^3$
as the set that captures all the \textit{state} constraints on the system
posed by the problem \eqref{problem} at $t \in \mathbb{R}_{\geq 0}$.
Therefore $\mathcal{Z}_{i,t}$ is such that:
\begin{align}
  \mathcal{Z}_{i,t} \triangleq \big\{\vect{z}_i(t) \in \mathbb{R}^{9}\times \mathbb{T}^3 : \
      & \|\vect{p}_i(t) - \vect{p}_j(t)\| > \underline{d}_{ij,a}, \forall j \in \mathcal{R}_i(t), \label{constraint:p_1}\\[2.5ex]
      & \|\vect{p}_i(t) - \vect{p}_j(t)\| < d_i, \forall j \in \mathcal{N}_i, \\[2.5ex]
      & \|\vect{p}_i(t) - \vect{p}_{\ell}\| > \underline{d}_{i\ell,o}, \forall \ell \in \mathcal{L}, \\[2.5ex]
      & \|\vect{p}_W - \vect{p}_i(t)\| < \overline{d}_{i,W}, \\[2.5ex]
      & - \frac{\pi}{2} < \theta_i(t) < \frac{\pi}{2} \label{constraint:p_5}\big\}
\end{align}
